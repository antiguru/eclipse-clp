% BEGIN LICENSE BLOCK
% Version: CMPL 1.1
%
% The contents of this file are subject to the Cisco-style Mozilla Public
% License Version 1.1 (the "License"); you may not use this file except
% in compliance with the License.  You may obtain a copy of the License
% at www.eclipse-clp.org/license.
%
% Software distributed under the License is distributed on an "AS IS"
% basis, WITHOUT WARRANTY OF ANY KIND, either express or implied.  See
% the License for the specific language governing rights and limitations
% under the License.
%
% The Original Code is  The ECLiPSe Constraint Logic Programming System.
% The Initial Developer of the Original Code is  Cisco Systems, Inc.
% Portions created by the Initial Developer are
% Copyright (C) 2006 Cisco Systems, Inc.  All Rights Reserved.
%
% Contributor(s):
%
% END LICENSE BLOCK
%
% $Id: umsexcept.tex,v 1.5 2013/02/09 20:27:58 jschimpf Exp $
%
% Created from umsexcept.tex
%

%----------------------------------------------------------------------
\chapter{Events and Interrupts}
%HEVEA\cutdef[1]{section}
%----------------------------------------------------------------------
\label{chapexcept}

The normal execution of a Prolog program may be interrupted by
events and interrupts:

\begin{quote}
\begin{description}
\item[Events]\mbox{}\\
  Events have the following properties:
    \begin{itemize}
    \item they may occur asynchronously (when posted by the environment)
        or synchronously (when raised by the program itself);
    \item they are handled synchronously by a handler goal that is inserted
        into the resolvent;
    \item the handler can cause the interrupted execution to fail or to abort;
    \item the handler can interact with the interrupted execution only via
        non-logical features (e.g., global variable or references);
    \item the handler can cause waking of delayed goals via symbolic triggers.
    \end{itemize}

\item[Errors]\mbox{}\\
    Errors can be viewed as a special case of events. They are raised by
    built-in predicates (e.g., when the arguments are of the wrong type)
    and usually pass the culprit goal to the error handler.

\item[Interrupts]\mbox{}\\
    Interrupts usually originate from the operating system, e.g., on a Unix
    host, signals are mapped to {\eclipse} interrupts.
    \begin{itemize}
    \item they occur asynchronously, but may be mapped into a sychronous event;
    \item certain predefined actions (like aborting) can be performed
        asynchronously.
%    \item in Unix, the handler can be executed asynchronously in a separate
%    {\eclipse} engine. This means that
%    \item the handler cannot interact with interrupted execution, except via
%       global variables, files and the like.
%    \item failure of the handler is ignored.
%    \item the development system catches and handles many operating system
%       signals as interrupts, user abort by typing \verb.^.C,
%       data arriving at sockets, memory protection faults, etc.
    \end{itemize}
\end{description}
\end{quote}


%----------------------------------------------------------------------
\section{Events}
%----------------------------------------------------------------------
\index{events}

\subsection{Event Identifiers and Event Handling}

Events are identified by names (atoms) or by anonymous handles.

When an event is raised, a call to the appropriate handler is inserted
into the resolvent (the sequence of executing goals).
The handler will be executed as soon as possible, which means at the
next synchronous point in execution, which is usually just before the
next regular predicate is invoked. Note that there are a few
built-in predicates that can run for a long time and will not allow
handlers to be executed until they return (e.g., \predspec{read/1},
\predspec{sort/4}).


\subsubsection{Creating Named Events}

A named event is created by defining a handler for it using
\bipref{set_event_handler/2}{../bips/kernel/event/set_event_handler-2.html}:
\begin{quote}
\begin{verbatim}
:- set_event_handler(hello, my_handler/1).
my_handler(Event) :-
    <code to deal with Event>
\end{verbatim}
\end{quote}
A handler for a named event can have zero or one arguments. When invoked,
the first argument is the event identifier, in this case the atom
\notation{hello}.
It is not possible to pass other information to the handler.

The handler for a defined event can be queried using
\bipref{get_event_handler/3}{../bips/kernel/event/get_event_handler-3.html}.


\subsubsection{Creating Anonymous Events}

An anonymous event is created with the built-in
\bipref{event_create/3}{../bips/kernel/event/event_create-3.html}:
\begin{quote}
\begin{verbatim}
..., event_create(my_other_handler(...), [], Event), ...
\end{verbatim}
\end{quote}
The built-in takes a handler goal and creates an anonymous event handle Event.
This handle is the only way to identify the event, and therefore must be
passed to any program location that wants to raise the event.
The handler goal can be of any arity and can take arbitrary arguments.
Typically, these arguments would include the Event handle itself and other
ground arguments (variables should not be passed because when the event
is raised, a copy of the handler goal with fresh variables will be executed).


\subsection{Raising Events}
Events can be raised in the following different ways:
\begin{itemize}
\item Explicitly by the {\eclipse} program itself, using
    \bipref{event/1}{../bips/kernel/event/event-1.html}.
\item By foreign code (C/C++) using the \notation{ec_post_event()} function.
\item Via signals/interrupts by setting the interrupt handler to
    \bipref{event/1}{../bips/kernel/event/event-1.html}.
\item Via I/O streams (e.g., queues can be configured to raise an event
    when they get written into).
\item Via timers, so-called after-events
\end{itemize}

\subsubsection{Raising Events Explicitly}
To raise an event from within {\eclipse} code, call
\bipref{event/1}{../bips/kernel/event/event-1.html} with the event
identifier as its argument. If no handler has been defined, a warning
will be raised:
\begin{quote}
\begin{verbatim}
?- event(hello).
WARNING: no handler for event in hello
Yes (0.00s cpu)
\end{verbatim}
\end{quote}
The event can be an anonymous event handle, e.g.,
\begin{quote}
\begin{verbatim}
?- event_create(writeln(handling(E)), [], E), event(E).
handling('EVENT'(16'edbc0b20))
E = 'EVENT'(16'edbc0b20)
Yes (0.00s cpu)
\end{verbatim}
\end{quote}
Raising events explicitly is mainly useful for test purposes, since
it is almost the same as calling the handler directly.

\subsubsection{Raising Events from Foreign Code}
To raise an event from within foreign C/C++ code, call
\begin{quote}
\begin{verbatim}
ec_post_event(ec_atom(ec_did("hello",0)));
\end{verbatim}
\end{quote}
This works both when the foreign code is called from {\eclipse} or when
{\eclipse} is embedded into a foreign code host program.


\subsubsection{Timed Events (``after'' events)}
\index{after events}
\index{timers}
\index{timed events}

An event can be triggered after a specified amount
of elapsed time. The event is then handled sychronously by {\eclipse}.
These events are known as ``after''
events, as they are set up so that the event occurs \emph{after} a
certain amount of elapsed time.
They are setup by one of the following predicates:

\begin{quote}
\begin{description}
\item[\biptxtref{event_after(+\pattern{EventId},~+\pattern{Time})}%
{event_after/2}{../bips/kernel/event/event_after-2.html}]
This sets up an event \about{EventId} so that the event is raised once after
\about{Time} seconds
of elapsed time from when the predicate is executed. \about{EventId} is an event
identifier and \about{Time} is a positive number.

\item[\biptxtref{event_after_every(+\pattern{EventId},~+\pattern{Time})}%
{event_after_every/2}{../bips/kernel/event/event_after_every-2.html}]
This sets up an event \about{EventId} so
that the event is raised repeatedly every \about{Time} seconds: first
\about{Time} seconds after the invocation of the predicate, then \about{Time}
seconds after that event was raised, and so on.

\item[\biptxtref{events_after(+\pattern{EventList})}%
{events_after/1}{../bips/kernel/event/events_after-1.html}]
This sets up a series of after events specified in the list \about{EventList},
which contains events of the form \pattern{EventId-Time}, or
\pattern{EventId-every(Time)} (specifying a single
event or a repeated event respectively).

The \about{Time} parameter is actually the minimum of elapsed time before the
event is raised. Factors constraining the actual time of raising of the
event include the granularity of the system clock, and also that {\eclipse}
must be in a state where it can \emph{synchronously} process the event, i.e.,
where it can make a procedure call.

Once an after event has been set up, it is pending until it is raised. In
the case of events caused by an invocation of \predspec{event_after_every/2},
the event will always be pending
because it is raised repeatedly. A pending event can be cancelled so that
it will not be raised.


\item[\biptxtref{%
cancel_after_event(+\pattern{EventId},~-\pattern{Cancelled})}%
{cancel_after_event/2}{../bips/kernel/event/cancel_after_event-2.html}]
This finds and cancels all pending after events with name \about{EventId} and
returns the actually cancelled ones in a list.

\item[\biptxtref{current_after_events(-\pattern{Events})}%
{current_after_events/1}{../bips/kernel/event/current_after_events-1.html}]
This returns a list of all pending after events.
\end{description}
\end{quote}

The after event mechanism allows multiple events to make use of the timing
mechanism independently of each other. The same event can be setup
multiple times with multiple calls to \predspec{event_after/2} and
\predspec{event_after_every/2}. The \predspec{cancel_after_event/2}
predicate will cancel all instances of an event.

By default, the after event feature uses the \notation{real} timer. The
timer can be switched to the \notation{virtual} timer, in which case the
elapsed time measured is user CPU time.\footnote{%
  This is time that the CPU spends on executing user code, i.e., the {\eclipse}
  program.}
This  setting is specified by the {\eclipse} environment flag
\notation{after_event_timer} (see \predspec{get_flag/2},
\predspec{set_flag/2}). Note that if the
timer is changed while some after event is still pending, these events
will no longer be processed. The timer should therefore not be changed
once after events are initiated.

Currently, the \notation{virtual} timer is not available on the Windows
platform. In addition, the users should not make use of these
timers for their own purposes if they plan to use the after event
mechanism.


\subsection{Events and Waking}

Using the suspension and event handling mechanisms together, a goal can be
added to the resolvent and executed after a defined elapsed time.
To achieve this, the goal is suspended and attached to a symbolic
trigger, which is triggered by an afer-event handler.  The goal behaves
``logically'', in that if the execution backtracks pass the point in which
the suspended goal is created, the goal will disappear from the resolvent
as expected and thus not be executed. The event will still be raised, but
there will not be a suspended goal to wake up. Note that if the execution
finishes before the suspended goal is due to be woken up, it will also not
enter the resolvent and will thus not be executed.

The following is an example of waking a goal with a timed event.
Once \verb+monitor(X)+ is called, the current value of X will be
printed every second until the query finishes or is backtracked over:
\begin{quote}
\begin{verbatim}
:- set_event_handler(monvar, trigger/1).

monitor(Var) :-
     suspend(m(Var), 3, trigger(monvar)),
     event_after_every(monvar, 1).

:- demon m/1.
m(Var) :- writeln(Var).

:- monitor(Var), <do_something>.
\end{verbatim}
\end{quote}
Note the need to declare
\predspec{m/1} as a demon: otherwise, once \predspec{m/1} is woken up once, it
will
disappear from the resolvent and the next \notation{monvar} event will not have
a suspended \predspec{m/1} to wake up.
Note also that it is necessary to connect the event machanism to
the waking mechanism by setting the event handler to
\bipref{trigger/1}{../bips/kernel/suspensions/trigger-1.html}.


\subsection{Aborting an Execution with Events}

Typically, event handlers would perform some action and then succeed,
letting the interrupted exectuion continue unharmed. Event handlers for
asynchronous events should never fail, because the failure will be inserted
in a random place in the resolvent, and the effect will be unpredictable.
It is however sometimes useful to allow an asynchronous event to abort
an execution (via
\bipref{throw/1}{../bips/kernel/control/throw-1.html}), e.g.,
to implement timeouts.\footnote{%
  Since implementing reliable timeouts is a
  nontrivial task, we recommend the use of
  \bipref{lib(timeout)}{../bips/lib/timeout/index.html} for this purpose.}

When dealing with events that occur asynchronously (in particular after-events),
and event handlers that cause the execution to abort, it is often a problem
that event handlers may be interrupted or preempted by other event handlers.
This can be avoided by use of the event-defer mechanism. An event can be
declared with the defer-property, which means that all further event handling
is temporarily suppressed as soon as the handling of this event begins.
In this case, the event handler is responsible for reenabling event handling
explicitly before returning by calling
\bipref{events_nodefer/0}{../bips/kernel/event/events_nodefer-0.html}.
For instance:
\begin{quote}
\begin{verbatim}
:- set_event_handler(my_event, defers(my_handler/0)).
my_after_handler :-       % event handling is deferred at this point
        <deal with event>,
        events_nodefer.   % allow other events to be handled again
\end{verbatim}
\end{quote}
In the presence of other event handlers which can cause aborts, this will
protect the handler code from being preempted.


%----------------------------------------------------------------------
\section{Errors}
%----------------------------------------------------------------------
\index{errors}
Error handling is one particular use of events.
The main property of error events is that they have a culprit goal,
i.e., the goal that detected or caused the error.
The error handler obtains that goal as an argument.

The errors that the system raises have numerical identifiers,
as documented in appendix \ref{chaperrors}.
User-defined errors have atomic names, they are the same as events.
Whenever an error occurs, the {\eclipse} system identifies the type of error,
and
calls the appropriate handler. For each type of error, it is possible
for the user to define a separate handler. This definition will replace
the default error handling routine for that particular error---all other
errors will still
be handled by their respective
handlers.
It is of
course possible to associate the same user defined error handler to more
than one error type.

When a goal is called and produces an error, execution of the goal
is aborted and the appropriate error handler is invoked.
This invocation of the error handler is seen as \emph{replacing}
the invocation of the erroneous goal:
\begin{itemize}
\item if the error handler fails it has the same effect as if the
erroneous goal failed;
\item if the error handler succeeds, possibly binding some variables,
the execution continues at the point behind the call of the erroneous goal;
\item if the handler calls \bipref{throw/1}%
{../bips/kernel/control/throw-1.html}, it has the same effect as if
this was done by the erroneous goal itself.
\end{itemize}
For errors that are classified as warnings the second point is somewhat
different: if the handler succeeds, the goal that raised the warning
is allowed to continue execution.

Apart from binding variables in the erroneous goal, error handlers can
also leave backtrack points. However, if the error was raised by
an external or a built-in that is implemented as an external, these
choicepoints are discarded.\footnote{%
  This is necessary because the compiler recognises simple predicates as
  deterministic at compile time and so if a simple predicate were to cause the
  invocation of a non-deterministic error handler, the generated code might no
  longer be correct.}


\subsection{Error Handlers}
The predicate \bipref{set_event_handler/2}%
{../bips/kernel/event/set_event_handler-2.html} is used to assign a procedure as
an error handler. The call
\begin{quote}
\begin{verbatim}
set_event_handler(ErrorId, PredSpec)
\end{verbatim}
\end{quote}
sets the event handler for error type \about{ErrorId} to the procedure specified
by \about{PredSpec}, which must be of the form \pattern{Name/Arity}.

The corresponding predicate
\bipref{get_event_handler/3}{../bips/kernel/event/get_event_handler-3.html}
may be used to identify
the current handler for a particular error. The call
\begin{quote}
\begin{verbatim}
get_event_handler(ErrorId, PredSpec, HomeModule)
\end{verbatim}
\end{quote}
will, provided \about{ErrorId} is a valid error identifier, unify
\about{PredSpec}
with the
specification of the current handler for error \about{ErrorId} in the form
\pattern{Name/Arity}, and \about{HomeModule} will be unified with the module
where the
error handler has been defined. Note that this error handler might not be
visible from every module and therefore may not be callable.

To re-install the system's error handler in case the user error handler is
no longer needed,
\bipref{reset_event_handler/1}%
{../bips/kernel/event/reset_event_handler-1.html}
should be used.
\bipref{reset_error_handlers/0}%
{../bips/kernel/event/reset_error_handlers-0.html} resets all error handlers to
their default values.

To enable the user to conveniently write predicates with error checking
the built-ins
\indextt{error/2}
\indextt{error/3}
\begin{quote}
\begin{verbatim}
error(ErrorId, Goal)
error(ErrorId, Goal, Module)
\end{verbatim}
\end{quote}
are provided to raise the error \about{ErrorId} (an error number or a name atom)
with the culprit \about{Goal}.
Inside tool procedures it is usually necessary to use
\bipref{error/3}{../bips/kernel/event/error-3.html}
in order to pass the caller module to the error handler.
Typical error checking code looks like this
\begin{quote}
\begin{verbatim}
increment(X, X1) :-
        ( integer(X) ->
            X1 is X + 1
        ;
            error(5, increment(X, X1))
        ).
\end{verbatim}
\end{quote}

The predicate
\bipref{current_error/1}{../bips/kernel/event/current_error-1.html}
can be used to yield all valid error numbers, a valid error is that one
to which an error message and an error handler are associated.
The predicate \bipref{error_id/2}{../bips/kernel/event/error_id-2.html}
gives the corresponding error message to the specified error number.
To ease the search for the appropriate error number,
the library \notation{util} contains the predicate
\indextt{list_error/3}
\begin{quote}
\begin{verbatim}
util:list_error(Text, N, Message)
\end{verbatim}
\end{quote}
which returns on backtracking all the errors whose error message
contains the string \about{Text}.

The ability to define any Prolog predicate as the error handler permits a
great deal of flexibility in error handling. However, this flexibility
should be used with caution. The action of an error handler could have side
effects altering the correctness of a program; indeed it could be responsible
for further errors being introduced. One particular area of danger is in the
use of input and output streams by error handlers.
%For example:
%a particular error handler may interact with
%the user at the terminal, to explain the nature of the error and ask for
%directions regarding what action should be taken. Care should be taken in
%such a case
%to ensure that the error handler does not affect the input to the program.
%If it does, since program execution continues normally after exit of the error
%handler, any input consumed by the error handler is lost.


\subsection{Arguments of Error Handlers}
An error handler has four optional arguments:
\begin{enumerate}
\item The first argument is the number or atom that identifies the error.
\item The second argument is
the culprit (a structure corresponding to the call which caused the
error).
For instance, if,
say, a type error occurs upon calling the second goal of the procedure
\notation{p(2, Z)}:
\begin{quote}
\begin{verbatim}
 p(X, Y) :- a(X), b(X, Y), c(Y).
\end{verbatim}
\end{quote}
the structure given to the error handler is \notation{b(2, Y)}.
Note that the handler could bind Y which would have the same effect
as if \predspec{b/2} had done the binding.

\item The third argument is only defined for a subset of the existing errors.
If the error occurred inside a tool body, it holds the caller module,
otherwise it is identical to the fourth argument.\footnote{%
  Note that some events are not errors but are used for different purposes.
  In those cases the second and third argument are sometimes used
  differently. See Appendix \ref{chaperrors} for details.}

\item The fourth argument is the lookup module for the culprit goal.  This
is needed for example when the handler wants to call the culprit reliably,
using a qualified call via \txtbipref{:/2}{(:)/2}{../bips/kernel/control/N-2.html}.
\end{enumerate}

The error handler is free to ignore some of these arguments,
i.e., it can have any arity from 0 to 4.
The first argument is provided for the case that the same procedure serves
as the handler for several error types---then it can distinguish
which is the actual error type.
\index{errors!handlers}
An error handler is just an ordinary Prolog procedure and thus within
it a call may be made to any other procedure, or any built in predicate;
this in particular means that a call to
\bipref{throw/1}{../bips/kernel/control/throw-1.html} may be
made (see the section on the
\bipref{catch/3}{../bips/kernel/control/catch-3.html}
predicate).\index{blocks} This will work
``through'' the call to the error handler, and so an exit may be made from
within the handler out of the current catch-block (i.e., back to the corresponding
call of the \bipref{catch/3}{../bips/kernel/control/catch-3.html} predicate).
Specifying the predicates \bipref{true/0}{../bips/kernel/control/true-0.html} or
\bipref{fail/0}{../bips/kernel/control/fail-0.html} as error handlers
will make the erroneous predicate succeed (without binding
any further variables) or fail respectively.

The following two templates are the most common for error handlers.
The first simply prints an error message and aborts:
\begin{quote}
\begin{verbatim}
my_error_handler(ErrorId, Goal, ContextModule) :-
        printf(error, "Error %w in %w in module %w%n",
                [ErrorId,Goal,ContextModule]),
        abort.
\end{verbatim}
\end{quote}
The following handler tries to repair the error and call the goal again:
\begin{quote}
\begin{verbatim}
my_error_repair_handler(ErrorId, Goal, ContextModule, LookupModule) :-
        % repair the error
        ... some code to repair the cause for the error ...
        % try call the erroneous goal again
        LookupModule : Goal @ ContextModule.
\end{verbatim}
\end{quote}


\subsection{User Defined Errors}
\label{user-errors}
\index{errors!user defined}
The following example illustrates the use of a user-defined error.
We declare a handler for the event \notation{Invalid command} and
raise the new error in the application code.
\begin{quote}
\begin{verbatim}
% Command error handler - output invalid command, sound bell and abort
command_error_handler(_, Command) :-
        printf("\007\nInvalid command: %w\n", [Command]),
        abort.

% Activate the handler
:- set_event_handler('Invalid command', command_error_handler/2).

% top command processing loop
go :-
        writeln("Enter command."),
        read(Command),
        ( valid_command(Command)->
            process_command(Command),
            go
        ;
            error('Invalid command',Command)  % Call the error handler
        ).

% Some valid commands
valid_command(start).
valid_command(stop).
\end{verbatim}
\end{quote}



%\subsection{Using Tools in Error Handlers}
%\index{errors!toolhandlers}
%\label{toolhandlers}
%If an error occurs in a tool predicate and the error handler
%has to fix the problem and continue the execution,
%it usually has to call directly the system tools
%using its module argument.
%As an example, here is the script of constructing the error handler
%for the error `procedure already dynamic'.
%This error occurs if the argument of the
%\bipref{dynamic/1}{../bips/kernel/dynamic/dynamic-1.html}
%predicate is already dynamic.
%The intended action is to erase the procedure so that
%when a file with this declaration is being compiled,
%the dynamic clauses are not added but they replace
%the existing procedure, which is what one wants
%during the debugging phase.
%We want to use the predicate
%\bipref{retract_all/1}{../bips/kernel/obsolete/retract_all-1.html} to remove
%all existing clauses, however we have to use its body procedure
%(it is a tool) to be able to access a predicate in an explicit module.
%
%\pagebreak[1]
%\begin{quote}
%\begin{verbatim}
%[eclipse 1]: use_module(library(util)),
%             list_error("dynamic", N, M).
%N = 63
%M = "procedure not dynamic"     More? (;)
%N = 64
%M = "procedure already dynamic"     More? (;)
%yes.
%       % find the tool body of retract_all/1
%
%[eclipse 2]: tool_body(retract_all/1, P, M).
%P = retract_all_body / 2
%M = sepia_kernel
%yes.
%       % to be able to use the body procedure, we must import it
%
%[eclipse 3]: import retract_all_body/2 from sepia_kernel.
%yes.
%       % Now we can define the error handling procedure
%
%[eclipse 4]: [user].
% dynamic_handler(_, dynamic(Name/Arity), Module) :-
%        !,
%        functor(F, Name, Arity),
%        retract_all_body(F, Module).
% dynamic_handler(N, OtherGoal, Module) :-
%        error(default(N), OtherGoal, Module).
% user compiled 348 bytes in 0.00 seconds
%
%yes.
%       % and set it as the error handler
%
%[eclipse 5]: set_event_handler(64, dynamic_handler/3).
%yes.
%       % Finally, we check whether it works as desired
%
%[eclipse 6]: [user].
% :- dynamic p/1.
% p(a). p(b).
% :- listing p/1.
%p(a).
%p(b).
% :- dynamic p/1.
% p(1).
% :- listing p/1.
%p(1).
%       % the previous clauses are indeed erased
%
% user compiled 0 bytes in 0.07 seconds
%\end{verbatim}
%\end{quote}
%\vspace*{\fill}


%----------------------------------------------------------------------
\section{Interrupts\label{sectinterrupts}}
%----------------------------------------------------------------------

\index{interrupts}
Operating systems such as Unix provide a notion of asynchronous interrupts
or signals. In a standalone {\eclipse} system, the signals can be handled by
defining interrupt handlers for them. In fact, a set of
default handlers is already predefined in this case.

In an embedded {\eclipse}, signals are usually handled by the host
application, and it is recommended to use the event mechanism described above
(the \notation{ec_post_event()} library function) to communicate between the
host
application and the {\eclipse} code.
However, even in this setting, {\eclipse} can also handle signals directly,
provided the programmer sets up a corresponding interrupt handler.

\vfill % <<<<<<<<<<<<<<<<<<<<<<<<<<

\subsection{Interrupt Identifiers}
Interrupts are identified either by their signal number (Unix) or
by a name which is derived from the name the signal has in the operating
system. Most built-ins understand both identifiers. It is usually
more portable to use the symbolic name.
The built-in
\bipref{current_interrupt/2}{../bips/kernel/event/current_interrupt-2.html} is
provided to check and/or
generate the valid interrupt numbers and their mnemonic names.


\subsection{Asynchronous handling}

When an interrupt happens, the {\eclipse} system
calls an interrupt handling routine in a manner very similar to
the case of event handling. The only argument to the handler is
the interrupt number.
Just as event handlers may be user defined, so it is possible to define
interrupt handlers. The goal
\begin{quote}
\preddef{set_interrupt_handler(\pattern{N},~\pattern{PredSpec})}%
\indextt{set_interrupt_handler/2}
\end{quote}
assigns the procedure specified by \about{PredSpec} as the interrupt handler for
the interrupt identified by \about{N} (a number or a name).
Some interrupts cannot be caught by the user (e.g., the \about{kill} signal),
trying to establish a handler for them yields an error message. Note that
\about{PredSpec} should be one of the predefined handlers. The use of general
user defined predicates is deprecated because of portability considerations.

To test interrupt handlers, the built-in
\bipref{kill/2}{../bips/kernel/opsys/kill-2.html} may be used to send
a signal to the own process.

The predicate
\bipref{get_interrupt_handler/3}%
{../bips/kernel/event/get_interrupt_handler-3.html}
may be used to find the
current interrupt handler for an interrupt N, in the same manner as
\predspec{get_event_handler}:
\begin{quote}
\preddef{%
get_interrupt_handler(\pattern{N},~\pattern{PredSpec},~\pattern{HomeModule})}%
\indextt{get_interrupt_handler/3}
\end{quote}

An interrupt handler has one optional argument, which is the interrupt
number.
There is no argument corresponding to the error culprit, since
the interrupt has no relation to the currently executed predicate.
A handler may be defined which takes no argument (such
as when the handler is defined for only one interrupt type).
If the handler has one argument, the identifier of the interrupt is passed
to the handler when it is called.

The following is the list of predefined interrupt handlers:
\begin{quote}
\begin{description}
\item[default/0]\indextt{default/0}\mbox{}\\
performs the standard UNIX handling of the specified interrupt (signal).
    Setting this handler is equivalent to calling \notation{signal(N, SIG_DFL)}
    on the C level.
    Thus e.g., specifying
    \begin{quote}
    \begin{verbatim}
    ?- set_interrupt_handler(int, default/0).
    \end{verbatim}
    \end{quote}
    will exit the {\eclipse} system when \verb:^C: is pressed.

\item[true/0]\indextt{true/0}\mbox{}\\
    This is equivalent to calling \notation{signal(N, SIG_IGN)} on the C level,
    i.e., the interrupt is ignored.

\item[throw/1]\indextt{throw/1}\mbox{}\\
    Invoke \predspec{throw/1} with the interupt's symbolic name.

\item[abort/0]\indextt{abort/0}\mbox{}\\
    Invoke \notation{throw(abort)}.

\item[halt/0]\indextt{halt/0}\mbox{}\\
    Terminate the {\eclipse} process.

\item[internal/0]\indextt{internal/0}\mbox{}\\
    Used by {\eclipse} to implement internal functionality like the
    profiler. This is not intended to be used by the user.

\item[event/1]\indextt{event/1}\mbox{}\\
    The signal is handled by posting a (synchronous) event. The event
    name is the symbolic name of the interrupt.
\end{description}
\end{quote}
Apart from these special cases, all other arguments will
result in the specified predicate being called when the appropriate
interrupt occurs. This general asynchronous interrupt handling
is not supported on all hardware/platforms,
neither in an embedded {\eclipse} (including the
\index{interrupts!tkeclipse} \tkeclipse{} development environment),
and is therefore deprecated.


%{\eclipse} provides special support for debugging interrupt handlers.
%They can be debugged independently of the interrupted program.
%To achieve this, every interrupt number has a flag that can take one of the
%values {\tt debug}, {\tt trace}, {\tt nodebug} or {\tt notrace}
%(the latter two are synonyms).
%It specifies if the corresponding interrupt handlers are executed
%with the debugger in leap mode, in creep mode or without debugger respectively.
%The flags can be set with the built-in
%\bipref{set_interrupt_flag/2}{../bips/kernel/event/set_interrupt_flag-2.html}
%and queried with
%\bipref{get_interrupt_flag/2}{../bips/kernel/event/get_interrupt_flag-2.html}.
%
%The remarks about care in the use of error handlers, especially in the
%matter of how they affect input and output, are equally applicable to
%interrupt handlers.

%\section{The block/3 Predicate}
%\index{blocks}
%\label{blocks}
%Another mechanism provided by {\eclipse} which may be used to provide error
%handling is the one proposed by the BSI and ISO standardisation committees.
%Is is called ``blocks'' in the standardisation context,
%but also known as ``Catch and Throw'' in some other systems.  There are
%two predicates involved here:
%\begin{quote}
%\begin{verbatim}
%block(Goal, Tag, Recovery)
%\end{verbatim}
%\begin{verbatim}
%exit_block(Tag)
%\end{verbatim}
%\end{quote}
%\index{block/3}
%\index{exit_block/1}
%{\it Goal} and {\it Recovery} must be callable Prolog terms,
%{\it Tag} must be a constant or a free variable.
%{\bf block(Goal,\ Tag,\ Recovery)} is very similar to {\bf call(Goal)}, but it
%also creates a label to which the program can return immediately by a suitable
%call of \bipref{exit_block/1}{../bips/kernel/control/exit_block-1.html}.
%
%{\bf block(Goal, Tag, Recovery)} first calls {\it Goal}, and if this succeeds
%then \bipref{block/3}{../bips/kernel/control/block-3.html} succeeds.
%If {\it Goal} fails then so does the call of
%\bipref{block/3}{../bips/kernel/control/block-3.html}.
%If, however, during the execution of
%{\it Goal} there is a call of {\bf exit_block(TagExit)}
%such that {\it TagExit} unifies with the {\it Tag} of a
%\bipref{block/3}{../bips/kernel/control/block-3.html}
%invocation,
%then \bipref{block/3}{../bips/kernel/control/block-3.html} calls the goal {\it
%Recovery}, and
%succeeds or fails according to whether {\it Recovery} succeeds or fails.
%
%If {\it Tag} does not unify with {\it TagExit}, the system continues
%looking for an earlier
%invocation of \bipref{block/3}{../bips/kernel/control/block-3.html}. (An error
%is raised if {\it TagExit} does
%not unify with a {\it Tag}  of any uncompleted call of
%\bipref{block/3}{../bips/kernel/control/block-3.html}.)
%
%One of the most likely uses for the block
%mechanism is exception handling.
%Since {\eclipse} provides the user with a sophisticated exception
%handling via user-definable handlers, the likely use of the block
%mechanism is more restricted than in other systems.
%Note that in contrast to the BSI proposal, the tag
%of the block cannot be a compound term.
%This restriction has been introduced because the semantics of
%having non-ground compound terms is not clearly defined.
%This does not mean that {\eclipse} is less powerful, though.
%If more information is to be passed from
%\bipref{exit_block/1}{../bips/kernel/control/exit_block-1.html} to
%\bipref{block/3}{../bips/kernel/control/block-3.html},
%communication using global variables can be used instead.
%

%HEVEA\cutend
