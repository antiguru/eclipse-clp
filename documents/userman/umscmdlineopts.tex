% BEGIN LICENSE BLOCK
% Version: CMPL 1.1
%
% The contents of this file are subject to the Cisco-style Mozilla Public
% License Version 1.1 (the "License"); you may not use this file except
% in compliance with the License.  You may obtain a copy of the License
% at www.eclipse-clp.org/license.
% 
% Software distributed under the License is distributed on an "AS IS"
% basis, WITHOUT WARRANTY OF ANY KIND, either express or implied.  See
% the License for the specific language governing rights and limitations
% under the License. 
% 
% The Original Code is  The ECLiPSe Constraint Logic Programming System. 
% The Initial Developer of the Original Code is  Cisco Systems, Inc. 
% Portions created by the Initial Developer are
% Copyright (C) 2006 Cisco Systems, Inc.  All Rights Reserved.
% 
% Contributor(s): 
% 
% END LICENSE BLOCK
%------------------------------------------------------------------------
\chapter{{\eclipse} Command Line Options}
%------------------------------------------------------------------------

The {\eclipse} system has several parameters which may be specified on the
command line at invocation time. All the parameters are available with
the tty {\tt eclipse}; with {\tt tkeclipse}, only the {\tt -g} and {\tt -l}
parameters are available.
The parameters are as follows:

\begin{description}
\item[$-$b bootfile] 
\index{command line options}
\index{command line options!--b}
\index{--b (command line option)}
Compile the file {\it bootfile} before starting the session.
Multiple -b options are allowed.
The file name is expected to be in the operating system's syntax.
The file is processed by
\bipref{ensure_loaded/1}{../bips/kernel/compiler/ensure_loaded-1.html},
i.e.\ it can be a precompiled file or a source file, and file extensions
are added as specified there.
%In this case the {\tt .eclipserc} file will not be compiled.

\item[$-$e goal]
\index{--e (command line option)}
\index{command line options!--e}
Instead of starting an interactive toplevel, the system will execute the
goal {\bf goal}. {\bf goal} is given in normal Prolog syntax, and has to be
quoted if it contains any characters that would normally be interpreted by the
shell. The -e option can be used together with the -b option and is executed
afterwards. Only one -e option is allowed.
%If the -e option is used together with the saved state option -s, then the
%goal will not be executed when making the saved state, but when restoring it.

The exit status of the {\eclipse} process reflects success or failure of the
\index{exit status}
executed Prolog goal (0 for success, 1 for failure, 2 for abort).

When you only have a runtime installation of eclipse, the -e option
is compulsory because a runtime system does not have an interactive
toplevel.


\item[$-$g size]
\index{--g (command line option)}
This option specifies to which limit the memory consumption of the
{\eclipse} global/trail stack can grow.
The size is specified in kilobytes (followed by an optional K), in megabytes
(followed by M) or in gigabytes (followed by G).
The default is 128M, ie.\ 128 Megabytes.
The amount required for this stack depends on the program's data
structures and may need to be increased for very large applications.

\item[$-$l size]
\index{--l (command line option)}
This option specifies to which limit the memory consumption of the
{\eclipse} local/control stack can grow.
The size is specified in kilobytes (followed by an optional K), in megabytes
(followed by M) or in gigabytes (followed by G).
The default is 128M, ie.\ 128 Megabytes.
The local/control stack is unlikely to require more than this default.
If it does, it is probably caused by a programming error.

\item[$-$D directory] 
\index{command line options}
\index{--D (command line option)}
This options allows to explicitly specify the {\eclipse} installation
directory, i.e.\ the directory under which the system tries to find
the {\eclipse} runtime system and libraries.  This option overrides
(and renders unnecessary) any setting of the ECLIPSEDIR environment
variable (Unix) or, respectively, an ECLIPSEDIR registry entry
(Windows) that may be in effect.

\item[$-$ $-$]
The {\eclipse} system will ignore this argument and everything that follows on
the commmand line. The Prolog program will only see the part of the
command line that follows this argument.
\end{description}


