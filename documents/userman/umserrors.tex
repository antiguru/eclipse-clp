% BEGIN LICENSE BLOCK
% Version: CMPL 1.1
%
% The contents of this file are subject to the Cisco-style Mozilla Public
% License Version 1.1 (the "License"); you may not use this file except
% in compliance with the License.  You may obtain a copy of the License
% at www.eclipse-clp.org/license.
%
% Software distributed under the License is distributed on an "AS IS"
% basis, WITHOUT WARRANTY OF ANY KIND, either express or implied.  See
% the License for the specific language governing rights and limitations
% under the License.
%
% The Original Code is  The ECLiPSe Constraint Logic Programming System.
% The Initial Developer of the Original Code is  Cisco Systems, Inc.
% Portions created by the Initial Developer are
% Copyright (C) 1996 - 2006 Cisco Systems, Inc.  All Rights Reserved.
%
% Contributor(s):
%
% END LICENSE BLOCK
%
% @(#)umserrors.tex	1.10 96/01/08
%
%%%%%%%%%%%%%%%%%%%%%%%%%%%%%%%%%%%%%%%%%%%%%%%%%%%%%%%%%%%%%%%%%%%%%%%%%%
% {\eclipse} Documentation
%
% umserrors.tex
%
% REL	DATE	AUTHOR		DESCRIPTION
% 2.10	060490	Eamon Falvey	Update for latex..
% 3.0	160590	Joachim Schimpf	Update for 3.0
% 3.0	020690	Micha Meier	Including the generated errors
%
%	Use the gen_errors/0 procedure in the file gen_error.pl
%	to generate the input files.
%
%%%%%%%%%%%%%%%%%%%%%%%%%%%%%%%%%%%%%%%%%%%%%%%%%%%%%%%%%%%%%%%%%%%%%%%%%%
\chapter{Events}
%HEVEA\cutdef[1]{section}
\label{errors}
\label{chaperrors}

We list here the {\eclipse} event types together with the default
event handlers and their description.
\index{event handlers}\index{handler!event}
\index{error handlers}\index{handler!error}
Unless otherwise specified, the arguments that the system passes
to the event handler are

\vspace{0.3cm}
\noindent
\begin{tabular}{p{4cm}p{4cm}p{4cm}}
\heading{First Argument} & \heading{Second Argument}
                                                  & \heading{Third Argument} \\
\hline
Event number & Culprit goal & Caller Module \\
\end{tabular}

\vspace{0.3cm}
\noindent
If the caller module is unknown, a free variable is passed.

%%%%%%%%%%
\section{Event Types}
%The types of events can be classified into the following sections:
%\begin{itemize}
%\item Argument Types and Values, Arithmetic
%\item Data and Memory Areas, Predicates, Modules, Operators
%\item Run-Time System, Compilation, Execution, Top-Level
%\item Macro transformation errors.
%\item I/O, Operating System, External Interface
%\item Advanced Features, Extensions, Debugging
%\item User-Defined Events
%\end{itemize}

\subsection{Argument Types and Values}
\begin{tabular}{|p{1.2cm}p{8cm}p{5cm}|}
\hline
\heading{Event} & \heading{Event Type} & \heading{Default Event Handler}\\
\hline
% BEGIN LICENSE BLOCK
% Version: CMPL 1.1
%
% The contents of this file are subject to the Cisco-style Mozilla Public
% License Version 1.1 (the "License"); you may not use this file except
% in compliance with the License.  You may obtain a copy of the License
% at www.eclipse-clp.org/license.
% 
% Software distributed under the License is distributed on an "AS IS"
% basis, WITHOUT WARRANTY OF ANY KIND, either express or implied.  See
% the License for the specific language governing rights and limitations
% under the License. 
% 
% The Original Code is  The ECLiPSe Constraint Logic Programming System. 
% The Initial Developer of the Original Code is  Cisco Systems, Inc. 
% Portions created by the Initial Developer are
% Copyright (C) 2006 Cisco Systems, Inc.  All Rights Reserved.
% 
% Contributor(s): 
% 
% END LICENSE BLOCK

1 & general error & error_handler / 2 \\
2 & term of an unknown type & error_handler / 2 \\
4 & instantiation fault & error_handler / 4 \\
5 & type error & error_handler / 4 \\
6 & out of range & error_handler / 4 \\
7 & string contains unexpected characters & error_handler / 2 \\
8 & bad argument list & error_handler / 2 \\

\hline
\end{tabular}

\vfill %<<<<<<<<<<<

\subsection{Arithmetic, Environment}
\begin{tabular}{|p{1.2cm}p{8cm}p{5cm}|}
\hline
\heading{Event} & \heading{Event Type} & \heading{Default Event Handler}\\
\hline
% BEGIN LICENSE BLOCK
% Version: CMPL 1.1
%
% The contents of this file are subject to the Cisco-style Mozilla Public
% License Version 1.1 (the "License"); you may not use this file except
% in compliance with the License.  You may obtain a copy of the License
% at www.eclipse-clp.org/license.
% 
% Software distributed under the License is distributed on an "AS IS"
% basis, WITHOUT WARRANTY OF ANY KIND, either express or implied.  See
% the License for the specific language governing rights and limitations
% under the License. 
% 
% The Original Code is  The ECLiPSe Constraint Logic Programming System. 
% The Initial Developer of the Original Code is  Cisco Systems, Inc. 
% Portions created by the Initial Developer are
% Copyright (C) 2006 Cisco Systems, Inc.  All Rights Reserved.
% 
% Contributor(s): 
% 
% END LICENSE BLOCK

15 & creating parallel choice point & fail / 0 \\
16 & failing to parallel choice point & fail / 0 \\
17 & recomputation failed & error_handler / 2 \\
20 & arithmetic exception & error_handler / 2 \\
21 & undefined arithmetic expression & error_handler / 4 \\
23 & comparison trap & compare_handler / 4 \\
24 & number expected & error_handler / 2 \\
25 & integer overflow & integer_overflow_handler / 2 \\
30 & trying to write a read-only flag & error_handler / 2 \\
31 & arity limit exceeded & error_handler / 2 \\
32 & no handler for event & warning_handler / 2 \\
33 & event queue overflow & error_handler / 2 \\

\hline
\end{tabular}

\vspace*{\fill}


\subsection{Data and Memory Areas, Predicates, Operators}
\begin{tabular}{|p{1.2cm}p{8cm}p{5.4cm}|}
\hline
\heading{Event} & \heading{Event Type} & \heading{Default Event Handler}\\
\hline
% BEGIN LICENSE BLOCK
% Version: CMPL 1.1
%
% The contents of this file are subject to the Cisco-style Mozilla Public
% License Version 1.1 (the "License"); you may not use this file except
% in compliance with the License.  You may obtain a copy of the License
% at www.eclipse-clp.org/license.
% 
% Software distributed under the License is distributed on an "AS IS"
% basis, WITHOUT WARRANTY OF ANY KIND, either express or implied.  See
% the License for the specific language governing rights and limitations
% under the License. 
% 
% The Original Code is  The ECLiPSe Constraint Logic Programming System. 
% The Initial Developer of the Original Code is  Cisco Systems, Inc. 
% Portions created by the Initial Developer are
% Copyright (C) 2006 Cisco Systems, Inc.  All Rights Reserved.
% 
% Contributor(s): 
% 
% END LICENSE BLOCK

40 & stale object handle & error_handler / 2 \\
41 & array or global variable does not exist & undef_array_handler / 3 \\
42 & redefining an existing array & make_array_handler / 4 \\
43 & multiple definition postfix/infix & error_handler / 2 \\
44 & record already exists & error_handler / 2 \\
45 & record does not exist & undef_record_handler / 2 \\
50 & trying to modify a read-only ground term & error_handler / 2 \\
60 & referring to an undefined procedure & error_handler / 4 \\
61 & inconsistent tool redefinition & error_handler / 4 \\
62 & inconsistent procedure redefinition & error_handler / 4 \\
63 & procedure not dynamic & error_handler / 4 \\
64 & procedure already dynamic & dynamic_handler / 3 \\
65 & procedure already defined & error_handler / 4 \\
66 & trying to modify a system predicate & error_handler / 4 \\
67 & procedure is not yet loaded & error_handler / 4 \\
68 & calling an undefined procedure & call_handler / 4 \\
69 & autoload event & autoload_handler / 4 \\
70 & accessing an undefined dynamic procedure & undef_dynamic_handler / 3 \\
71 & procedure already parallel & error_handler / 2 \\
72 & accessing an undefined operator & error_handler / 2 \\
73 & redefining an existing operator & true / 0 \\
74 & hiding an existing global operator & true / 0 \\
75 & referring to a deprecated predicate & declaration_warning_handler / 3 \\
76 & predicate declared but not defined & declaration_warning_handler / 3 \\
77 & predicate used but not declared or defined & declaration_warning_handler / 3 \\
78 & calling a procedure with a reserved name & error_handler / 2 \\

\hline
\end{tabular}

\subsection{Modules, Visibility}
\begin{tabular}{|p{1.2cm}p{8cm}p{5.4cm}|}
\hline
\heading{Event} & \heading{Event Type} & \heading{Default Event Handler}\\
\hline
% BEGIN LICENSE BLOCK
% Version: CMPL 1.1
%
% The contents of this file are subject to the Cisco-style Mozilla Public
% License Version 1.1 (the "License"); you may not use this file except
% in compliance with the License.  You may obtain a copy of the License
% at www.eclipse-clp.org/license.
% 
% Software distributed under the License is distributed on an "AS IS"
% basis, WITHOUT WARRANTY OF ANY KIND, either express or implied.  See
% the License for the specific language governing rights and limitations
% under the License. 
% 
% The Original Code is  The ECLiPSe Constraint Logic Programming System. 
% The Initial Developer of the Original Code is  Cisco Systems, Inc. 
% Portions created by the Initial Developer are
% Copyright (C) 2006 Cisco Systems, Inc.  All Rights Reserved.
% 
% Contributor(s): 
% 
% END LICENSE BLOCK

80 & not a module & error_handler / 2 \\
81 & module/1 can appear only as a directive & error_handler / 2 \\
82 & trying to access a locked module & locked_access_handler / 2 \\
83 & creating a new module & warning_handler / 2 \\
84 & referring to non-exported predicate & declaration_warning_handler / 3 \\
85 & referring to non-existing module & declaration_warning_handler / 3 \\
86 & lookup module does not exist & no_lookup_module_handler / 4 \\
87 & attempt to redefine an existing local item & warning_handler / 3 \\
88 & attempt to redefine an existing exported item & warning_handler / 3 \\
89 & attempt to redefine an already imported item & warning_handler / 3 \\
90 & procedure is already reexported & error_handler / 4 \\
91 & not a tool procedure & error_handler / 2 \\
92 & trying to redefine an existing local procedure & error_handler / 4 \\
93 & trying to redefine an existing exported procedure & error_handler / 4 \\
94 & trying to redefine an existing imported procedure & error_handler / 4 \\
96 & ambiguous import & ambiguous_import_resolve / 3 \\
97 & module already exists & error_handler / 2 \\
98 & key not correct & error_handler / 2 \\
99 & unresolved ambiguous import & ambiguous_import_warn / 3 \\
100 & accessing a procedure defined in another module & undef_dynamic_handler / 3 \\
101 & trying to erase a module from itself & error_handler / 2 \\

\hline
\end{tabular}
\vspace{0.5cm}

\vfill %<<<<<<<<<<<<<<<<<<<<<<

\subsection{Syntax Errors, Parsing}
\begin{tabular}{|p{1.2cm}p{8cm}p{4.5cm}|}
\hline
\heading{Event} & \heading{Event Type} & \heading{Default Event Handler}\\
\hline
% BEGIN LICENSE BLOCK
% Version: CMPL 1.1
%
% The contents of this file are subject to the Cisco-style Mozilla Public
% License Version 1.1 (the "License"); you may not use this file except
% in compliance with the License.  You may obtain a copy of the License
% at www.eclipse-clp.org/license.
% 
% Software distributed under the License is distributed on an "AS IS"
% basis, WITHOUT WARRANTY OF ANY KIND, either express or implied.  See
% the License for the specific language governing rights and limitations
% under the License. 
% 
% The Original Code is  The ECLiPSe Constraint Logic Programming System. 
% The Initial Developer of the Original Code is  Cisco Systems, Inc. 
% Portions created by the Initial Developer are
% Copyright (C) 2006 Cisco Systems, Inc.  All Rights Reserved.
% 
% Contributor(s): 
% 
% END LICENSE BLOCK

110 & syntax error:  & parser_error_handler / 2 \\
111 & syntax error: list tail ended improperly & parser_error_handler / 2 \\
112 & syntax error: illegal character in a quoted token & parser_error_handler / 2 \\
113 & syntax error: unexpected comma & parser_error_handler / 2 \\
114 & syntax error: unexpected token & parser_error_handler / 2 \\
115 & syntax error: unexpected end of file & parser_error_handler / 2 \\
116 & syntax error: numeric constant out of range & parser_error_handler / 2 \\
117 & syntax error: bracket necessary & parser_error_handler / 2 \\
118 & syntax error: unexpected fullstop & parser_error_handler / 2 \\
119 & syntax error: postfix/infix operator expected & parser_error_handler / 2 \\
120 & syntax error: wrong solo char & parser_error_handler / 2 \\
121 & syntax error: space between functor and open bracket & parser_error_handler / 2 \\
122 & syntax error: variable with multiple attributes & parser_error_handler / 2 \\
123 & illegal iteration specifier in do-loop & error_handler / 4 \\
124 & syntax error : prefix operator followed by infix operator & parser_error_handler / 2 \\
125 & syntax error : unexpected closing bracket & parser_error_handler / 2 \\
126 & syntax error : grammar rule head is not valid & parser_error_handler / 2 \\
127 & syntax error : grammar rule body is not valid & parser_error_handler / 2 \\
128 & syntax error : in source transformation & parser_error_handler / 2 \\
129 & syntax error: source transformation floundered & parser_error_handler / 2 \\

\hline
\end{tabular}

\vfill %<<<<<<<<<<<<<<<<<<<<<<


\subsection{Compilation, Run-Time System, Execution}
\begin{tabular}{|p{1.2cm}p{8cm}p{4.5cm}|}
\hline
\heading{Event} & \heading{Event Type} & \heading{Default Event Handler}\\
\hline
% BEGIN LICENSE BLOCK
% Version: CMPL 1.1
%
% The contents of this file are subject to the Cisco-style Mozilla Public
% License Version 1.1 (the "License"); you may not use this file except
% in compliance with the License.  You may obtain a copy of the License
% at www.eclipse-clp.org/license.
% 
% Software distributed under the License is distributed on an "AS IS"
% basis, WITHOUT WARRANTY OF ANY KIND, either express or implied.  See
% the License for the specific language governing rights and limitations
% under the License. 
% 
% The Original Code is  The ECLiPSe Constraint Logic Programming System. 
% The Initial Developer of the Original Code is  Cisco Systems, Inc. 
% Portions created by the Initial Developer are
% Copyright (C) 2006 Cisco Systems, Inc.  All Rights Reserved.
% 
% Contributor(s): 
% 
% END LICENSE BLOCK

130 & syntax error: illegal head & compiler_error_handler / 2 \\
131 & syntax error: illegal goal & compiler_error_handler / 2 \\
132 & syntax error: term of an unknown type & compiler_error_handler / 2 \\
133 & loading the library  & true / 0 \\
134 & procedure clauses are not consecutive & compiler_error_handler / 2 \\
135 & trying to redefine a protected procedure & compiler_error_handler / 2 \\
136 & trying to redefine a built-in predicate & compiler_error_handler / 2 \\
137 & trying to redefine a procedure with another type & compiler_error_handler / 2 \\
138 & singleton local variable in do-loop & singleton_in_loop / 2 \\
139 & compiled or dumped file message & compiled_file_handler / 3 \\
140 & undefined instruction & error_handler / 2 \\
141 & unimplemented functionality & error_handler / 2 \\
142 & built-in predicate not available on this system & error_handler / 2 \\
143 & compiled query failed & compiler_error_handler / 2 \\
144 & a cut is not allowed in a condition & compiler_error_handler / 2 \\
145 & procedure being redefined in another file & redef_other_file_handler / 2 \\
146 & start of compilation & true / 0 \\
147 & compilation aborted & compiler_abort_handler / 3 \\
148 & bad pragma & pragma_handler / 3 \\
149 & code unit loaded & unit_loaded_handler / 3 \\

\hline
\end{tabular}
\vspace{0.5cm}

\noindent
The handlers for these events receive the following arguments:

\noindent
\begin{tabular}{p{1.2cm}p{8cm}p{4.5cm}}
\heading{Event} & \heading{Second Argument} & \heading{Third Argument}\\
\hline
130 & Culprit clause & Module \\
131 & Culprit clause & Module \\
132 & Culprit clause & Module \\
133 & Library name (string) & undefined \\
134 & Procedure Name/Arity & Module \\
135 & Procedure Name/Arity & Module \\
136 & Procedure Name/Arity & Module \\
137 & Procedure Name/Arity & Module \\
138 & Variable name (atom) & undefined \\
139 & (File, Size, Time), see below & Module\\
140 & 'Emulate' & undefined \\
141 & Goal & Module \\
142 & Goal & Module \\
143 & Goal & Module \\
144 & Goal (if an execution error) or Culprit clause (if compiler error) &
      Module \\
145 & (Name/Arity, OldFile, NewFile) & Module \\
146 & File & Module \\
147 & File\\
148 & Clause & Module \\
\hline
\end{tabular}

\newpage
\noindent
The second argument for the event 139 depends on the predicate
where it was raised:
\begin{itemize}
\item
  \biprefni{compile/1,2}{../bips/kernel/compiler/compile-1.html}%
\indextt{compile/1}\indextt{compile/2}
  - (file name, code size, compile time)
\item
  \bipref{compile_stream/1}{../bips/kernel/compiler/compile_stream-1.html} -
  ('string', code size, compile time) with a string stream
\item
  \bipref{compile_stream/1}{../bips/kernel/compiler/compile_stream-1.html} -
  (file name, code size, compile time) with a stream associated to a file
\end{itemize}

\subsection{Top-Level}
\begin{tabular}{|p{1.2cm}p{8cm}p{4.5cm}|}
\hline
\heading{Event} & \heading{Event Type} & \heading{Default Event Handler}\\
\hline
% BEGIN LICENSE BLOCK
% Version: CMPL 1.1
%
% The contents of this file are subject to the Cisco-style Mozilla Public
% License Version 1.1 (the "License"); you may not use this file except
% in compliance with the License.  You may obtain a copy of the License
% at www.eclipse-clp.org/license.
% 
% Software distributed under the License is distributed on an "AS IS"
% basis, WITHOUT WARRANTY OF ANY KIND, either express or implied.  See
% the License for the specific language governing rights and limitations
% under the License. 
% 
% The Original Code is  The ECLiPSe Constraint Logic Programming System. 
% The Initial Developer of the Original Code is  Cisco Systems, Inc. 
% Portions created by the Initial Developer are
% Copyright (C) 2006 Cisco Systems, Inc.  All Rights Reserved.
% 
% Contributor(s): 
% 
% END LICENSE BLOCK

150 & start of eclipse execution & sepia_start / 0 \\
151 & eclipse restart & true / 0 \\
152 & end of eclipse execution & sepia_end / 0 \\
153 & toplevel: print prompt & toplevel_prompt / 2 \\
154 & toplevel: start of query execution & true / 0 \\
155 & toplevel: print values & print_values / 3 \\
156 & toplevel: print answer & tty_ask_more / 2 \\
157 & error exit & error_exit / 0 \\
158 & toplevel: entering break level & start_break / 3 \\
159 & toplevel: leaving break level & end_break / 3 \\

\hline
\end{tabular}

\vspace{0.5cm}
These events are not errors but rather hooks to allow users to modify
the behaviour of the {\eclipse} toplevel.
Therefore the arguments that are passed to the handler are not the
erroneous goal and the caller module but defined as follows:

\noindent
\begin{tabular}{p{1.2cm}p{8.5cm}p{4.5cm}}
\heading{Event} & \heading{Second Argument} & \heading{Third Argument}\\
\hline
150 & A free variable. If the handler binds the variable
to an atom, this name is used as the toplevel module name
& undefined \\
151 & undefined  & undefined \\
152 & The argument is the number that {\eclipse} will return to the
operating system & undefined \\
153 & current toplevel module & current toplevel module \\
154 & a structure of the form
\begin{center}
\notation{goal(\pattern{Goal},%
~\pattern{VarList},~\pattern{NewGoal},~\pattern{NewVarList})},
\end{center}
where \about{Goal} is the goal that is about to be executed and \about{VarList}
is the list
that associates the variables in \about{Goal} with their names
(like in \bipref{readvar/3}{../bips/kernel/ioterm/readvar-3.html}).
\about{NewGoal} and \about{NewVarList} are free variables. If the handler binds
\about{NewVarList}
then the toplevel will use \about{NewGoal} and \about{NewVarList} to replace
\about{Goal} and \about{VarList}
in the current query.
& current toplevel module \\
\hline
\end{tabular}
\newpage
\begin{tabular}{p{1.2cm}p{8cm}p{4.5cm}}
\heading{Event} & \heading{Second Argument} & \heading{Third Argument}\\
\hline
155 & A list associating the variable names with their values after the
query has been executed.
& current toplevel module \\
156 & An atom stating the answer to the query that was just executed.
The possible values are: \notation{yes}, \notation{last_yes} or \notation{no}
if the query had no variables,
\notation{more_answers}, \notation{last_answer} if the query contained
variables and
bindings were printed, \notation{no_answer} if a query containing variables
failed.
& current toplevel module \\
157 & undefined & undefined \\
158 & break level & current toplevel module \\
159 & break level & current toplevel module \\
\hline
\end{tabular}
\medskip

When the handler for event 152 (``end of eclipse execution'') calls
\bipref{throw/1}{../bips/kernel/control/throw-1.html},
{\eclipse}
is not exited. This is a way to prevent
accidental exits from the system. Failure of the handler is ignored.

\subsection{Macro Transformation Errors, Lexical Analyser}
\begin{tabular}{|p{1.2cm}p{8cm}p{5.4cm}|}
\hline
\heading{Event} & \heading{Event Type} & \heading{Default Event Handler}\\
\hline
% BEGIN LICENSE BLOCK
% Version: CMPL 1.1
%
% The contents of this file are subject to the Cisco-style Mozilla Public
% License Version 1.1 (the "License"); you may not use this file except
% in compliance with the License.  You may obtain a copy of the License
% at www.eclipse-clp.org/license.
% 
% Software distributed under the License is distributed on an "AS IS"
% basis, WITHOUT WARRANTY OF ANY KIND, either express or implied.  See
% the License for the specific language governing rights and limitations
% under the License. 
% 
% The Original Code is  The ECLiPSe Constraint Logic Programming System. 
% The Initial Developer of the Original Code is  Cisco Systems, Inc. 
% Portions created by the Initial Developer are
% Copyright (C) 2006 Cisco Systems, Inc.  All Rights Reserved.
% 
% Contributor(s): 
% 
% END LICENSE BLOCK

160 & global macro transformation already exists & error_handler / 4 \\
161 & macro transformation already defined in this module & macro_handler / 3 \\
162 & no macro transformation defined in this module & warning_handler / 2 \\
163 & illegal attempt to remove the last member of a character class & error_handler / 2 \\
164 & toplevel: print banner & tty_banner / 2 \\
165 & can't compile an attributed variable (use add_attribute/2,3) & error_handler / 2 \\
166 & file successfully processed & record_compiled_file_handler / 3 \\
167 & initialization/finalization goal failed or aborted & warning_handler / 3 \\

\hline
\end{tabular}

\medskip
The event 164 is raised whenever the toplevel loop is restarted.
\medskip

\noindent
\begin{tabular}{p{1.2cm}p{8cm}p{4.5cm}}
\heading{Event} & \heading{Second Argument} & \heading{Third Argument}\\
\hline
164 & the banner string \\
\hline
\end{tabular}
\vspace*{\fill}

\subsection{I/O, Operating System, External Interface}
\begin{tabular}{|p{1.2cm}p{8cm}p{4.5cm}|}
\hline
\heading{Event} & \heading{Event Type} & \heading{Default Event Handler}\\
\hline
% BEGIN LICENSE BLOCK
% Version: CMPL 1.1
%
% The contents of this file are subject to the Cisco-style Mozilla Public
% License Version 1.1 (the "License"); you may not use this file except
% in compliance with the License.  You may obtain a copy of the License
% at www.eclipse-clp.org/license.
% 
% Software distributed under the License is distributed on an "AS IS"
% basis, WITHOUT WARRANTY OF ANY KIND, either express or implied.  See
% the License for the specific language governing rights and limitations
% under the License. 
% 
% The Original Code is  The ECLiPSe Constraint Logic Programming System. 
% The Initial Developer of the Original Code is  Cisco Systems, Inc. 
% Portions created by the Initial Developer are
% Copyright (C) 2006 Cisco Systems, Inc.  All Rights Reserved.
% 
% Contributor(s): 
% 
% END LICENSE BLOCK

170 & system interface error & system_error_handler / 4 \\
171 & File does not exist :  & error_handler / 2 \\
172 & File is not open : & error_handler / 2 \\
173 & library not found & error_handler / 2 \\
174 & child process terminated due to signal & error_handler / 2 \\
175 & child process stopped & error_handler / 2 \\
176 & message passing error & error_handler / 2 \\
177 & shared library not found & error_handler / 2 \\
190 & end of file reached & eof_handler / 4 \\
191 & output error & output_error_handler / 4 \\
192 & illegal stream mode & error_handler / 2 \\
193 & illegal stream specification & error_handler / 2 \\
194 & too many symbolic names of a stream & error_handler / 2 \\
195 & yield on flush & io_yield_handler / 2 \\
196 & trying to modify a system stream & close_handler / 2 \\
197 & use 'input' or 'output' instead of 'user' & error_handler / 2 \\
198 & reading past the file end & past_eof_handler / 2 \\
210 & Remember() not inside a backtracking predicate & error_handler / 2 \\
211 & External function does not exist & error_handler / 2 \\
212 & External function returned invalid code & error_handler / 2 \\
213 & Error in external function & error_handler / 2 \\
214 & Licensing problem & error_handler / 2 \\

\hline
\end{tabular}

\vfill %<<<<<<<<<<<<<<<<<<<<<<

\subsection{Debugging, Object Files}
\begin{tabular}{|p{1.2cm}p{8cm}p{4.5cm}|}
\hline
\heading{Event} & \heading{Event Type} & \heading{Default Event Handler}\\
\hline
% BEGIN LICENSE BLOCK
% Version: CMPL 1.1
%
% The contents of this file are subject to the Cisco-style Mozilla Public
% License Version 1.1 (the "License"); you may not use this file except
% in compliance with the License.  You may obtain a copy of the License
% at www.eclipse-clp.org/license.
% 
% Software distributed under the License is distributed on an "AS IS"
% basis, WITHOUT WARRANTY OF ANY KIND, either express or implied.  See
% the License for the specific language governing rights and limitations
% under the License. 
% 
% The Original Code is  The ECLiPSe Constraint Logic Programming System. 
% The Initial Developer of the Original Code is  Cisco Systems, Inc. 
% Portions created by the Initial Developer are
% Copyright (C) 2006 Cisco Systems, Inc.  All Rights Reserved.
% 
% Contributor(s): 
% 
% END LICENSE BLOCK

230 & uncaught exception & error_handler / 2 \\
231 & default help/0 message & fail / 0 \\
249 & debugger new suspensions event & bip_delay / 0 \\
250 & debugger init event & trace_start_handler_tty / 0 \\
251 & debugger builtin fail event & bip_port / 4 \\
252 & debugger port event & trace_line_handler_tty / 2 \\
253 & debugger call event & ncall / 2 \\
254 & debugger exit event & nexit / 1 \\
255 & debugger redo event & redo / 5 \\
256 & debugger delay event & ndelay / 2 \\
257 & debugger wake event & resume / 2 \\
258 & debugger builtin call event & bip_port / 4 \\
259 & debugger builtin exit event & bip_port / 4 \\
260 & unexpected end of file & error_handler / 2 \\
261 & invalid saved state & error_handler / 2 \\
262 & can not allocate required space & error_handler / 2 \\
263 & can not save or restore from another break level than level 0 & error_handler / 2 \\
264 & not an eclipse object file  & compiled_file_handler / 3 \\
265 & bad eclipse object file version  & compiled_file_handler / 3 \\
267 & predicate not implemented in this version & error_handler / 2 \\
268 & predicate not supported in parallel session & error_handler / 2 \\

\hline
\end{tabular}
\vspace{0.5cm}

These handlers receive special arguments:

\noindent
\begin{tabular}{p{1.2cm}p{8cm}p{4.5cm}}
\heading{Event} & \heading{Second Argument} & \heading{Third Argument}\\
\hline
252 & trace_line\{port:Port,frame:Frame\} & undefined \\
264 & (File, [], []) & undefined \\
265 & (File, [], []) & undefined \\
\hline
\end{tabular}
\vspace*{\fill}

\subsection{Extensions}
\begin{tabular}{|p{1.2cm}p{8cm}p{4.5cm}|}
\hline
\heading{Event} & \heading{Event Type} & \heading{Default Event Handler}\\
\hline
% BEGIN LICENSE BLOCK
% Version: CMPL 1.1
%
% The contents of this file are subject to the Cisco-style Mozilla Public
% License Version 1.1 (the "License"); you may not use this file except
% in compliance with the License.  You may obtain a copy of the License
% at www.eclipse-clp.org/license.
% 
% Software distributed under the License is distributed on an "AS IS"
% basis, WITHOUT WARRANTY OF ANY KIND, either express or implied.  See
% the License for the specific language governing rights and limitations
% under the License. 
% 
% The Original Code is  The ECLiPSe Constraint Logic Programming System. 
% The Initial Developer of the Original Code is  Cisco Systems, Inc. 
% Portions created by the Initial Developer are
% Copyright (C) 2006 Cisco Systems, Inc.  All Rights Reserved.
% 
% Contributor(s): 
% 
% END LICENSE BLOCK

270 & undefined variable attribute & error_handler / 2 \\
271 & bad format of the variable attribute & error_handler / 2 \\
272 & delay clause may cause indefinite delay & warning_handler / 2 \\
273 & delayed goals left & delayed_goals_handler / 3 \\
274 & stack of woken lists empty & error_handler / 2 \\
280 & Found a solution with cost  & cost_handler / 2 \\

\hline
\end{tabular}

\vspace{0.5cm}

The handlers for these events receive the following arguments:

\noindent
\begin{tabular}{p{1.2cm}p{8cm}p{4.5cm}}
\heading{Event} & \heading{Second Argument} & \heading{Third Argument}\\
\hline
272 & Culprit clause & Module \\
273 & list of sleeping suspensions & undefined \\
280 & Cost, Goal & undefined \\
\hline
\end{tabular}

\section{Stack Overflows}
\index{stack!overflow}
\index{overflow, stack}
When a stack overflows, the system performs a
\bipref{throw/1}{../bips/kernel/control/throw-1.html}
with an appropriate exit tag, i.e.,
\begin{description}
\item[global_trail_overflow] for overflows of the global/trail stack
	that holds all the program's data structures.
\item[local_control_overflow] for overflows of the local/control stack
	that holds information related to the control flow.
\end{description}
These exits can be caught by wrapping a goal that is likely
to overflow the stacks into an appropriate
\bipref{catch/3}{../bips/kernel/control/catch-3.html}, e.g.,
\begin{quote}
\begin{verbatim}
..., catch(big_goal(X), global_trail_overflow, react_to_overflow), ...
\end{verbatim}
\end{quote}
In the debugger, you can locate the overflow by jumping to a LEAVE port
(z command).
See chapter \ref{chapmemory} for more details on memory usage.


\section{{\eclipse} Fatal Errors}

\index{fatal errors}\index{error!fatal}
A fatal error cannot be caught by the user.
When they occur, the system performs a warm restart.
The following fatal errors may be generated by {\eclipse}:
\begin{description}
\item[*** Fatal error: Out of memory - no more swap space]
The available memory (usually swap space) on the computer has been used up
either by the application or some external process.

\item[*** Fatal error: Internal error - memory corrupted]
This signals an inconsistency in the system's internal data structures.
The reason can be either a bug in the {\eclipse} system itself or in an
external predicate provided by the user.

\end{description}

\section{User-Defined Events}
User-defined events should use atomic event names rather than numbers.
See the description of
\bipref{set_event_handler/2}{../bips/kernel/event/set_event_handler-2.html}.


%\section{System Event Handlers}
%\index{error handlers}
%In the tables above the default event handlers for all the events are
%given. Here follows a short description of these handlers.
%Some of them only print error massages. Those can easily be redefined
%by the user.
%Others do more complex things to achieve a certain behaviour of
%the {\eclipse} system. If those are redefined by the user, this may have
%unexpected results.
%Note that the default handler can always be called (even while the active
%handler is redefined) by using
%\begin{verbatim}
%    error(default(Number), Goal [, Module])
%\end{verbatim}
%It is therefore not necessary to know the names of the default handlers.
%Moreover, most of them are not accessible for the user.
%\begin{description}
%\item[close_handler/2]
%\index{close_handler/2}
%prevents system stream from being closed.
%If an attempt is made to close a stream that a system stream
%is redirected to, the system stream is first reset to its standard value.
%
%\item[autoload_handler/3]
%\index{autoload_handler/3}
%Load the appropriate library and recall the autoloaded goal.
%
%\item[compare_handler/3]
%\index{compare_handler/3}
%applies arithmetic evaluation to the first two arguments of the goal,
%then re-calls the culprit goal with the evaluated arguments.
%
%\item[compiler_error_handler/2]
%\index{compiler_error_handler/2}
%prints error message with relevant line, then fails.
%
%\item[compiled_file_handler/2]
%\index{compiled_file_handler/2}
%prints the message about compiled or dumped file,
%possible with size and time indication
%
%\item[compiler_abort_handler/2]
%\index{compiler_abort_handler/2}
%prints the error message with the file name.
%If it can find out the line number, it is printed as well.
%
%\item[compiler_warning_handler/2]
%\index{compiler_warning_handler/2}
%prints error message with relevant line, then succeeds.
%
%\item[delayed_goals_handler/3]
%\index{delayed_goals_handler/3}
%prints a list of delayed goals and succeeds.
%
%\item[dynamic_handler/3]
%\index{dynamic_handler/3}
%retracts all clauses of the predicate that is already dynamic and succeeds.
%
%\item[eof_handler/2]
%\index{eof_handler/2}
%takes the appropriate action for reaching end of file,
%depending on the culprit goal, e.g., binding the result to {\bf end_of_file}
%if the goal was \bipref{read/1}{../bips/kernel/ioterm/read-1.html}.
%The handler fails for unknown goals.
%
%\item[error_handler/2]
%\index{error_handler/2}
%prints the error message and the culprit.
%Then it raises the event 157 (error exit) which by default
%aborts via {\bf throw(abort)}.
%
%\item[error_handler/3]
%\index{error_handler/3}
%used for errors inside tools.
%It is like {\bf error_handler/2} but it also prints the module
%used by the culprit.
%
%\item[integer_overflow_handler/2]
%\index{integer_overflow_handler/2}
%redo an overflowed word-sized-integer arithmetic operation with bignums.
%
%\item[locked_access_handler/2]
%\index{locked_access_handler/2}
%allows certain goals to be executed in spite of the module lock.
%
%\item[macro_handler/3]
%\index{macro_handler/3}
%prints a warning and redefines a macro.
%
%\item[make_array_handler/3]
%\index{make_array_handler/3}
%If the
%error number is 42 (redefining an existing array), it prints the warning,
%erases the existing array and replaces it by a new one.
%Otherwise, it calls the default handler.
%
%\item[message_handler/2]
%\index{message_handler/2}
%prints the error message followed by the second argument onto toplevel_output
%and succeeds.
%Note that the second argument is not necessarily the culprit goal,
%but rather just a string to be printed.
%This handler is used for events which are not errors.
%
%\item[output_error_handler/2]
%\index{output_error_handler/2}
%closes a related stream if necessary and calls {\bf system_error_handler/2}.
%
%\item[parser_error_handler/1]
%\index{parser_error_handler/1}
%prints the faulty input line and the corresponding error message, then fails.
%Used when the culprit is not important and when no abort should occur.
%
%\item[past_eof_handler/2]
%\index{past_eof_handler/2}
%closes the stream that has been read past end of file, then calls
%{\bf error_handler/2}.
%
%\item[system_error_handler/2]
%\index{system_error_handler/2}
%gets the operating system error number (from {\tt errno})
%and prints the corresponding
%error message.
%Then it raises the event 157 (error exit) which by default
%aborts via {\bf throw(abort)}.
%
%\item[undef_array_handler/3]
%\index{undef_array_handler/3}
%If the culprit was
%\bipref{setval/2}{../bips/kernel/storage/setval-2.html} with an atom as
%first argument, a local non-logical variable of that name is created using
%\bipref{variable/1}{../bips/kernel/storage/variable-1.html} and
%the culprit is re-called.  Otherwise like {\bf error_handler/2}.
%
%\item[undef_dynamic_handler/3]
%\index{undef_dynamic_handler/2}
%when a non-dynamic clause has been asserted, it makes it dynamic
%(if possible), then asserts it.
%
%\item[warning_handler/2]
%\index{warning_handler/2}
%prints the error message and the culprit and succeeds.
%
%\end{description}
%
%\section{System Interrupt Handlers}
%\index{interrupt handlers}
%Some of the signals (interrupts) are handled by built-in predicates
%\bipref{halt/0}{../bips/kernel/opsys/halt-0.html}, \bipref{abort/0}{../bips/kernel/control/abort-0.html} and \bipref{true/0}{../bips/kernel/control/true-0.html},
%others have special handlers:
%
%\begin{description}
%\item[interrupt_prolog/0]
%\index{interrupt_prolog/0}
%asks the user what to do - abort, start a break level, debug, continue or exit.
%
%\item[it_handler/0]
%\index{it_handler/0}
%only prints the signal number to the {\tt error} stream.
%
%\item[it_overflow/0]
%\index{it_overflow/0}
%prints the message "Segmentation violation - maybe machine stack overflow"
%and makes a warm restart.
%
%\item[it_reset/0]
%\index{it_reset/0}
%makes a warm restart.
%\end{description}

%HEVEA\cutend
