% BEGIN LICENSE BLOCK
% Version: CMPL 1.1
%
% The contents of this file are subject to the Cisco-style Mozilla Public
% License Version 1.1 (the "License"); you may not use this file except
% in compliance with the License.  You may obtain a copy of the License
% at www.eclipse-clp.org/license.
%
% Software distributed under the License is distributed on an "AS IS"
% basis, WITHOUT WARRANTY OF ANY KIND, either express or implied.  See
% the License for the specific language governing rights and limitations
% under the License.
%
% The Original Code is  The ECLiPSe Constraint Logic Programming System.
% The Initial Developer of the Original Code is  Cisco Systems, Inc.
% Portions created by the Initial Developer are
% Copyright (C) 1994 - 2006 Cisco Systems, Inc.  All Rights Reserved.
%
% Contributor(s):
%
% END LICENSE BLOCK
%
% @(#)umssocket.tex	1.5 94/01/14
%

% \comment{@(\#)text2.mss    20.3 9/19/88}
% \part{text2, root = `manual.mss'}
\chapter{Interprocess Communication}
%HEVEA\cutdef[1]{section}
\label{sockets}
{\eclipse} contains built-in predicates that support interprocess communications
using sockets.
Sockets implement bidirectional channels
that can connect multiple processes on different machines
in different networks.
The socket predicates are directly mapped to the system calls
and therefore detailed information can be found in the Unix manuals.

Sockets in general allow a networked communication among many processes,
where each packet sent by one process can be sent to different address.
In order to limit the number of necessary built-in predicates,
{\eclipse} supports only point-to-point communication based
on stream or datagram sockets, or many-to-one communication
based on datagrams.
Broadcasting as well as using one socket to send data
to different addresses is not supported, except that
datagram sockets can be re-connected, so that the same
socket is directed to another address.
Below we describe the basic communication types that are available
in {\eclipse}.

Note that the sockets streams in {\eclipse} are buffered like
all other streams, and so it is necessary to flush
the buffer in order to actually send the data to the socket.
This can be done either with the
\bipref{flush/1}{../bips/kernel/iostream/flush-1.html} predicate
or with the option \notation{\%b} in
\biprefni{printf/2,3}{../bips/kernel/ioterm/printf-2.html}.%
\indextt{printf/2}\indextt{printf/3}

\section{Socket Domains}
Currently there are two available domains, \notation{unix} and
\notation{internet}.
The communication in the \notation{unix} domain is limited
to a single machine running under an Unix operating system, and the sockets
are associated
to files in this machine's file system.

The \notation{internet} domain can be used to connect any two machines
which are connected through the network. It can also connect two processes
on the same machine.
The address of a socket is then identified
by the host name and the port number.
The host name is the same as obtained, e.g., with
\notation{get_flag(hostname,~Host)}.
The port identifies the channel on the host which is used
for the communication. This is available under both Unix and Windows
operating systems.


\section{Stream Connection (internet domain)}
This type of communication is very similar to pipes,
the stream communication is reliable and there are no boundaries
between the messages.
Stream sockets always require explicit connection from both
communicating processes.

After a socket is created with the
\bipref{socket/3}{../bips/kernel/iostream/socket-3.html} predicate,
one of the processes, the server, gives it a name
and waits for a connection.
The other process uses the same name when connecting to
the former process.
After the connection is established, both processes can read and write
on the socket and so the difference between the server and the
client disappears.
The socket addresses contain the host name and the port number.
Since one port number identifies the socket on a given host,
the process cannot itself specify the port number it wants to use
because it can be already in use by another process.
Therefore, the safe approach is to use the default and let the system
specify the port number, which is achieved by leaving the port
uninstantiated.
Since the host is always known,
it can also be left uninstantiated.
The client, however, has to specify both the host name and the port number:
\begin{quote}
\begin{verbatim}
server:
    [eclipse 10]: socket(internet, stream, s), bind(s, X).

    X = acrab5 / 3789
    yes.
    [eclipse 11]: listen(s, 1), accept(s, From, news).
    <blocks waiting for a connection>

client:
    [eclipse 26]: socket(internet, stream, s), connect(s, acrab5/3789).

    yes.
    [eclipse 27]: printf(s, "%w. %b", message(client)), read(s, Msg).

server:
    From = acrab4 / 1627
    yes.
    [eclipse 12]: read(news, Msg),
                  printf(news, "%w. %b", message(server)).

    Msg = message(client)
    yes.

client:
    Msg = message(server)
    yes.
\end{verbatim}
\end{quote}

\section{Datagram Connection (internet domain)}
This type of communication is the most general one offered by {\eclipse}.
It is based on packets sent from one process to another, perhaps across
a network. Any machine which is reachable over
the network can participate in the communication.

The communication protocol does not guarantee that the message
will always be delivered, but normally it will be.
Every packet represents a message which is read separately
at the system level, however at the Prolog level the packet
boundaries are not visible.\footnote{%
  The packet boundaries are not of much
  interest in Prolog because every Prolog term represents itself
  a message with clear boundaries.}
The difference to stream communication is that
there is no obligatory connection between the server and the client.
First the socket has to be created, and then the process which wants
to read from the it binds the socket to a name.
Any other process can then connect directly to this socket
using the \bipref{connect/2}{../bips/kernel/iostream/connect-2.html} predicate
and send data there.
This connection can be temporary, and after writing the message
to the socket the process can connect it to another socket,
or just disconnect it by calling \notation{connect(Socket,~0)}.

Such datagram connection works only in one direction, namely
from the process that called
\bipref{connect/2}{../bips/kernel/iostream/connect-2.html} to the process that
called
\bipref{bind/2}{../bips/kernel/iostream/bind-2.html}, however the connection in
the other direction
can be established in the same way.

Since {\eclipse} does not provide a link to the system call \notation{sendto()},
the address where the packet should be sent
to can be specified only using
\bipref{connect/2}{../bips/kernel/iostream/connect-2.html}.
If the next packet is to be sent to a different address, a new
\bipref{connect/2}{../bips/kernel/iostream/connect-2.html}
call can be used.
The socket can be disconnected by calling \notation{connect(s,~0/0)}.

The functionality of \notation{recvfrom()} is not available, i.e.,
the sender has to identify itself explicitly in the message
if it wants the receiver to know who the sender was.

Below is an example of a program that starts {\eclipse} on all
available machines which are not highly loaded and accepts
a hello message from them.
Note the use of \notation{rsh} to invoke the process on the remote machine
and pass it the host name and port address. Note that this example is Unix
specific.
\begin{quote}
\begin{verbatim}

% Invoke ECLiPSe on all available machines and accept a hello message
% from them.
connect_machines :-
    machine_list(List),        % make a list of machines from ruptime
    socket(internet, datagram, sigio(s)), % signal when data comes
    bind(s, Address),
    set_interrupt_handler(io, io_handler/0),
    connect_machines(List, Address).

% As soon as a message arrives to the socket, the io signal will
% be sent and the handler reads the message.
io_handler :-
    set_flag(enable_interrupts, off),
    read_string(s, "\n", _, Message),
    writeln(Message),
    set_flag(enable_interrupts, on).

% Invoke eclipse on all machines with small load and let them execute
% the start/0 predicate
connect_machines([info(RHost, UpTime, Users, L1, _, _)|Rest],
                  Host/Port
                ) :-
    UpTime > 0,        % it is not down
    L1 < 0.5,          % load not too high
    Users < 3,         % not too many users
    !,
    concat_string(, Command),
    exec(['rsh', RHost, 'eclipse', Host, Port, '-b',
       '/home/lp/micha/sepia4/up.pl', '-e', 'start'], [], _),
    connect_machines(Rest, Host/Port).
connect_machines([_|Rest], Address) :-
    connect_machines(Rest, Address).
connect_machines([], _).

% ECLiPSe on remote hosts is invoked with
%          eclipse host port -b file.pl -e start
% It connects to the socket of the main process,
% sends it a hello message and exits.
start :-
    is_built_in(socket/3),    % to ignore non-BSD machines
    argv(1, SHost),
    argv(2, SPort),
    atom_string(Host, SHost),
    number_string(Port, SPort),
    get_flag(hostname, LHost),
    socket(internet, datagram, s),   % create the socket
    connect(s, Host/Port),           % connect to the main process
    printf(s, "hello from %s\n%b", LHost).

% Invoke ruptime(1) and parse its output to a list of accessible
% machines in the form
%    info(Host, UpTime, Users, Load1, Load2, Load3).
machine_list(List) :-
    % exec/2 cannot be used as it could overflow
    % the pipe and then block
    exec(['ruptime', '-l'], [null, S], P),
    parse_ruptime(S, List),
    close(S),
    wait(P, _),
    !.
\end{verbatim}
\end{quote}

\begin{quote}
\begin{verbatim}
% Parse the output of ruptime
parse_ruptime(S, [Info|List]) :-
    parse_uptime_record(S, Info),
    !,
    parse_ruptime(S, List).
parse_ruptime(_, []).

% parse one line of the ruptime output
parse_uptime_record(S, info(Host, Time, Users, Load1, Load2, Load3)) :-
    read_token(S, Host, _),
    Host \== end_of_file,
    read_token(S, Up, _),
    (Up == up ->
        read_time(S, Time),
        read_token(S, ',', _),
        read_token(S, Users, _),
        read_token(S, _, _),
        read_token(S, ',', _),
        read_token(S, load, _),
        read_token(S, Load1, _),
        read_token(S, ',', _),
        read_token(S, Load2, _),
        read_token(S, ',', _),
        read_token(S, Load3, _)
    ;
        read_time(S, _),
        Time = 0
    ).

% Parse the up/down time and if the machine is down, return 0
read_time(S, Time) :-
    read_token(S, T1, _),
    (read_token(S, +, _) ->
        Days = T1,
        read_token(S, Hours, _),
        read_token(S, :, _)
    ;
        Days = 0,
        Hours = T1
    ),
    read_token(S, Mins, _),
    Time is ((24 * Days) + Hours) * 60 + Mins.
\end{verbatim}
\end{quote}

and here is a script of the session:

\begin{quote}
\begin{verbatim}
[eclipse 1]: [up].
up.pl      compiled traceable 4772 bytes in 0.08 seconds

yes.
[eclipse 2]: connect_machines.
sending to mimas3
sending to mimas8
sending to acrab23
sending to europa1
sending to europa5
sending to regulus2
sending to miranda5
sending to mimas2
sending to triton6
sending to europa2
sending to acrab7
sending to europa3
sending to sirius
sending to miranda6
sending to charon6
sending to acrab13
sending to triton1
sending to acrab20
sending to triton4
sending to charon2
sending to triton5
sending to acrab24
sending to acrab21
sending to scorpio
sending to acrab14
sending to janus5

yes.
[eclipse 3]: hello from mimas3
eclipse: Command not found.     % eclipse not installed here
hello from regulus2
hello from mimas8
hello from acrab20
hello from europa1
hello from mimas2
hello from miranda6
hello from miranda5
hello from europa3
hello from charon6
hello from charon2
hello from acrab24
hello from triton5
hello from acrab21
hello from janus5
hello from triton4
hello from triton6
hello from europa2
hello from europa5
hello from acrab23
hello from triton1
hello from acrab14
hello from acrab13
hello from acrab7
\end{verbatim}
\end{quote}


\section{Stream Connection (unix domain)}
The sequence of operations is the same as for the internet domain,
however in the unix domain the socket addresses are
the file names:

\begin{quote}
\begin{verbatim}
server:
    [eclipse 10]: socket(unix, stream, s), bind(s, '/tmp/sock').

    yes.
    [eclipse 11]: listen(s, 1), accept(s, _, news).
    <blocks waiting for a connection>

client:
    [eclipse 26]: socket(unix, stream, s), connect(s, '/tmp/sock').

    yes.
    [eclipse 27]: printf(s, "%w. %b", message(client)), read(s, Msg).

server:
    [eclipse 12]: read(news, Msg),
                  printf(news, "%w. %b", message(server)).

    Msg = message(client)
    yes.

client:
    Msg = message(server)
    yes.
\end{verbatim}
\end{quote}

\section{Datagram Connection (unix domain)}
This is similar to datagram connection in the internet domain, except that
it is limited to communications between two processes on the same Unix
machine.

Again, like in the internet domain, the connection must be established
in both directions if bi-direction communication is required:
\begin{quote}
\begin{verbatim}
server:
    % Make a named socket and read two terms from it
    [eclipse 10]: socket(unix, datagram, s), bind(s, '/tmp/sock').

    yes.
    [eclipse 11]: read(s, X), read(s, Y).

process1:
    % Connect a socket to the server and write one term
    [eclipse 32]: socket(unix, datagram, s), connect(s, '/tmp/sock').

    yes.
    [eclipse 33]: printf(s, "%w. %b", message(process1)).

process2:
    % Connect a named socket to the server and write another term
    [eclipse 15]: socket(unix, datagram, s), connect(s, '/tmp/sock'),
        bind(s, '/tmp/socka').

    yes.
    [eclipse 16]: printf(s, "%w. %b", message(process2)).

    yes.
    % And now disconnect the output socket from the server
    [eclipse 17]: connect(s, 0).

    yes.

server:
    % Now the server can read the two terms
    X = message(process1)
    Y = message(process2)
    yes.
    % and it writes one term to the second process on the same socket
    [eclipse 12]: connect(s, '/tmp/socka'),
        printf(s, "%w. %b", message(server)).

process2:
    %
    [eclipse 18]: read(s, Msg).

    Msg = message(server)
    yes.
\end{verbatim}
\end{quote}

%HEVEA\cutend
