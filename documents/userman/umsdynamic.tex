% BEGIN LICENSE BLOCK
% Version: CMPL 1.1
%
% The contents of this file are subject to the Cisco-style Mozilla Public
% License Version 1.1 (the "License"); you may not use this file except
% in compliance with the License.  You may obtain a copy of the License
% at www.eclipse-clp.org/license.
% 
% Software distributed under the License is distributed on an "AS IS"
% basis, WITHOUT WARRANTY OF ANY KIND, either express or implied.  See
% the License for the specific language governing rights and limitations
% under the License. 
% 
% The Original Code is  The ECLiPSe Constraint Logic Programming System. 
% The Initial Developer of the Original Code is  Cisco Systems, Inc. 
% Portions created by the Initial Developer are
% Copyright (C) 2006 Cisco Systems, Inc.  All Rights Reserved.
% 
% Contributor(s): 
% 
% END LICENSE BLOCK

%------------------------------------------------------------------------
\chapter{Dynamic Code}
%HEVEA\cutdef[1]{section}
%------------------------------------------------------------------------
\label{chapdynamic}

Support for dynamic code is provided partly for compatibility with
Prolog. Note that \eclipse provides much better primitives (see
chapter~\ref{chaparrays}) to support the non-logical storage of information
-- a major use for dynamic  predicates in Prolog. 

An {\eclipse} predicate can be made \emph{dynamic}.
That is, it can have clauses added and removed from its definition at run
time.
This chapter discusses how to do this, and what the implications are.

%------------------------------------------------------------------------
\section{Compiling Procedures as Dynamic or Static}
\label{compdynamic}

If it is intended that
a procedure be altered through the use of \bipref{assert/1}{../bips/kernel/dynamic/assert-1.html} and \bipref{retract/1}{../bips/kernel/dynamic/retract-1.html},
\index{assert/1}
\index{retract/1}
the system should be informed that the procedure will be dynamic,
since these predicates are
designed to work on dynamic procedures. 
If \bipref{assert/1}{../bips/kernel/dynamic/assert-1.html} is applied on a non-existing procedure, an error
is raised, however the default error handler for this error
only declares the procedure as dynamic and then makes the assertion.

A procedure is by default static unless it has been specifically declared as
dynamic.
Clauses of static procedures must always be consecutive,
they may not
be separated in one or more source files or by the user from the top level.
If the static procedure clauses are not consecutive, each of the
consecutive parts is taken as a separate procedure which redefines
the previous occurrence of that procedure, and so only the last one will
remain.
However, whenever the compiler encounters nonconsecutive clauses of a static
procedure in one file, it raises an exception whose default handler
prints a warning but it continues to compile the rest of the file.

If a procedure is to be dynamic the {\eclipse} system should be
given a specific {\it dynamic declaration}
A dynamic declaration takes the form
\begin{quote} \begin{verbatim}
:- dynamic SpecList.
\end{verbatim} \end{quote}
\index{dynamic/1}
The predicate \bipref{is_dynamic/1}{../bips/kernel/dynamic/is_dynamic-1.html} may be used to check if a procedure
is dynamic:
\begin{quote} \begin{verbatim}
is_dynamic(Name/Arity).
\end{verbatim} \end{quote}
\index{is_dynamic/1}
When the goal 
\begin{quote} \begin{verbatim}
compile(Somefile)
\end{verbatim} \end{quote}
is executed
and {\tt Somefile} contains clauses for procedures that have
already been defined
in the Prolog database, those procedures are treated in one of two ways:
If such a procedure is dynamic, its clauses compiled from {\tt Somefile}
are added to the database (just as would happen if they were asserted),
and the existing clauses are not affected.
For example, if the following
clauses have already been compiled:
\begin{quote}
\begin{verbatim}
:- dynamic city/1.

city(london).
city(paris).
\end{verbatim}
\end{quote}
and the file {\tt Somefile} contains the
following Prolog code:
\begin{quote}
\begin{verbatim}
city(munich).
city(tokyo).
\end{verbatim}
\end{quote}
then compiling {\tt Somefile} will cause adding
the clauses for {\bf city/1} to those
already compiled, as {\bf city/1} has been declared dynamic.
Thus the query  {\bf city(X}) will give:
\begin{quote}
\begin{verbatim}
[eclipse 5]: city(X).
X = london    More? (;)

X = paris    More? (;)

X = munich    More? (;)

X = tokyo
yes.
\end{verbatim}
\end{quote}

If, however, the compiled procedure is static,
the new clauses in {\tt Somefile} replace the old procedure.
Thus, if
the following clauses have been compiled:
\begin{quote}
\begin{verbatim}
city(london).
city(paris).
\end{verbatim}
\end{quote}
and the file {\tt Somefile} contains the following Prolog code:
\begin{quote}
\begin{verbatim}
city(munich).
city(tokyo).
\end{verbatim}
\end{quote}
when {\tt Somefile} is compiled, then the procedure {\bf city/1} is redefined.
Thus the query {\tt city(X}) will give:
\begin{quote}
\begin{verbatim}
[eclipse 5]: city(X).
X = munich    More? (;)

X = tokyo
yes.
\end{verbatim}
\end{quote}

When the \bipref{dynamic/1}{../bips/kernel/dynamic/dynamic-1.html} declaration is used on a procedure that is
already dynamic, which may happen for instance by recompiling a file
with this declaration inside, the system raises the error 64,
'procedure already dynamic'.
The default handler for this error, however, will only erase
all existing clauses for the specified procedure, so that
when such a file is recompiled several times during its debugging,
the system behaves as expected, the existing clauses
are always replaced.
The handler for this error can of course be changed if required.
If it is set to \bipref{true/0}{../bips/kernel/control/true-0.html}, for instance,
the \bipref{dynamic/1}{../bips/kernel/dynamic/dynamic-1.html} declaration is be just silently
accepted without erasing any clauses and without printing an error message.

%------------------------------------------------------------------------
\section{Altering programs at run time}

The Prolog database can be updated during the execution of a program.
\index{database}
{\eclipse} allows the user to modify procedures dynamically by adding
new clauses via \bipref{assert/1}{../bips/kernel/dynamic/assert-1.html} and by removing some clauses via \bipref{retract/1}{../bips/kernel/dynamic/retract-1.html}.
\index{assert/1}
\index{retract/1}
These predicates operate on dynamic procedures; if it is
required that the definition of a procedure be altered through
assertion and retraction, the procedure should therefore first be declared
dynamic (see the previous section). The effect of \bipref{assert/1}{../bips/kernel/dynamic/assert-1.html} and
\bipref{retract/1}{../bips/kernel/dynamic/retract-1.html} on static procedures is explained below.


The effect of the goal
\begin{quote}
\begin{verbatim}
assert(ProcClause)
\end{verbatim}
\end{quote}
where
{\tt ProcClause}\footnote{It should be remembered that because of the
definition of the syntax of a term, to assert a procedure of the form p
:- q,r it is necessary to enclose it in parentheses:
{\tt assert((p:-q,r))}.}
is a clause of the procedure {\tt Proc}, is as follows.
\begin{enumerate}
\item If {\tt Proc} has not been previously defined, the assertion
raises an exception, however the default handler for this exception
just declares the given procedure silently as dynamic and executes
the assertion.

\item If {\tt Proc} is already defined as a dynamic procedure,
the assertion adds {\it ProcClause}
to the database after any clauses already existing for {\tt Proc}.

\item If {\tt Proc} is already defined as a static procedure, then the assertion
raises an exception.
\end{enumerate}

\noindent
The goal 
\begin{quote}
\begin{verbatim}
retract(Clause)
\end{verbatim} 
\end{quote}
will unify {\tt Clause} with a clause on the dynamic database and remove it.
If {\tt Clause} does not specify a dynamic procedure, an exception is raised.

\index{logical update semantics}
{\eclipse}'s dynamic database features the so-called {\it logical update semantics}.
This means that any change in the database that occurs as a result of
executing one of the builtins of the abolish, assert or retract family
affects only those goals that start executing
afterwards. For every call to a dynamic procedure, the procedure is
virtually frozen at call time.

\section{Differences between static and dynamic code}

\begin{itemize}
\item Only dynamic procedures can have clauses added or removed at run time.
\item Matching clauses (section~\ref{matching}) are not supported by dynamic
  code. A runtime error (calling an undefined procedure $-\query->$/1) will
  be raised when executing dynamic code that has a matching clause head.
\item Clauses for a dynamic procedure need not be consecutive.
\item Source tracing is not supported for dynamic procedures.
\item assert/1, retract/1 and clause/1 does not perform clause
  transformation on the clause. If clause transformation is required,
  this can be done explicitly with 
  bipref{expand_clause/2}{../bips/kernel/database/expand_clause-2.html}
  before.
\item Internally, dynamic procedures are represented differently from static
  procedures. The execution of dynamic procedures will generally be slower
  than for static procedures.
\end{itemize}
   
%HEVEA\cutend
