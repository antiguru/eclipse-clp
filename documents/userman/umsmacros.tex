% BEGIN LICENSE BLOCK
% Version: CMPL 1.1
%
% The contents of this file are subject to the Cisco-style Mozilla Public
% License Version 1.1 (the "License"); you may not use this file except
% in compliance with the License.  You may obtain a copy of the License
% at www.eclipse-clp.org/license.
% 
% Software distributed under the License is distributed on an "AS IS"
% basis, WITHOUT WARRANTY OF ANY KIND, either express or implied.  See
% the License for the specific language governing rights and limitations
% under the License. 
% 
% The Original Code is  The ECLiPSe Constraint Logic Programming System. 
% The Initial Developer of the Original Code is  Cisco Systems, Inc. 
% Portions created by the Initial Developer are
% Copyright (C) 1996 - 2006 Cisco Systems, Inc.  All Rights Reserved.
% 
% Contributor(s): 
% 
% END LICENSE BLOCK
%
% @(#)umsmacros.tex	1.9 96/01/08 
%
% umsmacros.tex
%

\chapter{{\eclipse} Macros}
%HEVEA\cutdef[1]{section}
\label{chapmacros}

%----------------------------------------------------------------------
\section{Introduction}
%----------------------------------------------------------------------
\index{source transformation}
\index{macro expansion}
{\eclipse} provides a general mechanism to perform macro expansion
of Prolog terms.
Macro expansion can be performed in 3 situations:
\begin{description}

\item[read macros]
\index{read macros}
\index{macros!read}
they are expanded just after a Prolog term has been read by the
{\eclipse} parser. Note that the parser is not only used during
comilation but by all
\biptxtref{term-reading}{read/1}{../bips/kernel/ioterm/read-1.html}
predicates.

\item[compiler macros]
\index{compiler macros}
\index{macros!compiler}
they are expanded only during compilation and only when a term occurs
in a certain context (clause or goal).

\item[write macros]
\index{write macros}
\index{macros!write}
they are expanded just before a Prolog term is printed by one of the
output predicates

\end{description}
Macros are attached to classes of terms specified by their functors
or by their type.
Macros obey the module system's visibility rules.
They may be either
\biptxtref{local}{local/1}{../bips/kernel/modules/local-1.html}
or
\biptxtref{exported}{export/1}{../bips/kernel/modules/export-1.html}.
The macro expansion is performed by a user-defined Prolog predicate.


%----------------------------------------------------------------------
\section{Using the macros}
\label{usingmacros}
%----------------------------------------------------------------------

The following declarations and built-ins control macro expansion:
\begin{description}
\item[local macro(+TermClass, +TransPred, +Options)]
define a macro for the given {\it TermClass}. The transformation will
be performed by the predicate {\it TransPred}.
\item[export macro(+TermClass, +TransPred, +Options)]
as above, but available to other modules.
\item[erase_macro(+TermClass, +Options)]
erase a currently defined macro for {\it TermClass}. This can only be done
in the module where the definition was made.
\item[current_macro(?TermClass, ?TransPred, ?Options, ?Module)]
retrieve information about currently defined visible macros.
\end{description}
Macros are selectively applied only to terms of the specified class.
{\it TermClass} can take two forms:
\begin{description}
\item[Name/Arity] transform all terms with the specified functor
\index{type macros}
\index{macros!type}
\item[type(Type)] transform all terms of the specified type, where Type
is one of {\tt compound, string, integer, rational, float, breal, atom,
goal}\footnote{type(goal) stands for suspensions.}.
\end{description}
The {+TransPred} argument specifies the predicate that will perform the
transformation. It has to be of arity 2 or 3 and should have the form:
\begin{quote}\begin{verbatim}
trans_function(OldTerm, NewTerm [, Module]) :- ... .
\end{verbatim}\end{quote}
At transformation time, the system will call {\it TransPred} in the module
where \bipref{macro/3}{../bips/kernel/syntax/macro-3.html} was invoked.
The term to transform is passed as the first argument, the second is a free
variable which the transformation predicate should bind to the
transformed term, and the optional
third argument is the module where the term is read or written.

{\it Options} is a list which may be empty (in this case the macro defaults
to a local read term macro) or contain specifications from
the following categories:
\begin{itemize}
\item mode
\begin{description}
\index{macros!read}
\item[read:] This is a read macro and shall be applied after reading a
term (default).

\index{macros!write}
\item[write:] This is a write macro and shall be applied before printing
a term. 
\end{description}

\item type
\begin{description}
\index{macros!term}
\item[term:] Transform all terms (default).

\index{macros!clause}
\item[clause:] Transform only if the term is a program clause,
i.e. inside \bipref{compile/1}{../bips/kernel/database/compile-1.html}, \bipref{assert/1}{../bips/kernel/dynamic/assert-1.html} etc.
Write macros are applied using the 'C' option in the \bipref{printf/2}{../bips/kernel/ioterm/printf-2.html} predicate.

\index{macros!goal}
\item[goal:] Goal-read-macros are transformed only if the term is a
subgoal in the body of a program clause.
Goal-write macros are applied using the 'G' option in the
\bipref{printf/2}{../bips/kernel/ioterm/printf-2.html} predicate.
\end{description}

\item additional specification
\begin{description}
\index{macros!protect_arg}
\item[protect_arg:] Disable transformation of subterms (optional).
\index{macros!top_only}
\item[top_only:] Consider only the whole term, not subterms (optional).
\end{description}
\end{itemize}
The following shorthands exist:
\begin{description}
\item[local/export portray(+TermClass, +TransPred, +Options)]
    \bipref{portray/3}{../bips/kernel/syntax/portray-3.html}
    is like
    \bipref{macro/3}{../bips/kernel/syntax/macro-3.html},
    but the write-option is implied.
\item[inline(+PredSpec, +TransPred)]
    \bipref{inline/2}{../bips/kernel/database/inline-2.html}
    is the same as a goal-read-macro. The visibility is inherited
    from the transformed predicate.
\end{description}

Here is an example of a conditional read macro:
\begin{quote}
\begin{verbatim}
[eclipse 1]: [user].
 trans_a(a(X,Y), b(Y)) :-    % transform a/2 into b/1,
        number(X),           % but only under these
        X > 0.               % conditions

:- local macro(a/2, trans_a/2, []).
  user       compiled traceable 204 bytes in 0.00 seconds

yes.
[eclipse 2]: read(X).
        a(1, hello).

X = b(hello)                 % transformed
yes.
[eclipse 3]: read(X).
        a(-1, bye).

X = a(-1, bye)               % not transformed
yes.
\end{verbatim}
\end{quote}
If the transformation function fails, the term is not transformed. Thus, 
{\bf a(1, zzz)} is transformed into {\bf b(zzz)} but {\bf a(-1, zzz)} 
is not transformed.
The arguments are transformed bottom-up. It is possible to protect the 
subterms of a transformed term by specifying the flag {\tt protect_arg}.

A term can be protected against transformation by quoting it with 
the ``protecting functor'' (by default it is {\bf no_macro_expansion/1}):
\index{no_macro_expansion/1}
\index{macro!no_macro_expansion}
\begin{quote}
\begin{verbatim}
[eclipse 4]: read(X).
        a(1, no_macro_expansion(a(1, zzz))).
X = b(a(1, zzz)).
\end{verbatim}
\end{quote}
Note that the protecting functor is itself defined as a macro:
\begin{quote} \begin{verbatim}
trprotect(no_macro_expansion(X), X).
:- export macro(no_macro_expansion/1, trprotect/2, [protect_arg]).
\end{verbatim} \end{quote}

A local macro is only visible in the module where it has been defined.
When it is defined as exported, then it is copied to all
other modules that contain a
\bipref{use_module/1}{../bips/kernel/modules/use_module-1.html} or
\bipref{import/1}{../bips/kernel/modules/import-1.html}
for this module.
The transformation function should also be exported in this case.
There are a few global macros predefined by the system, e.g.\ for
% the following stops latex2html from turning --> into ->
{\tt -}{\tt ->/2} (grammar rules, see below) or {\tt with/2} and {\tt of/2}
(structure syntax, see section \ref{chapstruct}).
These predefined macros can be hidden by local macro definitions.

\index{macro_expansion}
The global flag {\bf macro_expansion} can be used to disable
macro expansion globally, e.g.\ for debugging purposes.
Use {\tt set_flag(macro_expansion, off)} to do so.

The next example shows the use of a type macro. Suppose we want to represent
integers as s/1 terms:
\begin{quote} \begin{verbatim}
[eclipse 1]: [user].
 tr_int(0, 0).
 tr_int(N, s(S)) :- N > 0, N1 is N-1, tr_int(N1, S).
 :- local macro(type(integer), tr_int/2, []).

yes.
[eclipse 2]: read(X).
        3.

X = s(s(s(0)))
yes.
\end{verbatim} \end{quote}
When we want to convert the s/1 terms back to normal integers so that they
are printed in the familiar form, we can use a write macro.
Note that we first erase the read macro for integers, otherwise we would get
unexpected effects since all integers occurring in the definition of
tr_s/2 would turn into s/1 structures:
\begin{quote} \begin{verbatim}
[eclipse 3]: erase_macro(type(integer)).

yes.
[eclipse 4]: [user].
 tr_s(0, 0).
 tr_s(s(S), N) :- tr_s(S, N1), N is N1+1.
 :- local macro(s/1, tr_s/2, [write]).

yes.
[eclipse 2]: write(s(s(s(0)))).
3
yes.
\end{verbatim} \end{quote}

%----------------------------------------------------------------------
\section{Definite Clause Grammars --- DCGs}
\label{dcg}
%----------------------------------------------------------------------
\index{DCG}
\index{definite clause grammar}
\index{grammar rules}
\index{$-->$/2}
Grammar rules are described in many standard Prolog texts (\cite{clocksin81}).
In {\eclipse} they are provided by a predefined global\footnote{
So that the user can redefine it with a local one.} macro for
% the following stops latex2html from turning --> into ->
{\tt -}{\tt ->/2}.
When the parser reads a clause whose main functor is {\tt -}{\tt ->/2}, it transforms 
it according to the standard rules.
The syntax for DCGs is as follows: 
\begin{quote}
\begin{verbatim}
grammar_rule --> grammar_head, ['-->'], grammar_body.

grammar_head --> non_terminal.
grammar_head --> non_terminal, [','], terminal.

grammar_body --> grammar_body, [','], grammar_body.
grammar_body --> grammar_body, [';'], grammar_body.
grammar_body --> grammar_body, ['->'], grammar_body.
grammar_body --> grammar_body, ['|'], grammar_body.
grammar_body --> iteration_spec, ['do'], grammar_body.
grammar_body --> ['-?->'], grammar_body.
grammar_body --> grammar_body_item.

grammar_body_item --> ['!'].
grammar_body_item --> ['{'], Prolog_goals, ['}'].
grammar_body_item --> non_terminal.
grammar_body_item --> terminal.
\end{verbatim}
\end{quote}
The non-terminals are syntactically identical to prolog goals (atom, compound
term or variable), the terminals are lists of prolog terms (typically
characters or tokens). Every 
term is transformed, unless it is enclosed in curly brackets. The control
constructs like conjunction {\tt ,/2}, disjunction ({\tt ;/2} or {\tt |/2}),
the cut ({\tt !/0}), the condition ({\tt ->/1}) and do-loops do not need to
be enclosed in curly brackets.

The grammar can be accessed with the built-in \bipref{phrase/3}{../bips/kernel/control/phrase-3.html}.
The first argument of \bipref{phrase/3}{../bips/kernel/control/phrase-3.html} is the name of the
grammar to be used, the 
second argument one is a list containing the input to be parsed. If the
parsing is successful the built-in will succeed.
For instance with the grammar
\begin{quote}
\begin{verbatim}
a --> [] | [z], a.
\end{verbatim}
\end{quote}
{\tt phrase(a, X, [])} will give on backtracking: {\tt X = [z] ; X = [z, z] ; X = [z, z, z] ; ...}.

\subsection{Simple DCG example}

The following example illustrates a simple grammar declared using the DCGs.

\begin{quote}
\begin{verbatim}
sentence --> imperative, noun_phrase(Number), verb_phrase(Number).

imperative, [you] --> [].
imperative --> [].

noun_phrase(Number) --> determiner, noun(Number).
noun_phrase(Number) --> pronom(Number).

verb_phrase(Number) --> verb(Number).
verb_phrase(Number) --> verb(Number), noun_phrase(_).

determiner --> [the].

noun(singular) --> [man].
noun(singular) --> [apple].
noun(plural) --> [men].
noun(plural) --> [apples].

verb(singular) --> [eats].
verb(singular) --> [sings].
verb(plural) --> [eat].
verb(plural) --> [sing].

pronom(plural) --> [you].
\end{verbatim}
\end{quote}
The above grammar may be successfully parsed
using \bipref{phrase/3}{../bips/kernel/control/phrase-3.html}. If the predicate
succeeds then the input has been parsed successfully.
\begin{quote}
\begin{verbatim}
[eclipse 1]: phrase(sentence, [the,man,eats,the,apple], []).

yes.
[eclipse 2]: phrase(sentence, [the,men,eat], []).

yes.
[eclipse 3]: phrase(sentence, [the,men,eats], []).

no.
[eclipse 4]: phrase(sentence, [eat,the,apples], []).

yes.
[eclipse 5]: phrase(sentence, [you,eat,the,man], []). 

yes.
\end{verbatim}
\end{quote}
The predicate \bipref{phrase/3}{../bips/kernel/control/phrase-3.html} may be used to return the point at which
parsing of a grammar fails --- if the returned list is empty then the
input has been successfully parsed.

\begin{quote}
\begin{verbatim}
[eclipse 1]: phrase(sentence, [the,man,eats,something,nasty],X).

X = [something, nasty]     More? (;) 

no (more) solution.
[eclipse 2]: phrase(sentence, [eat,the,apples],X).

X = [the, apples]     More? (;) 

X = []     More? (;) 

no (more) solution.
[eclipse 3]: phrase(sentence, [hello,there],X).

no (more) solution.
\end{verbatim}
\end{quote}

\subsection{Mapping to Prolog Clauses}
Grammar rule are translated to Prolog clauses by adding two arguments
which represent the input before and after the nonterminal which is
represented by the rule.
The effect of the transformation can be observed, e.g.\ by calling bipref{expand_clause/2}{../bips/kernel/database/expand_clause-2.html}:
\begin{quote} \begin{verbatim}
[eclipse 1]: expand_clause(p(X) --> q(X), Expanded).

X = X
Expanded = p(X, _250, _243) :- q(X, _250, _243)
Yes (0.00s cpu)
[eclipse 2]: expand_clause(p(X) --> [a], Expanded).

X = X
Expanded = p(X, _251, _244) :- 'C'(_251, a, _244)
Yes (0.00s cpu)
\end{verbatim} \end{quote}

\subsection{Parsing other Data Structures}

DCGs are in principle not limited to the parsing of lists.
The predicate \bipref{'C'/3}{../bips/kernel/termmanip/C-3.html} is responsible for reading resp.\ generating
the input tokens. The default definition is
\begin{quote}\begin{verbatim}
'C'([Token|Rest], Token, Rest).
\end{verbatim}\end{quote}
The first argument represents the parsing input before consuming
Token and Rest is the input after consuming Token.

By redefining 'C'/3, it is possible to apply a DCG to other
input sources than a list, e.g.\ to parse directly from an I/O stream:
\begin{quote}\begin{verbatim}
:- local 'C'/3.
'C'(Stream-Pos0, Token, Stream-Pos1) :-
        seek(Stream, Pos0),
        read_string(Stream, " ", _, TokenString),
        atom_string(Token, TokenString),
        at(Stream, Pos1).

 sentence --> noun, [is], adjective.
 noun --> [prolog] ; [lisp].
 adjective --> [boring] ; [great].
\end{verbatim}\end{quote}
This can then be applied to a string as follows:
\begin{quote}\begin{verbatim}
[eclipse 1]: String = "prolog is great", open(String, string, S),
             phrase(sentence, S-0, S-End).
...
End = 15
yes.
\end{verbatim}\end{quote}
Here is another redefinition of 'C'/3, using a similar idea, which allows
the direct parsing of {\eclipse} strings as sequences of characters:
\begin{quote}\begin{verbatim}
:- local 'C'/3.
'C'(String-Pos0, Char, String-Pos1) :-
        Pos0 =< string_length(String),
        string_code(String, Pos0, Char),
        Pos1 is Pos0+1.

anagram --> [].
anagram --> [_].
anagram --> [C], anagram, [C].
\end{verbatim}\end{quote}
This can then be applied to a string as follows:
\begin{quote}\begin{verbatim}
[eclipse 1]: phrase(anagram, "abba"-1, "abba"-5).
yes.
[eclipse 2]: phrase(anagram, "abca"-1, "abca"-5).
no (more) solution.
\end{verbatim}\end{quote}
Unlike the default definition, these redefinitions of 'C'/3 are not bi-directional.
Consequently, the grammar rules using them can only be used for parsing,
not for generating sentences.

Note that every grammar rule uses that definition of 'C'/3 which is visible in
the module where the grammar rule itself is defined.

%HEVEA\cutend
