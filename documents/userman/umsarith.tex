% BEGIN LICENSE BLOCK
% Version: CMPL 1.1
%
% The contents of this file are subject to the Cisco-style Mozilla Public
% License Version 1.1 (the "License"); you may not use this file except
% in compliance with the License.  You may obtain a copy of the License
% at www.eclipse-clp.org/license.
%
% Software distributed under the License is distributed on an "AS IS"
% basis, WITHOUT WARRANTY OF ANY KIND, either express or implied.  See
% the License for the specific language governing rights and limitations
% under the License.
%
% The Original Code is  The ECLiPSe Constraint Logic Programming System.
% The Initial Developer of the Original Code is  Cisco Systems, Inc.
% Portions created by the Initial Developer are
% Copyright (C) 1995 - 2006 Cisco Systems, Inc.  All Rights Reserved.
%
% Contributor(s):
%
% END LICENSE BLOCK
%
% @(#)umsarith.tex	1.8 95/03/17
%
%
%
% REL      DATE      AUTHOR             DESCRIPTION
% 2.10     080489    David Miller       Begin work on new arithmetic chapter
%                                       based on old builtins chapter.
%          250489    Joachim Schimpf    Modified for inline compiled arithmetic
%
\chapter{Arithmetic Evaluation\label{chaparith}}
\index{arithmetic}
%HEVEA\cutdef[1]{section}

\section{Built-Ins to Evaluate Arithmetic Expressions}
\index{arithmetic!built-ins}\index{built-ins!arithmetic}
Unlike other languages, Prolog usually interprets an arithmetic expression like
\notation{3 + 4} as a compound term with functor \notation{+} and two arguments.
Therefore a query like \notation{3 + 4 = 7} fails because a compound term does
not
unify with a number. The evaluation of an arithmetic expression has to be
explicitly requested by using one of the built-ins described below.

\index{comparison!arithmetic}\index{arithmetic!comparison}%
\index{arithmetic!built-ins!comparison}\index{built-ins!arithmetic!comparison}
The basic predicate for evaluating an arithmetic expression is
\bipref{is/2}{../bips/kernel/arithmetic/is-2.html}.
Apart from that only the 6 arithmetic comparison predicates evaluate
arithmetic expressions automatically.

\begin{quote}
\begin{description}
\item[\pattern{Result} is \pattern{Expression}]\indextt{is/2}
\about{Expression} is a valid arithmetic expression and \about{Result}
is an uninstantiated variable or a number.
The system evaluates \about{Expression} which yields a numeric result.
This result is then unified with \about{Result}.
An error occurs if \about{Expression} is not a valid arithmetic expression or
if the evaluated value and \about{Result} are of different types.

\item[\pattern{Expr1} $<$ \pattern{Expr2}]\indextt{</2}
succeeds if (after evaluation and type coercion) \about{Expr1} is less than
\about{Expr2}.

\item[\pattern{Expr1} $>=$ \pattern{Expr2}]\indextt{>=/2}
succeeds if (after evaluation and type coercion) \about{Expr1} is greater or
equal to \about{Expr2}.

\item[\pattern{Expr1} $>$ \pattern{Expr2}]\indextt{>/2}
succeeds if (after evaluation and type coercion) \about{Expr1} is greater than
\about{Expr2}.

\item[\pattern{Expr1} $=<$ \pattern{Expr2}]\indextt{=</2}
succeeds if (after evaluation and type coercion) \about{Expr1} is less or equal
to \about{Expr2}.

\item[\pattern{Expr1} =:= \pattern{Expr2}]\indextt{=:=/2}
succeeds if (after evaluation and type coercion) \about{Expr1} is equal to
\about{Expr2}.

\item[\pattern{Expr1} =\bsl= \pattern{Expr2}]%
\index{=\bsl=/2@\notation{=}\bsl\notation{=/2}}
succeeds if (after evaluation and type coercion) \about{Expr1} is not equal to
\about{Expr2}.
\end{description}
\end{quote}


\subsection{Arithmetic Evaluation vs Arithmetic Constraint Solving}

This chapter deals purely with the evaluation of arithmetic expressions
containing numbers. No uninstantiated variables must occur within the
expressions at the time they are evaluated. This is exactly like
arithmetic evaluation in procedural languages.

As opposed to that, in arithmetic constraint solving one can state
equalities and inequalities involving variables, and a constraint
solver tries to find values for these variables which satisfy these
constraints. Note that {\eclipse} uses the same syntax in both cases,
but different implementations providing different solving capabilities.
See the chapter \chapname{Common Solver Interface} in the
\ahref{../libman/libman.html}{Constraint Library Manual} for an overview.


\section{Numeric Types and Type Conversions}
\index{arithmetic!types}\index{types!arithmetic}
{\eclipse} distinguishes four types of numbers: \defnotionni{integers},
\defnotionni{rationals}, \defnotionni{floats} and \defnotionni{bounded reals}.
% we might have bounded integers at some point...

\subsection{Integers\label{intrep}}
\index{integers}
\index{bignum}
\index{type!integer}
The magnitude of integers is only limited by your available
memory. However, integers that fit into the word size of your computer are
represented more efficiently (this distinction is invisible to the user).
Integers are written in decimal notation or in base notation, for example:
\begin{quote}
\begin{verbatim}
0  3  -5  1024  16'f3ae  0'a  15511210043330985984000000
\end{verbatim}
\end{quote}

Note that integer range is unlimited if {\eclipse} was compiled with
bignum support. Otherwise, integers are restricted to that representable
in a machine word, and  \notation{max_integer flag} of
\bipref{get_flag/2}{../bips/kernel/env/get_flag-2.html}
returns the maximum integer value.

\subsection{Rationals}
\index{rational numbers}
\index{type!rational}
Rational numbers implement the corresponding mathematical domain,
i.e., ratios of two integers (numerator and denominator).
{\eclipse} represents rationals in a canonical form where the
greatest common divisor of numerator and denominator is 1 and the
denominator is positive. Rational constants are written as numerator
and denominator separated by an underscore, e.g.,
\begin{quote}
\begin{verbatim}
1_3  -30517578125_32768  0_1
\end{verbatim}
\end{quote}
Rational arithmetic is arbitrarily precise. When the global flag
\index{global flag!prefer_rationals}
\notationidx{prefer_rationals} is set, the system uses rational arithmetic
wherever possible. In particular, dividing two integers then yields a precise
rational rather than a float result.

Rationals are supported if {\eclipse} is compiled with bignum support.
If rationals are not supported, a type error will be raised when a rational is
required.

\subsection{Floating Point Numbers}
\index{floating point numbers}
\index{type!float}
Floating point numbers conceptually correspond to the mathematical
domain of real numbers, but are not precisely represented.
Floats are written with decimal point and/or an exponent, e.g.,
\begin{quote}
\begin{verbatim}
0.0  3.141592653589793  6.02e23  -35e-12  -1.0Inf
\end{verbatim}
\end{quote}
{\eclipse} uses double precision floats\footnote{%
  {\eclipse} versions older than 5.5 optionally supported single precision
  floats. This is no longer the case.}.


\subsection{Bounded Real Numbers}
\index{bounded reals}
\index{breal}
\index{type!breal}
It is a well known problem that floating point arithmetic suffers
from rounding errors.
To provide safe arithmetic over the real numbers, {\eclipse}
also implements bounded reals\footnote{%
  We have chosen to use the term \about{bounded real} rather than
  \about{interval} in order to avoid confusion with interval variables
  as used in the interval arithmetic constraint solvers}.
A bounded real consists of a pair of floating point numbers
which constitute a safe lower and upper bound for the real number that
is being represented. Bounded reals are written as two floating point
numbers separated by two underscores, e.g.,
\begin{quote}
\begin{verbatim}
-0.001__0.001  3.141592653__3.141592654  1e308__1.0Inf
\end{verbatim}
\end{quote}
A bounded real is a representation for a real number that definitely lies
somewhere between the two bounds, but the exact value cannot be determined\
\footnote{%
  This is in contrast to a floating point number, which represents
  a real number which lies somewhere in the vicinity of the float}.
Bounded reals are usually not typed in by the user, they are normally
the result of a computation or type coercion.

All computations with bounded reals give safe results, taking rounding
errors into account. This is achieved by doing \Index{interval arithmetic}
on the bounds and rounding the results outwards. The resulting
bounded real is then guaranteed to enclose the true real result.

Computations with floating point values result in uncertainties
about the correct result. Bounded reals make this uncertainty
explicit. A consequence of this is that sometimes it is conceptually
not possible to decide whether two bounded reals are identical or not.
This occurs when the bounds of the compared intervals overlap.
In this case, the arithmetic comparisons leave a (ground) delayed goal
behind which can then be inspected by the user to decide whether the
match is considered close enough. The syntactical comparisons like
\txtbipref{=/2}{(=)/2}{../bips/kernel/termcomp/E-2.html} and
\txtbipref{==/2}{(==)/2}{../bips/kernel/termcomp/EE-2.html} treat bounded reals
simply as a pair of bounds, and consider them equal when the bounds are
equal.


\subsection{Type Conversions}
Note that numbers of different types never unify, e.g., 3, 3_1, 3.0
and 3.0__3.0 are all different.
Use the arithmetic comparison predicates when you want to
compare numeric values.
When numbers of different types occur as arguments of an arithmetic
operation or comparison, the types are first made equal by converting
to the more general of the two types, i.e., the rightmost one in the sequence
\begin{quote}
integer $\rightarrow$ rational $\rightarrow$ float $\rightarrow$ bounded real
\end{quote}
The operation or comparison is then carried out with this type and the
result is of this type as well, unless otherwise specified.
Beware of the potential loss of precision in the
rational $\rightarrow$ float conversion!
Note that the system never does automatic conversions in the opposite direction.
Such conversion must be programmed explicitly using the
\biptxtref{integer}{integer/2}{../bips/kernel/arithmetic/integer-2.html},
\biptxtref{rational}{rational/2}{../bips/kernel/arithmetic/rational-2.html},
\biptxtref{float}{float/2}{../bips/kernel/arithmetic/float-2.html} and
\biptxtref{breal}{breal/2}{../bips/kernel/arithmetic/breal-2.html}
functions.

\section{Arithmetic Functions}
\index{arithmetic!functions}\index{functions!arithmetic}
\subsection{Predefined Arithmetic Functions}
\index{arithmetic!predefined arithmetic functions}
The following predefined arithmetic functions are available.
\about{E}, \about{E1} and \about{E2} stand for
arbitrary arithmetic expressions.

\vspace{2mm}
\noindent
{\small
\begin{tabular}{l l l l}
Function & Description & Argument Types & Result Type \\
\hline
+ E        & unary plus & number & number \\
-- E       & unary minus & number & number \\
abs(E)    & absolute value & number & number \\
sgn(E)    & sign value & number & integer \\
floor(E)   & round down to integral value & number & number \\
ceiling(E) & round up to integral value & number & number \\
round(E)   & round to nearest integral value & number & number \\
truncate(E) & truncate to integral value & number & number \\
E1 + E2    & addition & number $\times$ number & number \\
E1 -- E2   & subtraction & number $\times$ number & number \\
E1 * E2    & multiplication & number $\times$ number & number \\
E1 / E2    & division & number $\times$ number & see below \\
E1 // E2   & integer division (truncate) & integer $\times$ integer & integer \\
E1 rem E2  & integer remainder & integer $\times$ integer & integer \\
E1 div E2  & integer division (floor) & integer $\times$ integer & integer \\
E1 mod E2  & integer modulus & integer $\times$ integer & integer \\
gcd(E1,E2) & greatest common divisor & integer $\times$ integer & integer \\
lcm(E1,E2) & least common multiple & integer $\times$ integer & integer \\
E1 \char`\^\ E2 & power operation & number $\times$ number & number \\
min(E1,E2) & minimum of 2 values & number $\times$ number & number \\
max(E1,E2) & maximum of 2 values & number $\times$ number & number \\
\verb:\: E & bitwise complement & integer & integer \\
E1 \verb:/\: E2  & bitwise conjunction & integer $\times$ integer & integer \\
E1 \verb:\/: E2 & bitwise disjunction & integer $\times$ integer & integer \\
xor(E1,E2) & bitwise exclusive disjunction & integer $\times$ integer & integer \\
E1 $>>$ E2 & shift E1 right by E2 bits & integer $\times$ integer & integer \\
E1 $<<$ E2 & shift E1 left by E2 bits & integer $\times$ integer & integer \\
sin(E)     & trigonometric function & number & real \\
cos(E)     & trigonometric function & number & real \\
tan(E)     & trigonometric function & number & real \\
asin(E)    & trigonometric function & number & real \\
acos(E)    & trigonometric function & number & real \\
atan(E)    & trigonometric function & number & real \\
atan(E1,E1) & trigonometric function & number $\times$ number & real \\
exp(E)     & exponential function \( e^{x} \) & number & real \\
ln(E)      & natural logarithm & number & real \\
sqrt(E)    & square root & number & real \\
pi         & the constant pi = 3.1415926...  & --- & float \\
e          & the constant e = 2.7182818... & --- & float \\
fix(E)     & convert to integer (truncate) & number & integer \\
integer(E) & convert to integer (exact) & number & integer \\
float(E)   & convert to float & number & float \\
breal(E)   & convert to bounded real & number & breal \\
rational(E)   & convert to rational & number & rational \\
rationalize(E) & convert to rational & number & rational \\
numerator(E)   & extract numerator of a  rational & integer or rational &
                 integer \\
denominator(E)   & extract denominator of a rational & integer or rational &
                   integer \\
sum(L)   & sum of list elements & list & number \\
min(L)   & minimum of list elements & list & number \\
max(L)   & maximum of list elements & list & number \\
eval(E)   & evaluate runtime expression & term & number \\
\end{tabular}
}
\vspace{2mm}
\vfill       %<<<<<<<<<<<
\pagebreak   %<<<<<<<<<<<

\noindent
Argument types other than specified yield a type error.
As an argument type, \about{number} stands for \about{integer},
\about{rational}, \about{float} or \about{breal} with the type conversions as
specified above.
As a result type, \about{number} stands for
\emph{the more general of the argument types}, and \about{real} stands for
\about{float} or \about{breal}.
The division operator \notation{/} yields either a rational or a float result,
depending on the value of the global flag
\notationidx{prefer_rationals}.%
\index{global flag!prefer_rationals}\index{arithmetic!prefer_rationals}
The same is true for the result of \char`\^\ if an integer is raised to
a negative integral power.

The integer division \notation{//} rounds the result towards zero (truncates),
while the \notation{div} division rounds towards negative infinity (floor).
Each division function is paired with a corresponding remainder function:
(\notation{rem} computes the remainder corresponding to \notation{//}, and
\notation{mod}
computes the remainder corresponding to \notation{div}\
\footnote{%
  Caution: In {\eclipse} versions up to 5.8, \notation{mod} was the
  remainder corresponding to \notation{//}, i.e., behaved like \notation{rem}}.
The remainder results differ only in the case where the two arguments
have opposite signs.  The relationship between them is as follows:
\begin{quote}
\begin{verbatim}
X =:= (X rem Y) + (X // Y) * Y
X =:= (X mod Y) + (X div Y) * Y
\end{verbatim}
\end{quote}
This table gives an overview:
\begin{quote}
\begin{verbatim}
      10 x 3   -10 x 3   10 x -3   -10 x -3

//       3        -3       -3          3
rem      1        -1        1         -1
div      3        -4       -4          3
mod      1         2       -2         -1
\end{verbatim}
\end{quote}

\subsection{Evaluation Mechanism}
\index{arithmetic!expressions}

An arithmetic expression is a Prolog term that is made up of variables,
numbers, atoms and compound terms, e.g.,
\begin{quote}
\begin{verbatim}
3 * 1.5 + Y / sqrt(pi)
\end{verbatim}\end{quote}
Compound terms are evaluated by first evaluating their arguments and then
calling the corresponding evaluation predicate.
The evaluation predicate associated with a compound term
 \notation{func(a_1,..,a_n)}
is the predicate \predspec{func/}$(n+1)$. It receives
\notation{a_1},..,\notation{a_n} as its first
\about{n} arguments and returns a numeric result as its last argument.
This result is then used in the arithmetic computation.
For instance, the expression above would be evaluated by the goal sequence
\begin{quote}
\begin{verbatim}
*(3,1.5,T1), sqrt(3.14159,T2), /(Y,T2,T3), +(T1,T3,T4)
\end{verbatim}\end{quote}
where \about{T1}, \about{T2} etc. are auxiliary variables created by the system
to hold
intermediate results.

Although this evaluation mechanism is usually transparent to the user, it
becomes visible when errors occur, when subgoals are delayed, or
when inline-expanded code is traced.

\subsection{User Defined Arithmetic Functions}
\index{arithmetic!user defined arithmetic}
This evaluation mechanism outlined above is not restricted to the
predefined arithmetic functions shown in the table.
In fact it works for all atoms and compound terms.
It is therefore possible to define a new arithmetic operation by
just defining an evaluating predicate:
\index{factorial function}
\begin{quote}
\begin{verbatim}
[eclipse 1]: [user].
 :- op(200, yf, !).             % let's have some syntaxtic sugar
 !(N, F) :- fac(N, 1, F).
 fac(0, F0, F) :- !, F=F0.
 fac(N, F0, F) :- N1 is N-1, F1 is F0*N, fac(N1, F1, F).
 user       compiled traceable 504 bytes in 0.00 seconds

yes.
[eclipse 2]: X is 23!.       % calls !/2

X = 25852016738884976640000
yes.
\end{verbatim}
\end{quote}
Note that this mechanism is not only useful for user-defined predicates, but
can also be used to call {\eclipse} built-ins inside arithmetic expressions,
e.g.,
\begin{quote}
\begin{verbatim}
T is cputime - T0.
L is string_length("abcde") - 1.
\end{verbatim}\end{quote}
which call \bipref{cputime/1}{../bips/kernel/opsys/cputime-1.html} and
\bipref{string_length/2}{../bips/kernel/stratom/string_length-2.html}
respectively.
Any predicate that returns a number as its last argument
can be used in a similar manner.

However there is a difference compared to the evaluation of the predefined
arithmetic functions (as listed in the table above):
The arguments of the user-defined arithmetic expression are \emph{not
evaluated} but passed unchanged to the evaluating predicate.
For example, the expression \notation{twice(3+4)} is transformed into the goal
\notation{twice(3+4, Result)} rather than \notation{twice(7, Result)}.
This makes sense because otherwise it would not be possible to pass
any compound term to the predicate.
If evaluation is wanted, the user-defined predicate can explicitly call
\bipref{is/2}{../bips/kernel/arithmetic/is-2.html} or use \predspec{eval/1}.

\subsection{Runtime Expressions}
In order to enable efficient compilation of arithmetic expressions,
{\eclipse} requires that variables in compiled arithmetic expressions
must be bound to numbers at runtime, not symbolic expressions.
For example, in the following code \predspec{p/1} will only work when called
with a
numerical argument, else it will raise error 24:
\begin{quote}
\begin{verbatim}
p(Number) :- Res is 1 + Number, ...
\end{verbatim}
\end{quote}
To make it work even when the argument gets bound to a symbolic expression
at runtime, use \predspec{eval/1} as in the following example:
\begin{quote}
\begin{verbatim}
p(Expr) :- Res is 1 + eval(Expr), ...
\end{verbatim}
\end{quote}
If the expression is the only argument of \predspec{is/2}, the \predspec{eval/1}
may be omitted.


\section{Low Level Arithmetic Builtins}
The low level builtins (like
\txtbipref{+/3}{+/3}{../bips/kernel/arithmetic/P-3.html},
\bipref{sin/2}{../bips/kernel/arithmetic/sin-2.html} etc.)
which are used to evaluate
the predefined arithmetic functions can also be called directly,
but this is not recommended for portability reasons.
\index{compiler!arithmetic}
Moreover, there is no need to use them directly since the {\eclipse} compiler
will transform all arithmetic expressions into calls to the corresponding
low level builtins.

\section{The Multi-Directional Arithmetic Predicates}%
A drawback of arithmetic using
\bipref{is/2}{../bips/kernel/arithmetic/is-2.html} is that the right hand side
must
be fully instantiated at evaluation time.
Often it is desirable to have predicates that define true logic
relationships between their arguments like ``\about{Z} is the sum of \about{X}
and \about{Y}''.
For integer addition and multiplication this is provided as:
\begin{quote}
\begin{description}
\item[succ(X, Y)]\indextt{succ/2}
True if \about{X}and \about{Y} are natural numbers, and \about{Y} is one greater
than \about{X}.
At most one of \about{X}, \about{Y} can be a variable.

\item[plus(X, Y, Z)]\indextt{plus/3}
True if the sum of \about{X} and \about{Y} is \about{Z}. At most one of
\about{X}, \about{Y}, \about{Z} can be a variable.

\item[times(X, Y, Z)]\indextt{times/3}
True if the product of \about{X} and \about{Y} is \about{Z}. At most one of
\about{X}, \about{Y}, \about{Z} can be a variable.

\end{description}
\end{quote}

These predicates work only with integer arguments but any single argument
can be a variable which is then instantiated so that the relation holds.
If more than one argument is uninstantiated, an instantiation fault is produced.

Note that if one of the first two arguments is a variable,
a solution doesn't necessarily exist. For example, the following goal has
no integer solution :
\begin{quote}
\begin{verbatim}
[eclipse 1]: times(2, X, 3).

no (more) solution.
\end{verbatim}\end{quote}

Since any one of the arguments of these two predicates can be a variable,
it does not make much sense to use them in arithmetic expressions
where always the first arguments are taken as input and the last
one as output.

\section{Arithmetic and Coroutining}
\index{arithmetic!coroutining}
\index{coroutining!arithmetic}
\index{delay!arithmetic}
\label{arithdelay}

Arithmetic comparisons can be delayed until their arguments are
instantiated instead of generating an instantiation fault by passing the
comparison to the suspend solver (see section~\ref{suspendsolver}). This
gives a form of coroutining.

%HEVEA\cutend

