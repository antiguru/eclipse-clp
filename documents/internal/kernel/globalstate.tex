% BEGIN LICENSE BLOCK
% Version: CMPL 1.1
%
% The contents of this file are subject to the Cisco-style Mozilla Public
% License Version 1.1 (the "License"); you may not use this file except
% in compliance with the License.  You may obtain a copy of the License
% at www.eclipse-clp.org/license.
% 
% Software distributed under the License is distributed on an "AS IS"
% basis, WITHOUT WARRANTY OF ANY KIND, either express or implied.  See
% the License for the specific language governing rights and limitations
% under the License. 
% 
% The Original Code is  The ECLiPSe Constraint Logic Programming System. 
% The Initial Developer of the Original Code is  Cisco Systems, Inc. 
% Portions created by the Initial Developer are
% Copyright (C) 2006 Cisco Systems, Inc.  All Rights Reserved.
% 
% Contributor(s): 
% 
% END LICENSE BLOCK
\chapter{Global State}

%----------------------------------------------------------------------
\section{Memory Management}
Memory management in \eclipse\ is designed for a parallel
shared memory system (where several processes work on a piece of
shared memory), although this functionality is not
supported by IC-Parc.
The memory areas are:
\begin{description}
\item[shared heap]
    Most of the non-logical data is here:
    \begin{itemize}
    \item dictionary and everything related, e.g.
    \item abstract machine code
    \item heap terms
    \end{itemize}
\item[private heap]
    In a non-parallel \eclipse\ this is the same as the shared heap\footnote{
    In parallel \eclipse\ the private heap is private to each worker, while
    the shared heap is shared between workers}.
    \begin{itemize}
    \item temporary storage
    \item data that must be local to a process
    \item memory allocated by third party code, e.g. Cplex.
    \end{itemize}
\item[stack]
    The abstract machine stacks. They are allocated in pairs
    that grow towards each other from both ends of the allocated
    area:
    \begin{itemize}
    \item global-trail stack
    \item control-local stack
    \end{itemize}
\end{description}

\subsection{Heap Management}

The heap management (for shared memory, but the code is also used
in the non-parallel case) is in the subsystem {\bf Shm}.
To use the library\footnote{
In parallel \eclipse, this library was used to share memory between
workers and the worker manager process.},
include the following header files:
\begin{verbatim}
#include "memman.h"
\end{verbatim}
and link your program with libshm.a

The type generic_ptr is normally (void *). On compilers that don't
support it (char *) is used instead.
For the following, it is useful to know the ideas behind the UNIX
mmap() system call, on which the API is based.

\subsubsection{Initialisation}

\begin{verbatim}
char *shared_mem_init(
        int create_flag,        specifies if we create (!=0) or if we
                                    attach to (==0) existing shared memory.
        char* mapfile,          the filename of the mapfile or NULL for
                                    anonymous mapping into the swap space.
        char* start,            the address where the shared memory is
                                    mapped at. If NULL, the system chooses an
                                    address (during create) or uses the same
                                    address as the creator (during attach).
        int size,               maximum size of shared memory (0=default size),
                                    only used when create_flag is set.
        int increment,          minimum amount of bytes the shared memory is
                                    extended by on overflows (0=default).
        void (*panic_fct)()     function to call when out of memory. If NULL,
                                    the allocation functions return NULL
                                    instead.
        struct heap_descriptor *
                                this descriptor must exist and is filled
                                    by the function. It is used in further
                                    memory operations.
    );
\end{verbatim}

    The return value is equal to the specified start address if everything
    goes well, -1 otherwise.

    The first word of the shared memory is reserved for the application:
    The idea is to store here a pointer to some kind of toplevel data
    structure of the application.

    The panic function takes two descriptive arguments and may look
    as follows:

\begin{verbatim}
    void
    panic(what, where)
    char *what, *where;
    {
            printf("Panic: %s in %s\n", what, where);
            exit(-1);
    }
\end{verbatim}


\begin{verbatim}
void shared_mem_release(struct heap_descriptor *);
\end{verbatim}

    Should be called before a process using the shared memory exits.
    When the releasing process is the last attached one, the mapfile
    is removed.


\subsubsection{Header-less allocation}

These functions should be used when the caller can remember the size of the
allocated area, or when the size is implicit in the allocated data.
This allocation has no memory overhead, so e.g. allocating 8192 bytes
will use up exactly 8k of memory and no more.

\begin{verbatim}
generic_ptr     alloc_size(struct heap_descriptor *, int size);
\end{verbatim}

    Allocate size bytes of shared memory and return its address.
    If the allocation fails, the panic-function is called, if any,
    otherwise NULL is returned.

\begin{verbatim}
void            free_size(struct heap_descriptor *, generic_ptr addr, int size);
\end{verbatim}

    Free the memory at addr which was allocated previously with
    alloc_size(size). Size should be the same as on allocation,
    otherwise the result is undefined.

\begin{verbatim}
generic_ptr     realloc_size(struct heap_descriptor *,
                        generic_ptr addr, int oldsize, int newsize);
\end{verbatim}

    Resize an allocated area. The new address may differ from the old one,
    the old contents are copied.


\subsubsection{Header-full allocation}
This is a standard malloc()-style interface, where the allocated memory
block remembers its own size, so it does not have to be passed as an
argument when the block gets freed. This implies some bookkeeping overhead.
Implementation is on top of the header-less primitives above.

\begin{verbatim}
generic_ptr     h_alloc(struct heap_descriptor *, int size);
\end{verbatim}

    Allocate size bytes of shared memory and return its address.
    If the allocation fails, the panic-function is called, if any,
    otherwise NULL is returned.

\begin{verbatim}
void            h_free(struct heap_descriptor *, generic_ptr addr);
\end{verbatim}

    Free the memory at addr which was allocated previously with h_alloc().

\begin{verbatim}
generic_ptr     h_realloc(struct heap_descriptor *,
                        generic_ptr addr, int newsize);
\end{verbatim}

    Resize an allocated area. The new address may differ from the old one,
    the old contents are copied.


\subsubsection{Other}

\begin{verbatim}
int             alloc_statistics(struct heap_descriptor *, int what);
\end{verbatim}

    Return statistics about bytes of allocated memory.
    Values for what: HEAP_STAT_ALLOCATED and HEAP_STAT_USED.


\subsubsection{Spin locks}

Locks are of type a_mutex_t (declared in memman.h).
Functions are:

\begin{verbatim}
int             a_mutex_init(a_mutex_t *);
int             a_mutex_lock(a_mutex_t *);
int             a_mutex_unlock(a_mutex_t *);
int             a_mutex_destroy(a_mutex_t *);
\end{verbatim}


\subsubsection{Interrupts}

Motivation: Interrupts must be disabled before grabbing a lock,
otherwise the following can happen:
\begin{enumerate}
   \item grab a lock
   \item interrupt
   \item handler tries to get the lock again
   \item -> deadlock
\end{enumerate}

User interrupt handlers that might need locks should therefore
not be executed while a lock is being held.
The InterruptsDisabled condition flags this situation,
so these handlers should have the following structure:

\begin{verbatim}
signal_handler(n)
{
    if (InterruptsDisabled)
    {
        <remember n>
        Set_Interrupts_Pending();
        return;
    }
    <process the signal>
}

delayed_signal_handler()
{
    <get n back>
    Clr_Interrupts_Pending();
    <process the signal>
}
\end{verbatim}


The delayed signal handler is specified using the function

\begin{verbatim}
void irq_lock_init(void (*delayed_irq_fct)());
\end{verbatim}


A couple of macros is provided:

\begin{verbatim}
#define InterruptsDisabled      it_disabled_
#define Disable_Int()           it_disabled_++;
#define Enable_Int()    \
        { if (--it_disabled_ == 0 && delayed_it_) (*delayed_irq_func)(); }

#define InterruptsPending       delayed_it_
#define Set_Interrupts_Pending() delayed_it_ = 1;
#define Clr_Interrupts_Pending() delayed_it_ = 0;
\end{verbatim}

a_mutex_lock() internally calls Disable_Int().
a_mutex_unlock() internally calls Enable_Int().
Disable_Int() and Enable_Int() can also be called directly
delay interrupts.


\subsubsection{Private memory allocation}

Initialisation:

\begin{verbatim}
private_mem_init(void (*)() panic_fct);
\end{verbatim}

To allocate and free private memory, use the generic allocation functions
with the descriptor argument \&private_heap.

\subsection{Stack Management}

%----------------------------------------------------------------------
\section{Dictionary}
%----------------------------------------------------------------------
\section{Module System}
%----------------------------------------------------------------------
\section{Heap Terms}
%----------------------------------------------------------------------
