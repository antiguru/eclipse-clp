% BEGIN LICENSE BLOCK
% Version: CMPL 1.1
%
% The contents of this file are subject to the Cisco-style Mozilla Public
% License Version 1.1 (the "License"); you may not use this file except
% in compliance with the License.  You may obtain a copy of the License
% at www.eclipse-clp.org/license.
% 
% Software distributed under the License is distributed on an "AS IS"
% basis, WITHOUT WARRANTY OF ANY KIND, either express or implied.  See
% the License for the specific language governing rights and limitations
% under the License. 
% 
% The Original Code is  The ECLiPSe Constraint Logic Programming System. 
% The Initial Developer of the Original Code is  Cisco Systems, Inc. 
% Portions created by the Initial Developer are
% Copyright (C) 2006 Cisco Systems, Inc.  All Rights Reserved.
% 
% Contributor(s): 
% 
% END LICENSE BLOCK

%HEVEA\cutdef[1]{section}

\index{mathematical programming, interface to|(}
\index{mixed integer programming, interface to|(}
\index{linear programming, interface to|(}
\index{quadratic programming, interface to|(}
\index{simplex solver, interface to|(}
\section{Usage}
This library allows the use of an external mathematical programming 
(LP, MIP or quadratic) solver 
from within {\eclipse}. It provides a largely solver-independent API
to the programmer, so many programs will run with any supported external
solver.

%With the kind agreement of Dash Associates Ltd., the
%XPRESS-MP\footnote{XPRESS-MP is a product from Dash Associates
%Ltd. (www.dashoptimization.com)} solver is now included with the library,
%and is available for academic use under the {\eclipse} licence agreement.

See section~\ref{specificsolver} for more details on the supported solvers.

The most generic way to load the library is:
\begin{verbatim}
:- lib(eplex).
\end{verbatim}
\index{eplex}
This will try to load an appropriate external solver available on
the computer.

It is also possible to request a specific solver explicitly,
see section~\ref{specificsolver} for details.

Note that the {\eclipse} library described here is just an interface
to an external solver. In order to be able to use it, you need to have
access to a solver supported by the library. For commercial solvers, this
requires a licence for the solver on your machine. For more details,
see section \ref{specificsolver}. For open source solvers, the required
solver library may be distributed with {\eclipse} if its licence allows
this.


%\section{History}
%The latest major addition to this library is support for adding
%constraints to the solver, as far as this is supported by the
%external solver.
%
%The library has been ported to work with XPRESS-MP as well as CPLEX.
%
%The third version of this library uses standard range-variables
%provided by the \htmlref{range-library}{chaprange}, which should
%facilitate interfacing to other solvers.
%The interface to CPLEX has been extended such that more
%state information can be retrieved, e.g.\ constraint slacks,
%basis information, reduced costs etc.
%A quite generic solver demon is provided which makes it easy to
%use CPLEX within a data-driven CLP setting.
%
%The second major version made some things conceptually clearer.
%The notion of solver handles is new. It removes the ugly
%concept of a global state and hopefully encourages
%experiments with multiple solvers.
%The interface to the solver's state has been extended,
%allowing to modify and query many parameters and extract
%solution information.
%A pair of predicates has been added to allow reading and writing
%problem files in MPS or LP format.
%


%----------------------------------------------------------------------
\section{Eplex Instances}
%----------------------------------------------------------------------

In this chapter, the problem passed to the external solver will be referred
to as an {\it eplex problem}. An eplex problem consists of a set of linear
arithmetic constraints, whose variables have bounds and may possibly have
integrality constraints. The external solver will solve such a problem by optimising
these constraints with respect to an objective function. 

With the eplex library, it is possible to have more than one eplex problem
within one program. The simplest way to write such programs with the
library is through {\it Eplex Instances}. An eplex instance is an instance
of the eplex solver, to which an eplex problem can be sent. An external
{\it solver state} can be associated with each eplex instance, which can be
invoked to solve the eplex problem. Declaratively, an eplex instance can be
seen as a compound constraint consisting of all the variables, their bounds, and
constraints of the eplex problem.

Like other solvers, each eplex instance has its own module. To use an eplex
instance, it must first be declared, so that the module can be created.
This is done by:

\subsubsection{\biptxtref{eplex_instance(+Name)}{eplex_instance/1!epex}{../bips/lib/eplex/eplex_instance-1.html}}

This predicate will initialise an eplex instance {\tt Name}. Once
initialised, a {\tt Name} module will exist, to which the user can post the 
constraints for the eplex problem and setup and use the external solver
state to solve the eplex problem. Normally, this predicate should be issued
as a directive in the user's program, so that the program code can refer to
the instance directly in their code. For example:

\begin{verbatim}
   :- eplex_instance(instance).
\end{verbatim}


For convenience, the eplex library declares {\tt eplex} as an eplex instance 
when the library is loaded. 

\subsection{Linear Constraints}
\label{linear-constraints}
The constraints provided are equalities and inequalities over
linear expressions.
Their operational behaviour is as follows:
\begin{itemize}
\item When they contain no variables, they simply succeed or fail.
\item When they contain exactly one variable, they are translated into
a bound update on that variable, which may in turn fail, succeed,
or even instantiate the variable.
Note that if the variable's type is integer, the bound will be adjusted
to the next suitable integral value.
\item Otherwise, the constraint is transferred to the external solver state
if the state has been setup. If it has not, the constraint
delays and is transferred to the external solver state when it is setup.
This mechanism makes it possible to interface to a non-incremental
black-box solver that requires all constraints at once,
or to send constraints to the solver in batches
\end{itemize}


As with all predicates defined for an eplex instance, these constraints
should be module-qualified with the name of the eplex instance. In the following
they are shown qualified with the {\tt eplex} instance. Other instances can
be used if they have been declared using {\bf eplex_instance/1}.
 
\subsubsection{\biptxtref{EplexInstance: (X \$= Y)}{\$=/2!eplex}{../bips/lib/eplex/SE-2.html}}
X is equal to Y. X and Y are linear expressions.

\subsubsection{\biptxtref{EplexInstance: (X \$>= Y)}{\$$>$=/2!eplex}{../bips/lib/eplex/SGE-2.html}}
X is greater or equal to Y. X and Y are linear expressions.

\subsubsection{\biptxtref{EplexInstance: (X  \$=< Y)}{\$=$<$/2!eplex}{../bips/lib/eplex/SEL-2.html}}
X is less or equal to Y. X and Y are linear expressions.


\subsection{Linear Expressions}

The following arithmetic expression can be used inside the constraints:
\begin{description}
\item[{\bf X}]
Variables. If X is not yet a problem variable, it is turned into one
	via an implicit declaration {\tt X\ \$::\ -1.0Inf..1.0Inf}.

\item[{\bf 123, 3.4}]
Integer or floating point constants.

\item[{\bf +}Expr]
Identity.

\item[{\bf -}Expr]
Sign change.

\item[E1{\bf +}E2]
Addition.

\item[{\bf sum}(ListOfExpr)]
Equivalent to the sum of all list elements.

\item[E1{\bf -}E2]
Subtraction.

\item[E1{\bf *}E2]
Multiplication.

\item[ListOfExpr1{\bf *}ListOfExpr2]
Scalar product: The sum of the products of the corresponding
elements in the two lists.  The lists must be of equal length.
\end{description}

\subsection{Bounds}
Bounds for variables can be given to an eplex instance via the \verb'$::/2'
constraint:
\begin{description}
    \item[\biptxtref{EplexIntance: Vars \$:: Lo..Hi}{\$::/2!eplex}{../bips/lib/eplex/SNN-2.html}]
    \label{eplex-coloncolon2}
    Restrict the external solver to assign solution values for the eplex
    problem within the bounds specified by Lo..Hi.  
    Passes to the external solver the bounds for the variables in Vars.
    Lo, Hi are the lower and upper bounds, respectively. Note that the
    bounds are only passed to the external solver if they would narrow the
    current bounds, and failure will occur if the resulting interval is empty. 
    Note also that the external solver does not do any bound propagation
    and will thus not change the bounds on its own. The default bounds for
    variables are notionally -1.0Inf..1.0Inf (where infinity is actually
    defined as the solver's notion of infinity).
\end{description}

\subsection{Integrality}
The difference between using an LP vs.\ an MIP solver is made by
declaring integrality to the solver via the integers/1 constraint:
\begin{description}
    \item[\biptxtref{EplexInstance:integers(Vars)}{integers/1!eplex}{../bips/lib/eplex/integers-1.html}]
    \label{eplex-integers1}
	Inform the external solver to treat the variables Vars as integral.
	It does not impose the integer type on Vars. However, when a 
        typed_solution is retrieved (via lp_get/3 or
        lp_var_get/3), this will be rounded to the nearest integer.
        
        Note that unless eplex:integers/1 (or lp_add/3, see
        section~\ref{lp-add}) is invoked, any invocation
        of the eplex external solver (via lp_solve/2, lp_probe/3 or
	lp_demon_setup/5) will only solve a continuous relaxation, even
        when problem variables have been declared as integers in other
        solvers (e.g.\ ic).
        
\end{description}

Note that all the above constraints are local to the eplex instance; they
do not place any restrictions on the variables for other eplex instances or
solvers. Failure will occur only when inconsistency is detected within the
same eplex instance, unless the user explicitly try to merge the constraints
from different solvers/eplex instance.

A counterpart, \bipref{reals/1}{../bips/lib/eplex/reals-1.html} `constraint' also exists -- this simply declares the
variables specified are problem variables, and does not actually place any
other constraint on the variables.

\subsection{Solving Simple Eplex Problems}
\label{solving-eplex}
In order to solve an eplex problem, the eplex instance must be set up
for an external solver state. The solver state can then be invoked to
solve the problem. The simplest way to do this is to use:

\begin{description}
    \item[\biptxtref{EplexInstance:eplex_solver_setup(+Objective)}{eplex_solver_setup/1}{../bips/lib/eplex/eplex_solver_setup-1.html}]
    This predicate creates a new external solver state and associates it
    with the eplex instance. Any arithmetic, integrality and bound
    constraints posted for this eplex instance are collected to create the
    external solver state. After this, the solver state can be invoked to 
    solve the eplex problem.

    Objective is either min(Expr) or max(Expr) where Expr is a linear
    expression (or quadratic, if supported by the external solver).  

    \item[\biptxtref{EplexInstance:eplex_solve(-Cost)}{eplex_solve/1}{../bips/lib/eplex/eplex_solve-1.html}]
    Explicitly invokes the external solver state. Any new constraints
    posted are taken into account. If the external solver can find an
    optimal solution to the eplex problem, then the predicate succeeds and Cost is
    instantiated to the optimal value. If the problem is infeasible (has no
    solution) or unbounded (Cost is not bounded by the constraints), then 
    the predicate fails. 

\end{description}
\subsection{Examples}

Here is a simple linear program, handled by the predefined eplex instance 'eplex':
\begin{verbatim}
:- lib(eplex).

lp_example(Cost) :-
     eplex: eplex_solver_setup(min(X)),
     eplex: (X+Y $>= 3),
     eplex: (X-Y $= 0),
     eplex: eplex_solve(Cost).
\end{verbatim}

The same example using a user-defined eplex instance:
\begin{verbatim}
:- lib(eplex).
:- eplex_instance(my_instance).

lp_example(Cost) :-
     my_instance: eplex_solver_setup(min(X)),
     my_instance: (X+Y $>= 3),
     my_instance: (X-Y $= 0),
     my_instance: eplex_solve(Cost).
\end{verbatim}

Running the program gives the optimal value for Cost:

\begin{verbatim}
[eclipse 2]: lp_example(Cost).

Cost = 1.5
\end{verbatim}

Note that if the {\tt eplex} eplex instance is used instead of {\tt
my_instance}, then the {\tt eplex_instance/1} declaration is not
necessary.


By declaring one variable as integer, we obtain a Mixed Integer Problem:

\begin{verbatim}
:- lib(eplex).
:- eplex_instance(my_instance).

mip_example(Cost) :-
     my_instance: (X+Y $>= 3),
     my_instance: (X-Y $= 0),
     my_instance: integers([X]),
     my_instance: eplex_solver_setup(min(X)),
     my_instance: eplex_solve(Cost).

....
[eclipse 2]: mip_example(Cost).

Cost = 2.0
\end{verbatim}

The cost is now higher because X is constrained to be an integer. Note also
that in this example, we posted the constraints before setting up the 
external solver, whereas in the previous example we set up the solver
first. The solver set up and constraint posting can be done in
any order. If {\tt integers/1} constraints are only posted after problem
setup, the problem will be automatically converted from an LP to a MIP
problem. 

This section has introduced the most basic ways to use the eplex library. 
We will discuss more advanced methods of using the eplex instances in
section~\ref{eplex-instance-advanced}. 


%----------------------------------------------------------------------
\section{Advanced Use of Eplex Instances}
%----------------------------------------------------------------------
\label{eplex-instance-advanced}

\subsection{Obtaining Solver State Information}
\label{eplex-instance-solver-info}
The black-box interface binds both the objective value (Cost) and the
problem variables by bindings these variables. On the other hand, {\bf
  eplex_solve/1} binds the objective value, but does not bind the problem
variables. These values can be obtained by:

\begin{sloppypar}
\begin{description}
\item[\biptxtref{EplexInstance:eplex_var_get(+Var, +What, -Value)}{eplex_var_get/3}{../bips/lib/eplex/eplex_var_get-3.html}]

Retrieve information about the solver state associated with the eplex
instance for the variable {\tt Var}. If {\tt What} is {\tt solution} or
{\tt typed_solution}, then the value assigned to this variable by the
solver state to obtain the optimal solution is returned in {\tt
Value}. {\tt solution} returns the value as a float, and {\tt
typed_solution} returns the value as either a float or a rounded integer, depending
on if the variable was constrained to an integer in the eplex problem.

\item[\biptxtref{EplexInstance:eplex_get(+What, -Value)}{eplex_get/2}{../bips/lib/eplex/eplex_get-2.html}]

Retrieve information about solver state associated with the eplex instance.
This returns information such as the problem type, the constraints for the
eplex problem. See the reference manual for more details.

\item[\biptxtref{EplexInstance:eplex_set(+What, +Value)}{eplex_set/2}{../bips/lib/eplex/eplex_set-2.html}]
Set a solver option for the eplex instance. 
\item[\biptxtref{EplexInstance:eplex_write(+Format, +File)}{eplex_write/2}{../bips/lib/eplex/eplex_write-2.html}]
Write out the problem in the the eplex instance's solver state to the file
File in format Format. The writing is done by the external solver. Use the
use_var_name(yes) option in
\bipref{eplex_solver_setup/4}{../bips/lib/eplex/eplex_solver_setup-4.html}
so that the written file uses \eclipse variable names. Also the {\tt
write_before_solve} option of eplex_solver_setup/4 can be used to write out
a problem just before it is solved by the external solver: this allows
problem to be written in places where eplex_write/2 cannot be added
(e.g. for probing problems)..

\item[\biptxtref{EplexInstance:eplex_read(+Format, +File)}{eplex_read/2}{../bips/lib/eplex/eplex_read-2.html}]
Read a MP problem in the file File in format Format into a solver state,
and associate the solver with the eplex instance. No solver must already be
setup for the eplex instance. The solver state that is setup can only be
triggered explicitly.

\end{description}
\end{sloppypar}

So for the simple MIP example:
\begin{verbatim}

:- lib(eplex).
:- eplex_instance(my_instance).

mip_example2([X,Y], Cost) :-
     my_instance: (X+Y $>= 3),
     my_instance: (X-Y $= 0),
     my_instance: integers([X]),
     my_instance: eplex_solver_setup(min(X)),
     my_instance: eplex_solve(Cost),
     my_instance: eplex_var_get(X, typed_solution, X),
     my_instance: eplex_var_get(Y, typed_solution, Y).

....
[eclipse 2]: mip_example2([X,Y],C).

X = 2
Y = 2.0
C = 2.0

\end{verbatim}

In the example, only X is returned as an integer, as Y was not explicitly
constrained to be an integer.

Note that if there are multiple eplex instances, and a variable is shared
between the instances, then the solver state for each instance can have a
different optimal value to the variable.

\subsection{Creating Eplex Instances Dynamically}
\index{eplex!instance!eplex_instance/1}
So far, we have shown the use of {\tt eplex_instance/1} as a directive to
declare an eplex instance. For some applications, it might be necessary to
create eplex instances dynamically at run-time. The can be done by calling
{\tt eplex_instance/1} at run-time.
In this case, the instance name should {\em not} be used to module-qualify
any predicates in the code, since this will raise a compiler warning
complaining about an unknown module.

\begin{verbatim}
   new_pool(X,Y) :-  % INCORRECT
      eplex_instance(pool),
      pool: (X $>= Y), % will generate a warning
      ...
\end{verbatim}
\noindent
Of course, in the above code, the instance name {\tt pool} is already known
at compile time, so it can always be declared by a directive.

If the name is truly generated dynamically, this can be done as follows:
\begin{verbatim}
   new_pool(Pool,X,Y) :-
       eplex_instance(Pool),
       Pool: (X $>= Y),
       ....
\end{verbatim}

\subsubsection{Obtaining Bounds on the Objective Value}

The external solver does not always return the optimal objective
value, for example when the optimisation was aborted. However, even when
the solver returns an optimal solution, it may actually not be the exact
optimal, because of solver settings (e.g. for MIP problems, the MIP search
will terminate when the solution found is within certain tolerance of the
best possible solution value). In these cases, it may be useful to obtain
some bounds on the optimal objective value. The best and worst bounds on
the optimal objective can be obtained using the best_bound and worst_bound
options of \bipref{eplex_get/2}{../bips/lib/eplex/eplex_get-2.html},
respectively. 

\subsection{Interface for CLP-Integration: Solver Demons}

To implement hybrid algorithms where a run of a simplex/MIP solver is only
a part of the global solving process, the black-box model presented above
is not appropriate anymore. With eplex instances, we can call {\tt
eplex_solve/1} repeatedly to re-solve the problem, perhaps after adding
more constraints to the problem or after changes in the variable
bounds. However, the solver must be invoked explicitly. We require more
sophisticated methods of invoking the solver. This can be done by setting
up a solver demon, and specifying the conditions in which the demon is to
wake up and invoke the external solver.

\subsubsection{\biptxtref{EplexInstance:eplex_solver_setup(+Objective, -Cost, +ListOfOptions, +TriggerModes)}{eplex_solver_setup/4}{../bips/lib/eplex/eplex_solver_setup-4.html}}
This is a more sophisticated set up for a new solver state than
{\tt eplex_solver_setup/1} (in fact eplex_solver_setup/1 is a special case
of eplex_solver_setup/4).
The main idea is that a list of trigger conditions
are specified in {\tt TriggerModes}, and along with setting up the solver
state, a demon goal is created which is woken up when one of the
specified trigger condition is met. This demon goal will then invoke the
solver, with any 
constraints posted to the eplex instance since the solver was last invoked
taken into account, to re-solve the problem. 

The {\tt ListOfOptions} is a list of solver options for setting up the
solver state. Some of these affect the way the external solver solves the
problem, such as if presolve should be applied before solving the problem.
See the reference manual for \bipref{eplex_solver_setup/4}{../bips/lib/eplex/eplex_solver_setup-4.html} for
details on the available options and trigger modes.

As the solver is designed to be invoked repeatedly, it is inappropriate to
directly bind {\tt Cost} to the objective value. Instead, the objective
value is exported as a bound to Cost:
For a minimisation problem, each solution's
cost becomes a lower bound, for maximisation an upper bound on Cost.
This technique allows for repeated re-solving with reduced variable bounds
or added constraints. Note that the bound update is done only if the
solution is optimal. Note also that Cost is not automatically 
made a problem variable, and thus may not have bounds associated
with in. In order for the bounds information not to be lost, some bounds
should be given to Cost (e.g. making it a problem variable (but
this might introduce unnecessarily self-waking on bounds change), or via
another solver with bounds (e.g. ic)). 


\index{eplex!presolve}
Note that when a solver demon runs frequently on relatively small problems,
it can be important for efficiency to switch the external solver's
presolving off for this demon as part of the {\tt ListOfOptions} during the
setup of the problem to reduce overheads.

\subsubsection{Example}

The simplest case of having a simplex solver automatically cooperating
with a CLP program, is to set up a solver demon which will repeatedly
check whether the continuous relaxation of a set of constraints
is still feasible.
The code could look as follows (we use the eplex instance in this example):
\begin{verbatim}
simplex :-
    eplex:eplex_solver_setup(min(0), C, 
        [solution(no)], [bounds]).
\end{verbatim}
First, the constraints are normalised and checked for linearity.
Then a solver with a dummy objective function is set up. The option
{\tt solution(no)} indicates that we are not interested in solution values.
Then we start a solver demon which will re-examine the problem
whenever a change of variable bounds occurs.
The demon can be regarded as a compound constraint implementing the
conjunction of the individual constraints. It is able to detect
some infeasibilities that for instance could not be detected by a
finite domains solver, e.g.
\begin{verbatim}
[eclipse 2]: eplex:(X+Y+Z >= K), eplex:(X+Y+Z =< 1),
    eplex:eplex_solver_setup(min(0), C, 
        [solution(no)], [bounds]),
    K = 2.

No (0.00s cpu)
\end{verbatim}
In the example, the initial simplex is successful, but instantiating
K wakes the demon again, and the simplex fails this time.

A further step is to take advantage of the cost bound that the simplex
procedure provides. To do this, we need to give the objective 
The setup is similar to above, but we accept an objective function and
add a cost variable. The bounds of the cost variable will be updated
whenever a simplex invocation finds a better cost bound on the problem.
In the example below, an upper bound for the cost of 1.5 is found
initially:
\begin{verbatim}
[eclipse 5]: ic: (Cost $:: -1.0Inf..1.0Inf), 
      eplex:(X+Y $=< 1), eplex:(Y+Z $=< 1), eplex:(X+Z $=< 1),
      eplex:eplex_solver_setup(max(X+Y+Z), Cost, 
          [solution(no)], [bounds]).

X = X{-1e+20 .. 1e+20}
Y = Y{-1e+20 .. 1e+20}
Z = Z{-1e+20 .. 1e+20}
Cost = Cost{-1.0Inf .. 1.500001}


Delayed goals:
        lp_demon(prob(...), ...)
Yes (0.00s cpu)
\end{verbatim}
(Note that the ranges for X, Y and Z is -1e+20 .. 1e+20 as 1e+20 is this
external solver's notion of infinity). 

If the variable bounds change subsequently, the solver will be re-triggered,
improving the cost bound to 1.3:
\begin{verbatim}
[eclipse 6]: ic: (Cost $:: -1.0Inf..1.0Inf), 
      eplex:(X+Y $=< 1), eplex:(Y+Z $=< 1), eplex:(X+Z $=< 1),
      eplex:eplex_solver_setup(max(X+Y+Z), Cost, 
          [solution(no)], [bounds]), 
      eplex:(Y =< 0.3).

X = X{-1e+20 .. 1e+20}
Z = Z{-1e+20 .. 1e+20}
Cost = Cost{-1.0Inf .. 1.300001}
Y = Y{-1e+20 .. 0.3}


Delayed goals:
        lp_demon(prob(...), ...)
Yes (0.00s cpu)
\end{verbatim}

A further example is the implementation of a MIP-style branch-and-bound
procedure. Source code is provided in the library file mip.pl.

\subsection{Encapsulated Modification of the Problem: Probing}

The external mathematical programming solvers often provides the facility
for the user to change the problem being solved. This includes the addition
or removal of constraints, and the changing of the objective function.
We have already seen how extra constraints can be added. As {\eclipse} is a
logic programming language, removal of constraints is automatically
achieved by backtracking. We do not allow the user to explicitly remove
constraints that have been collected by the external solver, as this makes 
the problem non-monotonic.
For the same reason, we do not allow the objective function to be
changed.\footnote{However, some monotonic changes are allowed in the
low-level interface, for implementing column generation, see
section~\ref{coladd}.}
However, we do allow the problem (including the objective function) to be
{\it temporarily\/} changed in certain specified ways. This allows the
problem to be `probed' with these changes:

\subsubsection{\biptxtref{EplexInstance:eplex_probe(+Probes, -Cost)}{eplex_probe/2}{../bips/lib/eplex/eplex_probe-2.html}}
Similar to eplex_solve/1, but the problem is first temporarily modified as
specified in Probes before the optimisation. The Cost value
is instantiated to the objective value for this new modified problem, and
any solution state requested are also updated. 

\subsection{Destroying the Solver State}
\subsubsection{\biptxtref{EplexInstance:eplex_cleanup}{eplex_cleanup/0}{../bips/lib/eplex/eplex_cleanup-0.html}}

Destroy the specified solver, free all memory, etc.
Note that {\eclipse} will normally do the cleanup automatically,
for instance when execution fails across the solver setup, or
when a solver handle gets garbage collected. The solver is disassociated
with the eplex instance, and any outstanding constraints not yet collected
by the  solver are removed, with a warning to the user. In effect, the
eplex instance is reinitialised.

Note that this is a non-logical operation. Backtracking into code before 
{\tt eplex_cleanup/0} will not restore the solver state, and any attempt to
reuse the solver state will not be possible (the execution will abort with
an error). Normally, it is recommended to let {\eclipse} perform the cleanup 
automatically,
for instance when execution fails across the solver setup, or
when an unused solver state handle gets garbage collected.
However, calling eplex_cleanup/0 may
cause resources (memory and licence) to be freed earlier.

\subsection{Eplex Instance Interface Example: definition of optimize/2:}
\label{defoptimize}
A black-box setup-and-solve predicate {\bf optimize/2} can be defined as:
\begin{verbatim}
optimize(OptExpr, ObjVal) :-
        eplex:eplex_solver_setup(OptExpr),
        eplex:eplex_solve(ObjVal),
        eplex:eplex_get(vars, VArr),
        eplex:eplex_get(typed_solution, SolutionVector),
        VArr = SolutionVector,                  % do the bindings
        eplex:eplex_cleanup.
\end{verbatim}
A solver state is set up for the eplex instance {\tt eplex}, to allow
constraints that were previously posted to {\tt eplex} to be collected.
This happens once the solver is invoked by {\tt eplex_solve/1}. If there
is a solution, the solution vector is obtained, and the
variables are instantiated to those solutions.


%----------------------------------------------------------------------
\section{Low-Level Solver Interface}
%----------------------------------------------------------------------

For many applications, the facilities presented so far should be
appropriate for using Simplex/MIP through {\eclipse}.
However, sometimes it may be more convenient or efficient
to directly access the solver state
instead of going through the abstraction of the eplex instances. 
This section describes lower level operations like how to set up
solvers manually. In fact, these lower level predicates are used to
implement the predicates provided with eplex instances.

These predicates accesses the external solver state via a handle, which is
returned when the solver state is set up, and subsequently used to access a
particular solver state by the other predicates. The handle should be
treated as a opaque data structure that is used by the eplex library to
refer to a particular solver state.

\subsection{Setting Up a Solver State}

\subsubsection{\biptxtref{lp_demon_setup(+Objective, -Cost, +ListOfOptions, 
+TriggerModes, -Handle)}{lp_demon_setup/5}{../bips/lib/eplex/lp_demon_setup-5.html}}

This is used to set up a demon solver, and {\tt eplex_solver_setup/4} calls
this predicate. There is one extra argument compared to {\tt
  eplex_solver_setup/4}: the solver state handle {\tt Handle}, which is
returned by this predicate when the new solver state is created.
The other arguments are the same as in {\tt eplex_solver_setup/4}, except
that there is an additional option in {\tt ListOfOptions}: {\tt
  collect_from/1}. This is used to specify which, if any, eplex instance
the solver state should be collecting constraints from. If an eplex
instance is specified (as {\tt pool(Instance)}), then the solver state is
associated with that instance. If the eplex instance is {\it not\/} to be
associated with an eplex instance, {\tt none} should be given as the
argument to {\tt collect_from}. This allows a solver state to be set up
without the overhead of an eplex instance. The solver state will not
collect any constraints automatically when it is invoked; instead the
constraints must be added explicitly via the handle (using {\tt
  lp_add_constraints/3}).

By default, the external solver is invoked once after set up
by {\tt lp_demon_setup},
if any {\tt TriggerModes} is specified. Otherwise, the solver is not
invoked and the predicate returns after set up.


\subsubsection{\biptxtref{lp_setup(+NormConstraints, +Objective, +ListOfOptions, -Handle)}{lp_setup/4}{../bips/lib/eplex/lp_setup-4.html}}
\label{lpsetup}
This is an even lower-level primitive, setting up a solver state
without any automatic triggering.
It creates a new solver state for the set of constraints NormConstraints
(see \ahrefloc{constrcoll}{below} for how to obtain a set of
normalised constraints).
Apart from the explicitly listed constraints, the variable's ranges will
be taken into account as the variable bounds for the simplex algorithm.
Undeclared variables are implicitly declared as \bipref{reals/1}{../bips/lib/eplex/reals-1.html}.

However, when variables have been declared integers in other solvers (e.g.\
using \bipref{ic:integers/1}{../bips/lib/ic/integers-1.html}),
that is not taken into account by the solver by default.
This means that the solver will only work on the {\em relaxed problem}
(ie.\ ignoring the integrality constraints),
unless specified otherwise in the options.
Objective is either {\tt min(Expr)} or {\tt max(Expr)}
where Expr is a linear (or quadratic) expression.
ListOfOptions is a list of solver options, the same as for
\bipref{lp_demon_setup/5}{../bips/lib/eplex/lp_demon_setup-5.html} and \bipref{eplex_solver_setup/4}{../bips/lib/eplex/eplex_solver_setup-4.html}, except for the {\tt
  collect_from} and {\tt initial_solve} options, which are specific for the
demon solvers.

\subsection{Adding Constraints to a Solver State}

Constraints can be added directly to a solver state without posting them to
an eplex instance. This is done by:

\subsubsection{\biptxtref{lp_add_constraints(+Handle, +Constraints, +NewIntegers)}{lp_add_constraints/3}{../bips/lib/eplex/lp_add_constraints-3.html}}

Add new constraints (with possibly new variables) to the solver state
represented by Handle
The new constraints will be taken into account the next time the
solver is run.  The constraints will be removed on backtracking.

The constraints are first normalised, and simple constraints filtered out (as
discussed in section~\ref{linear-constraints}) before they are added
to the external solver (by calling lp_add/3 described below).

\subsubsection{\biptxtref{lp_add(+Handle, +NewNormConstraints, +NewIntegers)}{lp_add/3}{../bips/lib/eplex/lp_add-3.html}}
\label{lp-add}

This adds the constraints (both linear and integrality) to the
external solver represented by Handle. The linear arithmetic constraints
must be normalised. Note that it is possible to add trivial constraints,
which would be filtered out by the higher level {\tt lp_add_constraints/3}
using this predicate. Integrality constraints on non-problem variables are
filtered out and a warning given.

\subsubsection{\biptxtref{lp_add_vars(+Handle, +Vars)}{lp_add_vars/2}{../bips/lib/eplex/lp_add_vars-2.html}}

This adds the variables in Vars to the external solver state represented by
Handle. The variables should not contain variables which are already
problem variables. The variables are given the default bounds of
-infinity..infinity. 

\subsubsection{\biptxtref{lp_var_set_bounds(+Handle, +Var, ++Lo,++Hi)}{lp_var_set_bounds/4}{../bips/lib/eplex/lp_var_set_bounds-4.html}}

This updates the bounds for the problem variable Var in the external
solver state represented by Handle. Failure occurs if Var is not a problem
variable. 

%----------------------------------------------------------------------
\subsection{Running a Solver State Explicitly}
%----------------------------------------------------------------------


\subsubsection{\biptxtref{lp_solve(+Handle,
-Cost)}{lp_solve/2}{../bips/lib/eplex/lp_solve-2.html}}

Apply the external solver's LP or MIP solver to the problem represented by Handle.
Precisely which method is used depends on the options given to lp_setup/4.
lp_solve/2 fails if there is no solution or succeeds
if an optimal solution is found, returning the solution's cost in Cost
(unlike with lp_demon_setup/6, Cost gets instantiated to a number).
After a success, various solution and status information can be retrieved
using lp_get/3 and lp_var_get/4.

\begin{sloppypar}
The set of constraints considered by the solver is the one given when the
solver was created plus any new constraints that were added
(e.g  by lp_add_constraints/3) in the meantime.
\end{sloppypar}

If there was an error condition, or limits were exceeded,
lp_solve/2 raises an event. See section \ref{lpevents} for details.

\subsubsection{lp_probe(+Handle, +Probes, -Cost)}
\index{eplex!lp_probe/3}
Similar to lp_solve/2, but optimize for a modified problem as specified by
Probes. This is the
predicate called by \bipref{eplex_probe/2}{../bips/lib/eplex/eplex_probe-2.html}
\subsection{Accessing the Solver State}

In section~\ref{eplex-instance-solver-info}, we discussed how solver state
information can be accessed via the eplex instance. Here are the lower
level predicates that directly access this information via the solver
state's handle:

\subsubsection{\biptxtref{lp_get(+Handle, +What, -Value)}{lp_get/3}{../bips/lib/eplex/lp_get-3.html}}
Retrieve information about solver state and results. See the reference
manual description of {\tt lp_get/3} for a detailed description of the
available values for {\tt What}.

For example, it is possible to obtain the solution values from the last
successful invocation of the external solver using the following:

    \begin{verbatim}
    instantiate_solution(Handle) :-
        lp_get(Handle, vars, Vars),
        lp_get(Handle, typed_solution, Values),
        Vars = Values.
    \end{verbatim}


\subsubsection{\biptxtref{lp_var_get(+Handle,+Var, +What, -Value)}{lp_var_get/4}{../bips/lib/eplex/lp_var_get-4.html}}
Retrieve information about solver state represented by Handle,
related to a specific variable Var. Again, see the reference manual for the
available parameters.

\subsubsection{\biptxtref{lp_var_get_bounds(+Handle, +Var, -Lo, -Hi)}{lp_var_get_bounds/4}{../bips/lib/eplex/lp_var_get_bounds-4.html}}
Retrieve the bounds of the problem variable Var from the solver state
represented by Handle. 

\subsubsection{\biptxtref{reduced_cost_pruning(+Handle,?GlobalCost)}{reduced_cost_pruning/2}{../bips/lib/eplex/reduced_cost_pruning-2.html}}
This predicate implements a technique to prune variable bounds
based on a global cost bound and the reduced costs of some solution to
a problem relaxation.  The assumptions are that there is a global
problem whose cost variable is GlobalCost, and that Handle refers to
a linear relaxation of this global problem.
The pruning potentially affects all variables involved in the relaxed
problem.

\subsection{Expandable Problem and Constraints}
\label{coladd}

We provide low-level primitives to `expand' an eplex problem. Such a problem is
considered to have as yet unspecified components in the objective function
and posted constraints. These constraints are known as expandable constraints.
The as yet unspecified component involve variables
that have not yet been added to the problem. When these variables are
added, coefficients for the variables can be added to the expandable
constraints, as well as the objective function. These primitives are the
basis for implementing {\bf column generation}, and are used by the column
generation library, lib(colgen). 

These primitives modify an existing eplex problem {\it
non-monotonically}, and can only be used on problems that are not
represented by an eplex instance, and was not setup as a demon solver
(i.e. no trigger conditions are specified). 

\subsubsection{\biptxtref{lp_add_constraints(+Handle, +Constraints, +Ints, -Idxs)}{lp_add_constraints/4}{../bips/lib/eplex/lp_add_constraints-4.html}}
\index{column generation!lp_add_constraints/4}
This adds expandable constraints Constraints to the solver state
represented by Handle. The predicate returns a list of indices for these
constraints in Idxs. The indices are used to refer to the constraints when
new variables are added to expand the problem.

\subsubsection{\biptxtref{lp_add_columns(+Handle, +Columns)}{lp_add_columns/2}{../bips/lib/eplex/lp_add_columns-2.html}}
\index{column generation!lp_add_columns/4}
This expands the problem by adding new variables (columns) to the solver
state represented by Handle. Columns is a list of
variable:column-specification pair, where variable is the variable to be
added as a new column, and column-specification the specification for the
non-zero components of the column, i.e. coefficients for the expandable
constraints (referred to using the index obtained from
lp_add_constraints/4) and the objective for this variable.



\subsection{Changing Solver State Settings}
In addition to accessing information from the solver state, some options (a
subset of those specified during solver set up) can be changed by:
\subsubsection{\biptxtref{lp_set(+Handle, +What, +Value)}{lp_set/3}{../bips/lib/eplex/lp_set-3.html}}
This primitive can be used to change some of the initial options
even after setup. {\em Handle} refers to an existing solver state. See the
reference manual for details.

\subsection{Destroying a Solver State}
\subsubsection{\biptxtref{lp_cleanup(+Handle)}{lp_cleanup/1}{../bips/lib/eplex/lp_cleanup-1.html}}

Destroy the specified solver state, free all memory, etc. If the solver
state is associated with an eplex handle, the solver state is disassociated
with the eplex instance. However, unlike \bipref{eplex_cleanup/0}{../bips/lib/eplex/eplex_cleanup-0.html}, the
outstanding constraints not yet collected by the solver is not removed.

As with {\tt eplex_cleanup/0}, care should be taken before using this
non-logical predicate. 

\subsection{Miscellaneous Predicates}
\subsubsection{\biptxtref{lp_read(+File, +Format, -Handle)}{lp_read/3}{../bips/lib/eplex/lp_read-3.html}}

Read a problem from a file and setup a solver for it.  Format is
{\tt lp} or {\tt mps}.
The result is a handle similar to the one obtained by lp_setup/4.
\subsubsection{\biptxtref{lp_write(+Handle, +Format, +File)}{lp_write/3}{../bips/lib/eplex/lp_write-3.html}}

Write out the problem in the solver state represented by Handle to the file
File in format Format.

\aname{constrcoll}{}
\subsubsection{\biptxtref{normalise_cstrs(+Constraints, -NormConstraints, -NonlinConstr)}{normalise_cstrs/3}{../bips/lib/eplex/normalise_cstrs-3.html}}
where Constraints is a list of terms of the form
X \verb.$=. Y, X \verb.$>=. Y or X \verb.$=<. Y 
where X and Y are arithmetic expressions.
The linear constraints are returned in normalised form in NormConstraints,
the nonlinear ones are returned unchanged in NonlinConstr.

\section{Cutpool Constraints}

\index{cutpool constraints}
\index{global cuts}
In eplex, constraints added to a problem are removed on
backtracking. However, it is sometimes possible to discover constraints
that are valid for the whole problem, which the user wish to apply even
after backtracking -- such constraints are referred to as `global cuts'. 

In addition, the user may want to remove some constraints from
the problem being solved, because they do not help to constrain the
problem, but they slow down the solving of the problem. 

To support this, eplex allow constraints to be added to a
{\it cutpool} associated with a problem, instead of directly to the
problem. The main differences from the normal problem constraints are:
\begin{itemize}
\item they are {\it not\/} removed on backtracking. Once
added to a cutpool, a constraint exists until the problem itself is
destroyed. 
\item they are handled differently during solving, and the user has more
control on how the external solver takes the constraints into account.
\end{itemize}

\subsection{Solving a Problem with Cutpool Constraints}
Logically, cutpool constraints are always valid for the problem, and so
should be considered when the problem is solved.  Unlike normal
constraints, cutpool constraints are not necessarily added to the
solver's problem matrix, and if they are added, they are added only for the
solving, and removed from the problem matrix after the solving. 

When the external solver 
solves the problem, eplex ensures that the cutpool constraints are
consistent with the problem, i.e. none of the constraints are
violated. The cutpool constraints can either be added to the problem matrix
immediately, or they can be checked for violation after the solver solves
the problem. Any
violated constraints are then added to the problem, and the problem
resolved. This is repeated until either a fix-point is reached, where no
constraints are violated, or if the external solver is unable to solve the
problem. 

If the external solver does not produce a solution, then:
\begin{itemize}
\item if the problem is unbounded, any outstanding cutpool constraints are
added to the problem matrix without checking for violation,  and the
external solver is invoked for one more time. This is because the extra
constraints may make the problem bounded. 
\item if the problem is infeasible, then failure occurs as normal
\end{itemize}

This multiple invocation of the solver occurs within an eplex's call to the
external solver to solve a problem, and so the process should be
transparent to the user, except that the setting of the timeout applies to
each solver invocation, rather than to the whole solving process.

The user can specify how the cutpool constraints are treated: they can be
either added to the problem matrix before invoking the solver, or only
added if violated. In addition, cutpool constraints
can be made inactive, in which case they are not considered for adding to
the problem matrix at all (and are not checked for violations). This is
provided for efficiency reason -- if the user knows for certain
constraints would not be violated by the solution, they can be made
inactive. It is the user's responsibility to ensure the correctness in this
case. 

Unlike normal problem constraints, cutpool constraints cannot add new
variables to the problem, i.e. the constraint must only involve problem
variables that are present in the problem during solver set up. This is
because cutpool constraints are globally valid, and so cannot involve
variables that may not exist after backtracking. Variables can be added to
a problem before solver set up by posting constraints involving them,
including \bipref{reals/1}{../bips/lib/eplex/reals-1.html}, which simply
declares variables as problem variables. 

Additionally, each cutpool constraint belongs to a named
group, specified when the constraint is added.  This allows the user to
classify the cutpool constraints into different groups, and then
manipulate a groups of constraints as a whole, {\it e.g.\/} making them all
inactive. A default group is predefined, to which cutpool constraints
belongs unless specified otherwise. To add cutpool constraints to other
groups, the group name must first be created with the
{\tt cutpool_group} option of
\bipref{lp_get/3}{../bips/lib/eplex/lp_get-3.html}.

\subsection{Predicate-specific Support}
Constraints are added to the cutpool using:

\subsubsection{\biptxtref{lp_add_cutpool_constraints/4}{lp_add_cutpool_constraints/4!eplex}{../bips/lib/eplex/lp_add_cutpool_constraints-4.html}}
    Add the constraints to the cut-pool associated with the
    problem specified by the handle. By default, the constraints belong to
    the default group, and are active and have the `add initially' status
    set. These can be over-ridden by the Options argument. The predicate
    returns a list of indices for these constraints in Idxs.

Information on cutpool constraints can be obtained using the 
{\tt cutpool_info}
option of \bipref{lp_get/3}{../bips/lib/eplex/lp_get-3.html}. The status of
a cutpool constraint, such as its active status,  can be set using the 
{\tt cutpool_option} option of
\bipref{lp_set/3}{../bips/lib/eplex/lp_set-3.html} -- the change is
non-logical, i.e. it is not undone on backtracking.

Using lp_get/3 and lp_set/3, the user can program their own algorithms to
control how the cutpool constraints are treated when the problem is solved.
For example, the user may want to make a whole group of constraints
inactive because they seem to slow the solver down but do not produce
better solutions. In this case, the user can use lp_get/3 to obtain all the
current constraints in the group, and then use lp_set/3 to set these
constraints to inactive.

As cutpool constraints are not added directly to the problem matrix, this
affects the library predicates that deals with the problem state:

\begin{itemize}
\item row-wise solution states like dual and slack values include only the
cutpool constraints that are actually added to the problem. These are added
after the normal constraints, and their order in the matrix can be obtained
using the {\tt cutpool_info(last_added, index)} option of
\bipref{lp_get/3}{../bips/lib/eplex/lp_get-3.html}. 
\item the {\tt constraints} and {\tt constraints_norm} options of
\bipref{lp_get/3}{../bips/lib/eplex/lp_get-3.html} returns only the normal
constraints. Other options that returns information about the problem
(e.g. {\tt num_rows}) also do not include the cutpool constraints.  
\item 
\bipref{eplex_write/2}{../bips/lib/eplex/eplex_write-2.html} and 
\bipref{lp_write/3}{../bips/lib/eplex/lp_write-3.html} will dump all 
the active cutpool constraints with the problem. This may be different from
the actual problem solved by the external solver because not all active
cutpool constraints need be added to the problem, and the order of these
constraints could be different. To dump the exact problem solved by the
external solver, use the {\tt write_before_solve} option of the solver
instead. 
\end{itemize}

\section{Multiple Solver States}

This library allows multiple solver states to be maintained in the same
program. Each solver state represents an eplex problem. For the external
solver, each solver state is completely independent. For {\eclipse}, the
solver states may share variables in the constraints or objective
functions, but the constraints posted to the problem and to the cutpools
are specific to each solver state. The eplex library maintains separate
solution values for each solver state, and it is up to the user to
reconcile these solution values if they are different.

\index{unification!eplex variables}
When two eplex variables are unified, then the library ensures that the
now single variable maintains the eplex values from both variables. The one
exception is when two variables from the same solver state is unified.In
this case, eplex will merge its representations of the two columns into one:
the bounds of the columns are merged (and failure will occur if the
merged bound is empty); all the unified columns are constrained to integers
if one of them was constrained to integer previously, and 
an equality constraint between the two variables is sent to the
solver state, but the user can only obtain one eplex value from the unified
variable, even though in the external solver, the variable is still
represented as two variables (columns in the matrix). 

It is possible to turn off this automatic sending of the equality
constraints by specifying `no' for the option {\tt
post_equality_when_unified} (in solver setup, or via \verb'eplex_set/2'). 
The reason is that some solvers automatically perform unification
when they know that two variables are the same. For example, for
the constraint {\tt X \$= Y + Z}, if {\tt Y} becomes 0, then {\tt X} and
{\tt Z} may be unified by the solver maintaining the constraint. If the
same constraint was also posted to the eplex solver state, then there is no
need to send the redundant constraint. However, if the external solver
state did not have the constraint, then it can become inconsistent with
that of {\eclipse} if the equality constraint is not sent. Therefore, only
turn off sending of equality constraints if you are certain you know what
you are doing.

The merging of the bounds may trigger the solver if the bounds trigger
condition was specified. Note however that the deviating_bounds trigger
condition is not tested for, because there is no longer a valid solution
value for the merged columns. 

\section{External Solver Output and Log}
The external solver's output can be controlled using:
\begin{description}
\item[{\tt lp_set(SolverChannel, +(Stream))}]
Send output from SolverChannel to the \eclipse\ I/O stream Stream.
\item[{\tt lp_set(SolverChannel, -(Stream))}]
Stop sending output from SolverChannel to the \eclipse\ I/O stream Stream.
\end{description}
SolverChannel is one of
{\tt result_channel, error_channel, warning_channel, log_channel},
and Stream is an \eclipse\ stream identifier (e.g. {\tt output},
or the result of an open/3 operation).
By default, {\tt error_channel} is directed to \eclipse's {\tt error} stream,
{\tt warning_channel} to {\tt warning_output} stream,
while {\tt result_channel} and {\tt log_channel} are suppressed.
To see the output on these channels, do for instance
\begin{verbatim}
:- lp_set(result_channel, +output), lp_set(log_channel, +log_output).
\end{verbatim}
Similarly, to create a log file:
\begin{verbatim}
:- open("mylog.log", write, logstream), lp_set(log_channel, +logstream).
\end{verbatim}
and to stop logging:
\begin{verbatim}
:- lp_set(log_channel, -logstream), close(logstream).
\end{verbatim}


\section{Dealing with Large and Other Non-standard Numbers}

In many external solvers, infinities or very large numbers are not handled
directly. Instead, these solvers define a large (floating point) number to
be infinity. However, the problem that is sent to the external solver may
contain values greater than the solver's notion of infinity. This is
handled in the following way:

\begin{itemize}
\item If a variable's range extends beyond the solver's infinity, the range
is rounded down. 
\item If some coefficient (constant) in the problem is outside the solver's
range, an out of range error would be raised when this is detected (and the
problem is not passed to the external solver).
\end{itemize}

In addition, {\eclipse} supports numeric types that are not generally
available, e.g.\ bounded real and rational. These are converted into
floating point numbers before they are passed to the external solver.

\section{Error Handling}

The external solver's optimisation can abort without completely solving the
problem, because of some error, or some resource limit was reached. Eplex
classifies these into the following cases, with default ways of handling them:

\begin{description}
    \item[suboptimal]
    	This means that a solution was found but it may be suboptimal.
	The default behaviour is to print a warning and succeed.
    \item[unbounded]
	This means that the problem is unbounded.  The default
	behaviour is to bind Cost to infinity (positive or negative
	depending on the optimisation direction), print a warning and
	succeed.  CAUTION: No solution values are computed when the
	problem is unbounded, so unless the problem was set up with
	the solution(no) option, an error will occur when trying to
	continue as if the optimisation had succeeded.
    \item[unknown]
    	This means that due to the solution method chosen, it is unknown
	whether the problem is unbounded or infeasible. The default
	behaviour is to print a warning and fail (even though this
	may be logically wrong!).
    \item[abort]
    	Some other error condition occurred during optimisation.
	The default behaviour is to print an error and abort.
\end{description}

The default behaviours can be overridden for each problem by giving a user
defined goal to handle each case during problem setup in
eplex_solver_setup/4 (lp_setup/4, lp_demon_setup/5, or later with
eplex_set/2 or lp_set/3) as an option. If given, the user defined goal will
be executed instead. The user defined handler could for instance change
parameter settings and call lp_solve again. 

\label{lpevents}
The default behaviour is implemented by raising the events {\tt
eplex_suboptimal}, {\tt eplex_unbounded}, {\tt eplex_unknown} and {\tt
eplex_abort} for the different abort cases. These events can themselves be
redefined to change the default behaviours. However, as this changes the
behaviour globally, it is not recommended.

Note that in the user defined handlers, it may be possible to extract some
useful bound information on the optimal objective value using the
best_bound and worst_bound options of lp_get/3.

%----------------------------------------------------------------------
\section{Solver Behaviour Differences}
%----------------------------------------------------------------------
In general, an MP problem can have more than one optimal solution (i.e.\ 
different sets of assignments to the problem variables that gives the
optimal objective value). Any of these solutions is correct, and the
external solver will return one of them. It is possible for a different
solver (or even a different version of the same solver) to return another
of these solutions. If the user's program uses the solution values, then it
is possible that the detailed behaviour of the program could depend on the
solver being used. 

The solution that is returned can also depend on the detailed settings of
the floating point unit of the processor. Thus changing some of these
settings may change the solution that is returned. It is thus possible for
eplex to give different solutions on the same machine and solver if these
settings are changed (e.g.\ when {\eclipse} is embedded into a Java
application). 

%----------------------------------------------------------------------
\section{Solver Specific Information}
%----------------------------------------------------------------------
\label{specificsolver}

The external solvers currently supported by the eplex library are:
\begin{itemize}
\item XPRESS-MP, a product of Dash Optimisation (www.dashoptimization.com)
\item CPLEX, a product of ILOG (www.ilog.com)
\item Solvers accessed via OSI, an Open Source solver interface of the
COIN-OR project (www.coin-or.org). Various solvers are supported via OSI
using the generic interface. The following are currently distributed:
\begin{itemize}
\item clpcbc: COIN-OR's CLP linear solver in combination with the CBC
branch and cut framework for MIP problems. Eplex provides specialised
support for this combination, which enhance the features provided and
improve performance for solving larger MIP problems. 
\item symclp: COIN-OR's SYMPHONY MILP solver (with CLP as the linear
solver). Supported via the generic OSI API.
\end{itemize}
\end{itemize}
Both Dash and ILOG offer academic licences at discounted prices, or academic
partner programs.

To load a specific solver explicitly, use:

\begin{verbatim}
:- lib(eplex_cplex).
:- lib(eplex_xpress).
:- lib(eplex_osi_clpcbc). 
:- lib(eplex_osi_symclp). 
\end{verbatim}
\index{eplex_cplex}
\index{eplex_xpress}
\index{XPRESS-MP}
\index{CPLEX}
The first line explicitly requests the CPLEX solver, the second line
explicitly requests the XPRESS-MP solver. Note that these solvers must be
available for your machine for the above to work. The third line requests
the CLP/CBC solver, and the fourth line requests the SYMPHONY solver.

%To load a specific solver using ic for the interval representation, use:
%\begin{verbatim}
%:- lib(ic_eplex_cplex).
%:- lib(ic_eplex_xpress).
%\end{verbatim}
%\index{ic_eplex_cplex}
%\index{ic_eplex_xpress}

\subsection{Versions and Licences}

All the solvers supported by the library come in various versions.
The set of supported solver versions may vary between different
releases of {\eclipse}; please refer to the release notes.
%In addition,
%for XPRESS-MP, we distinguish versions included with {\eclipse}: the `OEM'
%versions, from versions obtained independently by the user. 

Depending on which solver licence you have (for the commercial solvers),
which version of it, and which hardware and operating system,
you need to use the matching version of this interface.\footnote{Note that
it is {\bf not} sufficient that you have a valid license and solver
libraries for a particular version of the solver. That solver version must
also be supported by the release of {\eclipse} you are using.}
Because an {\eclipse} installation can be shared between several
computers on a network, we have provided you with the possibility
to tell the system which licence you have on which machine.
To configure your local installation, simply add one line for
each computer with the appropriate licence to the file
\verb+<eclipsedir>/lib/eplex_lic_info.ecl+, where \verb+<eclipsedir>+
is the directory or folder where your {\eclipse} installation resides.
The file contains lines of the form
\begin{quote} \begin{verbatim}
licence(Hostname, Solver, Version, LicStr, LicNum).
\end{verbatim} \end{quote}
For example, if you have CPLEX version 7.5 on machine \verb+workhorse+,
%and both the OEM and non-OEM XPRESS-MP version 13.26 on machine
%\verb+mule+, and your internet
and XPRESS-MP version 15.20 (with the license file located in
\verb'/my/xpress/license') on machine \verb+mule+, and your Internet
domain is \verb'+icparc.ic.ac.uk',
you would add the lines
\begin{quote} \begin{verbatim}
licence('workhorse.icparc.ic.ac.uk', cplex, '75', '', 0).
licence('mule.icparc.ic.ac.uk', xpress, '1520', '/my/xpress/license', 0).         
\end{verbatim} \end{quote}
The hostname must match the result of get_flag(hostname,H),
converted to an atom (this is normally the complete Internet domain name,
rather than just the machine).
Version is formed from the concatenation of the major and minor version
numbers. 
%In the case of OEM XPRESS-MP, this is followed by the postfix
%\verb+icp+. 
The meaning of LicStr and LicNum depends on the optimiser:
For an open source solver, both LicStr and LicNum are unused, as no license
is required.
For CPLEX with normal licenses, they are unused (the environment
variable \verb'ILOG_LICENSE_FILE' should be set to the CPLEX
license file \verb'access.ilm' as usual). 
%LicStr is a string containing the environment settings
%for runtime licences, e.g.\ "CPLEXLICENSE=/usr/local/cplexlic.ptr",
%and LicNum is the serial number for runtime licences.
For XPRESS-MP,
%if the OEM version is used, LicStr should be the atom
%\verb+default+, as the licensing is handled internally by eplex. For other
%versions of the library,
LicStr is a string specifying the directory where
licence file is located (overrides value of XPRESS environment
variable). LicNum is unused. % in both cases.
If a machine has more than one licence and lib(eplex) is called, the
first one listed in \verb+eplex_lic_info.ecl+ will be used.

%----------------------------------------------------------------------
\subsection{Solver Differences}
%----------------------------------------------------------------------

While the eplex library allows the user to write code in a
solver-independent way, there are differences between the solvers that
the user should be aware of:

\begin{itemize}
\item Different solvers may support different features. In particular,
solvers may support different methods of solving the problem, and solving
 of problems with a quadratic objective is not supported for all solvers.
At the very minimum, solvers must be able to solve (optimise) linear
 problems with a linear objective. Currently, all supported solvers can
 solve linear and MIP problems, with a Simplex solver.
\item Some features may be poorly supported by a particular solver, and
 some feature (such as the relaxed probe of 
\bipref{eplex_probe/2}{../bips/lib/eplex/eplex_probe-2.html} may be
supported slightly differently).
\item Performance for specific problem may vary very significantly (orders
of magnitude) between different solvers. This does not necessarily indicate 
that one solver would consistently out perform another. In addition, the
different solvers may return a different optimal solution to specific
problems, i.e.\ with the same objective value, but different solution
values for the variables.
\item The solvers have different solver parameters to control/tune the
solver. These can be accessed from eplex, and is one of the only places
 where the user code may need to be solver specific.
\end{itemize}

\subsubsection{CPLEX}
CPLEX supports solving of linear and mixed integer problems, both with a
linear objective (LP and MILP), and also with a quadratic objective (QP and
MIQP). It supports various solving methods: simplex (primal and dual),
barrier, network simplex, and sifting. 

In recent versions of CPLEX, the relaxed and fixed probes are implemented 
by removing
the integer information from the problem and solving the relaxed problem,
and then adding the interger information back. This is because the CPLEX
API does not provide access to the root node of the MIP solve.

CPLEX have only global parameters.
  
\subsubsection{OSI}
The features provided by eplex OSI is determined by the actual
solver(s) used via OSI, and to a lesser extent by what the OSI API
supports. The OSI API is mainly designed to support solving of 
linear and, to a lesser extent, MILP problems. However, for specific
solvers, in particular the CLP/CBC combination, eplex
directly access the solvers' own API to provide some functionality not
supportable via OSI. Note however the OSI API is constantly evolving and
improving, so some of the features may be directly supported via OSI in the
future. 

Features {\bf not} supported by OSI: 
\begin{itemize}
\item changing of solving method (it does support specifying if the problem
should be solved as a primal or dual)
\item problems with a quadratic objective
\item time-outs from solving a problem.
\item obtaining detailed information about the MIP solving process,
especially when the MIP search was not complete, such as
the best MIP objective bound.
\item determining if an aborted solve have a suboptimal solution 
\item use of Special Order Set (SOS)
\end{itemize}

\paragraph{osi_clpcbc} Supports primal and dual simplex methods for
solving linear and MILP problems. It also supports time-outs, and is better 
at determining the state of an aborted problem than using OSI on its own.
For the MIP related functionality, the MIP solver CBC is accessed directly
rather than through OSI, and this 
provides better MIP support: more information on the MIP
solver state is available, and a more sophisticated MIP search (based on
sample code supplied with CBC) than the default is performed, generally
leading to faster MIP solves.

The problem representation is stored by CLP, and one performance issue 
when using CLP is that incrementally adding new constraints to
a problem after solver setup can be expensive, as the whole problem has to
be copied and expanded for each addition. It is therefore more efficient to 
either post the constraints before problem setup, or add a large number of
constraints in one go, e.g.\ by using 
\bipref{eplex_add_constraints/3}{../bips/lib/eplex/eplex_add_constraints-3.html}.
Unfortunately, this can be less memory efficient than incrementally adding
the constraints, if those constraints are only needed by eplex and not at
the {\eclipse} level.

\paragraph{osi_symclp} Supports primal and dual simplex methods for
solving linear and MILP problems. MILP is currently performed via the 
standard OSI API, and so has the same restrictions. Special Order Sets
(SOS) are not supported. Time-outs are not supported.
 
Another restriction is due to SYMPHONY, which does not allow the objective
coefficients to be modified after a problem have been solved once. Thus the
objective changing probes are not supported.

\subsubsection{XPRESS}
XPRESS supports solving of linear and mixed integer problems, both with a
linear objective (LP and MILP), and also with a quadratic objective (QP and
MIQP). It supports simplex (primal and dual) and barrier solving methods.

XPRESS does not maintain an optimisation direction with the problem;
instead it requires this to be specified each time the problem is
solved. As such, the optimisation direction, given in a LP format
specification of the problem, is ignored when a problem is read in, and
when writing a problem, minimisation is assumed. 



%----------------------------------------------------------------------
\subsection{Access to External Solver's Control Parameters}
%----------------------------------------------------------------------

The external solver has a number of control
parameters that affect the way it works.
These can be queried and modified using the
\bipref{lp_get/2}{../bips/lib/eplex/lp_get-2.html},  
\bipref{eplex_get/2}{../bips/lib/eplex/eplex_get-2.html},
\bipref{lp_get/3}{../bips/lib/eplex/lp_get-3.html}, and
\bipref{lp_set/2}{../bips/lib/eplex/lp_set-2.html}, 
\bipref{eplex_set/2}{../bips/lib/eplex/eplex_set-2.html}, 
\bipref{lp_set/3}{../bips/lib/eplex/lp_set-3.html} predicates respectively:

\subsubsection{\biptxtref{lp_get(+Handle, optimizer_param(+ParamName), -Value)}{lp_get/3}{../bips/lib/eplex/lp_get-3.html}}
Retrieve the value of a control parameter for the external solver for the
problem represented by Handle. These
parameters are solver specific; see
\bipref{lp_get/3}{../bips/lib/eplex/lp_get-3.html} for more details..

\subsubsection{\biptxtref{EplexInstance:eplex_get(optimizer_param(+ParamName), -Value)}{eplex:eplex_get/2}{../bips/lib/eplex/eplex_get-2.html}}
Like lp_get/3, but get a control parameter for the external solver
associated with the specified eplex instance. 

\subsubsection{\biptxtref{lp_get(optimizer_param(+ParamName), -Value)}{lp_get/2}{../bips/lib/eplex/lp_get-2.html}}
Retrieve the global value of a control parameter for the external solver. The
parameters and the exact meaning of `global' is solver specific: if the
solver does not have global parameters, this gets the global default value,
rather than the globally applicable value. The parameters are as in lp_get/3.

\subsubsection{\biptxtref{lp_set(+Handle, optimizer_param(+ParamName), +Value)}{lp_set/3}{../bips/lib/eplex/lp_set-3.html}}
Set a control parameter for the external solver for the problem represented
by Handle. If the external solver does not have problem specific
parameters, this will raise an unimplemented functionality exception. 
The parameters are as in lp_get/3.

\subsubsection{\biptxtref{EplexInstance:eplex_set(optimizer_param(+ParamName), +Value)}{eplex_set/2}{../bips/lib/eplex/eplex_set-2.html}}
Like lp_set/3, but set a control parameter for the external solver
associated with the specified eplex instance. 

\subsubsection{\biptxtref{lp_set(optimizer_param(+ParamName), +Value)}{lp_set/2}{../bips/lib/eplex/lp_set-2.html}}
Set a control parameter for the external solver for the problem globally.
If the external solver does not have global parameters, this will set the
global default for the parameter. The parameters are as in lp_get/3.

\subsubsection{\biptxtref{lp_get(optimizer, -Value)}{lp_get/2}{../bips/lib/eplex/lp_get-2.html}
and \biptxtref{lp_get(optimizer_version, -Value)}{lp_get/2}{../bips/lib/eplex/lp_get-2.html}}
Retrieve the name (currently 'cplex','xpress' or 'osi') and version of the external optimizer.
This can be used to write portable code even when using solver-specific settings:
\begin{verbatim}
    ( lp_get(optimizer, xpress) ->
        lp_set(Handler, optimize_param(maxnode), 100)
    ; lp_get(optimizer, cplex) ->
        lp_set(Handler, optimize_param(node_limit), 100)
    ; lp_get(optimizer, osi) ->
        (lp_get(optimizer_version, clpcbc) -> 
             lp_set(Handler, optimize_param(node_limit), 100)
        ;
             true
        )
    ), ...
\end{verbatim}

\index{mathematical programming, interface to|)}
\index{mixed integer programming, interface to|)}
\index{linear programming, interface to|)}
\index{quadratic programming, interface to|)}
\index{simplex solver, interface to|)}
%HEVEA\cutend
