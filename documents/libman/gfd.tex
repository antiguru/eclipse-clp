% BEGIN LICENSE BLOCK
% Version: CMPL 1.1
%
% The contents of this file are subject to the Cisco-style Mozilla Public
% License Version 1.1 (the "License"); you may not use this file except
% in compliance with the License.  You may obtain a copy of the License
% at www.eclipse-clp.org/license.
% 
% Software distributed under the License is distributed on an "AS IS"
% basis, WITHOUT WARRANTY OF ANY KIND, either express or implied.  See
% the License for the specific language governing rights and limitations
% under the License. 
% 
% The Original Code is  The ECLiPSe Constraint Logic Programming System. 
% The Initial Developer of the Original Code is  Cisco Systems, Inc. 
% Portions created by the Initial Developer are
% Copyright (C) 2012 Cisco Systems, Inc.  All Rights Reserved.
% 
% Contributor(s): Kish Shen
% 
% END LICENSE BLOCK

%HEVEA\cutdef[1]{section}

\section{Introduction}

   The GFD library is an interface for {\eclipse} to Gecode's finite domain constraint
   solver. Gecode ({\tt www.gecode.org}) is an open-source toolkit for 
   developing
   constraint-based systems in C++, and includes an  integer 
   finite domain constraint solver.

   This interface provides a high degree of code compatibility with the finite 
   domain portion of the IC library, and to a lesser extent, with the FD
   library as well. This means that programs originally written for the
   IC library should run with GFD with little modifications, beyond 
   renaming any explicit reference to the ic family of modules. For example,
   here is a GFD program for N-Queens:

\begin{quote}
\begin{verbatim}
:- lib(gfd).

queens_list(N, Board) :-
    length(Board, N),
    Board :: 1..N,
    (fromto(Board, [Q1|Cols], Cols, []) do
        ( foreach(Q2, Cols), param(Q1), count(Dist,1,_) do
            Q2 #\= Q1,
            Q2 - Q1 #\= Dist,
            Q1 - Q2 #\= Dist
        )
    ),
    labeling(Board).  
\end{verbatim}
\end{quote}

This version of the program is from an example IC version of N-Queens,
with just \verb':- lib(ic)' replaced by \verb':- lib(gfd)'. The
search is done in \eclipse, using GFD's \verb'labeling/1', which essentially
employs no heuristics in selecting the variable (input order) and choice of 
value to label the selected variable to (from minimum).



\section{Problem Modelling and Solving}

GFD provides facilities to model and solve problems over the 
finite integer domain, with Gecode as the solver. It supports the constraints
provided by Gecode -- and Gecode supports a large set of constraints. The 
search to solve the problem can be done at the 
{\eclipse} level (with support from Gecode for variable and value selections), 
or the whole search can be performed by Gecode itself using one of its 
search engines.   

Implementation-level differences (like Gecode's 
{\it re-computation based\/} model vs. \eclipse's {\it backtracking\/}
model) are largely invisible to the user,
as GFD automatically maintains the Gecode computational state 
to match \eclipse's.

\subsection{Usage}

To load the GFD library into your program, simply add the following directive
at an appropriate point in your code.

\begin{quote}
\begin{verbatim}
:- lib(gfd).
\end{verbatim}
\end{quote}

\subsection{Integer domain variables}

An (integer) domain variable is a variable which can be instantiated only to a
value from a given finite set of integer values. 

A variable becomes a domain variable when it first appears in a (GFD) 
constraint. If the constraint is a domain constraint, then the variable will
be given the domain specified by the constraint. Otherwise, the variable will
be given a default domain, which should be large enough for
most problem instances. 

The default domain is an interval, and the maximum and minimum values of this
interval can be changed using \bipref{gfd_set_default/2}{../bips/lib/gfd/gfd_set_default-2.html} (with the options 
{\tt interval_max} and {\tt interval_min} for the maximum and minimum values,
respectively).  You can also obtain the current values
of interval_max and interval_min using \bip{gfd_get_default/2}.

The Gecode documentation suggests that domain variables should be given as
small a domain as possible, and requires the user to explicitly specify a domain for 
all domain variables. While this is not required by GFD, following Gecode's
convention is still a good idea, since overly large domains can negatively
affect performance.  It is therefore recommended to make use of domain
constraints, and specify the domain for variables before using them in other
constraints.

A domain variable is mapped into a Gecode \verb'IntVar'.


\subsection{Constraints}

GFD supports the (integer finite domain) constraints implemented by Gecode.
Some of these constraints correspond to those in \eclipse's native finite 
domain solvers (IC and FD), but many do not. For those that are 
implemented in IC and/or FD, the same name and syntax is used by GFD 
(although details may differ, such as the types allowed for arguments).  
%One difference is that all these constraints are defined in the GFD
%library itself, and can thus be called without module qualification or
%loading additional libraries. 

For many constraints, Gecode supports a choice of consistency levels.
In GFD, these are provided as alternative implementations of these
constraints in one of three modules:
\begin{description}
\item[{\tt gfd_gac}] Domain consistent (Generalised Arc-Consistent), maps to {\tt ICL_DOM} in Gecode.
\item[{\tt gfd_bc}] Bound consistent, maps to {\tt ICL_BND} in Gecode.
\item[{\tt gfd_vc}] Value consistent, maps to {\tt ICL_VAL} in Gecode.
\end{description}
Constraints can either be imported from one of these modules, or explicitly
qualified with the module name, e.g.
\begin{quote}
\begin{verbatim}
gfd_gac:alldifferent(Xs)
\end{verbatim}
\end{quote}
Posting a constraint at a particular consistency level is supported only
if that consistency level is implemented for that constraint
in Gecode -- see the individual documentation for the constraints for details.
Note that the three consistency modules are implicitly created when library(gfd)
is loaded, and do not need to be loaded explicitly.
Posting a constraint unqualified (or qualified with {\tt gfd}) 
means posting the constraint at the default consistency level ({\tt ICL_DEF}). 

Even constraints that involve expressions may be posted at specific 
consistency levels. However, it is possible that some of
the sub-constraints and sub-expressions inside the expressions are not supported
at the given consistency level. In such cases, these sub-expressions will be
posted at the default consistency level.

For example, the N-Queens example
can post domain consistent versions of the \verb'#\=/2' constraints:
\begin{quote}
\begin{verbatim}
:- lib(gfd).

queens_list(N, Board) :-
    length(Board, N),
    Board :: 1..N,
    (fromto(Board, [Q1|Cols], Cols, []) do
        ( foreach(Q2, Cols), param(Q1), count(Dist,1,_) do
            gfd_gac: (Q2 #\= Q1),
            gfd_gac: (Q2 - Q1 #\= Dist),
            gfd_gac: (Q1 - Q2 #\= Dist)
        )
    ),
    label(Board).
\end{verbatim}
\end{quote}
In this particular example, using the stronger propagation actually results in
a reduction in performance, as there is no reduction in search space from the
stronger propagation, but an increase in the cost of doing the propagation, 


Gecode requires an explicit command to perform propagation. In GFD,
this command is implemented as a delayed goal at priority 9.
When a constraint is posted, GFD first adds any new domain variables in
 the constraint to Gecode, and then adds the constraint to Gecode,
 without performing any explicit propagation. The call to propagate is
then scheduled for execution. It is thus possible to post multiple
 constraints without propagation by posting the constraints at a more
urgent (i.e. numerically smaller) priority than 9 (see
\bipref{call_priority/2}{../bips/kernel/suspensions/call_priority-2.html}).
This could reduce the cost of performing the propagation.
 
Several constraints involve the use of indices. In \eclipse, indices starts
from 1, while Gecode, like C++ (the programming language it is implemented in), indices starts from 0. For compatibility 
with \eclipse, ``normal'' GFD constraints also have indices
that starts from 1. These constraints are mapped to the Gecode native 
indices in various ways, depending of the constraint, with the aim of 
minimising the overhead. GFD also supports versions of these constraints that 
uses Gecode's native indices, i.e. starting from 0, and these have an 
additional {\tt _g} in their name (e.g. {\tt bin_packing_g/3} is the Gecode native
index version of {\tt bin_packing/3}). These versions of the constraint do not
have the overhead of converting the index value, but may be
incompatible with the rest of \eclipse.

\subsubsection{Domain constraints}

The following domain constraints are supported by GFD:

\begin{description}
\item[\biptxtrefni{?Vars \#:: ++Domain}{\#::/2!gfd}{../bips/lib/gfd/HNN-2.html}]
Constrains Vars to have the domain Domain. A \biptxtrefni{reified}{\#::/3!gfd}{../bips/lib/gfd/HNN-3.html} version is also available.
\verb'::/2,3' are also supported as aliases.

\end{description}

  
\subsubsection{Arithmetic and logical expressions}

GFD supports expressions as arguments for relational and logical connective
constraints. Expressions can either evaluate to an integer (integer 
expression) or a truth value (constraint expression).

\begin{description}
\item[Relational Constraints]
These specify an arithmetic relationship between two integer expressions.
Constraint expressions are allowed as arguments of relational constraints,
with the truth value of the expression treated as the integer value 1 (true)
or 0 (false).

All relational constraints have their reified counterparts, which has an extra
boolean argument that specify if the constraint is entailed or not.

The  relational constraints are:
\biptxtrefni{\#</2,3 }{\#</2!gfd}{../bips/lib/gfd/HL-2.html} (less than),
\biptxtrefni{\#=/2,3}{\#=/2!gfd}{../bips/lib/gfd/HE-2.html} (equal), 
\biptxtrefni{\#=</2,3}{\#=</2!gfd}{../bips/lib/gfd/HEL-2.html} (less than or equal to),
\biptxtrefni{\#>/2,3}{\#>/2!gfd}{../bips/lib/gfd/HG-2.html} (greater than),
\biptxtrefni{\#>=/2,3}{\#>=/2!gfd}{../bips/lib/gfd/HGE-2.html} (greater than or equal to),
\biptxtrefni{\#\bsl=/2,3}{\#\\=/2!gfd}{../bips/lib/gfd/HRE-2.html} (not equal to).

\item[Logical Connective constraints]
Specifies a logical connection between  constraint expression(s). 
All logical connectives have their reified counterparts.
The available connectives are:

\biptxtrefni{<=>/2,3}{<=>/2!gfd}{../bips/lib/gfd/LEG-2.html} (equivalent),
\biptxtrefni{=>/2,3}{=>/2!gfd}{../bips/lib/gfd/EG-2.html} (implies),
\biptxtrefni{and/2,3}{and/2!gfd}{../bips/lib/gfd/and-2.html} (and),
\biptxtrefni{or/2,3}{or/2!gfd}{../bips/lib/gfd/or-2.html} (or),
\biptxtrefni{xor/2,3}{xor/2!gfd}{../bips/lib/gfd/xor-2.html} (exclusive or),
\biptxtrefni{neg/1,2}{neg/1!gfd}{../bips/lib/gfd/neg-1.html} (negation).

Constraints
which can be reified can occur as an argument of a logical connective, i.e.
as a constraint expression, evaluating to the reified truth value.

As relational constraints can be reified, and truth values of constraint
expressions can be evaluated as integer values, \texttt{\#=/2} can be used
instead of \texttt{<=>/2} and \texttt{\#\bsl=/2} can be used instead of
\texttt{xor}. This is also provided for compatibility with IC.


\end{description}

The syntax for the expressions closely follows that in IC. The 
following can be used inside expressions:

\begin{description}
\item[\texttt{X}]
            \emph{Variables}.  If \verb'X' is not yet a domain variable, it is turned 
            into one.

\item[\texttt{123}]
            Integer constants.

\item[\texttt{+Expr}]
            Identity.

\item[\texttt{-Expr}]
            Sign change.

\item[\texttt{abs(Expr)}]
    The absolute value of Expr.

\item[\texttt{E1+E2}]
    Addition.

\item[\texttt{E1-E2}]
    Subtraction.

\item[\texttt{E1*E2}]
    Multiplication.

\item[\texttt{E1//E2}]
    Integer division, truncating towards zero.

\item[\texttt{E1/E2}]
    Integer division, defined only if E2 evenly divides E1.

\item[\texttt{E1 rem E2}]
            Integer remainder, same sign as E1.

\item[\texttt{Expr}\textasciicircum{}{\texttt 2}]
            Square. Equivalent to {\texttt sqr(Expr)} (alias for compatibility).


\item[\texttt{min(E1,E2)}]
    Minimum.

\item[\texttt{max(E1,E2)}]
    Maximum.

\item[\texttt{sqr(Expr)}]
    Square. Logically equivalent to \verb|Expr*Expr|.

\item[\texttt{isqrt(Expr)}]
            Square root (always positive). Truncated towards zero.

\item[\texttt{sqrt(Expr)}]
            Square root, defined only if Expr is the square of an integer.

\item[\texttt{sum(ExprCol)}]
            Sum of a collection of expressions.

\item[\texttt{sum(IntCol*ExprCol)}]
            Scalar product of a collection of integers and expressions.
            \verb'IntCol' and \verb'ExprCol' must be the same size.

\item[\texttt{min(ExprCol)}]
            Minimum of a collection of expressions.

\item[\texttt{max(ExprCol)}]
            Maximum of a collection of expressions.

\item[\texttt{and}]
            Reified constraint conjunction.  e.g. \verb'X #> 3 and Y #< 8'.
            These are restricted to the top-level of an expression,
            and for reifiable expressions only,

\item[\texttt{or}]
            Reified constraint disjunction.  e.g. \verb'X #> 3 or Y #< 8'.
            These are restricted to the top-level of an expression,
            and for reifiable expressions only,

\item[\texttt{xor}]
            Reified constraint exclusive disjunction.  e.g. \verb'X #> 3 xor Y #< 8'.
            These are restricted to the top-level of an expression,
            and for reifiable expressions only,

\item[\texttt{=>}]
            Reified constraint implication.  e.g. \verb'X #> 3 => Y #< 8'.
            These are restricted to the top-level of an expression,
            and for reifiable expressions only,

\item[\texttt{neg}]
            Reified constraint negation.  e.g. \verb'neg X #> 3'
            These are restricted to the top-level of an expression,
            and for reifiable expressions only,

\item[\texttt{<=>}]
            Reified constraint equivalence.  e.g. \verb'X #> 3 <=> Y #< 8'.
            This is similar to {\texttt \#=} used in an expression context.
            These are restricted to the top-level of an expression,
            and for reifiable expressions only,

\item[
    \texttt{\#>}, \texttt{\#>=}, \texttt{\#=}, \texttt{\#=<},
 \texttt{\#<},
    \texttt{\#\bsl=}]

    Posted as a constraint, both the left- and right- hand arguments are
    expressions.

    Within the expression context, the constraint evaluates to its
    reified truth value.  If the constraint is entailed by the
    state of the constraint store then the (sub-)expression
    evaluates to \verb|1|.  If it is dis-entailed by the state of
    the constraint store then it evaluates to \verb|0|. If its
    reified status is unknown then it evaluates to an integral
    variable \verb|0..1|.

    Note: The simple cases (e.g.\ \verb|Bool #= (X #> 5)|) are
    equivalent to directly calling the reified forms of the basic
    constraints (e.g.\ \verb|#>(X, 5, Bool)|).

\item[\texttt{eval(Expr)}]
            Logically equivalent to \verb'Expr'.
            Should be used when \verb'Expr' is a compile-time variable
            which may get instantiated to an expression (rather than an
            integer) at runtime.

\item[\texttt{Functional/reified constraints}]
            Reified constraints (whose last argument is a 0/1 variable)
            and functional constraints (whose last argument is an integer
            variable) can be written without their last argument within
            an expression context.  The expression then effectively
            evaluates to the value of the missing (unwritten) argument.

\end{description}
 
The expressions allowed by GFD are a super-set of the expressions supported by 
Gecode (briefly, Gecode does not support functional and 
reified constraints in expressions, and collections of expressions are not 
supported). When an expression is posted, it is parsed and broken down into 
expressions and/or logical connectives supported by Gecode (more 
specifically, Gecode's MiniModel's {\tt IntRel} and {\tt BoolExpr}, along 
with any constraints). This is done to 
allow the user greater freedom in the code they write, and also to provide 
better compatibility with IC. 

Note that posting of complex expressions is relatively expensive: they are 
first parsed at the {\eclipse} level by GFD to extract the sub-expressions and 
any new domain variables, and these sub-expressions (in the form of 
{\eclipse} structures) are then parsed again at the GFD C++ level to convert
them to the appropriate Gecode data structures, which are then passed to
Gecode. Gecode itself will then convert these data structures
to the basic constraints that it supports. 


%%% Not sure which point we are trying to make here:
%As mentioned above, one of the reason for parsing the expressions is to
%extract new domain variables. This is another difference between GFD and
%Gecode: Gecode requires the user to explicitly initialise domain variables 
%(IntVar) with a domain before using them, while GFD will give any new 
%domain variables a default domain, so variables do not need to be 
%initialised  with a domain before use. This GFD behaviour is compatible with
%IC (except the default domain is a finite integer domain like FD).



\subsubsection{Arithmetic constraints}

These constraints impose some form of arithmetic relation between their
arguments. Some of these constraints can occur inside expressions, while
others are ``primitive'' versions of the constraint where the arguments
are domain variables (or integers).

%%% Not sure which point we are trying to make here:
%Many of the constraints listed here are only available in IC inside
%expressions, i.e. they are not available as independent constraints.
%In GFD, all ``operators'' allowed in expressions have a constraint
%counterpart that can be posted outside of expressions, partly because
%Gecode does provide these constraints.


\begin{description}
%%% Commented out until we sort out the name
%\item[\biptxtrefni{divmod(?X,?Y,?Q,?M)}{divmod/4!gfd}{../bips/lib/gfd/divmod-4.html}]
%Constrains Q to X // Y, and M to X rem Y.

\item[\biptxtrefni{all_eq(?Collection,?Y)}{all_eq/2!gfd}{../bips/lib/gfd/all_eq-2.html}]
Constrains each element of Collection to be equal to Y. Similar constraints
for the other relations: 
\biptxtrefni{all_ge/2)}{all_ge/2!gfd}{../bips/lib/gfd/all_ge-2.html} (greater than or equal to), 
\biptxtrefni{all_gt/2}{all_gt/2!gfd}{../bips/lib/gfd/all_gt-2.html} (greater than),
\biptxtrefni{all_le/2}{all_le/2!gfd}{../bips/lib/gfd/all_le-2.html} (less than or equal to),
\biptxtrefni{all_lt/2}{all_lt/2!gfd}{../bips/lib/gfd/all_lt-2.html} (less than), and
\biptxtrefni{all_ne/2}{all_ne/2!gfd}{../bips/lib/gfd/all_ne-2.html} (not equal).


\item[\biptxtrefni{max(+Collection,?Max)}{max/2!gfd}{../bips/lib/gfd/max-2.html}]
Constrains Max to be the maximum of the values in Collection. Similarly,
\biptxtrefni{min(+Collection,?Min)}{min/2!gfd}{../bips/lib/gfd/min-2.html}
for minimum.

%%% does not exist (would clash with arithmetic builtin)
%\item[\biptxtrefni{max(?X,?Y,?Max)}{max/3!gfd}{../bips/lib/gfd/max-3.html}]
%Constrains Max to be the maximum of X and Y. Similarly,
%\biptxtrefni{min(?X,?Y,?Min)}{min/3!gfd}{../bips/lib/gfd/min-3.html} for
% minimum.

\item[\biptxtrefni{mem(+Vars,?Member [,?Bool])}{mem/2!gfd}{../bips/lib/gfd/mem-2.html}]
Constrains Member to be the a member element in Vars. The 
\biptxtrefni{reified}{mem/3!gfd}{../bips/lib/gfd/mem-3.html} version has the
\verb'Bool' argument.

\item[\biptxtrefni{scalar_product(++Coeffs,+Collection,+Rel,?Sum [,?Bool])}{scalar_product/4!gfd}{../bips/lib/gfd/scalar_product-4.html}]
Constrains the scalar product of the elements of Coeffs
 and Collection to satisfy the relation {\em sum(Coeffs*Collection) Rel P}. 
\biptxtrefni{Reified}{scalar_product/5!gfd}{../bips/lib/gfd/scalar_product-5.html}
with \verb'Bool' argument.

\item[\biptxtrefni{sum(+Collection,?Sum)}{sum/2!gfd}{../bips/lib/gfd/sum-2.html}]
Constrains Sum to be the sum of the elements in Collection, or if the
argument is of the form IntCollection*Collection, the scalar product of the
commections.

\item[\biptxtrefni{sum(+Collection,+Rel,?Sum [,?Bool]}{sum/3!gfd}{../bips/lib/gfd/sum-3.html}]
Constrains the sum of the elements of Collection to satisfy the relation {\em sum(Collection) Rel Sum}.
\biptxtrefni{Reified}{sum/4!gfd}{../bips/lib/gfd/sum-4.html}
with \verb'Bool' argument.


\end{description}

\subsubsection{Ordering constraints}

These constraints impose some form of ordering relation on their arguments.

\begin{description}
\item[\biptxtrefni{lex_eq(+Collection1,+Collection2)}{lex_eq/2!gfd}{../bips/lib/gfd/lex_eq-2.html}]
Constrains Collection1 to be lexicographically equal to Collection2. 
Constraints for the other lexicographic relations:
\biptxtrefni{lex_ge/2}{lex_ge/2!gfd}{../bips/lib/gfd/lex_ge-2.html} (lexicographically greater or equal to),
\biptxtrefni{lex_gt/2}{lex_gt/2!gfd}{../bips/lib/gfd/lex_gt-2.html} (lexicographically greater than),
\biptxtrefni{lex_le/2}{lex_le/2!gfd}{../bips/lib/gfd/lex_le-2.html}
(lexicographically less or equal to), Collection2.
\biptxtrefni{lex_lt/2}{lex_lt/2!gfd}{../bips/lib/gfd/lex_lt-2.html},
(lexicographically less than),
\biptxtrefni{lex_neq/2}{lex_neq/2!gfd}{../bips/lib/gfd/lex_neq-2.html}
(lexicographically not equal to).

\item[\biptxtrefni{ordered(+Relation,+Collection)}{ordered/2!gfd}{../bips/lib/gfd/ordered-2.html}]
Constrains Collection to be ordered according to Relation.

\item[\biptxtrefni{precede(++Values,+Collection)}{precede/2!gfd}{../bips/lib/gfd/precede-2.html}]
Constrains each value in Values to precede its succeeding
value in Collection.

\item[\biptxtrefni{precede(+S,+T,+Collection)}{precede/3!gfd}{../bips/lib/gfd/precede-3.html}]
Constrains S to precede T in Collection.

\item[\biptxtrefni{sorted(?Unsorted, ?Sorted)}{sorted/2!gfd}{../bips/lib/gfd/sorted-2.html}]
Sorted is a sorted permutation of Unsorted.

\item[\biptxtrefni{sorted(?Unsorted, ?Sorted, ?Positions)}{sorted/3!gfd}{../bips/lib/gfd/sorted-3.html}]
Sorted is a sorted permutation (described by Positions) of Unsorted.

\end{description}

\subsubsection{Counting and data constraints}

These constraints impose restrictions either on the number of
 values that can be taken in one or more collections of domain
 variables, and/or on the positions of values in the collection.
\begin{description}
\item[\biptxtrefni{alldifferent(+Vars)}{alldifferent/1!gfd}{../bips/lib/gfd/alldifferent-1.html}]
Constrains all elements of Vars are different.

\item[\biptxtrefni{alldifferent_cst(+Vars,++Offsets)}{alldifferent_cst/2!gfd}{../bips/lib/gfd/alldifferent_cst-2.html}]
Constrains the values of each element plus corresponding offset to be pairwise different.

\item[\biptxtrefni{among(+Values, ?Vars, +Rel, ?N)}{among/4!gfd}{../bips/lib/gfd/among-4.html}]
The number of occurrences ({\em Occ}) in Vars of values taken from the set of
values specified in Values satisfies the relation {\em Occ Rel N}.

\item[\biptxtrefni{atleast(?N, +Vars, +V)}{atleast/3!gfd}{../bips/lib/gfd/atleast-3.html}]
At least N elements of Vars have the value V. Similarly 
\biptxtrefni{atmost(?N, +Vars, +V)}{atmost/3!gfd}{../bips/lib/gfd/atmost-3.html}.

\item[\biptxtrefni{count(+Value, ?Vars, +Rel, ?N)}{count/4!gfd}{../bips/lib/gfd/count-4.html}]
Constrains the number of occurrences of Value in Vars ({\em Occ}) to satisfy
the relation {\em Occ Rel N}.

\item[\biptxtrefni{count_matches(+Values, ?Vars, +Rel, ?N)}{count_matches/4!gfd}{../bips/lib/gfd/count_matches-4.html}]
The number of the elements in Vars that
 match their corresponding value in Values, {\em Matches}, satisfies the
 relation {\em Matches Rel N}.

\item[\biptxtrefni{element(?Index, +Collection, ?Value)}{element/3!gfd}{../bips/lib/gfd/element-3.html}]
Constrains Value to be the Index$^{th}$ element of the integer collection Collection.
 
\item[\biptxtrefni{gcc(+Bounds,+Vars)}{gcc/2!gfd}{../bips/lib/gfd/gcc-2.html}]
Constrains the number of occurrences of each Value in Vars according to the specification
in Bounds (global cardinality constraint).

\item[\biptxtrefni{nvalues(+Collection, +Rel, ?Limit)}{nvalues/3!gfd}{../bips/lib/gfd/nvalues-3.html}]
Constrains {\em N}, the number of distinct values occurring in 
Collection to satisfy the relation {\em N Rel Limit}.

\item[\biptxtrefni{occurrences(+Value,+Vars,?N)}{occurrences/3!gfd}{../bips/lib/gfd/occurrences-3.html}]
Constrains the value Value to occur N times in Vars.

\item[\biptxtrefni{sequence(+Low,+High,+K,+Vars,++Values)}{sequence/5!gfd}{../bips/lib/gfd/sequence-5.html}]
The number of values taken from Values is between Low and
High for all sequences of K variables in Vars. There is also a version for
binary (0/1) variables:
\biptxtrefni{sequence(+Low,+High,+K,+ZeroOnes)}{sequence/4!gfd}{../bips/lib/gfd/sequence-4.html}.
\end{description}

\subsubsection{Resource and scheduling constraints}
These constraints deal with scheduling and/or allocation of resources.

\begin{description}
\item[\biptxtrefni{bin_packing(+Items,++ItemSizes,+BinLoads)}{bin_packing/3!gfd}{../bips/lib/gfd/bin_packing-3.html}]
The one-dimensional bin packing constraint with loads: packing 
M items into N bins, each bin having a load specified in BinLoads.

\item[\biptxtrefni{bin_packing(+Items,++ItemSizes,+N,+BinSize)}{bin_packing/4!gfd}{../bips/lib/gfd/bin_packing-4.html}]
The one-dimensional bin packing constraint: packing M items
into N bins of size BinSize.

\item[\biptxtrefni{cumulative(+Starts,+Durations,+Usages,+ResourceLimit)}{cumulative/4!gfd}{../bips/lib/gfd/cumulative-4.html}]
Single-resource cumulative task scheduling constraint. A version with
optional tasks is also available:
\biptxtrefni{cumulative_optional(+StartTimes, +Durations, +Usages, +ResourceLimit, +Scheduled)}{cumulative_optional/5!gfd}{../bips/lib/gfd/cumulative_optional-5.html}.

\item[\biptxtrefni{cumulatives(+Starts,+Durations,+Heights,+Assigned,+Capacities)}{cumulatives/5!gfd}{../bips/lib/gfd/cumulatives-5.html}]
Multi-resource cumulatives constraint on specified tasks.

\item[\biptxtrefni{cumulatives_min(+Starts,+Durs,+Heights,+Assgn,+Mins)}{cumulatives_min/5!gfd}{../bips/lib/gfd/cumulatives_min-5.html}]
Multi-resource cumulatives constraint on specified tasks with
required minimum resource consumptions.

\item[\biptxtrefni{disjoint2(+Rectangles)}{disjoint2/1!gfd}{../bips/lib/gfd/disjoint2-1.html}]
Constrains the position (and possibly size) of the rectangles in
Rectangles so that none overlap. A version where placement of rectangles is 
optional is
\biptxtrefni{disjoint2_optional(+Rectangles)}{disjoint2_optional/1!gfd}{../bips/lib/gfd/disjoint2_optional-1.html}.

\item[\biptxtrefni{disjunctive(+StartTimes, +Durations)}{disjunctive/2!gfd}{../bips/lib/gfd/disjunctive-2.html}]
Constrains the tasks with specified start times and durations to not overlap in time. A version with optional tasks is also available:
\biptxtrefni{disjunctive_optional(+StartTimes, +Durations, +Scheduled)}{disjunctive_optional/3!gfd}{../bips/lib/gfd/disjunctive_optional-3.html}.

\end{description}

\subsubsection{Graph constraints}

In these constraints, the arguments represent a graph, and the
 constraint imposes some form of relation on the graph.

\begin{description}
\item[\biptxtrefni{circuit(+Succ)}{circuit/1!gfd}{../bips/lib/gfd/circuit-1.html}]
Constrains elements in Succ to form a Hamiltonian circuit.
A version allowing constant offsets is
\biptxtrefni{circuit_offset(+Succ,+Offset)}{circuit_offset/2!gfd}{../bips/lib/gfd/circuit_offset-2.html}.

\item[\biptxtrefni{circuit(+Succ,++CostMatrix,?Cost)}{circuit/3!gfd}{../bips/lib/gfd/circuit-3.html}]
Constrains elements in Succ to form a Hamiltonian circuit, with Cost
being the cost of the circuit, based on the edge cost matrix CostMatrix.
A version allowing constant offsets is
\biptxtrefni{circuit_offset(+Succ,+Offset,++CostMatrix,?Cost)}{circuit_offset/4!gfd}{../bips/lib/gfd/circuit_offset-4.html}.

\item[\biptxtrefni{circuit(+Succ,++CostMatrix,+ArcCosts,?Cost)}{circuit/4!gfd}{../bips/lib/gfd/circuit-4.html}]
Constrains elements in Succ to form a Hamiltonian circuit. ArcCosts
are the costs of the individual hops, and Cost their sum,
based on the edge cost matrix CostMatrix.
A version with constant offsets is available as
\biptxtrefni{circuit_offset(+Succ,+Offset,++CostMatrix,+ArcCosts,?Cost)}{circuit_offset/5!gfd}{../bips/lib/gfd/circuit_offset-5.html},

\item[\biptxtrefni{ham_path(?Start,?End,+Succ)}{ham_path/3!gfd}{../bips/lib/gfd/ham_path-3.html}]
Constrains elements in Succ to form a Hamiltonian path from Start to End.
A version with constant offsets is available as
\biptxtrefni{ham_path_offset(?Start,?End,+Succ,+Offset)}{ham_path_offset/4!gfd}{../bips/lib/gfd/ham_path_offset-4.html}.

\item[\biptxtrefni{ham_path(?Start,?End,+Succ,++CostMatrix,?Cost)}{ham_path/5!gfd}{../bips/lib/gfd/ham_path-5.html}]
Constrains elements in Succ to form a Hamiltonian path from Start to End,
with Cost being the cost of the path, based on the edge cost matrix CostMatrix.
A version with constant offsets is available as
\biptxtrefni{ham_path_offset(?Start, ?End, +Succ, +Offset, ++CostMatrix, ?Cost)}{ham_path_offset/6!gfd}{../bips/lib/gfd/ham_path_offset-6.html}.

\item[\biptxtrefni{ham_path(?Start,?End,+Succ,++CostMatrix,+ArcCosts,?Cost)}{ham_path/6!gfd}{../bips/lib/gfd/ham_path-6.html}]
Constrains elements in Succ to form a Hamiltonian path from Start to End.
ArcCosts are the costs of the individual hops, and Cost their sum,
based on the edge cost matrix CostMatrix.
A version with constant offsets is available as
\biptxtrefni{ham_path_offset(?Start, ?End, +Succ, +Offset, ++CostMatrix, +ArcCosts, ?Cost)}{ham_path_offset/7!gfd}{../bips/lib/gfd/ham_path_offset-7.html}.

\item[\biptxtrefni{inverse(+Succ,+Pred)}{inverse/2!gfd}{../bips/lib/gfd/inverse-2.html}]
Constrains elements of Succ to be the successors and
Pred to be the predecessors of nodes in a digraph. A version with offsets
is also available:
\biptxtrefni{inverse(+Succ,+SuccOffset,+Pred,+PredOffset)}{inverse/4!gfd}{../bips/lib/gfd/inverse-4.html}.

\end{description}
 
\subsubsection{Extensional constraints}
These are ``user defined constraints'' (also known as ad-hoc
 constraints), i.e. the allowable tuples of values for a
collection of domain variables is defined as part of the constraint. These
predicate differs in the way the allowable values are specified.

\begin{description}
\item[\biptxtrefni{regular(+Vars, ++RegExp)}{regular/2!gfd}{../bips/lib/gfd/regular-2.html}]
Constrains Vars' solutions to conform to that defined in the regular expression RegExp.

\item[\biptxtrefni{table(+Vars, ++Table)}{table/2!gfd}{../bips/lib/gfd/table-2.html}]
Constrain Vars' solutions to be those defined by the tuples in Table.
The variant
\biptxtrefni{table(+Vars, ++Table, +Option)}{table/3!gfd}{../bips/lib/gfd/table-3.html}
allows the specification of the algorithm used.

\item[\biptxtrefni{extensional(+Vars, ++Transitions, +Start, +Finals)}{extensional/4!gfd}{../bips/lib/gfd/extensional-4.html}]
Constrain Vars' solutions to conform to the finite-state 
automaton specified by Transitions with start state Start and  final states Finals.

\end{description}

\subsubsection{Other constraints}

Constraints that don't fit into the other categories.

\begin{description}

\item[\biptxtrefni{bool_channeling(?Var, +DomainBools, +Min)}{bool_channeling/3!gfd}{../bips/lib/gfd/bool_channeling-3.html}]
Channel the domain values of Vars to the 0/1 boolean variables in DomainBools.

\item[\biptxtrefni{integers(+Vars)}{integers/1!gfd}{../bips/lib/gfd/integers-1.html}]
Pseudo constraint (i.e. no constraint will be posted in Gecode):
Vars' domain is the integer numbers (within default bounds).

\end{description}

%----------------------------------------------------------------------
\subsection{Search Support}
%----------------------------------------------------------------------

GFD allows search to be performed in two ways: 
completely encapsulated in the external Gecode solver, or
in {\eclipse}, supported by GFD's variable selection 
and value choice predicates. 

\subsubsection{Performing search completely inside Gecode}
\label{searcheng}
Search can be performed in Gecode using one of its search engines. 
In this 
case, the search to produce a solution appears as an atomic step at
the {\eclipse} level, and backtracking into the search will produce the next 
solution (or fail if there are none), again as an atomic step.

This direct interface to Gecode's search engines is provided by
\biptxtrefni{gfd:search/6}{search/6!gfd}{../bips/lib/gfd/search-6.html},
and uses a syntax similar to that of the generic search/6 predicates
(in {\tt lib(gfd_search)} (see below), {\tt lib(ic)} and {\tt lib(fd_search)}). 

As the search is performed in Gecode, it should be more efficient than doing
the search in \eclipse, where the system has to repeatedly switch between
Gecode and {\eclipse} when doing the search. As the search is a single atomic
step from the {\eclipse} level, it is not suitable if your code needs to
interact with the search, e.g. if you are using constraints defined at the
{\eclipse} level, and/or if you are using other solvers during the search.

On the other hand, GFD's search/6 is less flexible than the generic search
-- you can only use the predefined variable 
selection and value choice methods, i.e. you cannot provide user-defined
predicates for the Select and Choice arguments. 

The search engine to use is specified by the Method argument in search/6. 
One method provided by Gecode is bb_min -- finding a minimal solution using
branch-and-bound, which is not provided by the generic search. 

Instead, branch-and-bound in {\eclipse} is provided by {\tt lib(branch_and_bound)}, which can
be used with generic search's {\tt search/6} to provide a similar functionality as
the {\tt bb_min} method of GFD's {\tt search/6}. The {\eclipse} branch-and-bound is more 
flexible, but is likely to be slower. Note that {\tt lib(branch_and_bound)} can
be used in combination with GFD's search/6, but this is probably not useful
unless you are doing some search in your own code in addition to that done by 
search/6, or if you want to see the non-optimal solutions generated by the
search.

There are some differences in how search is performed by Gecode and generic
search;
the most significant is that all the built-in choice-operators of the generic
search library make repeated choices on one variable until it becomes ground,
before proceeding and selecting the next variable.  Gecode's built-in
strategies on the other hand always interleave value choices with variable
selection.

Here is the N-Queens example using gfd's \texttt{search/6}:
\begin{quote}
\begin{verbatim}
:- lib(gfd).

queens_list(N, Board) :-
    length(Board, N),
    Board :: 1..N,
    (fromto(Board, [Q1|Cols], Cols, []) do
        ( foreach(Q2, Cols), param(Q1), count(Dist,1,_) do
            Q2 #\= Q1,
            Q2 - Q1 #\= Dist,
            Q1 - Q2 #\= Dist
        )
    ),
    search(Board, 0, input_order, indomain_min, complete, []).
\end{verbatim}
\end{quote}


\subsubsection{Search in {\eclipse} using GFD primitives}
\label{searchgfd}
The built-in Gecode search is appropriate when the problem consists
exclusively of GFD-variables and GFD-library-constraints, and when the
built-in search methods and search heuristics are sufficient to solve
the problem.

As soon as any user-defined constraints or any other {\eclipse}
solvers are involved, then the top-level search control should be
written in \eclipse, in order to allow non-gfd propagators to execute
between the labelling steps.  Also the implementation of problem-specific
search heuristics will usually make it necessary to lift the top-level
search control to the {\eclipse} level.
To make this possible, GFD provides primitives to support variable 
selection and value choice heuristics.

\begin{description}
\item[\biptxtrefni{gfd:select_var(-X, +Collection, -Rest, ++Arg, ++Select)}{select_var/5!gfd}{../bips/lib/gfd/select_var-5.html}]
Select a domain variable from Collection according to one of Gecode's
pre-defined selection criteria.  These include criteria not available in
other {\eclipse} solvers, like accumulated failure count.

\item[\biptxtrefni{gfd_search:delete(-X, +Collection, -Rest, ++Arg, ++Select)}{delete/5!gfd}{../bips/lib/gfd_search/delete-5.html}]
Select (and remove) a domain variable from Collection.  This is the
generic implementation, (compatible with IC and FD solvers), providing
a different choice of selection options, but likely to be less
efficient than select_var/5.

\item[\biptxtrefni{gfd:try_value(?Var, ++Method)}{try_value/2!gfd}{../bips/lib/gfd/try_value-2.html}]
This value choice predicate supports both Gecode-style binary choice and 
generic search's multi-way choice on the domain of a variable,
according to Method.
The binary-choice methods create two search alternatives, which reduce the variable domain
in complementary ways.  Because the variable is not necessarily instantiated,
this must be combined with a variable selection method that does not delete
the selected variable, such as select_var/5.

The multi-way choice methods make repeated choices on one variable as 
gfd_search's indomain/2.

\item[\biptxtrefni{gfd:indomain(?Var)}{indomain/1!gfd}{../bips/lib/gfd/indomain-1.html}]
Instantiate Var to elements in its domain, using a default method.

\item[\biptxtrefni{gfd_search:indomain(?Var, ++Method)}{indomain/2!gfd}{../bips/lib/gfd_search/indomain-2.html}]
A flexible way to nondeterministically assign values to finite domain
variables according to Method.  On success, Var is always instantiated.
This is the generic implementation,
(compatible with IC and FD solvers), providing a different choice of methods,
and likely to be less efficient than try_value/5.
\end{description}

A simple search using GFD's primitives can be defined in the following way: 
\begin{quote}
\begin{verbatim}
labeling(Vars, Select, Choice) :-
        ( select_var(V, Vars, Rest, 0, Select) ->
            try_value(V, Choice),
            labeling(Rest, Select, Choice)
        ;
            true
        ).
\end{verbatim}
\end{quote}
For binary choice methods of try_value/2, the search will 
mimic Gecode's built-in search (where a variable selection step is usually
interleaved with a binary choice on the variable domain).

For multi-way choice methods of try_value/2,  
(possibly several) value choices on a variable are made until the
variable is ground, before proceeding to select the next variable: On
backtracking to the try_value/2, alternative values for the variable
will be tried. This mimics the behaviour of gfd_search's indomain/2,
but try_value/2 is likely to be more efficient as it is specifically
tailored for GFD.

The same effect can be achieved by using select_var/5 and try_value/2
together with the generic
\biptxtrefni{gfd_search:search/6}{search/6!gfd_search}{../bips/lib/gfd_search/search-6.html}
predicate:
\begin{quote}
\begin{verbatim}
gfd_search:search(Vars, 0,
        select_var(Select), try_value(Choice), complete, [])
\end{verbatim}
\end{quote}
For even more complex user-defined heuristics, various properties associated
with a variable and its domain can be obtained using predicates described
in section~\ref{gfdvarquery}. Note that these include properties that are not
available in
\eclipse's solvers, such as weighted degree (a.k.a. accumulated failure count).



\section{User defined constraints and solver co-operation}
Like IC and FD solvers, GFD has facilities to allow the extension of the 
solver library so that GFD can co-operate with other solvers in solving a
problem, and also to allow the user to define their own constraints at the {\eclipse}
level. This is achieved by providing a suspension list with the gfd attribute,
which allows for the data-driven programming needed by solver co-operation and
constraint propagation, and a set of low-level predicates to process,
 query and  modify the domain of problem variables.

These facilities allow solver co-operation and user-defined 
constraints propagation at the {\eclipse} level, and not within Gecode directly.
So, search then must be done at the {\eclipse} level, i.e. not through Gecode's
search engines. The performance of constraints defined in this way will very
likely be less efficient than implementing the constraints directly in Gecode.

\subsection{The {\it gfd\/} attribute}

The GFD attribute is a meta-term which is attached to all GFD problem variables.
Many of the fields of the attribute are used for implementing the interface to
Gecode, and are of no interest to the user. The only field of interest is the
{\tt any} field, which is for the {\it any\/} suspension list, which is woken on 
any change in the domain of the variable:

\begin{verbatim}
gfd{
   ....
   any:SuspAny,
   ....
}
\end{verbatim}

The {\it any\/} suspension list has the same waking behaviour as the 
{\it any\/} suspension
list of FD, and is sufficient for implementing constraints -- the other 
suspension lists found in IC and FD are specialisations of the {\it any\/} 
suspension list, in that they provide more precise waking conditions. 
The reason that
only one suspension list is provided by GFD is to minimise the overhead in
normal use of the solver. 


In addition to waking the attribute's {\it any\/} suspension list, the 
{\it constrained\/}
suspension list will also be woken when a GFD variable's domain is changed,
and the {\it inst\/} suspension list will be woken if the variable is bound.

The suspension lists allow constraint propagation to be implemented at the
{\eclipse} level, which is distinct from the propagation of ``native'' Gecode
constraints, where each propagation phase (and in the case of using the 
search engine, the whole search) is an atomic step at the {\eclipse} level. 
This has a similar effect to running all ``native'' propagations
at a higher (more urgent) priority.
 
As only the {\it any\/} suspension list is provided, some rewriting of existing
user-defined constraints for IC and FD may be needed when such code is ported
for GFD.

\subsection{Modifying variable domains}

Like IC, GFD provides a set of predicates to modify the domain of GFD 
variables to support the writing of new constraints. Unlike normal constraints,
no Gecode level propagation or waking of other suspended goals (such as 
scheduled by other {\eclipse} level constraints) occurs with these predicates.

With the exception of
\bipref{impose_bounds/3}{../bips/lib/gfd/impose_bounds-3.html} none of
the goals call \bipref{wake/0}{../bips/kernel/suspensions/wake-0.html}, so
the programmer is free to do so at a convenient time.

Some of these predicates are provided for compatibility with IC, as these 
predicates have the same name and similar semantics to their IC counter-parts
(including the waking behaviour for {\tt impose_bounds/3}).
However, due to the difference in the way domains are represented in IC and
Gecode, these predicates may be inefficient for use with Gecode, particularly
if you need to modify multiple variables and/or multiple domain values. 
The predicates with names that begin with {\tt gfd_vars} are specific
to GFD and are designed to be more efficient than their IC compatible 
counter-parts.

The ``native'' primitives are:

\begin{description}
\item[\biptxtrefni{gfd_vars_exclude(+Vars,+Excl)}{gfd_vars_exclude/2!gfd}{../bips/lib/gfd/gfd_vars_exclude-2.html}]
Exclude the element Excl from the domains of Vars.

\item[\biptxtrefni{gfd_vars_exclude_domain(+Vars, ++Domain)}{gfd_vars_exclude_domain/2!gfd}{../bips/lib/gfd/gfd_vars_exclude_domain-2.html}]
Exclude the values specified in Domain from the domains of Vars.

\item[\biptxtrefni{gfd_vars_exclude_range(+Vars, +Lo, +Hi)}{gfd_vars_exclude_range/3!gfd}{../bips/lib/gfd/gfd_vars_exclude_range-3.html}]
Exclude the elements Lo..Hi from the domains of Vars.

\item[\biptxtrefni{gfd_vars_impose_bounds(+Vars, +Lo, +Hi)}{gfd_vars_impose_bounds/3!gfd}{../bips/lib/gfd/gfd_vars_impose_bounds-3.html}]
Update (if required) the bounds of Vars.

\item[\biptxtrefni{gfd_vars_impose_domain(+Vars,++Domain)}{gfd_vars_impose_domain/2!gfd}{../bips/lib/gfd/gfd_vars_impose_domain-2.html}]
Restrict (if required) the domain of Var to the domain specified  in Domain.

\item[\biptxtrefni{gfd_vars_impose_max(+Vars,+Bound)}{gfd_vars_impose_max/2!gfd}{../bips/lib/gfd/gfd_vars_impose_max-2.html}]
Update (if required) the upper bounds of Vars.

\item[\biptxtrefni{gfd_vars_impose_min(+Vars,+Bound)}{gfd_vars_impose_min/2!gfd}{../bips/lib/gfd/gfd_vars_impose_min-2.html}]
Update (if required) the lower bounds of Vars.

\end{description}

The IC-compatible primitives are:

\begin{description}
\item[\biptxtrefni{exclude(?Var, +Excl)}{exclude/2!gfd}{../bips/lib/gfd/exclude-2.html}]
Exclude the element Excl from the domain of Var.

\item[\biptxtrefni{exclude_range(?Var, +Lo, +Hi)}{exclude_range/3!gfd}{../bips/lib/gfd/exclude_range-3.html}]
Exclude the elements Lo..Hi from the domain of Var.

\item[\biptxtrefni{impose_bounds(?Var,+Lo,+Hi)}{impose_bounds/3!gfd}{../bips/lib/gfd/impose_bounds-3.html}]
Update (if required) the bounds of Var.

\item[\biptxtrefni{impose_domain(?Var,++Domain)}{impose_domain/2!gfd}{../bips/lib/gfd/impose_domain-2.html}]
Restrict (if required) the domain of Var to the domain of DomVar.

\item[\biptxtrefni{impose_max(?Var, +Hi)}{impose_max/2!gfd}{../bips/lib/gfd/impose_max-2.html}]
Update (if required) the upper bound of Var.

\item[\biptxtrefni{impose_min(?Var, +Lo)}{impose_min/2!gfd}{../bips/lib/gfd/impose_min-2.html}]
Update (if required) the lower bound of Var.

\end{description}


\subsection{Variable query predicates}
\label{gfdvarquery}

These predicates are used to retrieve various properties of a domain variable 
(and usually work on integers as well). 

In most cases, the property is obtained directly from Gecode. Many of these
properties are useful for selecting a variable for labelling. Here are some
examples:

\begin{description}
\item[\biptxtrefni{get_bounds(?Var, -Lo, -Hi)}{get_bounds/3!gfd}{../bips/lib/gfd/get_bounds-3.html}]
Retrieves the current bounds of Var.

\item[\biptxtrefni{get_constraints_number(?Var, -Number)}{get_constraints_number/2!gfd}{../bips/lib/gfd/get_constraints_number-2.html}]
Returns the number of propagators attached to the Gecode variable representing Var.

\item[\biptxtrefni{get_delta(?Var, -Width)}{get_delta/2!gfd}{../bips/lib/gfd/get_delta-2.html}]
Returns the width of the interval of Var.

\item[\biptxtrefni{get_domain(?Var, -Domain)}{get_domain/2!gfd}{../bips/lib/gfd/get_domain-2.html}]
Returns a ground representation of the current GFD domain of a variable.

\item[\biptxtrefni{get_domain_size(?Var, -Size)}{get_domain_size/2!gfd}{../bips/lib/gfd/get_domain_size-2.html}]
Returns the number of elements in the GFD domain of Var.

\item[\biptxtrefni{get_max(?Var,-Hi)}{get_max/2!gfd}{../bips/lib/gfd/get_max-2.html}]
Retrieves the current upper bound of Var. Similarly, \biptxtrefni{get_min/2)}{get_min/2!gfd}{../bips/lib/gfd/get_min-2.html} returns the lower bound.


\item[\biptxtrefni{get_median(?Var,-Median)}{get_median/2!gfd}{../bips/lib/gfd/get_median-2.html}]
Returns the median domain value of the GFD domain variable Var.

\item[\biptxtrefni{get_regret_lwb(?Var, -Regret)}{get_regret_lwb/2!gfd}{../bips/lib/gfd/get_regret_lwb-2.html}]
Returns the regret value for the lower bound of Var. Similarly, \biptxtrefni{get_regret_upb/2}{get_regret_upb/2!gfd}{../bips/lib/gfd/get_regret_upb-2.html}
for the upper bound.

\item[\biptxtrefni{get_weighted_degree(?Var,-WD)}{get_weighted_degree/2!gfd}{../bips/lib/gfd/get_weighted_degree-2.html}]
Returns the weighted degree (wdeg, accumulated failure count) of domain 
variable Var.

\item[\biptxtrefni{is_in_domain(+Val,?Var,[-Result])}{is_in_domain/2!gfd}{../bips/lib/gfd/is_in_domain-2.html}]
Succeeds iff Val is in the domain of Var. The \biptxtrefni{version}{is_in_domain/3!gfd}{../bips/lib/gfd/is_in_domain-3.html} with the \verb'+Result' argument
binds Result instead.


\item[\biptxtrefni{is_solver_var(?Term)}{is_solver_var/1!gfd}{../bips/lib/gfd/is_solver_var-1.html}]
Succeeds iff Term is an GFD domain variable.
\end{description}

\section{Low-level control of Gecode computation}
This section gives some information on the low-level workings of GFD and how
to adjust it. This information is not needed for most users, and can be 
skipped and consulted only when GFD is not behaving well with the user's
program.

GFD is designed so that the user can write programs that will run with
 Gecode without knowing any details about Gecode or how GFD interfaces
 to it. However, GFD does provide some parameters to control its behaviour.
While the default settings should work well under most circumstances, some
understanding of the inner workings of GFD is needed to change this default
behaviour. 

\subsection{Recomputation and Cloning}


Gecode implements search using a recomputation and cloning model,
which is fundamentally different from the backtracking model of \eclipse.
When failure occurs, Gecode does not backtrack to a previous computation
state; instead, the previous state is recomputed. To reduce the amount
of recomputation, the computation state is {\it cloned\/} periodically
during execution, and the recomputation would start from the nearest
such cloned state rather than from the start.

With GFD, when the search is done in Gecode (via GFD's 
\biptxtrefni{search/6}{seach/6!gfd}{../bips/lib/gfd/search-6.html}), the
recomputation and cloning is handled by Gecode. When the search is done
at the {\eclipse} level, GFD handles the recomputation and cloning automatically,
so that the user does not need to be aware of it.

GFD will create clones of the Gecode state periodically, and when the
current Gecode computation state becomes invalid through {\eclipse} backtracking,
GFD will recompute the new state from the closest cloned state. 
The frequency at which GFD create clones can be adjusted by the user --
the more frequently clones are created, the more memory will be used, but
the cost of recomputation will be less. Cloning itself will take time,
and depends on the size of the state. Normally, the cost of cloning is
quite low, so GFD by default will clone frequently during search, as 
this leads to faster execution times. However, if the program has a large
state (many variables and constraints), then frequent cloning may lead to
excessive memory consumption (and larger computation state will also be 
more expensive to clone), and reducing the frequency of cloning may
improve performance.

The frequency of cloning in GFD is controlled by the
\texttt{cloning_distance} parameter, which can be changed by
\bipref{gfd_set_default/2}{../bips/lib/gfd/gfd_set_default-2.html}
(and the current value can be accessed with
\bipref{gfd_get_default/2}{../bips/lib/gfd/gfd_get_default-2.html})
The \texttt{cloning_distance} specifies a threshold for the number of
changes GFD makes to the Gecode state before a clone is created.
Note that GFD does not create a clone simply when cloning_distance is
exceeded, as it only creates a new clone when the cloned state would
be the state just before a choice-point, so clones will normally only
be created during search phase of the user program.

While it is not necessary for the user to specify the creation of a clone,
under very unusual circumstances -- when the \texttt{cloning_distance} is
set very high -- GFD may not produce a clone at the right place. So
\bip{gfd_update/0} is provided to force the creation of a clone.
The expected usage is to call this predicate just before search starts.
For example, \texttt{labeling/3} can be written as:

\begin{quote}
\begin{verbatim}
labeling(Vs, Select, Choice) :-
    gfd_update,
    labeling1(Vs, Select, Choice, _).

labeling1(Vs, Select, Choice, VsH) :-
    ( select_var(V, Vs, 0, Select, VsH) ->
         indomain(V, Choice),
         labeling1(Vs, Select, Choice, VsH)
    ;
         true
    ).

\end{verbatim}
\end{quote}
Calling \texttt{gfd_update} before calling \texttt{labeling1} ensures that
GFD will only recompute inside the search.

\section{Main differences between GFD and IC}

Although GFD was designed to be code compatible with IC in terms of
syntax, there are still some unavoidable differences, because of differences 
between Gecode and IC. In addition, Gecode is 
implemented using very different implementation techniques from IC, and 
although such differences are mostly not visible in terms of syntax, there
are still semantics and performance implications. 

The main visible differences between GFD and IC are:
\begin{itemize}
       \item Real interval arithmetic and variables are not supported in GFD.

       \item Domain variables always have finite integer bounds, and the maximum 
       bounds are
       determined by Gecode. Like FD, default finite bounds are given to 
       domain variables that do not have explicit bounds, and the default
       settings for these bounds are below the maxima that Gecode allows.

       \item Constraint propagation is performed within Gecode, and each propagation
       phase is atomic at the {\eclipse} level. Posting of constraints and 
       propagation of their consequences are separate in Gecode. GFD uses a
       demon suspended goal to perform the propagation: after the posting
       of any constraint (and other changes to the problem that need
       propagation), this suspended goal is scheduled and woken. When the
       woken goal is executed, propagation is performed. 

       \item All constraints can be called from the gfd module, and in
       addition, some constraints can be called from modules that specify
       the consistency level: gfd_gac (generalised arc consistency, also
       known as domain consistency), gfd_bc (bounds consistency), gfd_vc (value
       consistency (naive)). The gfd module versions use the 
       default consistency 
       defined for the constraint by Gecode. These consistency levels map
       directly to those defined in Gecode for the constraints.

       \item gfd:search/6 interfaces to Gecode's search-engines, where the
       entire search is performed in Gecode, and the whole search appears
       atomic at the {\eclipse} level. 

       \item The suspension lists supported by GFD are different from IC.
       Currently, only the 'any' suspension list (for {\em any} changes to the
       variable's domain) found in FD but not IC, is supported. Note that
       the GFD constraints are implemented in Gecode directly, and therefore
       do not use GFD's suspension lists. 

      \item Constraint expressions are designed to be compatible with 
      IC's, and the arithmetic operators and logical connectives supported 
      by Gecode are supported, and these largely overlaps those of IC's.
      In addition, ``functional'' (where the last argument is a domain 
      variable) and reified constraints can appear in expressions without the
      last argument, as in IC.

      The differences from IC are:
      \begin{itemize}
        \item User defined constraints are not allowed in expressions. 
 
        \item The operators and connectives supported are those supported by
        Gecode, so most of the IC operators for real arithmetic are not 
        supported.

        %%% Is this correct?
        \item Only linear arithmetic (sub-)expressions are allowed between 
        logical connectives.

        \item GFD expressions are broken down into sub-expressions and 
       constraints that are supported natively by Gecode, where the additional
       sub-expressions are replaced by a domain variable in the original 
       expression. These domain variables are given the default bounds. 
      IC does something similar, but what constitutes additional sub-expressions
      will differ between GFD and IC, and the variables substituted
      for the sub-expressions would be given infinite bounds in IC.
 
      \end{itemize}

\item GFD variables cannot be copied to non-logical storage, and an error is 
raised if a GFD variable occurs in a term that is being copied for such purpose
(assert, non-logical variables, shelves, etc.). Note that this means that 
GFD is incompatible with Propia, as this library makes use of non-logical 
storage.
\end{itemize}

%HEVEA\cutend
