% BEGIN LICENSE BLOCK
% Version: CMPL 1.1
%
% The contents of this file are subject to the Cisco-style Mozilla Public
% License Version 1.1 (the "License"); you may not use this file except
% in compliance with the License.  You may obtain a copy of the License
% at www.eclipse-clp.org/license.
% 
% Software distributed under the License is distributed on an "AS IS"
% basis, WITHOUT WARRANTY OF ANY KIND, either express or implied.  See
% the License for the specific language governing rights and limitations
% under the License. 
% 
% The Original Code is  The ECLiPSe Constraint Logic Programming System. 
% The Initial Developer of the Original Code is  Cisco Systems, Inc. 
% Portions created by the Initial Developer are
% Copyright (C) 2012 Cisco Systems, Inc.  All Rights Reserved.
% 
% Contributor(s): Kish Shen
% 
% END LICENSE BLOCK

%HEVEA\cutdef[1]{section}

\section{Introduction}

   The GFD library is an interface for \eclipse to Gecode's finite domain constraint
   solver. Gecode ({\tt www.gecode.org}) is an open-source toolkit for 
   developing
   constraint-based systems in C++, and includes an  integer 
   finite domain constraint solver.

   This interface provides a high degree of code compatibility with the finite 
   domain portion of the IC library, and to a lesser extent, with the FD
   library as well. This means that programs originally written for the
   IC library should run with GFD with little modifications, beyond 
   renaming any explicit calls to the ic family of modules.

\subsection{Main differences between GFD and IC}

Although GFD was designed to be highly code compatible with IC in terms of
syntax, there are still some unavoidable differences, because of differences 
between Gecode and IC. In addition, Gecode is 
implemented using very different implementation techniques from IC, and 
although such differences are mostly not visible in terms of syntax, there
are still semantics and performance implications. 

The main visible differences between GFD and IC are:
\begin{itemize}
       \item Real interval arithmetic and variables are not supported in GFD.

       \item Domain variables have finite integer bounds, and the maximum 
       bounds are
       determined by Gecode. Like FD, default finite bounds are given to 
       domain variables that are not explicitly given bounds, and the default
       settings for these bounds are much smaller than what Gecode allows.

       \item Constraint propagation is performed in Gecode, and each propagation
       phase is atomic at the \eclipse level. Posting of constraints and 
       propagation of their consequences are separate in Gecode. GFD use a
       demon suspended goal to perform the propagation: after the posting
       of any constraint (and other changes to the problem that needs 
       propagation), the suspended goal is scheduled and woken. When the
       woken goal is executed, propagation is perform. The goal is suspended
       at priority 10, so if the posting of the constraint is executed at
       normal priority (12), the propagation will happen immediately. However,
       if the posting is done at a priority 10 or higher, then the propagation
       is delayed, allowing multiple constraints to be posted without
       propagation. 

       \item GFD supports constraints that are supported by Gecode,
       so the exact set of constraints supported is different from IC's. 
       However, the same basic arithmetic operators and relations are 
       supported, allowing for very similar arithmetic expressions. Also,
       many, if not most, of the constraints in IC are supported, along with
       some constraints not supported by IC. See the detailed documentation
       for more information.

       Similarly, the domain of a GFD variable is maintained inside Gecode,
       and provides predicates to access the properties of the variable that
       Gecode provides, which is a super-set of those provided by IC (and some 
       properties, such as the median of a domain, is defined differently from 
       IC's). 

       \item All constraints can be called from the gfd module, and in
       addition, some constraints can be called from modules that specify
       the consistency level: gfd_gac (generalised arc consistency, aka
       domain consistency), gfd_bc (bounds consistency), gfd_vc (value
       consistency (naive)). The calls to gfd module uses the default consistency 
       defined for the constraint by Gecode. These consistency levels maps 
       directly to those defined in Gecode for the constraints, so if Gecode supports 
       different consistency levels for a constraint, GFD supports it as 
       well. In  particular (and unlike IC), most arithmetic operations can 
       be bounds (the default) as well as domain consistent.

       \item gfd:search/6 interfaces to Gecode's search-engines, where the
       entire search is performed in Gecode, and the whole search appears
       atomic at the \eclipse level. The syntax for search/6 is designed to
       be as compatible with IC's search/6, although there are some 
       differences. The exact equivalent to IC's search/6, where the search
       is performed at the \eclipse level, is available via the gfd_search
       module (in fact it shares the same code with IC's search/6). This
       provides more flexibility, but is likely to be less efficient,
       because the search is done in \eclipse and is therefore not
       tightly integrated with gecode, and also because the 
       gfd_search is not optimised for use with Gecode. In addition,
       gfd also provide predicates for both variable selection and
       value choice that are optimised for gecode, which should be more
       efficient than those provided by gfd_search.


       \item The suspension lists supported by GFD are different from IC.
       Currently, only the 'any' suspension list (for any changes to the
       variable's domain) found in FD but not IC, is supported. Note that
       the GFD constraints are implemented in Gecode directly, and therefore
       do not use GFD's suspension lists. 

       The GFD suspension lists is provided to allow the user to write their 
       own  constraints for use with GFD at the \eclipse level, and also to 
       allow for co-operation of GFD with other solvers. However, because the 
       suspension lists are different from those of IC, code written in IC 
       that uses suspension lists will need some rewriting for GFD. 

      \item Constraint expressions are designed to be highly compatible with 
      IC's, and the arithmetic operators and logical connectives supported 
      by Gecode are supported, and these largely overlaps those of IC's.
      In addition, ``functional'' (where the last argument is a domain 
      variable) and reified constraints can appear in expressions without the
      last argument, as in IC.

      The differences from IC are:
      \begin{itemize}
        \item User defined constraints are not allowed in expressions. 
        Constraints that are allowed to appear are specifically recognised
        by GFD when the expression is parsed, and for details on which
        constraints are allowed in expressions, please refer to the individual 
        reference documentation for each constraint.
 
        \item The operators and connectives supported are those supported by
        Gecode, so most of the IC operators for real arithmetic are not 
        supported (and for those that are, like square-root, only integral 
        values are supported). In addition, only the square operator is 
        supported, and for compatibility, the \verb|^/2| operator is 
        allowed if the power is 2 (i.e. Expr\verb|^2|). 
 
        For logical connectives, \verb|xor/2| (exclusive or), not defined
        in IC, is supported, and \verb|<=>| (logical equivalence) is 
        available as an alternative to using IC's \verb'#=/2' in a reified 
        position (which is also supported by GFD).
   
        \item Only linear arithmetic (sub-)expressions are allowed between 
        logical connectives.

        \item GFD expressions are parsed at run-time when the expression is 
      posted, so IC's eval/1 is
      not needed. However, for compatibility reasons, it is allowed in
      arithmetic expressions, but does nothing. 
        \item GFD expressions are broken down into sub-expressions and 
       constraints that are supported natively by Gecode, where the additional
       sub-expressions are replaced by a domain variable in the original 
       expression. These domain variables are given the default bounds. 
      IC does something similar, but what constitute additional sub-expressions
      will of course differ between GFD and IC, and the substitute variables 
      for the sub-expressions would be given infinite bounds in IC.
 
      \end{itemize}

\item GFD variables cannot be copied for non-logical storage, and an error is 
raised if a GFD variable occurs in a term that is being copied for such purpose
(assert, non-logical variables, shelves, etc.). Note that this means that 
GFD is incompatible with Propia, as this library make use of non-logical 
storage.
\end{itemize}


GFD allows search to be performed at the \eclipse level, or inside Gecode. 
Search performed in Gecode has some difference with search performed in
\eclipse, the most significant is that variable selection and value choice is 
performed for 
{\it each\/} labelling step, and when the search backtracks to this step, 
variable selection is performed again (as well as value selection), 
so a different variable may be labelled. In contrast, in \eclipse, variable
selection is performed separately from value choice -- a variable is first
selected, and then different values are tried for the same variable on 
backtracking. In a backtracking system like \eclipse, it is 
difficult (and probably expensive) to perform the variable selection and 
value choice together; but with a recomputation system, since the state is
recomputed with each alternative labelling, there is no extra cost associated
with doing the variable selection together with the value choice. This 
difference means that, except for {\tt input_ordering} variable selection,
the search-space explored by search in Gecode and 
\eclipse could be different for the same heuristics, even without consider 
details of the heuristics. 

\section{Problem Modelling and Solving}

GFD provides facilities to model and solve problems over the 
finite integer domain, with Gecode as the solver. It supports the constraints
provided by Gecode. The search to solve the problem can be done at the 
\eclipse level (with support from Gecode for variable and value selections), 
or the whole search can be performed by Gecode itself using one of its 
search engines.   

In addition to the code-level differences, Gecode is implemented using a
{\it re-computation based\/} model, unlike the {\it backtracking\/}
 model of \eclipse. This has implications on many aspects of Gecode,
 and is discussed in more detail in section~\ref{gfdimpl}. 

\subsection{Usage}

To load the GFD library into your program, simply add the following directive
at an appropriate point in your code.

\begin{quote}
\begin{verbatim}
:- lib(gfd).
\end{verbatim}
\end{quote}

\subsection{Integer domain variables}

An (integer) domain variable is a variable which can be instantiated only to a
value from a given finite set of integer values. 

A variable becomes a domain variable when it first appear in a (GFD) 
constraint. If the constraint is a domain constraint, then the variable will
be given the domain specified by the constraint. Otherwise, the variable will
be given a default domain, which should be more than large enough for
most problem instances. 

The default domain is an interval, and the maximum and minimum values of this
interval can be changed using \bipref{gfd_set_default/2}{../bips/lib/gfd/gfd_set_default-2.html} (with the options 
{\tt interval_max} and {\tt interval_min} for the maximum and minimum values,
respectively. These values must be within the interval allowed by Gecode
(\bip{gfd_maxint/1} and \bip{gfd_minint/1}). You can also obtain the current values
of interval_max and interval_min using \bip{gfd_get_default/2}.

The Gecode documentation suggests that domain variables should be given as
small a domain as possible, and requires the user to specify the domain for 
all domain variables. This is not required by GFD, but following Gecode's
convention, it is recommended that the user use domain constraints to specify 
the domain for variables if possible before using them.

A domain variable is mapped into a Gecode \verb'IntVar'.

\subsubsection{Variable query predicates}
\label{gfdvarquery}

These predicates are used to retrieve various properties of a domain variable 
(and usually work on integers as well). 

In most cases, the property is obtained directly from Gecode. Many of these
properties are useful for selecting a variable for labeling.

\begin{description}
\item[\biptxtrefni{get_bounds(?Var, -Lo, -Hi)}{get_bounds/3!gfd}{../bips/lib/gfd/get_bounds-3.html}]
Retrieve the current bounds of Var.

\item[\biptxtrefni{get_constraints_number(?Var, -Number)}{get_constraints_number/2!gfd}{../bips/lib/gfd/get_constraints_number-2.html}]
Returns the number of propagators attached to the gecode variable representing Var.

\item[\biptxtrefni{get_delta(?Var, -Width)}{get_delta/2!gfd}{../bips/lib/gfd/get_delta-2.html}]
Returns the width of the interval of Var.

\item[\biptxtrefni{get_domain(?Var, -Domain)}{get_domain/2!gfd}{../bips/lib/gfd/get_domain-2.html}]
Returns a ground representation of the current GFD domain of a variable.

\item[\biptxtrefni{get_domain_as_list(?Var, -Domain)}{get_domain_as_list/2!gfd}{../bips/lib/gfd/get_domain_as_list-2.html}]
Returns a ground representation of the current GFD domain of a variable.

\item[\biptxtrefni{get_domain_size(?Var, -Size)}{get_domain_size/2!gfd}{../bips/lib/gfd/get_domain_size-2.html}]
Returns the number of elements in the GFD domain of Var.

\item[\biptxtrefni{get_finite_integer_bounds(?Var, -Lo, -Hi)}{get_finite_integer_bounds/3!gfd}{../bips/lib/gfd/get_finite_integer_bounds-3.html}]
Retrieve the current bounds of Var.

\item[\biptxtrefni{get_integer_bounds(?Var, -Lo, -Hi)}{get_integer_bounds/3!gfd}{../bips/lib/gfd/get_integer_bounds-3.html}]
Retrieve the current bounds of Var.

\item[\biptxtrefni{get_max(?Var,-Hi)}{get_max/2!gfd}{../bips/lib/gfd/get_max-2.html}]
Retrieve the current upper bound of Var.


\item[\biptxtrefni{get_median(?Var,-Median)}{get_median/2!gfd}{../bips/lib/gfd/get_median-2.html}]
Returns the median domain value of the GFD domain variable Var.

\item[\biptxtrefni{get_min(?Var,-Lo)}{get_min/2!gfd}{../bips/lib/gfd/get_min-2.html}]
Retrieve the current lower bound of Var.

\item[\biptxtrefni{get_regret_lwb(?Var, -Regret)}{get_regret_lwb/2!gfd}{../bips/lib/gfd/get_regret_lwb-2.html}]
Returns the regret value for the lower bound of Var.

\item[\biptxtrefni{get_regret_upb(?Var, -Regret)}{get_regret_upb/2!gfd}{../bips/lib/gfd/get_regret_upb-2.html}]
Returns the regret value for the upper bound of Var.

\item[\biptxtrefni{get_weighted_degree(?Var,-WD)}{get_weighted_degree/2!gfd}{../bips/lib/gfd/get_weighted_degree-2.html}]
Returns the weighted degree of domain variable Var.

\item[\biptxtrefni{is_exact_solver_var(?Term)}{is_exact_solver_var/1!gfd}{../bips/lib/gfd/is_exact_solver_var-1.html}]
Succeeds iff Term is an GFD domain variable.

\item[\biptxtrefni{is_in_domain(+Val,?Var)}{is_in_domain/2!gfd}{../bips/lib/gfd/is_in_domain-2.html}]
Succeeds iff Val is in the domain of Var.

\item[\biptxtrefni{is_in_domain(+Val, ?Var, -Result)}{is_in_domain/3!gfd}{../bips/lib/gfd/is_in_domain-3.html}]
Binds Result to indicate presence of Val in domain of Var.

\item[\biptxtrefni{is_solver_type(?Term)}{is_solver_type/1!gfd}{../bips/lib/gfd/is_solver_type-1.html}]
Succeeds iff Term is a GFD domain variable or an integer.

\item[\biptxtrefni{is_solver_var(?Term)}{is_solver_var/1!gfd}{../bips/lib/gfd/is_solver_var-1.html}]
Succeeds iff Term is an GFD domain variable.

\end{description}

\subsection{Constraints}

GFD supports the (integer finite domain) constraints implemented by Gecode.
Some of these constraints are also implemented in \eclipse's native finite 
domain solvers (IC and FD), but many are not. For those that are 
implemented in IC and/or FD, the same name and syntax is used by GFD 
(although the details may differ, such as the types allowed for arguments).  
One difference is that all these constraints are defined in the GFD
library itself, and can thus be called without module qualification or
loading additional libraries. 

However, the propagation consistency level for the constraint can be 
specified using a module-like syntax when posting the constraint:

\begin{verbatim}

<Propgation level>: <Constraint>

\end{verbatim}

where propagation level is one of:

\begin{description}
\item[{\tt gfd_gac}] Domain consistent (Generalised Arc-consistent). Map to {\tt ICL_DOM} in Gecode.
\item[{\tt gfd_bc}] Bound consistent. Map to {\tt ICL_BND} in Gecode.
\item[{\tt gfd_vc}] Value consistent. Map to {\tt ICL_VAL} in Gecode.
\end{description}

Although the propagation level is specified as a module, they do not need to
be loaded separately -- these modules are created when GFD is loaded.

Posting to a particular consistency level for a constraint is supported only
if that consistency level is implemented for that constraint
in Gecode. See the individual documentation for the constraints for details.
Note that posting the constraint unqualified (or qualified with {\tt gfd}) 
means posting the constraint at the default consistency level ({\tt ICL_DEF}). 

Note that posting of constraints with expressions at different 
consistency level is supported by GFD. However, it is possible that some of
the constraints and sub-expressions inside the expressions are not supported
at the given consistency level. In such cases, these sub-expressions will be
posted at the default consistency level.

Gecode requires an explicit command to perform propagation. In GFD,
the call to Gecode to perform propagation is implemented as a delayed
 goal at priority 10.
When a constraint is posted, GFD first add any new domain variables in
 the constraint to Gecode, and then add the constraint to Gecode,
 without performing any explicit propagation. The call to propagate is
then scheduled for execution. It is thus possible to post multiple
 constraints without propagation by posting the constraints at a more
urgent (i.e. numerically smaller) priority than 10. This could reduce
 the cost of performing the propagation.
 
Several constraints involve the use of indices. In \eclipse, indices starts
from 1, while Gecode, like C++, indices starts from 0. For compatibility 
with \eclipse, ``normal'' GFD constraints also have indices
that starts from 1. These constraints are mapped to the Gecode native 
indices in various ways, depending of the constraint, with the aim of 
minimising the overhead. GFD also support versions of these constraints that 
uses Gecode's native indices, i.e. starting from 0, and these have an 
additional {\tt _g} in their name (e.g. {\tt bin_packing_g/3} is the Gecode native
indices version of {\tt bin_packing/3}). These version of the constraint do not
have the overhead of converting the index value, but may be
incompatible with the rest of \eclipse.

\subsubsection{Domain constraints}

The domain constraints supported by GFD are:

\begin{description}
\item[\biptxtrefni{?Vars \#:: ++Domain}{\#::/2!gfd}{../bips/lib/gfd/HNN-2.html}]
Constrain Vars to have the domain Domain.
\item[\biptxtrefni{\#::(?Var, ++Domain, ?Bool)}{\#::/3!gfd}{../bips/lib/gfd/HNN-3.html}]
Reified {\tt \#::/2}.
\item[\biptxtrefni{?Vars :: ++Domain}{::/2!gfd}{../bips/lib/gfd/NN-2.html}]
Alias for {\tt \#::/2}.
\item[\biptxtrefni{::(?Var, ++Domain, ?Bool)}{::/3!gfd}{../bips/lib/gfd/NN-3.html}]
Alias for {\tt \#::/3}.

\end{description}

  
\subsubsection{Arithmetic and logical expressions}

Integer expressions can occur in relational and logical connective
constraints. These are:

\begin{description}
\item[\biptxtrefni{?ExprX \#< ?ExprY}{\#</2!gfd}{../bips/lib/gfd/HL-2.html}]
ExprX is less than ExprY.

\item[\biptxtrefni{\#<(?ExprX, ?ExprY, ?Bool)}{\#</3!gfd}{../bips/lib/gfd/HL-3.html}]
Reified ExprX is less than ExprY.

\item[\biptxtrefni{?ExprX \#= ?ExprY}{\#=/2!gfd}{../bips/lib/gfd/HE-2.html}]
ExprX is equal to ExprY.

\item[\biptxtrefni{\#=(?ExprX, ?ExprY, ?Bool)}{\#=/3!gfd}{../bips/lib/gfd/HE-3.html}]
Reified ExprX is equal to ExprY.

\item[\biptxtrefni{{?ExprX \#=< ?ExprY}}{\#=</2!gfd}{../bips/lib/gfd/HEL-2.html}]
ExprX is less than or equal to ExprY.

\item[\biptxtrefni{\#=<(?ExprX, ?ExprY, ?Bool)}{\#=</3!gfd}{../bips/lib/gfd/HEL-3.html}]
Reified ExprX is less than or equal to ExprY.

\item[\biptxtrefni{?ExprX \#> ?ExprY}{\#>/2!gfd}{../bips/lib/gfd/HG-2.html}]
ExprX is strictly greater than ExprY.

\item[\biptxtrefni{\#>(?ExprX, ?ExprY, ?Bool)}{\#>/3!gfd}{../bips/lib/gfd/HG-3.html}]
Reified ExprX is strictly greater than ExprY.

\item[\biptxtrefni{?ExprX \#>= ?ExprY}{\#>=/2!gfd}{../bips/lib/gfd/HGE-2.html}]
ExprX is greater than or equal to ExprY.

\item[\biptxtrefni{\#>=(?ExprX, ?ExprY, ?Bool)}{\#>=/3!gfd}{../bips/lib/gfd/HGE-3.html}]
Reified ExprX is greater than or equal to ExprY.

\item[\biptxtrefni{?ExprX \#\bsl= ?ExprY}{\#\\=/2!gfd}{../bips/lib/gfd/HRE-2.html}]
ExprX is not equal to ExprY.

\item[\biptxtrefni{\#\bsl=(?ExprX, ?ExprY, ?Bool)}{\#\\=/3!gfd}{../bips/lib/gfd/HRE-3.html}]
Reified ExprX is not equal to ExprY.

\item[\biptxtrefni{+ConX <=> +ConY}{<=>/2!gfd}{../bips/lib/gfd/LEG-2.html}]
Constraint ConX has the equivalent truth value as ConY.

\item[\biptxtrefni{<=>(+ConX,+ConY,?Bool)}{<=>/3!gfd}{../bips/lib/gfd/LEG-3.html}]
Bool is the reified truth of constraint ConX is equivalent to the truth of ConY.

\item[\biptxtrefni{+ConX => +ConY}{=>/2!gfd}{../bips/lib/gfd/EG-2.html}]
Constraint ConX  implies ConY.

\item[\biptxtrefni{=>(+ConX,+ConY,?Bool)}{=>/3!gfd}{../bips/lib/gfd/EG-3.html}]
Bool is the reified truth of constraint ConX implying the truth of ConY.

\item[\biptxtrefni{+ConX and +ConY}{and/2!gfd}{../bips/lib/gfd/and-2.html}]
Constraints ConX and ConY must both be true.

\item[\biptxtrefni{and(+ConX,+ConY,?Bool)}{and/3!gfd}{../bips/lib/gfd/and-3.html}]
Bool is the reified truth of both constraints ConX and ConY being true.

\item[\biptxtrefni{neg(+Con)}{neg/1!gfd}{../bips/lib/gfd/neg-1.html}]
Constraints Con is negated.

\item[\biptxtrefni{neg(+Con,?Bool)}{neg/2!gfd}{../bips/lib/gfd/neg-2.html}]
Bool is the logical negation of the reified truth constraints Con.

\item[\biptxtrefni{+ConX or +ConY}{or/2!gfd}{../bips/lib/gfd/or-2.html}]
At least one of the constraints ConX or ConY must be true.

\item[\biptxtrefni{or(+ConX,+ConY,?Bool)}{or/3!gfd}{../bips/lib/gfd/or-3.html}]
Bool is the reified truth of at least one of the constraints ConX or ConY being true.

\item[\biptxtrefni{+ConX xor +ConY}{xor/2!gfd}{../bips/lib/gfd/xor-2.html}]
Exactly one of the constraints ConX or ConY must be true.

\item[\biptxtrefni{xor(+ConX,+ConY,?Bool)}{xor/3!gfd}{../bips/lib/gfd/xor-3.html}]
Bool is the reified truth of exactly one of the constraints ConX or ConY being
 true.

\end{description}

The syntax for the expressions closely follows that in IC. The 
following can be used inside expressions:

\begin{description}
\item[\texttt{X}]
	    \emph{Variables}.  If \verb'X' is not yet a domain variable, it is turned 
	    into one.

   \item[\texttt{123}]
	    Integer constants.

   \item[\texttt{+Expr}]
	    Identity.

   \item[\texttt{-Expr}]
	    Sign change.

\item[\texttt{abs(Expr)}]
    The absolute value of Expr.

\item[\texttt{E1+E2}]
    Addition.

\item[\texttt{E1-E2}]
    Subtraction.

\item[\texttt{E1*E2}]
    Multiplication.

\item[\texttt{E1//E2}]
    Integer division. Truncate towards zero.

\item[\texttt{E1/E2}]
    Integer division (alias for compatibility).

\item[\texttt{E1 mod E2}]
	    Integer modulus.

\item[\texttt{divmod(E1,E2,E3)}]
	    Integer modulus, in addition {\texttt E3 \#= E1//E2}.

\item[\texttt{moddiv(E1,E2,E3)}]
	    Integer division, in addition {\texttt E3 \#= E1 mod E2}.

\item[\texttt{Expr}\textasciicircum{}{\texttt 2}]
	    Square. Equivalent to {\texttt sqr(Expr)} (alias for compatibility).


\item[\texttt{min(E1,E2)}]
    Minimum.

\item[\texttt{max(E1,E2)}]
    Maximum.

\item[\texttt{sqr(Expr)}]
    Square. Logically equivalent to \verb|Expr*Expr|.

\item[\texttt{sqrt(Expr)}]
	    Square root (always positive). Truncated towards zero.

\item[\texttt{sum(ExprCol)}]
	    Sum of a collection of expressions.

\item[\texttt{sum(IntCol*ExprCol)}]
	    Scalar product of a collection of integers and expressions.
            \verb'IntCol' and \verb'ExprCol' must be the same size.

\item[\texttt{min(ExprCol)}]
	    Minimum of a collection of expressions.

\item[\texttt{max(ExprCol)}]
	    Maximum of a collection of expressions.

\item[\texttt{and}]
	    Reified constraint conjunction.  e.g. \verb'X #> 3 and Y #< 8'.
            These are restricted to the top-level of an expression,
            and for reifiable expressions only,

\item[\texttt{or}]
	    Reified constraint disjunction.  e.g. \verb'X #> 3 or Y #< 8'.
            These are restricted to the top-level of an expression,
            and for reifiable expressions only,

\item[\texttt{xor}]
	    Reified constraint exclusive disjunction.  e.g. \verb'X #> 3 xor Y #< 8'.
            These are restricted to the top-level of an expression,
            and for reifiable expressions only,

\item[\texttt{=>}]
	    Reified constraint implication.  e.g. \verb'X #> 3 => Y #< 8'.
            These are restricted to the top-level of an expression,
            and for reifiable expressions only,

\item[\texttt{neg}]
	    Reified constraint negation.  e.g. \verb'neg X #> 3'
            These are restricted to the top-level of an expression,
            and for reifiable expressions only,

\item[\texttt{<=>}]
	    Reified constraint equivalence.  e.g. \verb'X #> 3 <=> Y #< 8'.
            This is an alias for {\texttt \#=} used in an expression context.
            These are restricted to the top-level of an expression,
            and for reifiable expressions only,

\item[
    \texttt{\#>}, \texttt{\#>=}, \texttt{\#=}, \texttt{\#=<},
 \texttt{\#<},
    \texttt{\#\bsl=}]

    Posted as a constraint, both the left- and right- hand arguments are
    expressions.

    Within the expression context, the constraint evaluates to its
    reified truth value.  If the constraint is entailed by the
    state of the constraint store then the (sub-)expression
    evaluates to \verb|1|.  If it is dis-entailed by the state of
    the constraint store then it evaluates to \verb|0|. If its
    reified status is unknown then it evaluates to an integral
    variable \verb|0..1|.

    Note: The simple cases (e.g.\ \verb|Bool #= (X #> 5)|) are
    equivalent to directly calling the reified forms of the basic
    constraints (e.g.\ \verb|#>(X, 5, Bool)|).

\item[\texttt{eval(Expr)}]
	    Equivalent to \verb'Expr'.

\item[\texttt{Functional/reified constraints}]
            Written without last argument, and within an expression context.

            The constraint evaluates to the value of the missing argument.
            Only reified constraints (whose (unwritten) last argument
            is a 0/1 Boolean representing the truth value of the constraint) 
            and constraints that can be written as 
            functions ((unwritten) last argument is a domain variable) are 
            allowed.

\end{description}
 
The expressions allowed by GFD is a super-set of the expressions supported by 
Gecode (briefly, Gecode does not support functional and 
reified constraints in expressions, and collections of expressions are not 
supported). When an expression is posted, it is parsed and broken down into 
expressions and/or logical connectives supported by Gecode (more 
specifically, Gecode's MiniModel's {\tt IntRel} and {\tt BoolExpr}, along 
with any constraints). This is done to 
allow the user greater freedom in the code they write, and also to provide 
better compatibility with IC. 

Note that posting of complex expressions is relatively expensive: they are 
first parsed at the \eclipse level by GFD to extract the sub-expressions and 
any new domain variables, and these sub-expressions (in the form of 
\eclipse structures) are then parsed again at the GFD C++ level to convert
them to the appropriate Gecode data structures, which are then passed to
Gecode. Gecode itself will probably parse these structures again
and convert them to the basic constraints that it supports. 

In addition,
unlike IC, the parsing of expressions is done entirely at run-time, when the
expression is posted. Thus, the cost of posting expressions, especially 
complex expressions, is likely to be higher than in IC. However, in normal
circumstances, especially for problems appropriate for constraint solving,  
the cost of posting the expressions is much less than the 
cost of solving the problem (through propagation and search), so the impact
on the overall performance in solving should normally be insignificant.

As mentioned above, one of the reason for parsing the expressions is to
extract new domain variables. This is another difference between GFD and
Gecode: Gecode requires the user to explicitly initialise domain variables 
(IntVar) with a domain before using them, while GFD will give any new 
domain variables a default domain, so variables do not need to be 
initialised  with a domain before use. This GFD behaviour is compatible with
IC (except the default domain is a finite integer domain like FD).



\subsubsection{Arithmetic constraints}

These constraints imposes some form of arithmetic relation between its
arguments. Some of these constraints can occur inside expressions, while
others are ``primitive'' versions of the constraint where the arguments
are domain variables (or integers), and are provided to  allow more efficient
posting of the simple arithmetic relation than using expressions (e.g.
{\tt lt(X,Y)} is more efficient to post than {\tt X \#< Y}).

Many of the constraints listed here are only available in IC inside
expressions, i.e. they are not available as independent constraints.
In GFD, all ``operator'' allowed in expressions have a constraint
counterpart that can be posted outside of expressions, partly because
Gecode does provide these constraints.


\begin{description}
\item[\biptxtrefni{abs(?X,?Y)}{abs/2!gfd}{../bips/lib/gfd/abs-2.html}]
Constrains Y to be the absolute value of X.

\item[\biptxtrefni{divide(?X,?Y,?Z)}{divide/3!gfd}{../bips/lib/gfd/divide-3.html}]
Constrains Z to X // Y (integer division, round towards 0).

\item[\biptxtrefni{divmod(?X,?Y,?Q,?M)}{divmod/4!gfd}{../bips/lib/gfd/divmod-4.html}]
Constrains Q to X // Y, and M to X mod Y.

\item[\biptxtrefni{eq(?X,?Y)}{eq/2!gfd}{../bips/lib/gfd/eq-2.html}]
Constrains X to be equal to Y.

\item[\biptxtrefni{eq(?Collection,?Y)}{eq/2!gfd}{../bips/lib/gfd/eq-2.html}]
Constrains each element of Collection to be equal to Y.

\item[\biptxtrefni{ge(?X,?Y)}{ge/2!gfd}{../bips/lib/gfd/ge-2.html}]
Constrains X to be greater than or equal to Y.

\item[\biptxtrefni{ge(?Collection,?Y)}{ge/2!gfd}{../bips/lib/gfd/ge-2.html}]
Constrains each element of Collection to be greater than or equal to Y.

\item[\biptxtrefni{gt(?X,?Y)}{gt/2!gfd}{../bips/lib/gfd/gt-2.html}]
Constrains X to be greater than Y.

\item[\biptxtrefni{gt(?Collection,?Y)}{gt/2!gfd}{../bips/lib/gfd/gt-2.html}]
Constrains each element of Collection to be greater than Y.

\item[\biptxtrefni{le(?X,?Y)}{le/2!gfd}{../bips/lib/gfd/le-2.html}]
Constrains X to be less than or equal to Y.

\item[\biptxtrefni{le(?Collection,?Y)}{le/2!gfd}{../bips/lib/gfd/le-2.html}]
Constrains each element of Collection to be less than or equal to Y.

\item[\biptxtrefni{lt(?X,?Y)}{lt/2!gfd}{../bips/lib/gfd/lt-2.html}]
Constrains X to be less than Y.

\item[\biptxtrefni{lt(?Collection,?Y)}{lt/2!gfd}{../bips/lib/gfd/lt-2.html}]
Constrains each element of Collection to be less than Y.

\item[\biptxtrefni{max(+Collection,?Max}{max/2!gfd}{../bips/lib/gfd/max-2.html}]
Alias for {\tt maxlist/2}: Constrains Max to be the maximum of the values in Collection.

\item[\biptxtrefni{max(?X,?Y,?Max)}{max/3!gfd}{../bips/lib/gfd/max-3.html}]
Constrains Max to be the maximum of X and Y.

\item[\biptxtrefni{maxlist(+Collection,?Max)}{maxlist/2!gfd}{../bips/lib/gfd/maxlist-2.html}]
Constrains Max to be the maximum of the values in Collection.

\item[\biptxtrefni{mem(+Vars,?Member)}{mem/2!gfd}{../bips/lib/gfd/mem-2.html}]
Constrains Member to be the a member element in Vars.

\item[\biptxtrefni{mem(+Vars,?Member,?Bool)}{mem/3!gfd}{../bips/lib/gfd/mem-3.html}]
Reified {\tt mem/2}.

\item[\biptxtrefni{min(+Collection,?Min)}{min/2!gfd}{../bips/lib/gfd/min-2.html}]
Alias for {\tt minlist/2}: Constrains Min to be the minimum of the values in Collection.

\item[\biptxtrefni{min(?X,?Y,?Min)}{min/3!gfd}{../bips/lib/gfd/min-3.html}]
Constrains Min to be the minimum of X and Y.

\item[\biptxtrefni{minlist(+Collection,?Min)}{minlist/2!gfd}{../bips/lib/gfd/minlist-2.html}]
Constrains Min to be the minimum of the values in Collection.

\item[\biptxtrefni{mod(?X,?Y,?Z)}{mod/3!gfd}{../bips/lib/gfd/mod-3.html}]
Constrains Z to X mod Y.

\item[\biptxtrefni{mult(?X,?Y,?Z)}{mult/3!gfd}{../bips/lib/gfd/mult-3.html}]
Constrains Z to X * Y.

\item[\biptxtrefni{ne(?X,?Y)}{ne/2!gfd}{../bips/lib/gfd/ne-2.html}]
Constrains X to be not equal to Y.

\item[\biptxtrefni{ne(?Collection,?Y)}{ne/2!gfd}{../bips/lib/gfd/ne-2.html}]
Constrains each element in Collection to be not equal to Y.

\item[\biptxtrefni{scalar_product(++Coeffs,+Collection,+Rel,?Sum)}{scalar_product/4!gfd}{../bips/lib/gfd/scalar_product-4.html}]
Constrains the scalar product of the elements of Coeffs
 and Collection to satisfy the relation sum(Coeffs*Collection) Rel P.

\item[\biptxtrefni{scalar_product(++Coeffs,+Collection,+Rel,?Sum,?Bool)}{scalar_product/5!gfd}{../bips/lib/gfd/scalar_product-5.html}]
Reified {\tt scalar_product/5}.

\item[\biptxtrefni{sqr(?X,?Y)}{sqr/2!gfd}{../bips/lib/gfd/sqr-2.html}]
Constrains Y to be the square of X (X*X).

\item[\biptxtrefni{sqrt(?X,?Y)}{sqrt/2!gfd}{../bips/lib/gfd/sqrt-2.html}]
Constrains Y to be the (non-negative integer) square root of X.

\item[\biptxtrefni{sum(+Collection,?Sum)}{sum/2!gfd}{../bips/lib/gfd/sum-2.html}]
Alias for {\tt sumlist/2}: Constrains Sum to be the sum (Collection) or scalar product (IntCollection*Collection) of
the Collection elements.

\item[\biptxtrefni{sum(+Collection,+Rel,?Sum)}{sum/3!gfd}{../bips/lib/gfd/sum-3.html}]
Constrains the sum of the elements of Collection to satisfy the relation sum(Collection) Rel Sum.

\item[\biptxtrefni{sum(+Collection,+Rel,?Sum,?Bool)}{sum/4!gfd}{../bips/lib/gfd/sum-4.html}]
Reified {\tt sum/3}.

\item[\biptxtrefni{sumlist(+Collection,?Sum)}{sumlist/2!gfd}{../bips/lib/gfd/sumlist-2.html}]
Constrains Sum to be the sum (Collection) or scalar product (IntCollection*Collection) of
the Collection elements.

\end{description}

\subsubsection{Ordering constraints}

These constraints imposes some form of ordering relation on its arguments.

\begin{description}
\item[\biptxtrefni{lex_eq(+Collection1+Collection2)}{lex_eq/2!gfd}{../bips/lib/gfd/lex_eq-2.html}]
Constrains Collection1 to be lexicographically equal to Collection2

\item[\biptxtrefni{lex_ge(+Collection1+Collection2)}{lex_ge/2!gfd}{../bips/lib/gfd/lex_ge-2.html}]
Constrains Collection1 to be lexicographically greater or equal to Collection2.

\item[\biptxtrefni{lex_gt(+Collection1+Collection2)}{lex_gt/2!gfd}{../bips/lib/gfd/lex_gt-2.html}]
Constrains Collection1 to be lexicographically greater than Collection2.

\item[\biptxtrefni{lex_le(+Collection1+Collection2)}{lex_le/2!gfd}{../bips/lib/gfd/lex_le-2.html}]
Constrains Collection1 to be lexicographically less or equal to Collection2.

\item[\biptxtrefni{lex_lt(+Collection1+Collection2)}{lex_lt/2!gfd}{../bips/lib/gfd/lex_lt-2.html}]
Constrains Collection1 to be lexicographically less than  Collection2.

\item[\biptxtrefni{lex_neq(+Collection1+Collection2)}{lex_neq/2!gfd}{../bips/lib/gfd/lex_neq-2.html}]
Constrains Collection1 to be lexicographically not equal to Collection2.

\item[\biptxtrefni{ordered(+Relation,+Collection)}{ordered/2!gfd}{../bips/lib/gfd/ordered-2.html}]
Constrains Collection to be ordered according to Relation.

\item[\biptxtrefni{precede(++Values,+Collection)}{precede/2!gfd}{../bips/lib/gfd/precede-2.html}]
Constrains each value in Values to precede its succeeding
value in Collection.

\item[\biptxtrefni{precede(+S,+T,+Collection)}{precede/3!gfd}{../bips/lib/gfd/precede-3.html}]
Constrains S to precede T in Collection.

\item[\biptxtrefni{sorted(?Unsorted, ?Sorted)}{sorted/2!gfd}{../bips/lib/gfd/sorted-2.html}]
Sorted is a sorted permutation of Unsorted.

\item[\biptxtrefni{sorted(?Unsorted, ?Sorted, ?Positions)}{sorted/3!gfd}{../bips/lib/gfd/sorted-3.html}]
Sorted is a sorted permutation (described by Positions) of Unsorted.
\item[\biptxtrefni{sorted_g(?Unsorted, ?Sorted, ?Positions)}{sorted_g/3!gfd}{../bips/lib/gfd/sorted_g-3.html}]
Gecode native indexing version of {\tt sorted/3}.

\end{description}

\subsubsection{Counting and data constraints}

These constraints imposes some restrictions either on the number of
 values that can be taken in one or more collection of domain
 variables, and/or on the positions of values in the collection.
\begin{description}
\item[\biptxtrefni{alldifferent(+Vars)}{alldifferent/1!gfd}{../bips/lib/gfd/alldifferent-1.html}]
Constrains all elements of Vars are different.

\item[\biptxtrefni{alldifferent_cst(+Vars,++Offsets)}{alldifferent_cst/2!gfd}{../bips/lib/gfd/alldifferent_cst-2.html}]
Constrains the values of each element plus corresponding offset to be pair-wised different.

\item[\biptxtrefni{among(+Values, ?Vars, +Rel, ?N)}{among/4!gfd}{../bips/lib/gfd/among-4.html}]
The number of occurrence (Occ) in Vars of values taken from the set of
values specified in Values satisfy  the relation Occ Rel N,

\item[\biptxtrefni{atleast(?N, +Vars, +V)}{atleast/3!gfd}{../bips/lib/gfd/atleast-3.html}]
At least N elements of Vars have the value V.

\item[\biptxtrefni{atmost(?N, +Vars, +V)}{atmost/3!gfd}{../bips/lib/gfd/atmost-3.html}]
At most N elements of Vars have the value V.

\item[\biptxtrefni{count(+Value, ?Vars, +Rel, ?N)}{count/4!gfd}{../bips/lib/gfd/count-4.html}]
Constrain the number of occurrence of Value in Vars (Occ) to satisfy
the relation Occ Rel N.

\item[\biptxtrefni{count_matches(+Values, ?Vars, +Rel, ?N)}{count_matches/4!gfd}{../bips/lib/gfd/count_matches-4.html}]
The number of the elements in Vars that
 match its corresponding value in Values, Matches, satisfies the
 relation Matches Rel N.

\item[\biptxtrefni{element(?Index, +Collection, ?Value)}{element/3!gfd}{../bips/lib/gfd/element-3.html}]
Constrains Value to be the Index^{th} element of the integer collection Collection.

\item[\biptxtrefni{element_g(?Index, +Collection, ?Value)}{element_g/3!gfd}{../bips/lib/gfd/element_g-3.html}]
Native Gecode indexing version of {\tt element/3}.
 
\item[\biptxtrefni{gcc(+Bounds,+Vars)}{gcc/2!gfd}{../bips/lib/gfd/gcc-2.html}]
Constrain the cardinality of each Value according to the specification
in Bounds.

\item[\biptxtrefni{nvalues(+Collection, +Rel, ?Limit)}{nvalues/3!gfd}{../bips/lib/gfd/nvalues-3.html}]
Constrains N, the number of distinct values assigned to 
Collection to satisfy the relation N Rel Limit.

\item[\biptxtrefni{occurrences(+Value,+Vars.?N)}{occurrences/3!gfd}{../bips/lib/gfd/occurrences-3.html}]
Constrain the value Value to occur N times in Vars.

\item[\biptxtrefni{sequence(+Low,+High,+K,+ZeroOnes)}{sequence/4!gfd}{../bips/lib/gfd/sequence-4.html}]
The number of occurrences of the value 1 is between Low and
High for all sequences of K variables in ZeroOnes

\item[\biptxtrefni{sequence(+Low,+High,+K,+Vars,++Values)}{sequence/5!gfd}{../bips/lib/gfd/sequence-5.html}]
The number of values taken from Values is between Low and
High for all sequences of K variables in Vars.

\end{description}

\subsubsection{Resource and scheduling constraints}
These constraints deal with scheduling and/or allocation of resources.

\begin{description}
\item[\biptxtrefni{bin_packing(+Items,++ItemSizes,+BinLoads)}{bin_packing/3!gfd}{../bips/lib/gfd/bin_packing-3.html}]
The one-dimensional bin packing constraint with loads: packing 
M items into N bins, each bin having a load specified in BinLoads.

\item[\biptxtrefni{bin_packing(+Items,++ItemSizes,+N,+BinSize)}{bin_packing/4!gfd}{../bips/lib/gfd/bin_packing-4.html}]
The one-dimensional bin packing constraint: packing M items
into N bins of size BinSize.

\item[\biptxtrefni{bin_packing_g(+Items,++ItemSizes,+BinLoads)}{bin_packing_g/3!gfd}{../bips/lib/gfd/bin_packing_g-3.html}]
Native Gecode indexing version of {\tt bin_packing/3}.

\item[\biptxtrefni{cumulative(+StartTimes,+Durations,+Usages,+ResourceLimit)}{cumulative/4!gfd}{../bips/lib/gfd/cumulative-4.html}]
Single resource cumulative constraint on scheduling tasks.

\item[\biptxtrefni{cumulative_optional(+StartTimes, +Durations, +Usages, +ResourceLimit, +Scheduled)}{cumulative_optional/5!gfd}{../bips/lib/gfd/cumulative_optional-5.html}]
Single resource cumulative constraint on scheduling optional tasks.

\item[\biptxtrefni{cumulatives(+StartTimes, +Durations, +Hights, +Assigned, +MachineCapacities)}{cumulatives/5!gfd}{../bips/lib/gfd/cumulatives-5.html}]
Multi-resource cumulatives constraint on specified tasks.

\item[\biptxtrefni{cumulatives_g(+StartTimes, +Durations, +Hights, +Assigned, +MachineCapacities)}{cumulatives_g/5!gfd}{../bips/lib/gfd/cumulatives_g-5.html}]
Native Gecode Indexing version of {\tt cumulatives/5}.

\item[\biptxtrefni{cumulatives_min(+StartTimes, +Durations, +Hights, +Assigned, +MachineConsumptions)}{cumulatives_min/5!gfd}{../bips/lib/gfd/cumulatives_min-5.html}]

Multi-resource cumulatives constraint on specified tasks with
required minimum resource consumptions.

\item[\biptxtrefni{cumulatives_min_g(+StartTimes, +Durations, +Hights, +Assigned, +MachineConsumptions)}{cumulatives_min_g/5!gfd}{../bips/lib/gfd/cumulatives_min_g-5.html}]
Native Gecode Indexing version of cumulatives_min/5.

\item[\biptxtrefni{disjoint2(+Rectangles)}{disjoint2/1!gfd}{../bips/lib/gfd/disjoint2-1.html}]
Constrains the position (and possibly size) of the rectangles in
Rectangles so that none overlaps.

\item[\biptxtrefni{disjoint2_optional(+Rectangles)}{disjoint2_optional/1!gfd}{../bips/lib/gfd/disjoint2_optional-1.html}]
Constrains the position (and possibly size) of the (possibly optional)
rectangles in Rectangles so that none overlaps.

\item[\biptxtrefni{disjunctive(+StartTimes, +Durations)}{disjunctive/2!gfd}{../bips/lib/gfd/disjunctive-2.html}]
Constrain the tasks with specified start times and durations to not overlap in time.

\item[\biptxtrefni{disjunctive_optional(+StartTimes, +Durations, +Scheduled)}{disjunctive_optional/3!gfd}{../bips/lib/gfd/disjunctive_optional-3.html}]
Constrain the optional tasks with specified start times and durations to
not overlap in time.

\end{description}

\subsubsection{Graph constraints}

In these constraints, the arguments represent a graph, and the
 constraint imposes some form of relation on the graph.

\begin{description}
\item[\biptxtrefni{circuit(+Succ)}{circuit/1!gfd}{../bips/lib/gfd/circuit-1.html}]
Constrains elements in Succ to form a Hamiltonian circuit.

\item[\biptxtrefni{circuit(+Succ,++CostMatrix,?Cost)}{circuit/3!gfd}{../bips/lib/gfd/circuit-3.html}]
Constrains elements in Succ to form a Hamiltonian circuit with cost
Cost.

\item[\biptxtrefni{circuit(+Succ,++CostMatrix,+ArcCosts,?Cost)}{circuit/4!gfd}{../bips/lib/gfd/circuit-4.html}]

\item[\biptxtrefni{circuit_g(+Succ)}{circuit_g/1!gfd}{../bips/lib/gfd/circuit_g-1.html}]
Native Gecode indexing version of {\tt circuit/1}.

\item[\biptxtrefni{circuit_g(+Succ,++CostMatrix,?Cost)}{circuit_g/3!gfd}{../bips/lib/gfd/circuit_g-3.html}]
Native Gecode indexing version of {\tt circuit/3}.

\item[\biptxtrefni{circuit_g(+Succ,++CostMatrix,+ArcCosts,?Cost)}{circuit_g/4!gfd}{../bips/lib/gfd/circuit_g-4.html}]
Native Gecode indexing version of {\tt circuit/4}.

\item[\biptxtrefni{circuit_offset(+Succ,+Offset)}{circuit_offset/2!gfd}{../bips/lib/gfd/circuit_offset-2.html}]
Constrains elements (offset by Offset) in Succ to form a Hamiltonian
circuit.

\item[\biptxtrefni{circuit_offset(+Succ,+Offset,++CostMatrix,?Cost)}{circuit_offset/4!gfd}{../bips/lib/gfd/circuit_offset-4.html}]
Constrains elements in Succ (offset by Offset) to form a Hamiltonian
circuit with cost Cost.

\item[\biptxtrefni{circuit_offset(+Succ,+Offset,++CostMatrix,+ArcCosts,?Cost)}{circuit_offset/5!gfd}{../bips/lib/gfd/circuit_offset-5.html}]
Constrains elements in Succ (offset by Offset) to form a Hamiltonian
circuit with cost Cost.

\item[\biptxtrefni{circuit_offset_g(+Succ,+Offset)}{circuit_offset_g/2!gfd}{../bips/lib/gfd/circuit_offset_g-2.html}]
Native Gecode indexing version of {\tt circuit_offset/2}.

\item[\biptxtrefni{circuit_offset_g(+Succ,+Offset,++CostMatrix,?Cost)}{circuit_offset_g/4!gfd}{../bips/lib/gfd/circuit_offset_g-4.html}]
Native Gecode indexing version of {\tt circuit_offset/4}.

\item[\biptxtrefni{circuit_offset_g(+Succ,+Offset,++CostMatrix,+ArcCosts,?Cost)}{circuit_offset_g/5!gfd}{../bips/lib/gfd/circuit_offset_g-5.html}]
Native Gecode indexing version of {\tt circuit_offset/5}.

\item[\biptxtrefni{ham_path(?Start,?End,+Succ)}{ham_path/3!gfd}{../bips/lib/gfd/ham_path-3.html}]
Constrains elements in Succ to form a Hamiltonian path from Start to
End.

\item[\biptxtrefni{ham_path(?Start,?End,+Succ,++CostMatrix,?Cost)}{ham_path/5!gfd}{../bips/lib/gfd/ham_path-5.html}]
Constrains elements in Succ to form a Hamiltonian path from Start to End with cost Cost.

\item[\biptxtrefni{ham_path(?Start,?End,+Succ,++CostMatrix,+ArcCosts,?Cost)}{ham_path/6!gfd}{../bips/lib/gfd/ham_path-6.html}]
Constrains elements in Succ to form a Hamiltonian path from Start to End, with cost Cost,

\item[\biptxtrefni{ham_path_g(?Start,?End,+Succ)}{ham_path_g/3!gfd}{../bips/lib/gfd/ham_path_g-3.html}]
Native Gecode indexing version of {\tt ham_path/3}.

\item[\biptxtrefni{ham_path_g(?Start,?End,+Succ,++CostMatrix,?Cost)}{ham_path_g/5!gfd}{../bips/lib/gfd/ham_path_g-5.html}]
Native Gecode indexing version of {\tt ham_path/5}.

\item[\biptxtrefni{ham_path_g(?Start,?End,+Succ,++CostMatrix,+ArcCosts,?Cost)}{ham_path_g/6!gfd}{../bips/lib/gfd/ham_path_g-6.html}]
Native Gecode indexing version of {\tt ham_path/6}.

\item[\biptxtrefni{ham_path_offset(?Start,?End,+Succ,+Offset)}{ham_path_offset/4!gfd}{../bips/lib/gfd/ham_path_offset-4.html}]
Constrains elements (offset by Offset) in Succ to form a Hamiltonian
path from Start to End.

\item[\biptxtrefni{ham_path_offset(?Start,?End,+Succ,+Offset,++CostMatrix,?Cost)}{ham_path_offset/6!gfd}{../bips/lib/gfd/ham_path_offset-6.html}]

\item[\biptxtrefni{ham_path_offset(?Start,?End,+Succ,+Offset,++CostMatrix,+ArcCosts,?Cost)}{ham_path_offset/7!gfd}{../bips/lib/gfd/ham_path_offset-7.html}]
Constrains elements in Succ (offset by Offset) to form a Hamiltonian
path from Start to End with cost Cost.

\item[\biptxtrefni{ham_path_offset_g(?Start,?End,+Succ,+Offset)}{ham_path_offset_g/4!gfd}{../bips/lib/gfd/ham_path_offset_g-4.html}]
Native Gecode indexing version of {\tt ham_path_offset/4}.

\item[\biptxtrefni{ham_path_offset_g(?Start,?End,+Succ,+Offset,++CostMatrix,?Cost)}{ham_path_offset_g/6!gfd}{../bips/lib/gfd/ham_path_offset_g-6.html}]
Native Gecode indexing version of {\tt ham_path_offset/6}.

\item[\biptxtrefni{ham_path_offset(?Start,?End,+Succ,+Offset,++CostMatrix,+ArcCosts,?Cost)}{ham_path_offset_g/7!gfd}{../bips/lib/gfd/ham_path_offset_g-7.html}]
Native Gecode indexing version of {\tt ham_path_offset/7}.

\item[\biptxtrefni{inverse(+Succ,+Pred)}{inverse/2!gfd}{../bips/lib/gfd/inverse-2.html}]
Constrains elements of Succ to be the successors and
Pred to be the predecessors of nodes in a digraph.

\item[\biptxtrefni{inverse(+Succ,+SuccOffset,+Pred,+PredOffset)}{inverse/4!gfd}{../bips/lib/gfd/inverse-4.html}]
Constrains elements of Succ (with SuccOffset) to be the successors and
Pred (with PredOffset) to be the predecessors of nodes in a digraph.

\item[\biptxtrefni{inverse_g(+Succ,+Pred)}{inverse_g/2!gfd}{../bips/lib/gfd/inverse_g-2.html}]
Native Gecode indexing version of {\tt inverse/2}.

\item[\biptxtrefni{inverse_g(+Succ,+SuccOffset,+Pred,+PredOffset)}{inverse_g/4!gfd}{../bips/lib/gfd/inverse_g-4.html}]
Native Gecode indexing version of {\tt inverse/4}.

\end{description}
 
\subsubsection{Extensional constraints}
These are ``user defined constraints'' (also known as ad-hoc
 constraints), i.e. the allowable sequence of values for a
collection of domain variables is defined as part of the constraint. These
predicate differs in the way the allowable values are specified.

\begin{description}
\item[\biptxtrefni{regular(+Vars, ++RegExp)}{regular/2!gfd}{../bips/lib/gfd/regular-2.html}]
Constrain Vars' solutions to conform to that defined in the regular expression RegExp.

\item[\biptxtrefni{table(+Vars, ++Table)}{table/2!gfd}{../bips/lib/gfd/table-2.html}]
Constrain Vars' solutions to be those defined by the tuples in Table.

\item[\biptxtrefni{table(+Vars, ++Table,+Option)}{table/3!gfd}{../bips/lib/gfd/table-3.html}]
Constrain Vars' solutions to be those defined by the tuples in Table.

\item[\biptxtrefni{extensional(+Vars, ++Transitions, +Start, +Finals)}{extensional/4!gfd}{../bips/lib/gfd/extensional-4.html}]
Constrain Vars' solutions to conform to the finite-state 
automaton specified by Transitions with start state Start and  final states Finals.

\end{description}

\subsubsection{Other constraints}

Constraints that don't fit into the other categories.

\begin{description}

\item[\biptxtrefni{bool_channeling(?Var, +DomainBools, +Min)}{bool_channeling/3!gfd}{../bips/lib/gfd/bool_channeling-3.html}]
Channel the domain values of Vars to the 0/1 boolean variables in DomainBools

\item[\biptxtrefni{integers(+Vars)}{integers/1!gfd}{../bips/lib/gfd/integers-1.html}]
Pseudo constraint (i.e. no constraint will be posted in Gecode): Vars' domain is the integer numbers (within default bounds).

\end{description}


\subsection{Search Support}

GFD allows search to be performed in three ways:

\begin{enumerate}
\item in Gecode. 
\item in \eclipse supported by GFD's variable selection 
and value choice predicates. 
\item in \eclipse using \eclipse's ``generic search'' 
facilities (search/6 and related predicates used by IC and FD). 
\end{enumerate}

\subsubsection{Performing search using Gecode}
\label{searcheng}
Search can be performed in Gecode using one of its search engines. 
In this 
case, the search to produce a solution appears as an atomic step at
the \eclipse level, and backtracking into the search will produce the next 
solution (or fail if there are none), again as an atomic step.

GFD uses \biptxtrefni{search/6}{search/6!gfd}{../bips/lib/gfd/search-6.html}
to interface to Gecode's search engines. This predicate is
designed to provide a similar syntax to the search/6 of the generic search 
interface (in {\tt lib(ic)} and {\tt lib(fd_search)}). 

As the search is performed in Gecode, it should be more efficient than doing
the search in \eclipse, where the system have to repeatedly switch between
Gecode and \eclipse when doing the search. As the search is a single atomic
step from the \eclipse level, it is not suitable if your code needs to
interact with the search, e.g. if you are using constraints defined at the
\eclipse level, and/or if you are using other solvers during the search.

Another difference between GFD's search/6 and the generic search is that 
GFD's is  less flexible -- you can only use the predefined variable 
selection and value choice methods, i.e. you cannot provide user-defined
predicates for the Select and Choice arguments. 

The search engine to use is specified by the Method argument in search/6. 
One method provided by Gecode is bb_min -- finding a minimal solution using
branch-and-bound, which is not provided by the generic search. 

Instead, branch-and-bound in \eclipse is provided by {\tt lib(branch_and_bound)}, which can
be used with generic search's {\tt search/6} to provide a similar functionality as
the {\tt bb_min} method of GFD's {\tt search/6}. The \eclipse branch-and-bound is more 
flexible, but is likely to be slower. Note that {\tt lib(branch_and_bound)} can
be used in combination with GFD's search/6, but this is probably not useful
unless you are doing some additional search in your own code to that done by 
search/6, or if you want to see the non-optimal solutions generated by the
search.

There is some differences in how search is performed by Gecode and \eclipse;
the most significant is that variable selection and value choice is performed
together for the labelling of every value, instead of what is done in \eclipse,
where the variable choice is performed first, and then value choice to chose
a value is performed for the {\it same} variable. 


\subsubsection{Search using GFD primitives}
\label{searchgfd}
GFD provides various predicates to perform variable choice
and value selection at the \eclipse level, so that the user can write their
own search routine in \eclipse. 

In addition, various properties associated
with a variable and its domain, which can be used in choosing the variable
and the value, can be obtained using predicates described
in section~\ref{gfdvarquery}. Note that these include properties that are not
available in
 \eclipse's solvers, such as weighted degree (aka accumulated failure count).

Using these facilities, the search will be performed
at the \eclipse level, thus allowing more flexibility than using Gecode's 
search engines. On the other hand, it is likely to be less efficient, both
because the search is performed in \eclipse, and also because there is
repeated transition between \eclipse and GFD at every step of the search.

However, these primitives are coded specifically for Gecode, and many are
likely to be code more efficiently than the equivalent available via 
generic search. Also, some are simply not implemented in generic search, 
such as the selection of variable based on the weighted degree (wdeg in the
literature) of the variables. 

\begin{description}
\item[\biptxtrefni{select_var(-X, +Collection, ++Arg, ++Select, ?Handle)}{select_var/5!gfd}{../bips/lib/gfd/select-5.html}]
Choose a domain variable from Collection according to pre-defined selection criteria.

\item[\biptxtrefni{delete(-X, +List, -R, ++Arg, ++Select)}{delete/5!gfd}{../bips/lib/gfd/delete-5.html}]
This is mainly provided for compatibility. Choose a domain variable from List according to selection criteria.

\item[\biptxtrefni{indomain(?Var)}{indomain/1!gfd}{../bips/lib/gfd/indomain-1.html}]
Instantiate Var to an element in its domain.

\item[\biptxtrefni{indomain(?Var, ++Method)}{indomain/2!gfd}{../bips/lib/gfd/indomain-2.html}]
A flexible way to assign values to finite domain variables specified
 by Method.

\item[\biptxtrefni{labeling(+Vars)}{labeling/1!gfd}{../bips/lib/gfd/labeling-1.html}]
Instantiates all variables in a collection to elements of their domains.

\item[\biptxtrefni{labeling(+Vars,+Select,+Choice)}{labeling/3!gfd}{../bips/lib/gfd/labeling-3.html}]
Instantiates all variables in a collection to elements of their domains, using 
Select variation selection method and Choice for value.
\end{description}

Note that the GFD specific {\tt select_var/5} as well as IC compatible 
{\tt delete/5} is provided for variable selection. {\tt delete/5} is provided 
mainly for compatibility, as {\tt select_var/5} should be more efficient in
most cases, as the variable selection is done in a single low-level call,
rather than at the \eclipse level.

Note that {\tt delete/5} 
is also available from generic search, except that
some 'native' Gecode selection methods, which are either not found in the generic search version,  or are implemented less efficiently, are available in this 
'native' version. 

\subsubsection{Generic search}
Generic search provide predicates that supports searching in \eclipse's 
finite domain solvers. In GFD, these predicates are available through the
{\tt gfd_search} library. To use these predicates, gfd_search library must
be loaded, and the calls to these predicates have to be module-qualified
with {\tt gfd_search:} as predicates with the same name and arity are defined
in GFD itself.

The main predicates are:

\begin{description}
\item[\biptxtrefni{search(+L, ++Arg, ++Select, +Choice, ++Method, +Option)}{search/6!gfd_search}{../bips/lib/gfd_search/search-6.html}] 
Perform search.

\item[\biptxtrefni{indomain(?Var, ++Method)}{indomain/2!gfd_search}{../bips/lib/gfd_search/indomain-2.html}] 
A flexible way to assign values to finite domain variables specified
 by Method.

\item[\biptxtrefni{delete(-X, +List, -R, ++Arg, ++Select)}{delete/5!gfd_search}{../bips/lib/gfd_search/delete-5.html}]
Choose a domain variable from List according to selection criteria.

\end{description}

These predicates are ``generic'' in the sense that the implementation is shared
by all \eclipse's finite domain solvers (IC, FD and GFD), so they should be
highly compatible with the versions of these predicates in these solvers.

GFD also have ``native'' (i.e. GFD-specific) versions of these predicates
in {\tt lib(gfd)}. These native versions interfaces more directly to Gecode,
and should be more efficient than the ``generic'' versions, but are less 
compatible with the generic versions in other solvers -- there are differences
in the available variable selection and value choice options, and for 
{\tt search/6}, the native version performs the search entirely within
Gecode (see section~\ref{searcheng}) instead of \eclipse.

These generic search predicates are provided mainly for compatibility with
the other solvers, and we recommend the use of the native versions of these
predicates. If the search performed is simple (i.e. no user-defined search)
and is entirely done by the 
built-in constraints of the finite domain solver (i.e. with no user-defined 
constraints or other solvers involved), then using the native 
\biptxtrefni{search/6}{seach/6!gfd}{../bips/lib/gfd/search-6.html} should give the best results. If the search
has to be done in \eclipse, then consider changing the code to use the native 
versions of the search-related primitives such as {\tt indomain/2} and {\tt delete/5}.


\section{User defined constraints and solver co-operation}
Like IC and FD solvers, GFD has facilities to allow the extension of the 
solver library so that GFD can co-operate with other solvers in solving a
problem, and also for the user to define their own constraints at the \eclipse
level. This is achieved by providing a suspension list with the gfd attribute,
which allows for the data-driven programming needed by solver co-operation and
constraint propagation, and a set of low-level predicates to process,
 query and  modify the domain of problem variables.

These facilities allow solver co-operation and user-defined 
constraints propagation at the \eclipse level, and not within Gecode directly,
so the search must be done at the \eclipse level, i.e. not through Gecode's
search engines. In addition, the performance of such constraints will very
likely be less efficient than implementing the constraints directly in Gecode.

\subsection{The {\it gfd\/} attribute}

The GFD attribute is a meta-term which is attached to all GFD problem variables.
Many of the fields of the attribute is used for implementing the interface to
Gecode, and is of no interest to the user. The only field of interest is the
any field, which is for the {\it any\/} suspension list, which is woken on 
any change in the domain of the variable:

\begin{verbatim}
gfd{
   ....
   any:SuspAny,
   ....
}
\end{verbatim}

The {\it any\/} suspension list has the same waking behaviour as the 
{\it any\/} suspension
list of FD, and is sufficient for implementing constraints -- the other 
suspension lists found in IC and FD are optimisations on the {\it any\/} 
suspension list, i.e. to provide for more precise waking conditions. 
The reason that
only one suspension list is provided by GFD is to minimise the overhead in
normal use of the solver. 


In addition to waking the attribute's {\it any\/} suspension list, the 
{\it constrained\/}
suspension list will also be woken when a GFD variable's domain is changed,
and the {\it inst\/} suspension list will be woken if the variable is bound.

The suspension list allow constraint propagation to be implemented at the
\eclipse level, which is distinct from the propagation of ``native'' Gecode
constraints, where each propagation phase (and in the case of using the 
search engine, the whole search) is an atomic step at the \eclipse level. 
This has a similar effect to if all the ``native'' propagations are scheduled 
at a higher (more urgent) priority.
 
As only the {\it any\/} suspension list is provided, some rewriting of existing
user-defined constraints for IC and FD will be needed when such code is ported
for GFD.

\subsection{Modifying variable domains}

Like IC, GFD provide a set of predicates to modify the domain of GFD 
variables to support the writing of new constraints. Unlike normal constraints,
no Gecode level propagation or waking of other suspended goals (such as 
scheduled by other \eclipse level constraints) occurs with these predicates.

With the exception of
\bipref{impose_bounds/3}{../bips/lib/gfd/impose_bounds-3.html} none of
the goals call \bipref{wake/0}{../bips/kernel/suspensions/wake-0.html}, so
the programmer is free to do so at a convenient time.

Some of these predicates are provided for compatibility with IC, as these 
predicates have the same name and similar semantics to their IC counter-parts
(including the waking behaviour for {\tt impose_bounds/3}).
However, due to the difference in the way domains are represented in IC and
Gecode, these predicates may be inefficient for use with Gecode, particularly
if you need to modify multiple variables and/or multiple domain values. 
The predicates with names that begins with {\tt gfd_vars}, which are specific
to GFD, are designed to be more efficient than their IC compatible 
counter-parts.

The ``native'' primitives are:

\begin{description}
\item[\biptxtrefni{gfd_vars_exclude(+Vars,+Excl)}{gfd_vars_exclude/2!gfd}{../bips/lib/gfd/gfd_vars_exclude-2.html}]
Exclude the element Excl from the domains of Vars.

\item[\biptxtrefni{gfd_vars_exclude_domain(+Vars, ++Domain)}{gfd_vars_exclude_domain/2!gfd}{../bips/lib/gfd/gfd_vars_exclude_domain-2.html}]
Exclude the values specified in Domain from the domains of Vars.

\item[\biptxtrefni{gfd_vars_exclude_range(+Vars, +Lo, +Hi)}{gfd_vars_exclude_range/3!gfd}{../bips/lib/gfd/gfd_vars_exclude_range-3.html}]
Exclude the elements Lo..Hi from the domains of Vars.

\item[\biptxtrefni{gfd_vars_impose_bounds(+Vars, +Lo, +Hi)}{gfd_vars_impose_bounds/3!gfd}{../bips/lib/gfd/gfd_vars_impose_bounds-3.html}]
Update (if required) the bounds of Vars.

\item[\biptxtrefni{gfd_vars_impose_domain(+Vars,++Domain)}{gfd_vars_impose_domain/2!gfd}{../bips/lib/gfd/gfd_vars_impose_domain-2.html}]
Restrict (if required) the domain of Var to the domain specified  in Domain.

\item[\biptxtrefni{gfd_vars_impose_max(+Vars,+Bound)}{gfd_vars_impose_max/2!gfd}{../bips/lib/gfd/gfd_vars_impose_max-2.html}]
Update (if required) the upper bounds of Vars.

\item[\biptxtrefni{gfd_vars_impose_min(+Vars,+Bound)}{gfd_vars_impose_min/2!gfd}{../bips/lib/gfd/gfd_vars_impose_min-2.html}]
Update (if required) the lower bounds of Vars.

\end{description}

The IC-compatible primitives are:

\begin{description}
\item[\biptxtrefni{exclude(?Var, +Excl)}{exclude/2!gfd}{../bips/lib/gfd/exclude-2.html}]
Exclude the element Excl from the domain of Var.

\item[\biptxtrefni{exclude_range(?Var, +Lo, +Hi)}{exclude_range/3!gfd}{../bips/lib/gfd/exclude_range-3.html}]
Exclude the elements Lo..Hi from the domain of Var.

\item[\biptxtrefni{impose_bounds(?Var,+Lo,+Hi)}{impose_bounds/3!gfd}{../bips/lib/gfd/impose_bounds-3.html}]
Update (if required) the bounds of Var.

\item[\biptxtrefni{impose_domain(?Var,++Domain)}{impose_domain/2!gfd}{../bips/lib/gfd/impose_domain-2.html}]
Restrict (if required) the domain of Var to the domain of DomVar.

\item[\biptxtrefni{impose_max(?Var, +Hi)}{impose_max/2!gfd}{../bips/lib/gfd/impose_max-2.html}]
Update (if required) the upper bound of Var.

\item[\biptxtrefni{impose_min(?Var, +Lo)}{impose_min/2!gfd}{../bips/lib/gfd/impose_min-2.html}]
Update (if required) the lower bound of Var.

\end{description}


\section{Implementation considerations}
\label{gfdimpl}

GFD is designed so that the user can write programs that will run with
 Gecode without knowing any details about Gecode or how GFD interfaces
 to it. However, in order to fully exploit GFD, especially when
 changing some of the default settings, some understanding of the
 relevant implementation issues is needed.This section will cover some
of these.

\subsection{Recomputation and Copying}

In Gecode, search -- the ability to try different alternatives -- is 
implemented in a different way from \eclipse. In \eclipse, search is
implemented using the technique of backtracking -- sufficient
 information is maintained during forward execution so that the
 computation state can be `rolled
 back' (or backtracked) to a previous state so that an alternative can be 
tried. In Gecode, the system does not roll back the computation state,
 but instead the old state is recomputed. To reduce the amount of
 recomputation needed, clones or copies of the computation state are made
 during forward execution, so that a previous state can be recomputed
 from the closest clone. So a fundamental issue for implementing GFD
 is how to integrate this recomputation/copying based scheme into the 
backtracking based scheme of \eclipse.

The cloning and recomputation applies to the Gecode computational state, 
referred to as a {\it `space'}.Gecode provides facilities to clone a space,
and for handling the recomputation and cloning when search is
 performed inside its search engines. Thus, cloning and recomputation is
handled by Gecode if
 Gecode's search engine is used to performed the search, but if the search
is performed by \eclipse, GFD has to handle the recomputation and
 cloning, such that after backtracking, the correct Gecode computation state is available
 without any explicit programming from the user.

During execution, GFD maintains
a chain of spaces -- a current space for the current state, and a
number of 'ancestor' spaces representing earlier states. GFD creates
a new ancestor space from time to time during forward execution 
by cloning the current space.

The current space will become invalid if the system backtrack pass 
any changes to the space. A new valid current space can then be recomputed
by cloning the closest ancestor space -- i.e.\ the logically last ancestor
 -- the {\it parent\/} space, and recomputing all the changes
 made to the space since the parent. This is done by
maintaining a list of {\it events}, where
an event is any change made to the Gecode space. The event list
 records events performed on the current space since the creation of the
parent space. As this event list is 
maintained logically at the \eclipse level, it will always represent the 
changes that need to be performed to recompute the current state from
 the parent. Thus,
recomputing simply means `replaying' or re-executing the events in the 
correct order. To speed up the recomputation, no propagation is
 performed until all the events have been replayed.

A Gecode space becomes invalid when \eclipse backtracks pass the last change
made to the space, i.e.\ the last event. When this happens, the space is marked
as invalid. Recomputation of the new space is done lazily, and not when
the current space becomes invalid, i.e.\ only when the space is needed again,
e.g.\ adding a new event to the space, or if the space is otherwise
accessed.

Each ancestor space consumes memory, which
is (normally) only released when GFD backtrack pass the creation of
 that space. On the 
other hand, while cloning itself also takes some computation time, it is
generally more time-efficient than recomputation, so in general,
the more frequent the cloning, the faster will be the overall execution,
but also the more memory will be required. The frequency of cloning
can be controlled via the option {\tt cloning_distance} (see \bipref{gfd_set_default/2}{../bips/lib/gfd/gfd_set_default-2.html}).

GFD will only create a new parent space when it is potentially useful
 to do so, when the following conditions are met:

\begin{enumerate}
\item there should be some difference with the last clone, i.e. 
there has been some new event performed since then. 
\item a new parent is only needed when there are alternatives branches 
in the search-space to explore since the current parent, i.e.
there is at least one choice-point since then. This is because if a
 parent is create before there are any choice-points, then backtracking will simply 
backtrack pass this new parent.
\item if cloning is done and then failure
occurs before the new state can be modified, the cloning will be a waste, so
cloning should only be done before a new event is performed.
\item to reduce the amount of recomputation, the new parent should be created
at late as possible. A number of events can be performed between choice-points,
but it is wasteful to clone before the last event before a choice-point, 
because the current state after backtracking is that after the last event before a choice-point,
and extra recomputation will be performed needlessly if cloning is done earlier.
\end{enumerate}

Thus, just before performing a new event, GFD checks to see if a new
 parent should be created. A new parent should be created if
the new event is the first event after 
the most recent choice-point, and if
the number of events since the last parent exceeds the cloning distance.
The new parent space is created by cloning the current space, and
making the current space the new parent, while the newly cloned space
 becomes the new current space. This means the new parent state will
still be available when \eclipse backtracks to the choice-point to try 
for alternatives -- while the new current state would have been
 discarded as the cloning is done after the choice-point.

In theory, an unlimited number of events can be kept on the events
 list, but in practice, a large number of events will consume
 significant amounts of memory, making it unavailable for other uses,
and execution performance can also be degraded, though the need to
 swap memory. The {\it cloning_distance\/} will normally limit the
number of events on the event list, but as outlined, a condition
for creating a new parent is if
there is a choice-point since the last event. So if the execution
 is deterministic, which is common during the modelling phase of a
constraint program, it is possible to accumulate a large number of
events. To avoid the list becoming excessively long, GFD will create a 
new parent clone
 ``early''(i.e.\ even if there is no choice-points since the last
 event), if the number of events exceed the {\it max_events\/}
 threshold, which can be changed via 
\bipref{gfd_set_default/2}{../bips/lib/gfd/gfd_set_default-2.html}.
The creation of this parent space is slightly different from the 
'normal' parent crated after a choice-point: in that if there
are no choice-points between the old parent and the current state
(i.e. the state of the new parent), the new parent can replace the
old parent, i.e. the old parent can be immediately discarded. 

As outlined, GFD handles cloning implicitly, so there is no need for
the user to explicitly specify when a new parent should be created.
However, GFD does provide a predicate to explicitly create a new parent
-- \bip{gfd_update}, as there are some situations where creating a 
new parent could improve the subsequent performance, but GFD will 
not always create the parent because the conditions for doing so are
 not met. The most obvious (and perhaps the only) use case for this
predicate is to ensure the creation of a new parent just before the
search phase of the program starts.

Creating a new parent just before the search mean that only
events during the search phase will be recomputed, regardless of the
setting of cloning_distance. Note that this is equivalent to the 
behaviour of Gecode, where recomputation is only defined during
search, because branching, the equivalent of backtracking and trying an 
alternative, is only possible within search, and a clone of the space
before search starts is always passed in for the search.

In addition to performing pre-search events redundantly (as they
 cannot be changed by the search),
an additional reason why it is desirable to only recomputing 'search' events
is that, generally, events during search tend to be quite simple
 -- labelling a variable to a value; while events more likely to be
performed during
 the modelling phase can be quite complex (e.g.\ posting a complex 
constraint expression) and thus more time consuming to recompute. 

Unlike Gecode, where search is only done within a search-engine, there
 is no clear-cut distinction in \eclipse between the modelling and
 search phases, as both are programmed using the same language. So if
the search is programmed by the user, 
there is no way of automatically detecting the start of the search
 phase of the program. However, under most conditions, a
 new  parent will be created
at the start of the search phase, as the first event of the search will be
after the first choice of the search, so one of the conditions for
 creating a new parent, that the new event
is the first event after a choice-point is always met. The other condition,
that the number of events in the event list is above the 
{\tt cloning_distance} is also likely to be met,  
as the modelling phase of constraint programs is normally just before
the search phase and is deterministic, so there should be a large
 number of events in the events list.

However, this is not guaranteed, particularly if {\tt
 cloning_distance} is set to 
a large value. In particular, if a parent was created  during the
modelling phase, e.g.\ because events_max was exceeded, or if the
 modelling phase was not totally deterministic, then the events list
 will be reset with the new parent, thus reducing the number of events
 in the events list. 

The intended use of \bip{gfd_update} is for the user to call
this predicate just before the start of search. The native labeling predicates
of GFD (\biptxt{labeling/1}{gfd:labeling/1} and 
\biptxt{labeling/2}{labeling/2}) both call this predicate before the
search (labelling) starts. As discussed, calling this predicate is not needed,
but if the user intend to mostly recompute during search by setting a
large {\tt cloning_distance}, using this predicate may improve the
 performance during search.

\subsubsection{Recomputation and value choice during search}

Using recomputation has some implications for performing search in 
Gecode.
Firstly, the value choice strategy {\tt in_domain} where the domain values
of a variable are tried in sequence, is expensive, because it requires 
remembering the current domain of the variable, which cannot be recomputed. 
Instead, {\tt in_domain_min}, which tries the values in the same order as 
{\tt in_domain}, is more efficient, because the domain does not need to 
be remembered; instead, extra constraints are posted to exclude the previous
tried value from the domain. 

Thus, for programming a value choice strategy in GFD, it is probably
 more efficient to change the choosing of a value to a series of binary
 choices -- first setting the variable to the chosen value, and in the
binary alternative, excluding the value, and then choosing the next
 value in a new choice-point.

\subsection{Gecode Variables}

Each GFD problem variable maps to a Gecode {\tt IntVar}, and all the IntVars
are stored in an array associated with the Gecode space.
At the \eclipse level, there is a corresponding (one dimensional)
array of the GFD problem
variables, so that the \eclipse variable and its corresponding {\tt IntVar}
have the same index in their 
arrays\footnote{The {\tt IntVar} at index 0 in the Gecode array is not used, as \eclipse indices starts at 1}. 

\eclipse arrays cannot be expanded, so when execution is started, an
array of some default size is created, and the entries in the array
filled in as new GFD problem variables are created. When the array
 becomes full, a new array that is double of the original size is
 created to replace the old array (with the existing entries copied).
The initial size of the array can be specified using the 
{\tt array_size} option of \bip{gfd_set_default/2}. Setting a large
 value for this will reduce the cost of copying and expanding the array, 
but may waste some memory for the unused entries -- ideally, the size of
this array should be the same as the number of problem variables in your
problem.

Gecode distinguishes between ``normal'' ({\tt IntVar})
and ``boolean'' domain variables {\tt BoolVar}). Boolean variables can only take. 
the values true and false, represented as 1 (true) and 0
 (false).In \eclipse's domain solver, boolean variables are just ``normal''
domain variables with a 0/1 domain. For compatibility with these
 solvers, GFD also does not distinguish between normal and boolean
 domain variables at the user level. Implementationally, a problem variable
 which needs to be a boolean domain variable in Gecode is represented by
a normal {\tt IntVar} in the Gecode variable array, and an additional
 {\tt BoolVar} that is linked to the {\tt IntVar} via a {\tt channel}
constraint which synchronise their domains. GFD access the variable
 via the {\tt IntVar}, while the {\tt BoolVar} is used for Gecode API calls
that requires a {\tt BoolVar}.

Default variable domain, max domain, variable domain for aux variables
 (GFD, Gecode)

\section{An Example}

Here we show various different ways of writing a simple constraint program --
N-Queens -- in GFD.

\begin{quote}
\begin{verbatim}
:- lib(gfd).

queens_list(N, Board) :-
        length(Board, N),
        Board :: 1..N,
        (fromto(Board, [Q1|Cols], Cols, []) do
            ( foreach(Q2, Cols), param(Q1), count(Dist,1,_) do
                Q2 #\= Q1,
                Q2 - Q1 #\= Dist,
                Q1 - Q2 #\= Dist
            )
        ),
        labeling(Board).  

\end{verbatim}
\end{quote}

This version of the program is from an example IC version of N-Queens,
with just \verb':- lib(ic)' replaced by \verb':- lib(gfd)'. The
search is done in \eclipse, using GFD's \verb'labeling/1', which essentially
employs no heuristics in selecting the variable (input order) and choice of 
value to label the selected variable to (from minimum).

To perform the search in Gecode, \verb'labeling(Board)' can be
 replaced by GFD's {\tt search/6}: 

\begin{quote}
\begin{verbatim}

:- lib(gfd).

queens_list(N, Board) :-
        length(Board, N),
        Board :: 1..N,
        (fromto(Board, [Q1|Cols], Cols, []) do
            ( foreach(Q2, Cols), param(Q1), count(Dist,1,_) do
                Q2 #\= Q1,
                Q2 - Q1 #\= Dist,
                Q1 - Q2 #\= Dist
            )
        ),
        search(Board, 0, input_order, indomain_min, complete, []).

\end{verbatim}
\end{quote}
We have to specify the variable selection and value choice heuristics here,
and the same ones as used in \verb'labelling/1' is used here, so the difference
in performance between this and the previous version of the program should be
due to performing the search in Gecode compared to \eclipse.

The search can also be performed using the  \verb'search/6' in the 
\verb'gfd_search' library:

\begin{quote}
\begin{verbatim}

:- lib(gfd), lib(gfd_search).

queens_list(N, Board) :-
        length(Board, N),
        Board :: 1..N,
        (fromto(Board, [Q1|Cols], Cols, []) do
            ( foreach(Q2, Cols), param(Q1), count(Dist,1,_) do
                Q2 #\= Q1,
                Q2 - Q1 #\= Dist,
                Q1 - Q2 #\= Dist
            )
        ),
        gfd_search:search(Board, 0, input_order, indomain_min, complete, []).

\end{verbatim}
\end{quote}
As \verb'lib(gfd_search)' is implemented using generic code shared with other
finite domain solvers, it is not very efficient, and should only be used to
quickly run code written for these other libraries in GFD, and replaced with
native GFD predicates if possible.

For example, to do the search in \eclipse, but with a more flexible labelling
predicate than \verb'labeling/1', \verb'labeling/3' can be used:

\begin{quote}
\begin{verbatim}
:- lib(gfd).

queens_list(N, Board) :-
        length(Board, N),
        Board :: 1..N,
        (fromto(Board, [Q1|Cols], Cols, []) do
            ( foreach(Q2, Cols), param(Q1), count(Dist,1,_) do
                Q2 #\= Q1,
                Q2 - Q1 #\= Dist,
                Q1 - Q2 #\= Dist
            )
        ),
        labeling(Board, input_order, indomain_min).  

\end{verbatim}
\end{quote}

Instead of using a built-in labelling predicate, it is also possible to write
the seach code, using GFD primitives:


\begin{quote}
\begin{verbatim}
    
:- lib(gfd).

queens_list(N, Board) :-
        length(Board, N),
        Board :: 1..N,
        (fromto(Board, [Q1|Cols], Cols, []) do
            ( foreach(Q2, Cols), param(Q1), count(Dist,1,_) do
                Q2 #\= Q1,
                Q2 - Q1 #\= Dist,
                Q1 - Q2 #\= Dist
            )
        ),
        my_labeling(Board).

my_labeling(Board) :-
       gfd_update,
       (fromto(Board, B1,B2, []) do
            delete(V, B1, B2, 0, input_order),
            indomain(V, indomain_min)
       ).

\end{verbatim}
\end{quote}
In this example, \bip{gfd_update} is called just before the search starts,
to ensure that no recomputation is done for the modelling part of the code
(posting of the constraints), and is only needed if the \verb'cloning_distance' 
is changed from the default to a large value.

There are also changes that can be made to the modelling code. For example,
the propagation strength of the constraints can be specified, e.g. as domain
consistent:
\begin{quote}
\begin{verbatim}
:- lib(gfd).

queens_list(N, Board) :-
        length(Board, N),
        Board :: 1..N,
        (fromto(Board, [Q1|Cols], Cols, []) do
            ( foreach(Q2, Cols), param(Q1), count(Dist,1,_) do
                gfd_gac: Q2 #\= Q1,
                gfd_gac: Q2 - Q1 #\= Dist,
                gfd_gac: Q1 - Q2 #\= Dist
            )
        ),
        labeling(Board).  

\end{verbatim}
\end{quote}
In this particular example, using the stronger propagation actually results in
a reduction in performance, as there are no reduction in search space from the
stronger propagation, but an increase in the cost of doing the propagation, 

Finally, we can post the 'primitive' form of the constraints instead of using
constrained expressions. The cost of posting these primitives should be cheaper
than using constrained expressions, but is less convenient:
\begin{quote}
\begin{verbatim}
:- lib(gfd).

queens_list(N, Board) :-
        length(Board, N),
        Board :: 1..N,
        (fromto(Board, [Q1|Cols], Cols, []) do
            ( foreach(Q2, Cols), param(Q1), count(Dist,1,_) do
                ne(Q2, Q1),
                scalar_product([1,-1],[Q2,Q1], (#\=), Dist),
                scalar_product([1,-1],[Q1,Q2], (#\=), Dist)
            )
        ),
        labeling(Board).  

\end{verbatim}
\end{quote}

%HEVEA\cutend
