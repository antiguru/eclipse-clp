% BEGIN LICENSE BLOCK
% Version: CMPL 1.1
%
% The contents of this file are subject to the Cisco-style Mozilla Public
% License Version 1.1 (the "License"); you may not use this file except
% in compliance with the License.  You may obtain a copy of the License
% at www.eclipse-clp.org/license.
% 
% Software distributed under the License is distributed on an "AS IS"
% basis, WITHOUT WARRANTY OF ANY KIND, either express or implied.  See
% the License for the specific language governing rights and limitations
% under the License. 
% 
% The Original Code is  The ECLiPSe Constraint Logic Programming System. 
% The Initial Developer of the Original Code is  Cisco Systems, Inc. 
% Portions created by the Initial Developer are
% Copyright (C) 2006 Cisco Systems, Inc.  All Rights Reserved.
% 
% Contributor(s): Joachim Schimpf, IC-Parc
% 
% END LICENSE BLOCK
%
% $Id: fdglobal.tex,v 1.2 2013/02/09 14:59:59 jschimpf Exp $
%
% Author: Joachim Schimpf
%

\chapter{Global Finite Domain Constraints}
\label{chapglobconstr}
%HEVEA\cutdef[1]{section}
\section{Various Constraints over Lists and Arrays}

The libraries
\biptxtref{ic_global}{library(ic_global)}{../bips/lib/ic_global/index.html}
and
\biptxtref{ic_global_gac}{library(ic_global_gac)}{../bips/lib/ic_global_gac/index.html}
implement a number of constraints over lists of integer variables.
They are loaded using a directive like
\begin{quote}\begin{verbatim}
:- use_module(library(ic_global)).	% or :- lib(ic_global).
\end{verbatim}\end{quote}
For details of the implemented constraints, refer to the Reference Manual
for the libraries.

These constraints are commonly known as \aboutidx{global constraints},
which simply indicates that they can constrain a large number of variables
at once, without breaking up the constraint into more primitive constraints.
This can on one hand facilitate problem modelling, and on the other hand
lead to more effective constraint propagation.

Note that some constraints are implemented in both of the libraries:
in these cases, ic_global_gac implements a "globally arc consistent"
version, while ic_global implements a less computationally expensive
version with weaker propagation.

Among the constraints provided are the following (refer to the
Reference Manual for the complete list):

\begin{description}
\item[\biptxtref{alldifferent(+List)}{ic_global:alldifferent/1}{../bips/lib/ic_global/alldifferent-1.html}]\ \\
\index{alldifferent/1}
A version of alldifferent/1 with strong propagation.

\item[\biptxtref{alldifferent(+List, ++Capacity)}{ic_global:alldifferent/2}{../bips/lib/ic_global/alldifferent-2.html}]\ \\
\index{alldifferent/2}
Like alldifferent/1, but every value may occur Capacity times.

\item[\biptxtref{alldifferent_matrix(+Matrix)}{alldifferent_matrix/1}{../bips/lib/ic_global_gac/alldifferent_matrix-1.html}]\ \\
Constrains the rows and columns of Matrix to be different values.

\item[\biptxtref{gcc(++Bounds,+Vars)}{gcc/2}{../bips/lib/ic_global_gac/gcc-2.html}]\ \\
The \aboutidx{global cardinality constraint} makes sure certain values
occur with a certain frequency in Vars.  This is a generalisation of the
occurrences-constraint.
See also \bipref{gcc_matrix/3}{../bips/lib/ic_global_gac/gcc_matrix-3.html}.

\item[\biptxtref{inverse(+Succ, +Pred)}{inverse/2}{../bips/lib/ic_global_gac/inverse-2.html}]\ \\
Constrains elements of Succ to be the successors and Pred to be the
predecessors of nodes in a digraph.

\item[\biptxtref{lex_le(+List1, +List2)}{lex_le/2}{../bips/lib/ic_global/lex_le-2.html}]\ \\
\index{lexico_le/2}
Imposes a lexicographic ordering between the two lists.

\item[\biptxtref{lex_lt(+List1, +List2)}{lex_lt/2}{../bips/lib/ic_global/lex_lt-2.html}]\ \\
Imposes a strict lexicographic ordering between the two lists.

\item[\biptxtref{ordered(++Relation, +List)}{ordered/2}{../bips/lib/ic_global/ordered-2.html}]\ \\
\index{ordered/2}
Constrains List to be ordered according to Relation.
Relation is one of the atoms \lt, =\lt, \gt, \gt=, = .

\item[\biptxtref{ordered_sum(++List, +Sum)}{ordered_sum/2}{../bips/lib/ic_global/ordered_sum-2.html}]\ \\
\index{ordered_sum/2}
The list elements are ordered and their sum is Sum.  A combination of
ordered and sum constraint with stronger propagation.

\item[\biptxtref{occurrences(++Value, +List, ?N)}{occurrences/3}{../bips/lib/ic_global/occurrences-3.html}]\ \\
\index{occurrences/3}
The value Value occurs in List N times.
Operationally: N gets updated to reflect the number of
possible occurrences in the List. List elements may get
instantiated to Value, or Value may be removed from their
domain if required by N.

\item[\biptxtref{same(+List1, +List2)}{same/2}{../bips/lib/ic_global_gac/same-2.html}]\ \\
List2 is a permutation of List1.

\item[\biptxtref{sequence(+Low, +High, +K, +Vars, ++Values)}{sequence/5}{../bips/lib/ic_global_gac/sequence-5.html}]\ \\
Every subsequence of Vars of length K contains at least Low and at most High
occurrences of Values.
Also \bipref{sequence/4}{../bips/lib/ic_global_gac/sequence-4.html}.

\item[\biptxtref{sorted(?List, ?Sorted)}{ic_global:sorted/2}{../bips/lib/ic_global/sorted-2.html}]\ \\
\index{sorted/2}
Sorted is a sorted permutation of List.

\item[\biptxtref{sorted(?List, ?Sorted, ?Positions)}{ic_global:sorted/3}{../bips/lib/ic_global/sorted-3.html}]\ \\
\index{sorted/3}
Sorted is a sorted permutation of List and Positions is a list whose
elements indicating the position of each unsorted list element within
the sorted list.

\item[\biptxtref{sumlist(+List, ?Sum)}{ic_global:sumlist/2}{../bips/lib/ic_global/sumlist-2.html}]\ \\
\index{sumlist/2}
The sum of the list elements is Sum. This constraint
is more efficient than a general IC sum constraint
if the list is long and Sum is not constrained frequently.

\end{description}


\section{Cumulative Constraint and Resource Profiles}

The library {\bf cumulative} implements the cumulative scheduling constraint.
It is based on the IC library and is loaded using one of 
\begin{quote}\begin{verbatim}
:- use_module(library(ic_cumulative)).
:- lib(ic_cumulative).
\end{verbatim}\end{quote}


\begin{description}
\item[\biptxtref{cumulative(+StartTimes, +Durations, +Resources, ++ResourceLimit)}{ic_cumulative:cumulative/4}{../bips/lib/ic_cumulative/cumulative-4.html}]\ \\
\index{cumulative/4}
A cumulative scheduling constraint. StartTimes, Durations and Resources
are lists of equal length N of integer variables or integers.
ResourceLimit is an integer. The declarative meaning is:
If there are N tasks, each starting at a certain start time, having
a certain duration and consuming a certain (constant) amount of
resource, then the sum of resource usage of all the tasks does not
exceed ResourceLimit at any time.

\item[\biptxtref{profile(+StartTimes, +Durations, +Resources, -Profile)}{ic_cumulative:profile/4}{../bips/lib/ic_cumulative/profile-4.html}]\ \\
\index{profile/4}
StartTimes, Durations, Resources and Profile
are lists of equal length N of integer variables or integers
with the same meaning as in cumulative/4.
The list Profile indicates the level of resource usage at the
starting point of each task.
\end{description}


\section{Edge-finder}

The libraries {\bf ic_edge_finder} and {\bf ic_edge_finder3}
implement stronger versions of the
disjunctive and cumulative scheduling constraints. They employ
a technique known as edge-finding to derive stronger bounds on
the starting times of the tasks.
The library is loaded using either
\begin{quote}\begin{verbatim}
:- use_module(library(ic_edge_finder)).
\end{verbatim}\end{quote}
to get a weaker variant with quadratic complexity, or
\begin{quote}\begin{verbatim}
:- use_module(library(ic_edge_finder3)).
\end{verbatim}\end{quote}
to get a stronger variant with cubic complexity.

\begin{description}
\item[disjunctive(+StartTimes,+Durations)]\ \\
\index{disjunctive/2}
A disjunctive scheduling constraint. StartTimes and Durations
are lists of equal length N of integer variables or integers.
The declarative meaning is that the N tasks with certain start times
and duration do not overlap at any point in time.

\item[cumulative(+StartTimes,+Durations,+Resources,++ResourceLimit)]\ \\
\index{cumulative/4}
A cumulative scheduling constraint. StartTimes, Durations and Resources
are lists of equal length N of integer variables or integers.
ResourceLimit is an integer. The declarative meaning is:
If there are N tasks, each starting at a certain start time, having
a certain duration and consuming a certain (constant) amount of
resource, then the sum of resource usage of all the tasks does not
exceed ResourceLimit at any time.
This constraint can propagate more information than the implementation
in library(cumulative).

\item[cumulative(+StartTimes,+Durations,+Resources,+Area,++ResourceLimit)]\ \\
\index{cumulative/5}
In this variant, an area (the product of duration and resource usage of
a task) can be specified, e.g.\ if duration or resource usage are not
known in advance. The edge-finder algorithm can make use of this information
to derive bound updates.
\end{description}
%HEVEA\cutend