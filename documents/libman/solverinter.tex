% BEGIN LICENSE BLOCK
% Version: CMPL 1.1
%
% The contents of this file are subject to the Cisco-style Mozilla Public
% License Version 1.1 (the "License"); you may not use this file except
% in compliance with the License.  You may obtain a copy of the License
% at www.eclipse-clp.org/license.
% 
% Software distributed under the License is distributed on an "AS IS"
% basis, WITHOUT WARRANTY OF ANY KIND, either express or implied.  See
% the License for the specific language governing rights and limitations
% under the License. 
% 
% The Original Code is  The ECLiPSe Constraint Logic Programming System. 
% The Initial Developer of the Original Code is  Cisco Systems, Inc. 
% Portions created by the Initial Developer are
% Copyright (C) 2006 Cisco Systems, Inc.  All Rights Reserved.
% 
% Contributor(s): Kish Shen, IC-Parc
% 
% END LICENSE BLOCK

% $Id: solverinter.tex,v 1.1 2006/09/23 01:49:34 snovello Exp $
% Author:		Kish Shen


\chapter{Common Solver Interface}
%HEVEA\cutdef[1]{section}
\index{common solver interface|(}

\section{Introduction}
\eclipse\ now provides a common syntax for the main arithmetic constraints
provided by different constraint solvers.
The basic idea is that the name and syntax of the constraint determines the
declarative meaning, while the operational semantics (the algorithmic
constraint behaviour) is determined by the module which implements the
constraint.
This principle simplifies the development of applications that use
hybrid solution methods. Constraints can be passed easily to different,
even multiple, solvers.


\section{Common constraints}

The constraints can be divided into the following groups:
\begin{itemize}
\item the numeric type constraints reals/1 and integers/1.
    Note that in this context, integers are considered a subset of the reals.

\item the range constraints ::/2, \verb'#::' and \$::/2, which give upper
    and lower bounds to their variables. In addition, ::/2 and \verb'#::'
    can also imply integrality.

\item arithmetic equality, inequality and disequality over the mathematical
    real numbers, e.g.
    \verb'$=', \verb'$>=', \verb'>', \verb'$\=' (and their symnonyms
    \verb'=:=', \verb'>=', \verb'>', \verb'=\=').
    Note that in this context,
    integers are considered a subset of the reals and can therefore
    occur in these constraints.

\item arithmetic equality, inequality and disequality which in addition to
    the above constrain all variables within their arguments to integers.
    Syntactically, these generally have a leading \verb'#',
    e.g.\ \verb'#=', \verb'#\=', \verb'#<'.
\end{itemize}


\begin{table}
%\begin{minipage}{\textwidth}
\begin{center}
\begin{toimage}
\begin{tabular}{|l||c|c|c|c|c|c|c|c|}
\hline
       &              &              &              &              &             & \verb'#::/2' &                   & \\
       &              &              &              &              &             & \verb'#=/2'  &                   & \\
       &              &              &              &              &             & \verb'#>=/2' &                   & \\
       &              & \verb'$::2'  &              &              &             & \verb'#=</2' &                   & \\
       & \verb'=:=/2' & \verb'$=/2'  & \verb'>/2'   & \verb'$>/2'  &             & \verb'#>/2'  &                   & \\
       & \verb'>=/2'  & \verb'$>=/2' & \verb'</2'   & \verb'$</2'  &             & \verb'#</2'  &                   & \\
       & \verb'=</2'  & \verb'$=</2' & \verb'=\=/2' & \verb'$\=/2' & \verb'::/2' & \verb'#\=/2' & \verb'integers/1' & \verb'reals/1' \\
\hline
\hline
suspend	& yes & yes & yes & yes & yes & yes & yes & yes \\
\hline                          
ic	& yes & yes & yes & yes & yes & yes & yes & yes \\
\hline                          
eplex	& yes & yes & --- & --- & yes & --- & yes & yes \\
\hline                          
colgen	& yes & yes & --- & --- & yes & --- & --- & yes \\
\hline                          
(arith)	&(yes)& --- &(yes)& --- & --- & --- & --- & --- \\
\hline
\end{tabular}
\end{toimage}\imageflush
\end{center}
{\small
    \begin{itemize}
    \item If integer bounds are given to the eplex version of \verb'::/2' 
    the external solver does not consider this as an integrality constraint
    and only solves the continuous relaxation which can then be rounded
    to the next integer.  To make the external solver solve a mixed
    integer problem, use the eplex version of integers/1.
    \end{itemize}
}
%\end{minipage}
\caption{Supported constraints for various arithmetic solvers}
\label{commoncons}
\end{table}
Not all constraints are supported by all the solvers. For example, the
eplex solver does not support any strict inequality constraints.
Table~\ref{commoncons} shows the constraints that are available
from the various constraint solvers.
In the table, a `yes' entry indicates that the
particular constraint is supported by the particular solver. Note
that some further restrictions may apply for a particular solver. For
example, the eplex solver can only handle linear expressions. Refer
to the documentation for each individual solver to see what restrictions
might apply.

Note that the last line, labelled `arith', is not really a constraint
solver but represents just the standard arithmetic tests which require
all variables to be instantiated. This behaviour is provided by the
(automatically imported) module {\tt eclipse_language}.

It can be somewhat confusing that these standard arithmetic tests have the same
names as the corresponding constraints. One one hand, they have the same declarative
meaning. On the other hand, they are not really interchangeable because they
can only be used as tests, not as active constraints. 
The following synonyms are therefore provided to make the distinction visible
where needed, and to reduce the need for module-qualification:

\begin{center}
\begin{tabular}{|c|c|}
\hline
\verb'$=/2'  & \verb'=:=/2' \\
\verb'$\=/2' & \verb'=\=/2' \\
\verb'$>=/2' & \verb'>=/2'  \\
\verb'$=</2' & \verb'=</2'  \\
\verb'$>/2'  & \verb'>/2'  \\
\verb'$</2'  & \verb'</2'  \\
\hline
\end{tabular}
\end{center}


\section{Using the constraints}

To use the constraints, \eclipse\ needs to know which solver to pass a
particular constraint to. The easiest method for doing this is to module
qualify the constraint. For example,

\begin{verbatim}
    ..., ic: (A #>= B), ...
\end{verbatim}

\noindent
passes the constraint \verb'A #>= B' to the ic solver. The solver must be
loaded first (e.g. via lib/1) before any constraint can be passed to it.

A constraint can also be passed to more than one solver by specifying a
list in the module qualification. For example,

\begin{verbatim}
    ..., [ic, suspend]: (A #>= B), ...
\end{verbatim}

\noindent
will pass the constraint to both the ic and suspend solvers. 

Module qualification is {\em not} needed if the constraint is defined by an
imported module, and there is no other imported module
which defines the same constraint. For example, if ic is the only imported
module which defines \verb'#>=/2', then \verb'#>=/2' can be used without 
qualification:

\begin{verbatim}
    ..., A #>= B, ...
\end{verbatim}

Note that for constraints that are defined for \verb'eclipse_language',
such as \verb'>=' (the standard arithmetic test), the default behaviour
when an unqualified call to such a
constraint is made is to pass it to \verb'eclipse_language',
even if another solver which defines the constraint is imported.
Thus, for example
\begin{verbatim}
    ..., A >= B, ...
\end{verbatim}

\noindent
will by default have standard (i.e.\ non-suspending) test semantics, even
if, e.g.\ the \verb'ic' library (which also defines \verb'>=/2') is
imported. To access the \verb'ic' version, module qualification should
be added:
\begin{verbatim}
    ..., ic:(A >= B), ...
\end{verbatim}
Alternatively, the synonymous \verb'$>=/2' constraint could be used:
\begin{verbatim}
    ..., A $>= B, ...
\end{verbatim}

In general, module qualifications are recommended if the programmer
wants to ensure a particular constraint behaviour regardless of which
other modules might be loaded.  On the other hand, if the intention is
to switch easily between different solvers by simply loading a
different library, module qualification is best omitted.

Finally, it is also possible to let the running program determine which
solver to use. In this case, the program has a variable in the module
position, which will only be bound at runtime:
\begin{verbatim}
    ..., Solver:(A #>= B), ...
\end{verbatim}
This will however prevent the solver from performing any compile-time
preprocessing on the constraint.



\section{The Solvers}

\begin{description}
\item[\biptxtref{suspend}{lib(suspend)}{../bips/lib/suspend/index.html}]
    This is the simplest possible 'solver'. Its behaviour is to wait until
    all variables in a constraint have been instantiated to numbers.
    Then it performs a test to check whether the constraint is satisfied,
    and fails if this is not the case.

\item[\biptxtref{ic}{lib(ic)}{../bips/lib/ic/index.html}]
    A new hybrid solver, combining integer and real interval constraint
    solving.  This solver is intended to replace the FD (and the already
    discontinued RIA) solver.
    For more information, please see chapter \ref{chapic}.

\item[\biptxtref{eplex}{lib(eplex)}{../bips/lib/eplex/index.html}]
    An interface to an LP or MIP solver, i.e.\ it implements linear
    constraints over reals or integers.

%\item[\biptxtref{fdplex}{lib(fdplex)}{../bips/lib/fdplex/index.html}]
%    An example of a hybrid solver that combines the fd and eplex solvers.

\item[arith]
    This is not really a solver, but just the implementation of simple
    arithmetic tests in module {\tt eclipse_language}. These require
    that all variables are instantiated when the test is invoked.
    The reason to list it here is that the proper solvers use the
    same syntax and can be considered generalisations of the
    traditional tests.
\end{description}
\index{common solver interface|)}

%HEVEA\cutend