% BEGIN LICENSE BLOCK
% Version: CMPL 1.1
%
% The contents of this file are subject to the Cisco-style Mozilla Public
% License Version 1.1 (the "License"); you may not use this file except
% in compliance with the License.  You may obtain a copy of the License
% at www.eclipse-clp.org/license.
% 
% Software distributed under the License is distributed on an "AS IS"
% basis, WITHOUT WARRANTY OF ANY KIND, either express or implied.  See
% the License for the specific language governing rights and limitations
% under the License. 
% 
% The Original Code is  The ECLiPSe Constraint Logic Programming System. 
% The Initial Developer of the Original Code is  Cisco Systems, Inc. 
% Portions created by the Initial Developer are
% Copyright (C) 2006 Cisco Systems, Inc.  All Rights Reserved.
% 
% Contributor(s): 
% 
% END LICENSE BLOCK

\chapter{\eclipse{} Programming}
%HEVEA\cutdef[1]{section}

%----------------------------------------------------------------------
\section{Structure Notation}
\label{structures}
In \eclipse, structure fields can be given names.
This makes it possible to write structures in a
more readable and maintainable way.
Such structures first need to be declared by specifying a template like:   
\begin{code}
:- local struct( book(author, title, year, publisher) ).
\end{code}
Structures with the functor book/4 can then be written as   
\begin{quote}\begin{verbatim}
book{}
book{title:'tom sawyer'}
book{title:'tom sawyer', year:1876, author:twain}
\end{verbatim}\end{quote}
which, in canonical syntax, correspond to the following:
\begin{quote}\begin{verbatim}
book(_, _, _, _)
book(_, 'tom sawyer', _, _)
book(twain, 'tom sawyer', 1876, _)
\end{verbatim}\end{quote}
There is absolutely no semantic difference between the two syntactical forms.
The special struct-syntax with names has the advantage that
\begin{itemize}
\item the arguments can be written in any order
\item ``dummy'' arguments with anonymous variables do not need to be written
\item the arity of the structure is not implied (and can be changed
by changing the declaration and recompiling the program)
\end{itemize}
Sometimes it is necessary to refer to the numerical position of a
structure field within the structure, e.g. in the arg/3 predicate:
\begin{quote}\begin{verbatim}
arg(3, B, Y)
\end{verbatim}\end{quote}
When the structure has been declared as above, we can write instead:
\begin{quote}\begin{verbatim}
arg(year of book, B, Y)
\end{verbatim}\end{quote}

Declared structures help readability, and make programs easier to modify.
In order not to lose these benefits, one should always use curly-bracket and
of-syntax when working with them, and never write them in canonical
syntax or referring to argument positions numerically.

\See{See also the
\bipref{update_struct/4}{../bips/kernel/termmanip/update_struct-4.html}
built-in predicate.}


%----------------------------------------------------------------------
\section{Loops}
\label{sec:loops}
To reduce the need for auxiliary recursive predicates, \eclipse{} allows
the use of an iteration construct
\begin{quote}\begin{verbatim}
( IterationSpecs do Goals )
\end{verbatim}\end{quote}
Typical applications are: Iteration over a list
\begin{quote}\begin{verbatim}
?- ( foreach(X,[1,2,3]) do writeln(X) )
1
2
3
Yes (0.00s cpu)
\end{verbatim}\end{quote}
Process all elements of one list and construct another:
\begin{quote}\begin{verbatim}
?- ( foreach(X,[1,2,3]), foreach(Y,List) do Y is X+3 ).
List = [4, 5, 6]
Yes (0.00s cpu)
\end{verbatim}\end{quote}
Process a list to compute the sum of its elements:
\begin{quote}\begin{verbatim}
?- ( foreach(X,[1,2,3]), fromto(0,In,Out,Sum) do Out is In+X ).
Sum = 6
Yes (0.00s cpu)
\end{verbatim}\end{quote}
Note that the variables X, Y, In and Out are local variables in the loop,
while the input list and Sum are shared with the context.

If a parameter remains constant across all loop iterations it must
be specified explicitly (via {\bf param}),
for example when iterating over an array:
\begin{quote}\begin{verbatim}
?- Array = [](4,3,6,7,8),
   (
       for(I,1,5),
       fromto(0,In,Out,Sum),
       param(Array)
   do
       Out is In + Array[I]
   ).
\end{verbatim}\end{quote}
\quickref{Iteration Specifiers for Loops}{
\begin{description}
\item[fromto(First,In,Out,Last)]\ \\
     iterate Goals starting with In=First until Out=Last.
\item[foreach(X,List)]\ \\
     iterate Goals with X ranging over all elements of List.
\item[foreacharg(X,StructOrArray)]\ \\
     iterate Goals with X ranging over all arguments of StructOrArray.
\item[foreacharg(X,StructOrArray,Idx)]\ \\
     same as before, but Idx is set to the argument position of X in
     StructOrArray.
\item[foreachelem(X,Array)]\ \\
     like foreacharg/2, but iterates over all elements of an array
     of arbitrary dimension.
\item[foreachelem(X,Array,Idx)]\ \\
     same as before, but Idx is set to the index position of X in
     Array.
\item[foreachindex(Idx,Array)]\ \\ 
     like foreachelem/3, but returns just the index position and not the
     element.
\item[for(I,MinExpr,MaxExpr)]\ \\
     iterate Goals with I ranging over integers from MinExpr to MaxExpr.
\item[for(I,MinExpr,MaxExpr,Increment) ]\ \\
     same as before, but Increment can be specified (it defaults to 1). 
\item[multifor(List,MinList,MaxList)]\ \\
     like for/3, but allows iteration over multiple indices (saves
     writing nested loops).
\item[multifor(List,MinList,MaxList,IncrementList)]\ \\
     same as before, but IncrementList can be specified (i.e.\ how
     much to increment each element of List by).
\item[count(I,Min,Max)]\ \\
     iterate Goals with I ranging over integers from Min up to Max.
\item[param(Var1,Var2,...)]\ \\
     for declaring variables in Goals global, i.e.\ shared with the context.
\end{description}
}
\See{For details and more examples see the description of the
\bipref{do/2}{../bips/kernel/control/do-2.html}
built-in predicate. Additional background can be found in \cite{loops02}.}

%----------------------------------------------------------------------

\section{Working with Arrays of Items}

For convenience, \eclipse{} has some features for facilitating working with
arrays of items.
Arrays can be of any dimension, and can be declared with the
\bipref{dim/2}{../bips/kernel/termmanip/dim-2.html}
predicate:
\begin{quote}\begin{verbatim}
?- dim(M,[3,4]).
M = []([](_131, _132, _133, _134),
       [](_126, _127, _128, _129),
       [](_121, _122, _123, _124))
yes.
\end{verbatim}\end{quote}
\bipref{dim/2}{../bips/kernel/termmanip/dim-2.html} can also be used to
query the dimensions of an array:
\begin{quote}\begin{verbatim}
?- dim(M,[3,4]), dim(M,D).
...
D = [3, 4]
yes.
\end{verbatim}\end{quote}

\Note{Note that arrays are just structures, and that the functor is not
important.}

To access a specific element of an array in an expression, specify the index
list of the desired element, e.g.\
\begin{quote}\begin{verbatim}
?- M = []([](2, 3, 5),
          [](1, 4, 7)),  X is M[1, 2] + M[2, 3].
X = 10
M = []([](2, 3, 5), [](1, 4, 7))
yes.
\end{verbatim}\end{quote}

\quickref{Array notation}{
\begin{itemize}
\item Arrays are just structures
\item The functor is not important
\item Declare or query array size with
        \bipref{dim/2}{../bips/kernel/termmanip/dim-2.html}
\item Access elements in expressions by specifying their index list
        (e.g.\ \texttt{A[7]}, \texttt{M[2,3]})
\item Indices start at 1
\end{itemize}
}

\See{For further details see the Array Notation section of the User Manual.}

%----------------------------------------------------------------------


\section{Storing Information Across Backtracking}

In pure logic programming, the complete state of a computation is
reset to an earlier state on backtracking.
The all-solutions predicates introduced in section \ref{all-solutions}
provide a way to collect solutions across backtracking.

The following section presents \eclipse's lower-level primitives for storing
information across failures: bags and shelves.
Both bags and shelves are referred to by handle, not by name,
so they make it easy to write robust, reentrant code.
Bags and shelves disappear when the system backtracks over their
creation, when the handle gets garbage collected, or when they are
destroyed explicitly.


\subsection{Bags}

A bag is an anonymous object which can be used to store information
across failures.  A typical application is the collection of
alternative solutions.

A bag is an unordered collection, referred to by a handle.
A bag is created using bag_create/1, terms can be added to a bag using
bag_enter/2, and the whole contents of the bag can be retrieved
using bag_retrieve/2 or bag_dissolve/2.
A simple version of the findall/3 predicate from section \ref{all-solutions}
can be implemented like:
\begin{code}
simple_findall(Goal, Solutions) :-
        bag_create(Bag),
        (
            call(Goal),
            bag_enter(Bag, Goal),
            fail
        ;
            bag_dissolve(Bag, Solutions)
        ).
\end{code}


\subsection{Shelves}

A shelf is an anonymous object which can be used to store information
across failures. A typical application is counting of solutions,
keeping track of the best solution, 
aggregating information across multiple solutions etc. 

A shelf is an object with multiple slots whose contents survive
backtracking. The content of each slot can be set and retrieved
individually, or the whole shelf can be retrieved 
as a term. Shelves are referred to by a handle.

A shelf is initialized using shelf_create / 2 or shelf_create /
3. Data is stored in the slots (or the shelf as a whole) with
shelf_set / 3 and retrieved with shelf_get / 3.  

For example, here is a meta-predicate to count the number of solutions
to a goal: 
\begin{code}
count_solutions(Goal, Total) :-
    shelf_create(count(0), Shelf),
    (
        call(Goal),
        shelf_get(Shelf, 1, Old),
        New is Old + 1,
        shelf_set(Shelf, 1, New),
        fail
    ;
        shelf_get(Shelf, 1, Total)
    ),
    shelf_abolish(Shelf).
\end{code}


%----------------------------------------------------------------------


%----------------------------------------------------------------------
\section{Input and Output}
%----------------------------------------------------------------------

\subsection{Printing {\eclipse} Terms}
\quickref{Builtins for writing}{
\begin{description}
\item[write(+Stream, ?Term)]\ \\
        write one term in a default format.
\item[write_term(+Stream, ?Term, +Options)]\ \\
        write one term, format options can be selected.
\item[printf(+Stream, +Format, +ArgList)]\ \\
        write a string with embedded terms, according to a format string.
\item[writeq(+Stream, ?Term), write_canonical(+Stream, ?Term)]\ \\
        write one term so that it can be read back.
\item[put(+Stream, +Char)]\ \\
        write one character.
\end{description}
}
The predicates of the write-group are generic in the sense that they
can print any {\eclipse} data structure.
The different predicates print slightly different formats.
The {\tt write/1} predicate is intended to be most human-readable, 
while {\tt writeq/1} is designed so that the
printed data can be read back by the predicates of the read-family.
If we print the structured term \verb.foo(3+4, [1,2], X, 'a b', "string").
the results are as follows:
\begin{quote}\begin{verbatim}
write:             foo(3 + 4, [1, 2], X, a b, string)
writeq:            foo(3 + 4, [1, 2], _102, 'a b', "string")
\end{verbatim}\end{quote}
The write-format is the shortest, but some information is missing,
e.g. that the sequence \verb.a b. is an atomic unit and that \verb.string.
is a string and not an atom. The writeq-format quotes items properly,
moreover, the variables are printed with unique numbers, so different
variables are printed differently and identical ones identically.

Single characters, encoded in ascii, can be output using {\tt put/1},
for example: 
\begin{quote}\begin{verbatim} 
[eclipse: 1] put(97).
a
yes.
\end{verbatim}\end{quote}

\subsection{Reading {\eclipse} Terms}
\quickref{Builtins for reading}{
\begin{description}
\item[read(+Stream, -Term)]\ \\
        read one fullstop-terminated \eclipse term.
\item[read_term(+Stream, -Term, +Options)]\ \\
        read one fullstop-terminated \eclipse term.
\item[get(+Stream, -Char)]\ \\
        read one character.
\item[read_string(+Stream, +Terminator, -Length, -String)]\ \\
        read a string up to a certain terminator character.
\item[read_token(+Stream, -Token, -Class)]\ \\
        read one syntactic token (e.g.\ a number, an atom, a bracket, etc).
\end{description}
}
If the data to be read is in Prolog syntax, it can be read using
{\tt read(?Term)}.
This predicate reads one fullstop-terminated
\eclipse term from stream Stream.
A fullstop is defined as a dot followed by a layout character like
blank space or newline.
Examples:
\begin{quote}\begin{verbatim}
[eclipse 4]: read(X).
 123,a.
X = 123, a
yes.

[eclipse 6]: read(X).
 [3,X,foo(bar),Y].
X = [3, X, foo(bar), Y]
yes.

\end{verbatim}\end{quote}

Single characters can be input using {\tt get/1}, which gets their
ascii encoding, for example:
\begin{quote}\begin{verbatim} 
[eclipse: 1] get(X).
a
X=97
yes.
\end{verbatim}\end{quote}


\subsection{Formatted Output}
The printf-predicate is similar to the printf-function in C, with some
\eclipse{}-specific format extensions.
Here are some examples of printing numbers:
\begin{quote}\begin{verbatim}
?- printf("%d", [123]).
123
yes.
?- printf("%5d,%05d", [123,456]).
  123,00456
yes.
?- printf("%6.2f", [123]).
type error in printf("%6.2f", [123])
?- printf("%6.2f", [123.4]).
123.40
yes.
?- printf("%6.2f", [12.3]).
 12.30
yes.
\end{verbatim}\end{quote}
The most important \eclipse{}-specific format option is \%w, which
allows to print like the predicates of the write-family:
\begin{quote}\begin{verbatim}
?- printf("%w", [foo(3+4, [1,2], X, 'a b', "string")]).
foo(3 + 4, [1, 2], X, a b, string)
\end{verbatim}\end{quote}
The \%w format allows a number of modifiers in order to access all the
existing options for the printing of \eclipse{} terms.

\See{For details see the
\bipref{write_term/2}{../bips/kernel/ioterm/write_term-2.html}
and
\bipref{printf/2}{../bips/kernel/ioterm/printf-2.html}
predicates.}

%----------------------------------------------------------------------

\subsection{Streams}

\eclipse{} I/O is done from and to named channels called streams. 
The following streams are always opened when \eclipse{} is running:
{\bf input} (used by the input predicates that do not have
an explicit stream argument, e.g.\ \bipref{read/1}{../bips/kernel/ioterm/read-1.html}
\index{read/1}\index{input}),
{\bf output} (used by the output predicates that do not have
an explicit stream argument, e.g.\ \bipref{write/1}{../bips/kernel/ioterm/write-1.html}
\index{write/1}\index{output}),
{\bf error} (output for error messages and all messages about exceptional states
\index{error}),
{\bf warning_output} (used by the system to output warning messages
\index{warning_output}),
{\bf log_output} (used by the system to output log messages, e.g.\ messages about garbage
collection activity
\index{log_output}),
{\bf null} (\index{null}
a dummy stream, output to it is discarded, on input it always
gives end of file).

Data can be read from a specific stream using
\biptxtref{read(+Stream,
?Term)}{read/2}{../bips/kernel/ioterm/read-2.html},
and written to a specific stream using
\biptxtref{write(+Stream,
?Term)}{write/2}{../bips/kernel/ioterm/write-2.html}.
If no particular stream is specified, input predicates read from {\bf input}
and output predicates write to {\bf output}.

New streams may be opened onto various I/O devices, see figure \ref{ioopen}.
\begin{figure}
\begin{center}
\begin{tabular}{|c|l|}
\hline
I/O device      &       How to open             \\
\hline
\hline
tty             &       implicit (stdin,stdout,stderr) or
                        \bipref{open/3}{../bips/kernel/iostream/open-3.html} of a device file \\
\hline
file            &       \biptxtref{open(FileName, Mode, Stream)}{open/3}{../bips/kernel/iostream/open-3.html}           \\
\hline
string          &       \biptxtref{open(string(String), Mode, Stream)}{open/3}{../bips/kernel/iostream/open-3.html}             \\
\hline
queue           &       \biptxtref{open(queue(String), Mode, Stream)}{open/3}{../bips/kernel/iostream/open-3.html}              \\
\hline
pipe            &       \bipref{exec/2}{../bips/kernel/opsys/exec-2.html},
                        \bipref{exec/3}{../bips/kernel/opsys/exec-3.html} and
                        \bipref{exec_group/3}{../bips/kernel/opsys/exec_group-3.html}   \\
\hline
socket          &       \bipref{socket/3}{../bips/kernel/iostream/socket-3.html} and
                        \bipref{accept/3}{../bips/kernel/iostream/accept-3.html}        \\
\hline
null            &       implicit (null stream)  \\
\hline
\end{tabular}
\end{center}
\caption{How to open streams onto the different I/O devices}
\label{ioopen}
\end{figure}

All types of streams are closed using
\biptxtref{close(+Stream)}{close/1}{../bips/kernel/iostream/close-1.html}.
\See{See the complete description of the
stream-related built-in predicates in the Reference Manual}

For network communication over sockets, there is a full set of predicates
modelled after the BSD socket interface:
\bip{socket/3},
\bip{accept/3},
\bip{bind/2},
\bip{listen/2},
\bip{select/3}.
See the reference manual for details.

Output in \eclipse{} is usually buffered, i.e.\ printed text goes into
a buffer and may not immediately appear on the screen, in a file, or
be sent via a network connection. Use
\biptxtref{flush(+Stream)}{flush/1}
{../bips/kernel/iostream/flush-1.html}
to empty the buffer and write all data to the underlying device.


%------------------------------------------------------------------


%----------------------------------------------------------------------
\section{Matching}
In \eclipse{} you can write clauses that use {\bf matching} (or one-way
unification) instead of head unification. 
Such clauses are written with the {\bf ?-} functor instead of {\bf :-}.
Matching has the property that no variables in the caller will be bound.
For example
\begin{code}
p(f(a,X)) ?- writeln(X).
\end{code}
will fail for the following calls:
\begin{quote}\begin{verbatim}
?- p(F).
?- p(f(A,B)).
?- p(f(A,b)).
\end{verbatim}\end{quote}
and succeed (printing b) for
\begin{quote}\begin{verbatim}
?- p(f(a,b)).
\end{verbatim}\end{quote}
Moreover, the clause
\begin{code}
q(X,X) ?- true.
\end{code}
will fail for the calls
\begin{quote}\begin{verbatim}
?- q(a,b).
?- q(a,B).
?- q(A,b).
?- q(A,B).
\end{verbatim}\end{quote}
and succeed for
\begin{quote}\begin{verbatim}
?- q(a,a).
?- q(A,A).
\end{verbatim}\end{quote}



%----------------------------------------------------------------------
\newpage
\section{List processing}
%----------------------------------------------------------------------

Lists are probably the most heavily used data structure in Prolog and
\eclipse{}. Apart from unification/matching, the most commonly used
list processing predicates are: append/3, length/2, member/2 and sort/2.
The append/3 predicate can be used to append lists or to split lists:
\begin{quote}\begin{verbatim}
?- append([1, 2], [3, 4], L).
L = [1, 2, 3, 4]
Yes (0.00s cpu)
?- append(A, [3, 4], [1, 2, 3, 4]).
A = [1, 2]
More (0.00s cpu)
No (0.01s cpu)
?- append([1, 2], B, [1, 2, 3, 4]).
B = [3, 4]
Yes (0.00s cpu)
\end{verbatim}\end{quote}
The length/2 predicate can be used to compute the length of a list
or to construct a list of a given length:
\begin{quote}\begin{verbatim}
?- length([1, 2, 3, 4], N).
N = 4
Yes (0.00s cpu)
?- length(List, 4).
List = [_1693, _1695, _1697, _1699]
Yes (0.00s cpu)
\end{verbatim}\end{quote}
The member/2 predicate can be used to check membership in a list
(but memberchk/2 should be preferred for that purpose),
or to backtrack over all list members:
\begin{quote}\begin{verbatim}
?- memberchk(2, [1, 2, 3]).
Yes (0.00s cpu)
?- member(X, [1, 2, 3]).
X = 1
More (0.00s cpu)
X = 2
More (0.01s cpu)
X = 3
Yes (0.01s cpu)
\end{verbatim}\end{quote}
The sort/2 predicate can sort any list and remove duplicates:
\begin{quote}\begin{verbatim}
?- sort([5, 3, 4, 3, 2], Sorted).
Sorted = [2, 3, 4, 5]
Yes (0.00s cpu)
\end{verbatim}\end{quote}
\See{For more list processing utilities, see the documentation for library(lists).}


%----------------------------------------------------------------------
\section{String processing}

\eclipse{} (unlike many Prolog systems) provides a string data type
and the corresponding string manipulation predicates, e.g.
string_length/2, concat_string/2, split_string/4, substring/4,
and conversion from and to other data types, e.g.
string_list/2, atom_string/2, number_string/2, term_string/2.
\begin{quote}\begin{verbatim}
?- string_length("hello", N).
N = 5
Yes (0.00s cpu)
?- concat_string([abc, 34, d], S).
S = "abc34d"
Yes (0.00s cpu)
?- string_list("hello", L).
L = [104, 101, 108, 108, 111]
Yes (0.00s cpu)
?- term_string(foo(3, bar), S).
S = "foo(3, bar)"
Yes (0.00s cpu)
\end{verbatim}\end{quote}


%----------------------------------------------------------------------
\section{Term processing}

Apart from unification/matching, there are a number of generic built-in
predicates that work on arbitrary data terms.
The \verb/=../ predicate converts structures into lists and vice versa:
\begin{quote}\begin{verbatim}
?- foo(a, b, c) =.. List.
List = [foo, a, b, c]
Yes (0.00s cpu)
?- Struct =.. [foo, a, b, c].
Struct = foo(a, b, c)
Yes (0.00s cpu)
\end{verbatim}\end{quote}
The arg/3 predicate extracts an argument from a structure:
\begin{quote}\begin{verbatim}
?- arg(2, foo(a, b, c), X).
X = b
Yes (0.00s cpu)
\end{verbatim}\end{quote}
The functor/3 predicate extracts functor name and arity from a structured term,
or, conversely, creates a structured term with a given functor name and arity:
\begin{quote}\begin{verbatim}
?- functor(foo(a, b, c), N, A).
N = foo
A = 3
Yes (0.00s cpu)
?- functor(F, foo, 3).
F = foo(_1696, _1697, _1698)
Yes (0.00s cpu)
\end{verbatim}\end{quote}
The term_variables/2 predicate extracts all variables from an arbitrarily
complex term:
\begin{quote}\begin{verbatim}
?- term_variables(foo(X, 3, Y, X), Vars).
Vars = [Y, X]
\end{verbatim}\end{quote}
The copy_term/2 predicate creates a copy of a term with fresh variables:
\begin{quote}\begin{verbatim}
?- copy_term(foo(3, X), Copy).
Copy = foo(3, _864)
Yes (0.00s cpu)
\end{verbatim}\end{quote}


%----------------------------------------------------------------------
\section{Module System}
\label{secmodules}
%----------------------------------------------------------------------
\subsection{Overview}
The \eclipse{} module system controls the visibility of
predicate names,
syntax settings (structures, operators, options, macros),
and non-logical store names (records, global variables).
Predicates and syntax items can be declared local or
they can be exported and imported.
Store names are always local.

%All of what a module exports can be imported by invoking
%\begin{quote} \begin{verbatim}
%:- use_module(a_module).
%\end{verbatim} \end{quote}
%or individual predicates can be imported using e.g.
%\begin{quote} \begin{verbatim}
%:- import p/3 from a_module.
%\end{verbatim} \end{quote}

%----------------------------------------------------------------------
\subsection{Making a Module}
A source file can be turned into a module by starting it with a 
module directive. A simple module is:
\begin{code}
:- module(greeting).
:- export hello/0.

hello :-
        who(X),
        printf("Hello %w!%n", [X]).

who(world).
who(friend).
\end{code}
This is a module which contains two predicates. One of them, hello/0
is exported and can be used by other modules. The other, who/1 is
local and not accessible outside the module.

\subsection{Using a Module}
There are 3 ways to use hello/0 from another module.
The first possibility is to import the whole ''greeting'' module.
This makes everything available that is exported from ''greeting'':
\begin{code}
:- module(main).
:- import greeting.

main :-
        hello.
\end{code}
The second possibility is to selectively only import the hello/0
predicate:
\begin{code}
:- module(main).
:- import hello/0 from greeting.

main :-
        hello.
\end{code}
The third way is not to import, but to module-qualify the call to hello/0:
\begin{code}
:- module(main).

main :-
        greeting:hello.
\end{code}


\subsection{Qualified Goals}

The module-qualification using \verb.:/2. is also used to resolve
name conflicts,
i.e.\ in the case where a predicate of the same name is defined
in more than one imported module.
In this case, none of the conflicting
predicates is imported - an attempt to call the unqualified predicate
raises an error.
The solution is to qualify every reference with the module name:
\begin{quote}\begin{verbatim}
:- lib(ic).       % exports $>= / 2
:- lib(eplex).    % exports $>= / 2

    ..., ic:(X $>= Y), ...
    ..., eplex:(X $>= Y), ...
\end{verbatim}\end{quote}
A more unusual feature, which is however very appropriate for
constraint programming, is the possibility to call several versions
of the same predicate by specifying several lookup modules:
\begin{quote}\begin{verbatim}
    ..., [ic,eplex]:(X $>= Y), ...
\end{verbatim}\end{quote}
which has exactly the same meaning as
\begin{quote}\begin{verbatim}
    ..., ic:(X $>= Y), eplex:(X $>= Y), ...
\end{verbatim}\end{quote}
Note that the modules do not have to be known at compile time, i.e. it
is allowed to write code like
\begin{quote}\begin{verbatim}
    after(X, Y, Solver) :-
        Solver:(X $>= Y).
\end{verbatim}\end{quote}
This is however likely to be less efficient because it prevents
compile-time optimizations.




\subsection{Exporting items other than Predicates}
The most commonly exported items, apart from predicates,
are structure and operator declarations.
This is done as follows:
\begin{code}
:- module(data).
:- export struct(employee(name,age,salary)).
:- export op(500, xfx, reports_to).
...
\end{code}
Such declarations can only be imported by importing the whole
module which exports them, i.e. using {\tt import data.}.


\See{For more details see the User Manual chapter on Modules.}


%----------------------------------------------------------------------

\section{Exception Handling}

It is sometimes necessary to exit prematurely from an executing
procedure, for example because some situation was detected which
makes continuing impossible.
In this situation, one wants to return to some defined state and
perform some kind of recovery action.
This functionality is provided by block/3 and exit_block/1.
\quickref{Exception Handling}{
\begin{description}
\item[block(Goal, BTag, Recovery)]\ \\
    like {\tt call(Goal)}, except that in addition a Recovery goal is set up, which
     can be called by {\tt exit_block} from anywhere inside the
     call to Goal. When {\tt exit_block(ETag)} is called, then if {\tt ETag} 
     unifies with a {\tt BTag} from an enclosing {\tt block}, the
     recovery goal associated with that {\tt block} is called, with the system
     immediately failing back to where the {\tt block} was called.  In
     addition, {\tt ETag} can be used to pass information to the recovery 
     goal, if {\tt BTag} occurs as an argument of {\tt Recovery}. 
\item[exit_block(ETag)]\ \\
     will transfer control to the innermost enclosing block/3 whose
     {\tt BTag} argument unifies with {\tt ETag}.
\end{description}
}
By wrapping a predicate call into block/3, any irregular termination
can be caught and handled, e.g.
\begin{code}
protected_main(X,Y,Z) :-
    block(
        main(X,Y,Z),
        Problem,
        printf("Execution of main/3 aborted with %w%n", [Problem])
    ).

main(X,Y,Z) :-
    ...,
    ( test(...) -> ... ; exit_block(test_failed) ),
    ...,
\end{code}
When built-in predicates raise errors, this results in the predicate
being exited with the tag \verb.abort., which can also be caught:
\begin{quote}\begin{verbatim}
?- block(X is 1//0, T, true).
arithmetic exception in //(1, 0, X)
X = X
T = abort
Yes (0.00s cpu)
\end{verbatim}\end{quote}
Note that timeouts and stack overflows also lead to exits and can be
caught this way.


%----------------------------------------------------------------------
%\section{Using the Debugger (2)}

%----------------------------------------------------------------------
\section{Time and Memory}

\subsection{Timing}
Timings are available via the built-in predicates
\bipref{cputime/1}
and
\bipref{statistics/2}.
To obtain the CPU time consumption of a (succeeding) goal, use the scheme
\begin{quote} \begin{verbatim}
cputime(StartTime),
my_goal,
TimeUsed is cputime-StartTime,
printf("Goal took %.2f seconds%n", [TimeUsed]).
\end{verbatim} \end{quote}

\ignore{
\subsection{Profiling}
The profiling tool can be used any time with any compiled Prolog code.
It uses interrupt-driven sampling and is therefore not available
on all hardware.

When the goal succeeds or fails, the profiler prints so
and then it prints the statistics about the time spent
in every encountered procedure:
\begin{quote} \begin{verbatim}
[eclipse 5]: profile(boyer).
rewriting...
proving...
goal succeeded

                PROFILING STATISTICS
                --------------------

Goal:             boyer
Total user time:  10.65s

Predicate             Module         %Time  Time
-------------------------------------------------
rewrite           /2  eclipse        52.3%  5.57s
garbage_collect   /0  sepia_kernel   23.1%  2.46s
rewrite_args      /2  eclipse        16.6%  1.77s
equal             /2  eclipse         4.7%  0.50s
remainder         /3  eclipse         1.5%  0.16s
...
plus              /3  eclipse         0.1%  0.01s

yes.
\end{verbatim} \end{quote}
}

%----------------------------------------------------------------------
%\section{Memory and Garbage collection}
The
\bipref{statistics/2}{../bips/kernel/env/statistics-2.html} and \bipref{statistics/0}{../bips/kernel/env/statistics-0.html}
commands can also be used to obtain memory usage information.
The memory areas used by \eclipse{} are:
\begin{description}
\item[Shared and private heap] for compiled code, non-logical store (
    bags and shelves, findall)
    dictionary of functors, various tables and buffers.
\item[Global stack] for most \eclipse{} data like lists, structures, suspensions.
        This is likely to be the largest consumer of memory.
\item[Local stack] for predicate call nesting and local variables.
\item[Control and trail stack] for data needed on backtracking.
\end{description}
Automatic garbage collection is done on the global and trail stack,
and on the dictionary. Garbage collection parameters can be set using
\bipref{set_flag/2}{../bips/kernel/env/set_flag-2.html}
and an explicit collection can be requested using
\bipref{garbage_collect/0}{../bips/kernel/env/garbage_collect-0.html}.

%----------------------------------------------------------------------
%\section{Macros}
%    \subsection{Read/write Macros}
%    \subsection{Goal Macros}

%----------------------------------------------------------------------
%\section{Compiler}

%----------------------------------------------------------------------
\ignore{
\section{Events}

%\subsection{Event Identifiers}
Events are identified by names (user defined events)
or by small numbers (\eclipse{} errors).

%\subsection{Handling Events}
When an event occurs, a call to the appropriate handler is inserted
into the resolvent (the sequence of executing goals).
The handler will be executed as soon as possible, which means at the
next synchronous point in execution, which is usually just before the
next regular predicate is invoked.

A handler is defined using a call like this
\begin{code}
my_handler(Event) :-
    <code to deal with Event>

:- set_event_handler(hello, my_handler/1).
\end{code}
\index{set_event_handler/2}
The handler's first argument is the event identifier, in this case the
atom 'hello'.

%\subsection{Raising Events}
Events are normally posted to the \eclipse{} engine from its software
environment, e.g.\ from a C program using
\begin{quote}\begin{verbatim}
ec_post_event(ec_atom(ec_did("hello",0)));
\end{verbatim}\end{quote}
\index{ec_post_event}

An event can also be raised by the running program itself, using
\bipref{event/1}{../bips/kernel/event/event-1.html}:
\begin{quote}\begin{verbatim}
    ..., event(hello), ...
\end{verbatim}\end{quote}
}

\section{Exercises}

\begin{enumerate}

\item

Using a \texttt{do} loop, write a predicate which, when given a 1-d array,
returns a list containing the elements of the array in reverse order.


\item

Write a predicate \texttt{transpose(Matrix, Transpose)} to transpose a 2-d
array.

Can you make it work backwards?  (i.e.\ if \texttt{Transpose} is specified,
can you make it return a suitable \texttt{Matrix}?)

\end{enumerate}

%HEVEA\cutend
