% BEGIN LICENSE BLOCK
% Version: CMPL 1.1
%
% The contents of this file are subject to the Cisco-style Mozilla Public
% License Version 1.1 (the "License"); you may not use this file except
% in compliance with the License.  You may obtain a copy of the License
% at www.eclipse-clp.org/license.
% 
% Software distributed under the License is distributed on an "AS IS"
% basis, WITHOUT WARRANTY OF ANY KIND, either express or implied.  See
% the License for the specific language governing rights and limitations
% under the License. 
% 
% The Original Code is  The ECLiPSe Constraint Logic Programming System. 
% The Initial Developer of the Original Code is  Cisco Systems, Inc. 
% Portions created by the Initial Developer are
% Copyright (C) 2006 Cisco Systems, Inc.  All Rights Reserved.
% 
% Contributor(s): 
% 
% END LICENSE BLOCK
%
% @(#)umscmacros.tex    
%

%%%%%%%%%%%%%%%%%%%%%%%%%%%%%%%%%%%%%%%%%%%%%%%%%%%%%%%%%%%%%%%%%%%%%%%%%%
% {\eclipse} Documentation
%
% embed.tex
%
%%%%%%%%%%%%%%%%%%%%%%%%%%%%%%%%%%%%%%%%%%%%%%%%%%%%%%%%%%%%%%%%%%%%%%%%%%
\chapter{Calling {\eclipse} from  C++}
\label{chap1embed}
%HEVEA\cutdef[1]{section}

This chapter describes how {\eclipse} can be included in an external
program as a library, how to start it, and how to
communicate with it. Code examples are given in C++.
For the equivalent C functions, please refer to chapter~\ref{chapsumc}.

%---------------------------------------------------------------------
\section{To get started}
%---------------------------------------------------------------------

This section is about the prerequisites for working with
{\eclipse} in your development environment. The directory structure,
the libraries and the include files are described.

\subsection{Directories}
\index{directories}

The libraries and include files needed to use {\eclipse} as an
embedded component are available under the {\eclipse}
\index{installation directory}
directory which was set-up during installation. If you have access
to a stand-alone {\eclipse} it can be found using the following query
at the {\eclipse} prompt:

\begin{verbatim}
    [eclipse 1]: get_flag(installation_directory,Dir).

    Dir = "/usr/local/eclipse"
    yes.
    [eclipse 2]
\end{verbatim}

We will assume from here that {\eclipse} was installed in "/usr/local/eclipse".

You would find the include files in "/usr/local/eclipse/include/\$ARCH"
and the the libraries in "/usr/local/eclipse/lib/\$ARCH" where
"\$ARCH" is a string naming the architecture of your machine.
\index{ARCH}
\index{architecture}
This can be found using the following {\eclipse} query:

\begin{verbatim}
    [eclipse 2]: get_flag(hostarch,Arch).

    Arch = "sun4"
    yes.
    [eclipse 3]:
\end{verbatim}

You will need to inform your C or C++ compiler and linker about these
directories so that these tools can include and link the appropriate files.
A make file "Makefile.external" can be found together with the libraries.
\index{Makefile}
The definitions in that makefile may  have to be updated according to
your operating system environment.

A set of example C and C++ programs can be found in
"/usr/local/eclipse/doc/examples".

When delivering an application you will have to include with it the
contents of the directory 
"/usr/local/eclipse/lib" without which {\eclipse}
cannot work. Normally this would be copied into the directory structure
of the delivered application. The interface can set different values
for this directory, enabling different applications to have different
sets of libraries.


\subsection{Definitions}

To include the definitions needed for calling the {\eclipse} library
in a C program use:
\index{include files}
\begin{verbatim}
    #include <eclipse.h>
\end{verbatim}

For C++ a more convenient calling convention can be used based on
some classes wrapped around these C definitions. To include these use:

\begin{verbatim}
    #include <eclipseclass.h>
\end{verbatim}



\subsection{Compiling, linking and running on Unix/Linux}

Assuming that the environment variable ECLIPSEDIR is set to the
{\eclipse} installation directory and the environment variable ARCH is set to
the architecture/operating system name,
an application can be built as follows:
\begin{verbatim}
gcc -I$ECLIPSEDIR/include/$ARCH eg_c_basic.c -L$ECLIPSEDIR/lib/$ARCH -leclipse
\end{verbatim}
This will link your application with the shared library \verb+libeclipse.so+.

At runtime, your application must be able to locate \verb+libeclipse.so+.
This can be achieved by adding $ECLIPSEDIR/lib/$ARCH to your
\verb+LD_LIBRARY_PATH+ environment variable.

The embedded {\eclipse} finds its own support files (e.g.\ {\eclipse} libraries)
through the \verb+ECLIPSEDIR+ environment variable.
This must be set to the location where {\eclipse} is installed,
e.g.\ \verb+/usr/local/eclipse+.
Alternatively, the application can invoke \verb+ec_set_option+ to specify
the \verb+ECLIPSEDIR+ location before initialising the embedded {\eclipse}
with \verb+ec_init+.


\subsection{Static linking (Unix/Linux)}

If your operating system only supports static linking, or if you want
to link statically for some reason, you have to link explicitly with
\verb+libeclipse.a+ and the necessary support libraries must be specified, e.g.
\begin{verbatim}
gcc -I$ECLIPSEDIR/include/$ARCH eg_c_basic.c $ECLIPSEDIR/lib/$ARCH/libeclipse.a
    -L$ECLIPSEDIR/lib/$ARCH -lgmp -lshm -ldummies -ldl -lnsl -lsocket -lm
\end{verbatim}
The libraries \verb+gmp+, \verb+shm+ and \verb+dummies+ are {\eclipse}
support libraries and must be specified in that order.
The others are Unix libraries.

It is recommended that you copy the makefile "Makefile.external"
provided in your  installation directory under lib/\$ARCH and adapt it
for your purposes.
\index{Makefile}
% If the makefile is not used, the command to compile a C++ source
% with {\eclipse} library calls looks similar to this:
% \begin{verbatim}
%     % g++ -I/usr/local/eclipse/include/sparc_sunos5 -c eg_cc_main.cc
% \end{verbatim}
% The linking of an executable with embedded {\eclipse} looks like
% \begin{verbatim}
%     % g++ eg_cc_main.o -L/usr/local/eclipse/lib/sparc_sunos5 -leclipse
%                        -lgmp -lshm -ldummies -ldl -lnsl -lsocket -lm
% \end{verbatim}
% The details may vary depending on what compiler and operating system
% you use.  Refer to the Makefile.external for details.


\subsection{Compiling, linking and running on Windows}

You need to have the following additional setting in your C/C++
compiler/development system:
\begin{itemize}
\item In the {\bf C/C++ Preprocessor} settings,
specify the {\eclipse} include directory as an additional
include directory, e.g.\ \verb+C:/Eclipse/include/i386_nt+.
\item In the {\bf Link} settings,
specify \verb+eclipse.lib+ as an additional object/library
module, and the location of this library, e.g.\
\verb+C:/Eclipse/lib/i386_nt+ as an additional library path.
\end{itemize}

At runtime, your application must be able to locate \verb+eclipse.dll+,
i.e.\ you should either
\begin{itemize}
\item copy \verb+eclipse.dll+ into the folder where your application is located, or
\item copy \verb+eclipse.dll+ into one of Windows' standard library folders, or
\item add the path to the folder where \verb+eclipse.dll+ can be found
	to your PATH environment variable.
\end{itemize}
The \verb+eclipse.dll+ finds its own support files (e.g.\ {\eclipse} libraries)
through the \verb+ECLIPSEDIR+ registry entry under the registry key
\verb+HKEY_LOCAL_MACHINE\SOFTWARE\IC-Parc\ECLiPSe\X.Y+ (X.Y is the version
number).
This must be set to the location where {\eclipse} is installed,
e.g.\ \verb+C:/Eclipse+.
Alternatively, the application can invoke \verb+ec_set_option+ to specify
the \verb+ECLIPSEDIR+ location before initialising the embedded {\eclipse}
with \verb+ec_init+.


%---------------------------------------------------------------------
\section{Creating an {\eclipse} context}
%---------------------------------------------------------------------

{\eclipse} runs as a special thread (we will call it {\eclipse} engine) within
your application,
maintaining its own execution state. This section is about when and
how to initialise it. There are parameters to be applied before
initialisation, but these are usually acceptable. These parameters are
described in Appendix~\ref{chapecoptions}.

Although it is useful to think of {\eclipse} as a thread, it is not an
\index{thread}
operating system thread, but is rather implemented as a set of C
functions that maintain some state. This state is the complete
execution state of the {\eclipse} engine, its stack of goals, its
success and failure continuations and its global stack of all
constructed data.

At particular points during its execution {\eclipse} will yield control
back to the C level. This is implemented by returning from a function.
{\eclipse} can then be restarted from the exact point it left off. This
is implemented by a function call.

\subsection{Initialisation}

You initialise {\eclipse} by calling the parameterless function
\index{initialisation}
\begin{verbatim}
    int ec_init();
\end{verbatim}

A process should do this just once. \verb.ec_init. returns 0 on success or
-1 if an error occurred. It is possible to influence the initialisation of
{\eclipse} by setting initialisation options as described in
Appendix~\ref{chapecoptions}.

None of the functions of the interface work before this initialisation. In
particular in C++, if you use static variables which are
constructed by calling {\eclipse} functions you must arrange for the
initialisation to occur before the constructors are called.


%---------------------------------------------------------------------
\section{Control flow}
%---------------------------------------------------------------------

{\eclipse} and a C or C++ main program are like threads running in a
single process. Each maintains its state and methods for exchanging
\index{thread}
\index{state}
data and yielding control to the other thread.

The main method of sending data from C++ to {\eclipse} is by posting
\index{posting goals}
goals for it to solve. All posted goals are solved in conjunction
with each other, and with any previously posted goals that have
succeeded.

Data is passed back by binding logical variables within the goals.
\index{logical variable}

\index{resume}
Control is explicit in C++. After posting some goals, the C++ program
calls the \verb.EC_resume(). function and these goals are all
solved. A return code says whether they were successfully solved
or whether a failure occurred.


In {\eclipse} control is normally implicit. Control returns to C++ when all
goals have been solved.

\begin{verbatim}
    #include        "eclipseclass.h"

    main()
    {
        ec_init();

        /* writeln("hello world"), */
        post_goal(term(EC_functor("writeln",1),"hello world"));
        EC_resume();
        ec_cleanup(0);
    }
\end{verbatim}
The above is an example program that posts a goal and executes it.

\subsection{Control flow and search}

Using this model of communication it is possible to construct programs where
execution of C++ code and search within the {\eclipse} are interleaved.

\index{posting goals}
\index{failure}
\index{backtracking}
If you post a number of goals (of which some are non-deterministic) and
resume the {\eclipse} execution and the goals succeed, then control returns
to the C++ level. By posting a goal that fails, the {\eclipse} execution
will fail back into the previous set of goals and these will succeed with
a different solution. 

\begin{verbatim}
    #include        "eclipseclass.h"

    main()
    {
        ec_init();
        EC_ref Pred;

        post_goal(term(EC_functor("current_built_in",1),Pred));
        while (EC_succeed == EC_resume())
        {
            post_goal(term(EC_functor("writeln",1),Pred));
            post_goal(EC_atom("fail"));
        }
        ec_cleanup(0);
    }
\end{verbatim}
The above example prints all the built ins available in {\eclipse}.
When \verb.EC_resume(). returns \verb.EC_succeed. there is a solution
to a set of posted goals, and we print out the value of \verb.Pred..
otherwise \verb.EC_resume(). returns \verb.EC_fail. to indicate
that no more solutions to any set of posted goals is available.

\index{cut}
It is possible also to cut such search. So for example one could modify
the example above to only print the 10th answer. Initially one simply
fails, but at the tenth solution one cuts further choices. Then
one prints the value of 'Pred'.

\begin{verbatim}
    #include        "eclipseclass.h"

    main()
    {
        ec_init();
        EC_ref Pred,Choice;
        int i = 0;

        post_goal(term(EC_functor("current_built_in",1),Pred));
        while (EC_succeed == EC_resume(Choice))
        {
            if (i++ == 10)
            {
                Choice.cut_to();
                break;
            }
            post_goal(term(EC_atom("fail")));
        }
        post_goal(term(EC_functor("writeln",1),Pred));
        EC_resume():
        ec_cleanup(0);
    }
\end{verbatim}

\index{choicepoint}
When \verb.EC_resume(). is called with an \verb.EC_ref. argument, this
is for data returned by the \verb.EC_resume(). If the return code is
\verb.EC_succeed. The \verb.EC_ref. is set to a choicepoint identifier
which can be used for cutting further choices at that point.

\subsection{Asynchronous events}

\index{events}
\index{posting events}
The posting of goals and building of any {\eclipse} terms in general
cannot be done asynchronously to the {\eclipse} execution. It has to
be done after the \verb.EC_resume(). function has returned.

\index{asynchronous}
Sometimes it may be necessary to signal some asynchronous event to
{\eclipse}, for example to implement a time-out. To do this one
posts a named event to {\eclipse}. At the next synchronous point
within the eclipse execution, the handler for that event is
invoked.
\begin{verbatim}
    /* C++ code, possibly within a signal handler */
    post_event(EC_atom("timeout"));

    /* ECLiPSe code */
    handle_timeout(timeout) :-
        <appropriate action>

    :- set_event_handler(timeout, handle_timeout/1).
\end{verbatim}


\subsection{The yield-resume model}

\index{passing data}
\index{yield}
\index{resume}
Although implicitly yielding control when a set of goals succeeds
or fails is often enough, it is possible to explicitly yield
control to the C++ level. This is done with the
\bipref{yield/2}{../bips/kernel/externals/yield-2.html}
predicate.  This yields control to the calling C++ program.
The arguments are used for passing data to C++ and from C++.

When \bipref{yield/2}{../bips/kernel/externals/yield-2.html} is called within {\eclipse} code, the \verb.EC_resume().
function returns the value \verb.EC_yield. so one can recognise this case.
The data passed out via the first argument of \bipref{yield/2}{../bips/kernel/externals/yield-2.html}
can  be accessed from C++ via the \verb.EC_ref. argument to \verb.EC_resume()..
The data received in the second argument of \bipref{yield/2}{../bips/kernel/externals/yield-2.html} is either
the list of posted goals, or an \verb.EC_word. passed as an input
argument to \verb.EC_resume()..

\begin{verbatim}
    yield(out(1,2),InData),
\end{verbatim}

In this example the compound term \verb.out(1,2). is passed to C++.
If we had previously called:
\begin{verbatim}
    EC_ref FromEclipse;
    result = EC_resume(FromEclipse);
\end{verbatim}
then \verb.result. would be \verb.EC_yield. and \verb.FromEclipse. would
refer to \verb.out(1,2).. If then we resumed execution with:
\begin{verbatim}
    result = EC_resume(EC_atom("ok"),FromEclipse);
\end{verbatim}
then the variable \verb.InData. on the {\eclipse} side
would be set to the atom 'ok'.


\subsection{Summary of EC_resume() arguments}

\index{passing data}
\verb.EC_resume(). can be called with two optional arguments. An
input argument that is an \verb.EC_word. and an output that is an
\verb.EC_ref..

If the input argument is omitted, input is taken as the list of posted
goals. Otherwise the input to {\eclipse} is exactly that argument.

If the output argument is present, its content depends on the value
returned by \verb.EC_resume().. If it returns \verb.EC_succeed. it is
\index{choicepoint}
the choicepoint identifier. If it returns \verb.EC_yield. It is the
first argument to the  \bipref{yield/2}{../bips/kernel/externals/yield-2.html} goal. If it returns \verb.EC_fail.
it is not modified.
%HEVEA\cutend


%---------------------------------------------------------------------
\chapter{Managing Data and Memory in Mixed-Language Applications}
\label{chapmixed}
%HEVEA\cutdef[1]{section}
%---------------------------------------------------------------------

{\eclipse} is a software engine for constraint propagation and search tasks.
As such, it represents its data in a form that is different
from how it would be represented in a traditional C/C++ program.
\index{search}
\index{garbage collection}
\index{memory management}
In particular,
the {\eclipse} data representation supports automatic memory management
and garbage collection, modifications that can be undone in a search context,
referential transparency and dynamic typing.

In a mixed-language application, there are two basic ways of communicating
information between the components coded in the different languages:
\index{passing data}
\begin{description}
\item[Conversion:] When data is needed for processing in another language,
        it can be converted to the corresponding representation.
        This technique is appropriate for simple data types (integers, strings),
        but can have a lot of overhead for complex structures.
\item[Sharing:] The bulk of the data is left in its original representation,
        referred to by a handle, and interface functions (or methods) provide
        access to its components when required.
\end{description}
Both techniques are supported by the {\eclipse}/C and {\eclipse}/C++ interface.


%--------------------------------
\section{Constructing {\eclipse} data}
%--------------------------------
\subsection{{\eclipse} atoms and functors}

\index{passing data}
\index{atom}
\index{functor}
\begin{verbatim}
    /* ECLiPSe code */
    S = book("Gulliver's Tales","Swift",hardback,fiction),
\end{verbatim}

In the above structure 'hardback' and 'fiction' are atoms. 'book'
is the functor of that structure, and it has an arity (number
of arguments) of 4.

Each functor and atom is entered into a dictionary, and is always
referred to by its dictionary entry. Two classes, \verb.EC_atom.
and \verb.EC_functor. are used to access such dictionary entries.

The 'Name' method applies to both, to get their string form.
The 'Arity' method can be used to find out how many arguments
a functor has.

\begin{verbatim}
    /* C++ code */
    EC_functor book("book",4);
    EC_atom hardback("hardback");

    if (book.Arity()) == 4) .. /* evaluates to true */
    if (book == hardback) ..   /* evaluates to false */
    s = hardback.Name();       /* like s = "hardback"; */
\end{verbatim}


\subsection{Building {\eclipse} terms}

\index{pword}
\index{EC_word}
\index{term}
The \verb.pword. C data type is used to store {\eclipse} terms. In C++ the
\verb.EC_word. data type is used. This is used for any C type as well
as for {\eclipse} structures and lists. The size remains fixed in all
cases, since large terms are constructed on the {\eclipse} global stack.

\index{garbage collection}
The consequences of this are that terms will be garbage collected
or moved so terms do not survive the execution of {\eclipse}. In
particular, one cannot build such terms asynchronously while
{\eclipse} is running, for example this precludes building terms from
within a signal handler unless it can make sure that {\eclipse} has
yielded when it is running.

\subsection{Building atomic {\eclipse} terms}
It is possible to simply cast from a number of simple C++ types to
build an \verb.EC_word. In addition, functions exist for creating
\index{logical variable}
new variables, and for the nil which terminates {\eclipse} lists.
In C++ you can just cast.
\begin{verbatim}
    /* making simple terms in C++ */
    EC_word w;
    EC_atom hardback("hardback");
    w = (EC_word) "Swift";
    w = (EC_word) hardback;
    w = (EC_word) 1.002e-7;
    w = (EC_word) 12345;
    w = (EC_word) nil();
    w = (EC_word) newvar();

    /* ECLiPSe equivalent code */
    P1 = "Swift",
    P2 = hardback,
    P3 = 1.002e-7,
    P4 = 12345,
    P5 = [],
    P6 = _,
\end{verbatim}

\subsection{Building {\eclipse} lists}
\index{list}
The \verb.list(head,tail). function builds a list out of two terms. Well
formed lists have lists as their tail term and a nil ("[]") at the end, or a
variable at the end for difference lists.

\begin{verbatim}
    /* making the list [1, "b", 3.0] in C++ */
    EC_word w = list(1, list("b", list(3.0, nil())));
\end{verbatim}

The following example shows how you can write functions to build
variable length lists.

\begin{verbatim}
/* function to build a list [n,n+1,n+2,.....,m-1,m] */
EC_word fromto(int n, int m)
{
    EC_word tail = nil();
    for(int i = m ; i >= n ; i--)
        tail = list(i,tail);
    return tail;
}
\end{verbatim}

The list is constructed starting from the end, so at all points during its
construction you have a valid term. The interface is designed to
make it hard to construct terms with uninitialised sub-terms, which is
what you would need if you were to construct the list starting with
the first elements.

\subsection{Building {\eclipse} structures}
\index{term}
\index{structure}
\index{compound term}
The \verb%term(functor,args..)% function is used to build {\eclipse}
structures. A number of different functions each with a different
number of arguments is defined so as not to disable C++ casting
which would be the case if we defined a function with variable
arguments.

\begin{verbatim}
    /* making s(1,2,3) in C++ */
    EC_functor s_3("s",3);
    EC_word w = term(s_3,1,2,3);
\end{verbatim}

The above interface is convenient for terms with small fixed arities,
for much larger terms an array based interface is provided.

\begin{verbatim}
    /* making s(1,2,..,n-1,n) */
    EC_word args[n];
    for(int i=0 ; i<n ; i++)
        args[i] = i+1;
    EC_word w = term(EC_functor("s",n),args);
\end{verbatim}


%--------------------------------
\section{Converting {\eclipse} data to C data}
%--------------------------------

\index{type testing}
There are several aspects to examining the contents of a term. These
include decomposing compound terms such as lists and structures,
converting simple terms to C data types and testing the types of terms.

The functions for decomposing and converting check that the type
is appropriate. If it is they return \verb.EC_succeed. if not
they return a negative error code.

\subsection{Converting simple {\eclipse} terms to C data types}

To convert from an {\eclipse} term to a C type you first have to
declare a variable with that type. For fixed size data types
(you can convert to \verb.double., \verb.long. and \verb.dident.
fixed size data types) you are responsible for allocating the
memory. For strings you declare a \verb.char*. variable and
on conversion it will point to the internal {\eclipse} string.
\index{string}

In the following example we see how one can try to convert to
different types. Of course normally you will know what type
you are expecting so only one of these functions need be called.
\begin{verbatim}
    EC_word term;
    double r;
    long i;
    EC_atom did;
    char *s;
    if (EC_succeed == term.is_double(&d))
        cout << d << "\n";
    else if (EC_succeed == term.is_long(&i))
        cout << i << "\n";
    else if (EC_succeed == term.is_atom(&did))
        cout << did.Name() << "\n";
    else if (EC_succeed == term.is_string(&s))
        cout << s << "\n";
    else
        cout << "not a simple type\n";
\end{verbatim}

The term is converted by the function which returns \verb.EC_success..
The functions that fail to convert will return a negative error number.

\index{string}
\index{garbage collection}
Care has to be taken with strings, these pointers point to the internal
{\eclipse} string which may move or be garbage collected during an {\eclipse}
execution. As a result if a string is to be kept permanently one should
copy it first.


\subsection{Decomposing {\eclipse} terms}

The function \verb.ec_get_arg(index,term,&subterm). is used to get
the index'th subterm of a structure. The index varies from 1 to
arity of \verb.term.. A list can also be decomposed this way, where
the head is at index 1 and the tail at index 2.
\index{list}
\index{structure}

Below we see how we would write a function to find the nth element of
a list.
\begin{verbatim}
    int nth(const int n,const EC_word list, EC_word& el)
    {
        EC_word tail = list;
        for (int i=1, i<n, i++)
            if (EC_fail == tail.arg(2,tail))
                return EC_fail;
        return tail.arg(1,el);
    }
\end{verbatim}
The above function actually is not limited to lists but could work
on any nested structure.


%--------------------------------
\section{Referring to {\eclipse} terms}
%--------------------------------

The terms constructed so far (as EC-words) have been volatile, that is they
do not survive an {\eclipse} execution (due to eg.\ garbage collection),
\index{garbage collection}
It is possible to create safe terms
that have been registered with the {\eclipse} engine and which do
survive execution. The \verb.EC_ref.  and \verb.EC_refs. classes
are provided for this purpose. \verb.EC_refs. are vectors of
safe terms.

When you declare an \verb.EC_ref. it will contain free variables.
\begin{verbatim}
EC_ref X; /* declare one free variable */
EC_refs Tasks(10); /* declare 10 free variables */
\end{verbatim}


\index{logical variables}
\verb.EC_ref.s work like logical variables. When {\eclipse} fails during search
they are reset to old values. They are always guaranteed to refer to
something i.e.\ they never contain dangling references.
If {\eclipse} backtracks to a point in the execution
older than the point at which the references were created, they
return to being free variables, or take on their initial values.

\index{references}
It is possible to declare references, giving them an initialiser
but this must be an atomic type that fits into a single word. That
restricts you to atoms, integers and nil.

You can freely assign between an \verb.EC_ref. and a \verb.EC_word..

One point to take care of is that assigning such a variable
is not like unification since assignment cannot fail. It
just overwrites the old value. Assignment is very similar
to the \bipref{setarg/3}{../bips/kernel/termmanip/setarg-3.html} built-in in the {\eclipse} language.


%---------------------------------------------------------------------
\section{Passing generic C or C++ data to {\eclipse}}
%---------------------------------------------------------------------

\index{references}
It is possible to include any C or C++ data in an {\eclipse} term. To do this
it is wrapped into a handle to tell {\eclipse} that this is external data.
You also have to supply a method table, which is a set of functions
that are called when {\eclipse} wants to make common operations that it
assumes can be done on any data (eg.\ comparing, printing).


\subsection{Wrapping and unwrapping external data in an {\eclipse} term}

To create an {\eclipse} wrapper for a C/C++ object,
the function \verb.handle(). is used. It takes a pointer to any C or C++
data, and a pointer to a suitable method table (\verb.t_ext_type. structure)
and creates an {\eclipse} handle term which refers to them.
Method tables for the common case of arrays of char, long or double
\index{array}
are predefined. For example a handle for a double array is made like this
\begin{verbatim}
    double my_array[5] = {1.1, 2.2, 3.3, 4.4, 5.5};
    EC_word w = handle(&ec_xt_double_arr, my_array);
\end{verbatim}
where \verb.ec_xt_double_arr. is a predefined method table for arrays of doubles.
To convert back from an {\eclipse} term \verb.is_handle(). is used.
The method table passed in indicates the sort of data you expect to get.
If the {\eclipse} handle contains the wrong sort, the function returns
TYPE_ERROR:
\begin{verbatim}
    if ((EC_succeed == w.is_handle(&ec_xt_double_arr, &obj))
        obj->my_method();
    else
        cerr << "unexpected type\n";
\end{verbatim}


\subsection{The method table}
\index{method table}
Apart from the predefined method tables ec_xt_double_arr,
ec_xt_long_arr and ec_xt_char_arr, new ones can easily be defined.
The address of the table is used as a type identifier, so when you
get an external object back from {\eclipse} you can check its type
to determine the kinds of operations you can do on it.
You can choose not to implement one or more of these functions, by
leaving a null pointer (\verb.(void*)0.) in its field.

\begin{verbatim}
    typedef void *t_ext_ptr;

    typedef struct {
        void       (*free)       (t_ext_ptr obj);
        t_ext_ptr  (*copy)       (t_ext_ptr obj);
        void       (*mark_dids)  (t_ext_ptr obj);
        int        (*string_size)(t_ext_ptr obj, int quoted);
        int        (*to_string)  (t_ext_ptr obj, char *buf, int quoted);
        int        (*equal)      (t_ext_ptr obj1, t_ext_ptr obj2);
        t_ext_ptr  (*remote_copy)(t_ext_ptr obj);
        EC_word    (*get)        (t_ext_ptr obj, int idx);
        int        (*set)        (t_ext_ptr obj, int idx, EC_word data);
    } t_ext_type;
\end{verbatim}

\begin{description}
\item[void free(t_ext_ptr obj)]
\index{references}
This is called by {\eclipse} if it loses a reference to the external
data. This could happen if the {\eclipse} execution were to fail
to a point before the external object was created, or if a
permanent copy was explicitly removed with built-ins like
\bipref{setval/2}{../bips/kernel/arrays/setval-2.html},
\bipref{erase/2}{../bips/kernel/record/erase-2.html}
or \bipref{bag_dissolve/2}{../bips/kernel/arrays/bag_dissolve-2.html}.
Note that an invocation of this function only means that {\em one}
reference has been deleted (while the copy function indicates that
a reference is added).

\item[t_ext_ptr copy(t_ext_ptr obj)]
This is called by {\eclipse} when it wants to make a copy of an object.
This happens when calling {\eclipse} built-ins like \bipref{setval/2}{../bips/kernel/arrays/setval-2.html} or
\bipref{recordz/2}{../bips/kernel/record/recordz-2.html} which make permanent copies of data. The return value is
the copy.
If no copy-method is specified, these operations will not be possible
with terms that contain an object of this type.
A possible implementation is to return a pointer to the original and
e.g.\ increment a reference counter (and decrement the counter in
the free-method correspondingly).

\item[void mark_dids( t_ext_ptr obj)]
This is called during dictionary garbage collection. If an external
object contains references to the dictionary (\verb.dident.) then
it needs to mark these as referenced.

\item[int string_size(t_ext_ptr obj, int quoted)]
\item[int to_string(t_ext_ptr obj,char *buf, int quoted)]
When {\eclipse} wants to print an external object it calls \verb.string_size().
to get an estimate of how large the string would be that represents it.
This is used by {\eclipse} to allocate a buffer. The string representation must
be guaranteed to fit in the buffer.

Finally  the \verb.to_string(). function is called. This should write the
string representation of the object into the buffer, and return the actual
size of that string.

\item[int equal(t_ext_ptr obj1, t_ext_ptr obj2)]
This is called when two external objects are unified or compared.
Prolog views the external object as a ground, atomic element.
It should return non-zero if the objects are considered equal.

\item[t_ext_ptr remote_copy(t_ext_ptr obj)]
This is called by parallel {\eclipse} when it needs to make a copy of an
object in another worker. If the object is in shared memory, this method
can be the same as the copy method.

\item[EC_Word get(t_ext_ptr obj, int idx)]
Returns the value of a field of the C++ object.
This methods gets invoked when the {\eclipse} predicate \bipref{xget/3}{../bips/kernel/externals/xget-3.html} is called.
The mapping of index values to fields is defined by the get/set-method pair.

\item[int set(t_ext_ptr obj, int idx, EC_word data)]
Set the value of a field of the C++ object.
This methods gets invoked when the {\eclipse} predicate \bipref{xset/3}{../bips/kernel/externals/xset-3.html} is called.
The mapping of index values to fields is defined by the get/set-method pair.
\end{description}

Example of the simplest possible user-defined method table:
\begin{verbatim}
    t_ext_type my_type = {NULL,NULL,NULL,NULL,NULL,NULL,NULL,NULL,NULL};
    my_struct data_in;
    ...
    // creating a handle for data_in
    EC_word w = handle(&my_type, &data_in);
    ...
    // checking a handle and extracting the data pointer
    my_struct *data_out;
    if ((EC_succeed == w.is_handle(&my_type, &data_out))
        data_out->my_method();
    else
        cerr << "unexpected type\n";
\end{verbatim}

%HEVEA\cutend


