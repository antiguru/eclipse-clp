% BEGIN LICENSE BLOCK
% Version: CMPL 1.1
%
% The contents of this file are subject to the Cisco-style Mozilla Public
% License Version 1.1 (the "License"); you may not use this file except
% in compliance with the License.  You may obtain a copy of the License
% at www.eclipse-clp.org/license.
% 
% Software distributed under the License is distributed on an "AS IS"
% basis, WITHOUT WARRANTY OF ANY KIND, either express or implied.  See
% the License for the specific language governing rights and limitations
% under the License. 
% 
% The Original Code is  The ECLiPSe Constraint Logic Programming System. 
% The Initial Developer of the Original Code is  Cisco Systems, Inc. 
% Portions created by the Initial Developer are
% Copyright (C) 2006 Cisco Systems, Inc.  All Rights Reserved.
% 
% Contributor(s): 
% 
% END LICENSE BLOCK

%%%%%%%%%%%%%%%%%%%%%%%%%%%%%%%%%%%%%%%%%%%%%%%%%%%%%%%%%%%%%%%%%%%%%%%%%%
% {\eclipse} Documentation
%
% $Id: embfunc.tex,v 1.1 2006/09/23 01:48:59 snovello Exp $
%
%%%%%%%%%%%%%%%%%%%%%%%%%%%%%%%%%%%%%%%%%%%%%%%%%%%%%%%%%%%%%%%%%%%%%%%%%%
\chapter{External Predicates in C and C++}
\label{chapext}
%HEVEA\cutdef[1]{section}

\section{Coding External Predicates}

External Predicates are C/C++ functions that can be called like predicates
from \eclipse.
Two following extra interface functions are provided for this purpose:
\begin{description}
\item[EC_word EC_arg(int i)]
	returns the i'th argument of the predicate call.

\item[pword ec_arg(int i)]\ \\
	same for C.

\item[int unify(EC_word, EC_word)]\ \\
	unify two pwords.  The return code indicates success or
	failure.  Note however, that if attributed variables are
	involved, their handlers have not been invoked yet (this
	happens after the external predicate returns).

\item[int EC_word::unify(EC_word)]\ \\
	same as method.

\item[int ec_unify(pword, pword)]\ \\
	same for C.
\end{description}

Apart from that, all functions for constructing, testing and decomposing
{\eclipse} data can be used in writing the external predicate
(see chapter \ref{chapmixed}).
Here are two examples working with lists, the first one constructing
a list in C:
\begin{quote}\begin{verbatim}
#include "eclipse.h"
int
p_string_to_list()         /* string_to_list(+String, -List) */
{
    pword  list;
    char *s;
    long len;
    int res;

    res = ec_get_string_length(ec_arg(1), &s, &len);
    if (res != PSUCCEED) return res;

    list = ec_nil();    /* build the list backwards */
    while (len--)
        list = ec_list(ec_long(s[len]), list);

    return ec_unify(ec_arg(2), list);
}
\end{verbatim}\end{quote}
The next example uses an input list of integers and sums up the numbers.
It is written in C++:
\begin{quote}\begin{verbatim}
#include "eclipseclass.h"
extern "C" int
p_sumlist()
{
    int res;
    long x, sum = 0;
    EC_word list(EC_arg(1));
    EC_word car,cdr;

    for ( ; list.is_list(car,cdr) == EC_succeed; list = cdr)
    {
        res = car.is_long(&x);
        if (res != EC_succeed) return res;
        sum += x;
    }
    res = list.is_nil();
    if (res != EC_succeed) return res;
    return unify(EC_arg(2), EC_word(sum));
}
\end{verbatim}\end{quote}
The source code of these examples can be found in directory
doc/examples within the {\eclipse} installation.

\section{Compiling and loading}
It is strongly recommended to copy the makefile "Makefile.external"
provided in your  installation directory under lib/\$ARCH and adapt it
for your purposes.
\index{Makefile}
If the makefile is not used, the command to compile a C source
with {\eclipse} library calls looks something like this:
\begin{verbatim}
    % cc -G -I/usr/local/eclipse/include/sparc_sunos5
                -o eg_externals.so eg_externals.c
\end{verbatim}
or
\begin{verbatim}
    % cc -shared -I/usr/local/eclipse/include/i386_linux
                -o eg_externals.so eg_externals.c
\end{verbatim}
If the external is to be used in a standalone {\eclipse},
it is possible to dynamically load it
using the \bipref{load/1}{../bips/kernel/externals/load-1.html}
predicate:
\begin{verbatim}
    load("eg_externals.so")
\end{verbatim}
On older UNIX platforms without dynamic loading, the following method
may work. Compile the source using
\begin{verbatim}
    % cc -c -I/usr/local/eclipse/include/sparc_sunos5 eg_externals.c
\end{verbatim}
and load it with a command like
\begin{verbatim}
    load("eg_externals.o -lg -lm")
\end{verbatim}
The details may vary depending on what compiler and operating system
you use.  Refer to the Makefile.external for details.

Once the object file containing the C function has been loaded into
{\eclipse}, the link between the function and a predicate name
is made with
\bipref{external/2}{../bips/kernel/externals/external-2.html}
\begin{verbatim}
    external(sumlist/2, p_sumlist)
\end{verbatim}
The new predicate can now be called like other predicates.
Note that the 
\bipref{external/2}{../bips/kernel/externals/external-2.html}
declaration must precede any call to the declared predicate,
otherwise the {\eclipse} compiler will issue an {\em inconsistent
redefinition} error. Alternatively, the
\bipref{external/1}{../bips/kernel/externals/external-1.html}
forward declaration can be used to prevent this.

If the external is needed in the context of an {\eclipse} which is itself
embedded in a C/C++ host program, then the external code can be
compiled/linked together with the host program, and the link between function
and predicate name can alternatively be made by calling the C function
ec_external(), e.g.
\begin{verbatim}
    ec_external(ec_did("sumlist",2), p_sumlist, ec_did("eclipse"))
\end{verbatim}
This must be done after the embedded {\eclipse} has been initialised
(and after the module that is supposed to contain the external predicate
has already been created).


\section{Restrictions and Recommendations}

It is neither supported nor recommended practice to call ec_resume()
from within an external predicate, because this would invariably lead
to programs which are hard to understand and to get right.

Currently, it is also not possible to post goals from within an
external predicate, but that is a sensible programming style and
will be supported in forthcoming releases.
Posting events however is already possible now.

%HEVEA\cutend
